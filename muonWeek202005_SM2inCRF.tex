\documentclass[usenamees,dvipsnames]{beamer}
\usetheme{Madrid}
\usecolortheme{spruce}
\definecolor{green(pigment)}{rgb}{0.0, 0.65, 0.31}
\setbeamercolor*{item}{fg=green(pigment)}
\definecolor{Green}{rgb}{0.00, 1.00, 0.00}
\definecolor{Red}{rgb}{1.00, 0.00, 0.00}
\definecolor{Blue}{rgb}{0.00, 0.00, 1.00}
\usepackage[utf8]{inputenc}
\usepackage{amsmath}
\usepackage{amsfonts}
\usepackage{amssymb}
\usepackage[german]{babel}
\usepackage{graphicx}
\usepackage{rotating}
\usepackage{textcomp}
\usepackage{multirow,bigdelim,dcolumn,booktabs}
%\usepackage{beamerthemeshadow}
\usepackage{subfigure} 
\usepackage{siunitx}
\usepackage{appendixnumberbeamer}
\usepackage{hyperref}
\usepackage{mdframed}
\usepackage{xcolor}
\usepackage[absolute,overlay]{textpos}
%\beamersetuncovermixins{\opaqueness<1>{25}}{\opaqueness<2->{15}}
\beamertemplatenavigationsymbolsempty

\usepackage{tikz}
\usetikzlibrary{decorations.text}
%\usetikzlibrary{trees}
\usetikzlibrary{decorations.pathmorphing}
\usetikzlibrary{decorations.markings}
\usetikzlibrary{patterns}

\graphicspath{
	{pictures/}
	{pictures/passivationComparison/}
%	{/home/m/Maximilian.Herrmann/Bilder/forLatex/plots/}
%	{/home/m/Maximilian.Herrmann/Bilder/forLatex/sketches/}
%	{/home/m/Maximilian.Herrmann/Bilder/forLatex/pictures/}
}

\title[SM2 in CRF]{SM2 Micromegas Modules in the Cosmic Ray Facility}  
\author[M. Herrmann]{Maximilian Herrmann}
\institute[LMU Munich]{Ludwig-Maximilians-Universit\"at M\"unchen - Lehrstuhl Schaile}
\date[12.05.2020]{Muon Week 12.05.2020} 

\begin{document}

\frame{
	\titlepage
}

\frame{
	\frametitle{Validation using Cosmic Muons - \textbf{C}osmic \textbf{R}ay \textbf{F}acility}
	
	\begin{columns}
		\column{.40\textwidth}
			\centering
			\includegraphics[width=0.95\textwidth]{CRFprinciple3-crop.pdf}
		\column{.60\textwidth}
			\centering
			\scriptsize
			\only<1>{
				\begin{tabular}{ll}
					trigger & scintillator hodoscope
					\\
					& resolution $\sim$ \SI{10}{cm}
					\\
					track reconstruction & Monitored Drift Tubes (MDTs)
					\\
					& resolution $\sim$ \SI{0.2}{mm}
					\\
					active area & \SI{2.2}{m} $\times$ \SI{4}{m}
					\\
					angular acceptance & $\pm$ \SI{30}{\degree}
					\\
					energy cut & iron plate $\to E_{\mu} > $ \SI{600}{MeV} 
					\\
					readout (full module) & 12288 channels
					\\
					readout rate & 100 Hz (online zerosuppression)
				\end{tabular}
			}
	\end{columns}
	\vspace{5mm}		
		
	\begin{columns}
		\column{.50\textwidth}
			\centering
			\includegraphics[width=0.75\textwidth]{CRFwModuleNdoublet.jpg}
		\column{.50\textwidth}
			\centering
			\includegraphics[width=0.85\textwidth]{moduleInCRF.jpg}
	\end{columns}
}

%\frame{\frametitle{\large Pulse Height and Efficiency (Ar:CO$_2$ 93:7 vol\%, $U_{\mathrm{amp}} =$ \SI{580}{V})}
%	\footnotesize
%	
%	\begin{columns}
%		\column{.5\textwidth}
%			\centering
%			full area map
%		\column{.5\textwidth}
%			\centering
%			VS amplification voltage
%	\end{columns}
%	\vspace{2mm}
%
%	\begin{columns}
%		\column{.5\textwidth}
%			\centering
%			\rotatebox{90}{$\;\;\;\;\;\;\;\;\;\;\;\;$\textbf{signal height}}
%			\includegraphics[width=0.85\textwidth]{MMS200024L1gain.png}
%		\column{.5\textwidth}
%			\centering
%			\includegraphics[width=0.8\textwidth]{MMS200024L1ClusterQampScan.png}
%	\end{columns}
%	\vspace{3mm}
%	
%	\begin{columns}
%		\column{.5\textwidth}
%			\centering
%			\rotatebox{90}{$\;\;\;\;\;\;\;\;\;\;\;\;$\textbf{efficiency}}
%			\includegraphics[width=0.85\textwidth]{MMS200024L1efficiency.png}
%		\column{.5\textwidth}
%			\centering
%			\includegraphics[width=0.8\textwidth]{MMS200024L1ampScan.png}
%	\end{columns}
%	
%}

\frame{\frametitle{\large Pulse Height and Efficiency (Ar:CO$_2$ 93:7 vol\%) Module 21}

	\begin{columns}
		\column{.25\textwidth}
			\centering
			\includegraphics[width=0.9\textwidth]{MMS200021L1gain.png}
		\column{.25\textwidth}
			\centering
			\includegraphics[width=0.9\textwidth]{MMS200021L2gain.png}
		\column{.25\textwidth}
			\centering
			\includegraphics[width=0.9\textwidth]{MMS200021L3gain.png}
		\column{.25\textwidth}
			\centering
			\includegraphics[width=0.9\textwidth]{MMS200021L4gain.png}
	\end{columns}
	
	\vspace{2mm}
	
	\begin{columns}
		\column{.25\textwidth}
			\centering
			\includegraphics[width=0.9\textwidth]{MMS200021L1efficiency.png}
		\column{.25\textwidth}
			\centering
			\includegraphics[width=0.9\textwidth]{MMS200021L2efficiency.png}
		\column{.25\textwidth}
			\centering
			\includegraphics[width=0.9\textwidth]{MMS200021L3efficiency.png}
		\column{.25\textwidth}
			\centering
			\includegraphics[width=0.9\textwidth]{MMS200021L4efficiency.png}
	\end{columns}
		
	\vspace{2mm}
	
	\begin{columns}
		\column{.25\textwidth}
			\centering
			\includegraphics[width=0.9\textwidth]{MMS200021L1ClusterQampScan.png}
		\column{.25\textwidth}
			\centering
			\includegraphics[width=0.9\textwidth]{MMS200021L2ClusterQampScan.png}
		\column{.25\textwidth}
			\centering
			\includegraphics[width=0.9\textwidth]{MMS200021L3ClusterQampScan.png}
		\column{.25\textwidth}
			\centering
			\includegraphics[width=0.9\textwidth]{MMS200021L4ClusterQampScan.png}
	\end{columns}
		
	\vspace{2mm}
	
	\begin{columns}
		\column{.25\textwidth}
			\centering
			\includegraphics[width=0.9\textwidth]{MMS200021L1ampScan.png}
		\column{.25\textwidth}
			\centering
			\includegraphics[width=0.9\textwidth]{MMS200021L2ampScan.png}
		\column{.25\textwidth}
			\centering
			\includegraphics[width=0.9\textwidth]{MMS200021L3ampScan.png}
		\column{.25\textwidth}
			\centering
			\includegraphics[width=0.9\textwidth]{MMS200021L4ampScan.png}
	\end{columns}
	
}

\frame{\frametitle{Typical Pulse-Height Distributions (M27)}
	\footnotesize

	\begin{columns}
		\column{.6\textwidth}
			\centering
			\includegraphics[width=0.9\textwidth]{m27_eo_580V_chargeVSstrip.png}
		\column{.4\textwidth}
			\centering
			single APV (128 strips)
			\vspace{2mm}
			
			\includegraphics[width=0.95\textwidth]{m27_eo_580V_stripCharge.pdf}
	\end{columns}

	\vspace{3mm}

	\begin{columns}
		\column{.33\textwidth}
			\centering
			efficient cluster
			\vspace{2mm}
			
			\includegraphics[width=\textwidth]{m27_ei_580V_clusterQboards.pdf}
		\column{.33\textwidth}
			\centering
			mean map
			\vspace{2mm}
			
			\includegraphics[width=\textwidth]{m27_ei_580V_meanClusterQmap.pdf}
		\column{.33\textwidth}
			\centering
			MPV map
			\vspace{2mm}
			
			\includegraphics[width=\textwidth]{m27_ei_580V_MPVclusterQmap.pdf}
	\end{columns}
		
}
	
\frame{\frametitle{CRF measurements for Pulse-Height Comparisons}
	\scriptsize
	\vspace{5mm}
	
%	\hspace{-5mm}
	\setlength\tabcolsep{1.5pt}
	\begin{tabular}{ccccccccc}
		\hline
		\hline
		module &  & \hspace{1mm} & \multicolumn{2}{c}{before} & \hspace{1mm} & \multicolumn{2}{c}{after} & modification
		\\
		 & voltage [V] & \hspace{1mm} & measurement & events ($10^6$) & \hspace{1mm} & measurement & events ($10^6$) &
		\\
		\hline
		12 & 570 & & 19.06.-20:09 & 4.8 & & 30.08.-11:29 & 9.8 & passivated
		\\
		13 & 570 & & 28.08.-09:32 & 3.9 & & 02.10.-16:34 & 7.0 & taped
		\\
		17 & 570 & & 18.09.-08:12 & 3.9 & & 12.10.-18:26 & {\color{red}0.5} & taped
		\\
		18 & 580 & & 09.09.-20:01 & N/A & & 07.10.-19:52 & 4.6 & taped
		\\
		\hline
		\hline
	\end{tabular}
	\vspace{5mm}

	problems : 
	\begin{itemize}
		\item[]
			m12 : upside down (before) , flushing (after) , other mesh on outer layers
		\item[]
			m13 : slowcontrol stopped (after)
		\item[]
			m17 : not all sectors at nominal voltage (before)
		\item[]
			m18 : wrongly connected $\Rightarrow$ mapping has to be adjusted (before)
		\item[]
			for all measurements : pressure yet not considered
	\end{itemize}

}

\frame{
	\textbf{M12 @ 570 V}
	\centering 
	
	\textbf{before passivation} 
	\vspace{1mm}
	
	\begin{columns}
		\column{.25\textwidth}
			\centering
			\includegraphics[width=\textwidth]{M12_570before/MMS200012L1gain.png}
		\column{.25\textwidth}
			\centering
			\includegraphics[width=\textwidth]{M12_570before/MMS200012L2gain.png}
		\column{.25\textwidth}
			\centering
			\includegraphics[width=\textwidth]{M12_570before/MMS200012L3gain.png}
		\column{.25\textwidth}
			\centering
			\includegraphics[width=\textwidth]{M12_570before/MMS200012L4gain.png}
	\end{columns}
	\vspace{1mm}
	918 - 920 mBar
	\vspace{5mm}
	
	\textbf{after passivation} 
	\vspace{1mm}
	
	\begin{columns}
		\column{.25\textwidth}
			\centering
			\includegraphics[width=\textwidth]{M12_570after/MMS200012L1gain.png}
		\column{.25\textwidth}
			\centering
			\includegraphics[width=\textwidth]{M12_570after/MMS200012L2gain.png}
		\column{.25\textwidth}
			\centering
			\includegraphics[width=\textwidth]{M12_570after/MMS200012L3gain.png}
		\column{.25\textwidth}
			\centering
			\includegraphics[width=\textwidth]{M12_570after/MMS200012L4gain.png}
	\end{columns}
	\vspace{1mm}
	927 - 933 mBar

}

\frame{
	\textbf{M13 @ 570 V}
	\centering 
	
	\textbf{before taping}  
	\vspace{1mm}
	
	\begin{columns}
		\column{.25\textwidth}
			\centering
			\includegraphics[width=\textwidth]{M13_570before/MMS200013L1gain.png}
		\column{.25\textwidth}
			\centering
			\includegraphics[width=\textwidth]{M13_570before/MMS200013L2gain.png}
		\column{.25\textwidth}
			\centering
			\includegraphics[width=\textwidth]{M13_570before/MMS200013L3gain.png}
		\column{.25\textwidth}
			\centering
			\includegraphics[width=\textwidth]{M13_570before/MMS200013L4gain.png}
	\end{columns}
	\vspace{1mm}
	927 - 929 mBar
	\vspace{5mm}
	
	\textbf{after taping}  
	\vspace{1mm}
	
	\begin{columns}
		\column{.25\textwidth}
			\centering
			\includegraphics[width=\textwidth]{M13_570after/MMS200013L1gain.png}
		\column{.25\textwidth}
			\centering
			\includegraphics[width=\textwidth]{M13_570after/MMS200013L2gain.png}
		\column{.25\textwidth}
			\centering
			\includegraphics[width=\textwidth]{M13_570after/MMS200013L3gain.png}
		\column{.25\textwidth}
			\centering
			\includegraphics[width=\textwidth]{M13_570after/MMS200013L4gain.png}
	\end{columns}
	\vspace{1mm}
	922 - 931 mBar

}

\frame{
	\textbf{M17 @ 570 V}
	\centering 
	
	\textbf{before taping}  
	\vspace{1mm}
	
	\begin{columns}
		\column{.25\textwidth}
		\centering
		\includegraphics[width=\textwidth]{M17_570before/MMS200017L1gain.png}
		\column{.25\textwidth}
		\centering
		\includegraphics[width=\textwidth]{M17_570before/MMS200017L2gain.png}
		\column{.25\textwidth}
		\centering
		\includegraphics[width=\textwidth]{M17_570before/MMS200017L3gain.png}
		\column{.25\textwidth}
		\centering
		\includegraphics[width=\textwidth]{M17_570before/MMS200017L4gain.png}
	\end{columns}
	\vspace{1mm}
	934 - 936 mBar
	\vspace{5mm}
	
	\textbf{after taping}  
	\vspace{1mm}
	
	\begin{columns}
		\column{.25\textwidth}
		\centering
		\includegraphics[width=\textwidth]{M17_570after/MMS200017L1gain.png}
		\column{.25\textwidth}
		\centering
		\includegraphics[width=\textwidth]{M17_570after/MMS200017L2gain.png}
		\column{.25\textwidth}
		\centering
		\includegraphics[width=\textwidth]{M17_570after/MMS200017L3gain.png}
		\column{.25\textwidth}
		\centering
		\includegraphics[width=\textwidth]{M17_570after/MMS200017L4gain.png}
	\end{columns}
	\vspace{1mm}
	926 - 927 mBar
	\vspace{6mm}

	stereo\_out board8 right repaired after this measurement

}

\frame{
	\textbf{M18 @ 580 V}
	\centering 

	\textbf{before taping} 
	\vspace{1mm}
	
	\begin{columns}
		\column{.25\textwidth}
			\centering
			\includegraphics[width=\textwidth]{M18_580before_remapped/MMS200018L1gain.png}
		\column{.25\textwidth}
			\centering
			\includegraphics[width=\textwidth]{M18_580before_remapped/MMS200018L2gain.png}
		\column{.25\textwidth}
			\centering
			\includegraphics[width=\textwidth]{M18_580before_remapped/MMS200018L3gain.png}
		\column{.25\textwidth}
			\centering
			\includegraphics[width=\textwidth]{M18_580before_remapped/MMS200018L4gain.png}
	\end{columns}
	\vspace{1mm}
	N/A
	\vspace{5mm}

	\textbf{after taping}  
	\vspace{1mm}
	
	\begin{columns}
		\column{.25\textwidth}
			\centering
			\includegraphics[width=\textwidth]{M18_580after/MMS200018L1gain.png}
		\column{.25\textwidth}
			\centering
			\includegraphics[width=\textwidth]{M18_580after/MMS200018L2gain.png}
		\column{.25\textwidth}
			\centering
			\includegraphics[width=\textwidth]{M18_580after/MMS200018L3gain.png}
		\column{.25\textwidth}
			\centering
			\includegraphics[width=\textwidth]{M18_580after/MMS200018L4gain.png}
	\end{columns}
	\vspace{1mm}
	925 - 929 mBar

}
	
\frame{\frametitle{Pulse-Height Pressure Dependence (M3 long term test)}
%	\centering

	\begin{columns}
		\scriptsize
		\column{.4\textwidth}
			\centering
			pressure
			
			\includegraphics[width=0.95\textwidth]{m3_pressureVStime_Xmas.pdf}
			
			\vspace{2mm}
			mean cluster charge
			
			\includegraphics[width=0.95\textwidth]{m3_eo_b8_meanClusterQvsTime_Xmas.pdf}
		\column{.6\textwidth}
			\centering
			\includegraphics[width=1.0\textwidth]{m3_570V_xmas_meanClusterQvsPressure_allBoards_otherColors.pdf}
	\end{columns}
	
	\vspace{5mm}
		
	$\Rightarrow$ clear pressure dependence, as expected
			
	\vspace{3mm}
	
	$\Rightarrow$ dependence on pillar-height NOT straight forward
	
	\vspace{5mm}
	\scriptsize
	(pressure biased: offset about 30 mBar)
}
	
\frame{\frametitle{Pulse-Height Amplification-Gap Dependence}
%	\centering
	\small
	
	\begin{columns}
		\column{.5\textwidth}
			\centering
			M8 and Eta3-Doublet
			
			(same measurement period)
		\column{.5\textwidth}
			\centering
			modules 1,3,6,7
	\end{columns}
	\vspace{3mm}

	\begin{columns}
		\column{.45\textwidth}
			\centering
			\includegraphics[width=0.95\textwidth]{m8_9307_A570V_C300V_clusterQvsPillarHeight.pdf}
		\column{.55\textwidth}
			\centering
			\includegraphics[width=0.95\textwidth]{meanClusterQvsPillarHeight_etaNstereo_wFit.png}
	\end{columns}
	
	\vspace{5mm}
	
	$\Rightarrow$ for a single measurement period a good correlation is given
		
	\vspace{3mm}
	
	$\Rightarrow$ comparison of different periods require pressure compensation
	
}
	
\frame{
	\centering \textbf{cluster charge} @ 580 V
	
	(modules 4 , 13 , 14, 15 , 17 , 18 , 20 - 25)
%	\vspace{2mm}
	
	\begin{columns}
		\column{.04\textwidth}
		\column{.48\textwidth}
			\centering
			mean
		\column{.48\textwidth}
			\centering
			MPV
	\end{columns}
%	\vspace{1mm}
	
	\begin{columns}
		\column{.04\textwidth}
			\rotatebox{90}{pillar height}
		\column{.48\textwidth}
			\centering
			\includegraphics[width=\textwidth]{meanClusterQvsPillarHeight_passivatedModules.pdf}
		\column{.48\textwidth}
			\centering
			\includegraphics[width=\textwidth]{MPVclusterQvsPillarHeight_passivatedModules.pdf}
	\end{columns}
%	\vspace{3mm}

	\begin{columns}
		\column{.04\textwidth}
			\rotatebox{90}{pressure}
		\column{.48\textwidth}
			\centering
			\includegraphics[width=\textwidth]{meanClusterQvsPressure_passivatedModules.pdf}
		\column{.48\textwidth}
			\centering
			\includegraphics[width=\textwidth]{MPVclusterQvsPressure_passivatedModules.pdf}
	\end{columns}
%	\vspace{3mm}

	\begin{columns}
		\column{.04\textwidth}
			\rotatebox{90}{product}
		\column{.48\textwidth}
			\centering
			\includegraphics[width=\textwidth]{meanClusterQvsPressurePillarHeight_passivatedModules.pdf}
		\column{.48\textwidth}
			\centering
			\includegraphics[width=\textwidth]{MPVclusterQvsPressurePillarHeight_passivatedModules.pdf}
	\end{columns}

}

\frame{\frametitle{Alignment Reconstruction Using Cosmic Muon Tracks}
	\footnotesize		
		
	\begin{columns}
		\column{.50\textwidth}
			\centering
			difference of reconstructed hit 
			
			and reference track interpolation
			\vspace{1mm}
			
			\includegraphics[width=0.8\textwidth]{m1_ei_residaul_nearZero.pdf}
			\vspace{1mm}
				
			deviations of inter-strip distance
			\vspace{1mm}
			
			\includegraphics[width=0.8\textwidth]{eta_in_resMeanVSmdtY_woAdapterBoard_wCorPitch_wCenter.png}
		\column{.50\textwidth}
			\centering
			board rotation and strip deformations
			\vspace{1mm}
			
			\includegraphics[width=0.8
			\textwidth]{m3_eo_resMeanVSscinX_allBoards.pdf}
			\vspace{1mm}
				
			bend strip shape
			\vspace{1mm}
				
			\includegraphics[width=0.95\textwidth]{ROboardsBendStrips-crop.png}
			\vspace{13mm}
	\end{columns}
}

\frame{\frametitle{\large Alignment Investigation Eta-panel 3 $\Rightarrow$ Consistent Reconstruction}
	\centering
	
	\begin{columns}
		\column{.05\textwidth}
		\column{.3\textwidth}
			\centering
			September 2018, 
			
			in M3
		\column{.3\textwidth}
			\centering
			January 2019, 
			
			as doublet
		\column{.3\textwidth}
			\centering
			May 2019, 
			
			as doublet
	\end{columns}
	\vspace{2mm}
	
	\begin{columns}
		\column{.05\textwidth}
			\rotatebox{90}{board 8}
		\column{.3\textwidth}
			\centering
			\includegraphics[width=0.9\textwidth]{eta3inM3_b8_resMeanVsscinX.pdf}
		\column{.3\textwidth}
			\centering
			\includegraphics[width=0.9\textwidth]{eta3withM6_b8_resMeanVsscinX.pdf}
		\column{.3\textwidth}
			\centering
			\includegraphics[width=0.9\textwidth]{eta3withM8_b8_resMeanVsscinX.pdf}
	\end{columns}
	\vspace{2mm}
	
	\begin{columns}
		\column{.05\textwidth}
			\rotatebox{90}{board 7}
		\column{.3\textwidth}
			\centering
			\includegraphics[width=0.9\textwidth]{eta3inM3_b7_resMeanVsscinX.pdf}
		\column{.3\textwidth}
			\centering
			\includegraphics[width=0.9\textwidth]{eta3withM6_b7_resMeanVsscinX.pdf}
		\column{.3\textwidth}
			\centering
			\includegraphics[width=0.9\textwidth]{eta3withM8_b7_resMeanVsscinX.pdf}
	\end{columns}
	\vspace{2mm}
	
	\begin{columns}
		\column{.05\textwidth}
			\rotatebox{90}{board 6}
		\column{.3\textwidth}
			\centering
			\includegraphics[width=0.9\textwidth]{eta3inM3_b6_resMeanVsscinX.pdf}
		\column{.3\textwidth}
			\centering
			\includegraphics[width=0.9\textwidth]{eta3withM6_b6_resMeanVsscinX.pdf}
		\column{.3\textwidth}
			\centering
			\includegraphics[width=0.9\textwidth]{eta3withM8_b6_resMeanVsscinX.pdf}
	\end{columns}

}

\frame{\frametitle{Alignment Comparison Production Modules}
	\footnotesize

	\begin{columns}
		\column{.5\textwidth}
			\centering
			shifts
			
			\includegraphics[width=0.95\textwidth]{shiftsWRTboard7_summary.pdf}
		\column{.5\textwidth}
			\centering
			rotations
			
			\includegraphics[width=0.95\textwidth]{rotationsWRTboard7_summary_100umPERm.pdf}
	\end{columns}

	\vspace{3mm}

	\begin{columns}
		\column{.5\textwidth}
			\centering
			pitch deviations
			
			\includegraphics[width=0.95\textwidth]{pitchDeviations_summary.pdf}
		\column{.5\textwidth}
			\centering
			bendings
			
			\includegraphics[width=0.95\textwidth]{bendings_summary.pdf}
	\end{columns}
		
}

\frame{\frametitle{Timing Investigation using FEC-APV-Readout}
	\footnotesize

	\begin{columns}
		\column{.5\textwidth}
			\centering
			signal evaluation points
			
			\includegraphics[width=\textwidth]{m16_eo_FEC0_earliestStripTime_signalScan.pdf}
		\column{.5\textwidth}
			\centering
			capacitive coupling correction
			
			\includegraphics[width=\textwidth]{m16_eo_FEC0_earliestStripTime_CCCtest.pdf}
	\end{columns}

	\vspace{3mm}

	\begin{columns}
		\column{.5\textwidth}
			\centering
			effect of jitter correction
			
			\includegraphics[width=\textwidth]{m16_eo_FEC0_earliestStripTime_jitterScan_zoomed.pdf}
		\column{.5\textwidth}
			\centering
			magnitude of jitter correction
			
			\includegraphics[width=\textwidth]{m16_earliestStripTimeWidth_subtractedVSwoJitter_eachAdapterBoard.pdf}
	\end{columns}

	\vspace{3mm}

	$\Rightarrow$ narrowest time distribution : inflection , without CCC , jitter subtracted
	
	\vspace{2mm}
	
	$\Rightarrow$ jitter correction : about 3\% or about 0.039 $\cdot$ \SI{25}{ns}

}

\frame{\frametitle{Timing Investigation using FEC-APV-Readout}
	\centering 
		
	layer cabeling
	
	\vspace{1mm}
	
	\begin{columns}
		\column{.5\textwidth}
			\centering
			odd (stereo\_out , eta\_in)
			
			\includegraphics[width=0.7\textwidth]{ROboardFECmappingColored-crop.pdf}
		\column{.5\textwidth}
			\centering
			even (stereo\_in , eta\_out)
			
			\includegraphics[width=0.7\textwidth]{ROboardFECmappingColored_odd-crop.pdf}
	\end{columns}
		
	\vspace{4mm}
	
	\begin{columns}
		\column{.5\textwidth}
			\centering
			all layer FEC3 (same direction)
			
			\includegraphics[width=\textwidth]{m16_earliestStripTimeVSposition_allLayers_FEC3.pdf}
		\column{.5\textwidth}
			\centering
			alternating FECs (left-right)
			
			\includegraphics[width=\textwidth]{m16_earliestStripTimeVSposition_allLayers_alternatingFECs.pdf}
	\end{columns}

}

\frame{
	\centering 
	expected : \SI{6}{ns / m} $=$ 0.24 $\cdot$ \SI{25}{ns / m}
	
	opposite readout $\Rightarrow$ \textbf{0.48} $\cdot$ \SI{25}{ns / m}
	
	\vspace{8mm}
	
	\begin{columns}
		\column{.5\textwidth}
			\centering
			layer time difference
			
			\includegraphics[width=\textwidth]{m16_eo_FEC4_earliestTimeDiffVSposition_sameNotherFEC_fit.pdf}
		\column{.5\textwidth}
			\centering
			signal propagation time
			
			\includegraphics[width=\textwidth]{m16_signalPropagationTime_layerVSfec.pdf}
	\end{columns}
	
	\vspace{8mm}

	$\Rightarrow$ mean reconstructed : $\left( \mathrm{\textbf{0.10}} \pm 0.03 \right) \cdot$ \SI{25}{ns / m}

}

\frame{\frametitle{Summary}
	\scriptsize
	investigation of the full active area using reference tracker for cosmic muons
	\begin{itemize}
		\item
			information about pulse-height and efficiency send together with modules to CERN
		\item
			comparison of measurements for same modules show
			\begin{itemize}
				\scriptsize
				\item
					similar features $\Rightarrow$ consistent reconstruction
				\item
					inhomogeneities arise mainly due to pillar-height variations
			\end{itemize}
		\item
			pulse-height dependence:
			\begin{itemize}
				\scriptsize
				\item
					amplification gap size (pillar-height)
				\item
					pressure
				\item
					micro-mesh
			\end{itemize}
		\item
			timing-evaluation difficult using local setup
			
			$\Rightarrow$ systematic effects under investigation
	\end{itemize}

	\centering
	\includegraphics[width=0.6\textwidth]{CRFintegratedSeriesProduction_moduleNumbers_finish.pdf}
	
}

\appendix

\frame[noframenumbering]{
	\Huge
	\centering
	Backup
}

\frame{\frametitle{SRS-based Data-Acquisition}

	\begin{columns}
		\column{.25\textwidth}
			\centering
			\includegraphics[width=0.9\textwidth]{APV25.png}
		\column{.25\textwidth}
			\centering
			\includegraphics[width=0.8\textwidth]{FECcabling.jpg}
		\column{.25\textwidth}
			\centering
			\includegraphics[width=0.9\textwidth]{eventdisplay_m5_570V.pdf}
		\column{.25\textwidth}
			\centering
			\includegraphics[width=0.9\textwidth]{pulseheight_differentSignaltimes.pdf}
	\end{columns}
	
	\vspace{4mm}
	
	\begin{columns}
		\column{.5\textwidth}
			\centering
			\includegraphics[width=\textwidth]{CRF_TDAQ-crop.pdf}
		\column{.5\textwidth}
			\centering
			\includegraphics[width=\textwidth]{CRFdataStreams-crop.pdf}
	\end{columns}

}

\frame{\frametitle{Precision Reconstruction of Geometrical Properties}
	Boards of the Readout Anode can have:

	\hspace{20mm} $\Rightarrow$ {\color{red}rotations} and {\color{orange}shifts} w.r.t. each other
	
	\hspace{20mm} $\Rightarrow$ {\color{blue}non-straight strip shape} and {\color{OliveGreen}pitch deviation}
	
	\vspace{3mm}
	
	\begin{columns}
		\column{.50\textwidth}
			{\centering
			\textbf{residual mean VS \\ non-precision direction}
			
			\includegraphics[width=0.80\textwidth]{stripShapeNrotation-crop.pdf}
			}
			
			\scriptsize
			$\Rightarrow$ slope indicates rotations
			
			$\Rightarrow$ strip shape is given by 
			
			\hspace{3mm} deviation from straight line
		\column{.50\textwidth}
			{\centering
			\textbf{residual mean VS \\ precision direction}
			
			\vspace{7mm}
			
			\includegraphics[width=0.9\textwidth]{boardAlignmentNpitchError-crop.pdf}
			}
			
			\scriptsize
			$\Rightarrow$ shift between boards is given by 
			
			\hspace{3mm} difference of the centers
			
			$\Rightarrow$ slope indicates deviation to nominal pitch 
	\end{columns}
}

\frame{\frametitle{CRF hit distribution}
	\centering
	
	\includegraphics[width=0.7\textwidth]{m3_CRFhits_Xmas.pdf}
	
	\vspace{2mm}
	
	M3 eta\_out, 5 mm efficient
	\vspace{1mm}
	
	\includegraphics[width=0.7\textwidth]{m3_5mmHits_Xmas.pdf}

}

\frame{\frametitle{Theoretical Gain Description}

	\scriptsize
	\begin{columns}
		\column{.5\textwidth}
			Gain by Townsend (1910) :
			\vspace{1mm}
			
			$G = \exp\left[\boldsymbol{d} \cdot \alpha \right]$
			\vspace{3mm}
			
			$\alpha$ by Rose and Korff (1941) :
			\vspace{1mm}
			
			$
				G = 
					\exp\left[ \;\; 
						\textcolor{red}{d} \cdot
						\dfrac{ \textcolor{red}{p} A }{T} 
						\; \cdot \; \exp\!\left(\; -\dfrac{\textcolor{red}{d p} B}{T \textcolor{blue}{U}} \; \right) 
						\;\; 
					\right]
			$
			\vspace{3mm}
			
			effective description :
			\vspace{1mm}
			
			$G = \mathrm{f}\left(\textcolor{red}{d p} ; \textcolor{blue}{U}\right)$
		\column{.5\textwidth}
			\centering
			fieldsimulation, Kuger (2017)
			\includegraphics[width=\textwidth]{micromegasFieldSimulation_Kuger2017.png}
	\end{columns}
	\vspace{5mm}
	
	paper by Giomataris mentioning the prospects of a large sample 
	
	of gap widths ($d \mathrel{\hat=}$ pillar heights):
	\vspace{1mm}
	
	\url{https://www.slac.stanford.edu/pubs/icfa/fall99/paper1/paper1.pdf}

}

\frame{
	\centering 

	\begin{columns}
		\column{.5\textwidth}
			\centering
			525 V
			
			\includegraphics[width=\textwidth]{meanClusterQvsPressurePillarHeight_passivatedModules_525V_ranged.pdf}
		\column{.5\textwidth}
			\centering
			540 V
			
			\includegraphics[width=\textwidth]{meanClusterQvsPressurePillarHeight_passivatedModules_540V_ranged.pdf}
	\end{columns}

	\vspace{8mm}

	\begin{columns}
		\column{.5\textwidth}
			\centering
			560 V
			
			\includegraphics[width=\textwidth]{meanClusterQvsPressurePillarHeight_passivatedModules_560V_ranged.pdf}
		\column{.5\textwidth}
			\centering
			570 V
			
			\includegraphics[width=\textwidth]{meanClusterQvsPressurePillarHeight_passivatedModules_570V_ranged.pdf}
	\end{columns}

}
	
\frame{\frametitle{CRF measurements with module 3 (high statistics)}
%	\scriptsize
	\vspace{5mm}
%	\centering
	
%	\hspace{-5mm}
%	\setlength\tabcolsep{1.5pt}
	\begin{tabular}{ccc}
		\hline
		\hline
		measurement & events ($10^6$) & remark
		\\
		\hline
		20.12.19-15:21 & 19.2 &
		\\
		22.12.19-11:32 & 20.0 &
		\\
		24.12.19-10:19 & 20.0 & slowcontrol stopped
		\\
		26.12.19-18:26 & 18.0 &
		\\
		28.12.19-13:30 & 8.8 &
		\\
		29.12.19-10:47 & 19.6 &
		\\
		31.12.19-10:06 & 20.0 &
		\\
		02.01.20-14:07 & 8.7 &
		\\
		03.01.20-10:54 & 20.0 &
		\\
		\hline
		\hline
	\end{tabular}
	\vspace{8mm}

	$\Rightarrow$ in total \textbf{154} $\cdot 10^6$ events
	\vspace{2mm}
	
	$\Rightarrow$ usable after MDT-analysis \textbf{70} $\cdot 10^6$ events

}
	
\frame{\frametitle{\normalsize CRF measurements with module 16 (drift scan @ $U_{\mathrm{amp}} =$ 570 V)}
%	\scriptsize
%	\vspace{5mm}
	\centering
	
%	\hspace{-5mm}
%	\setlength\tabcolsep{1.5pt}
	\begin{tabular}{ccc}
		\hline
		\hline
		drift voltage [V] & measurement & events ($10^6$) 
		\\
		\hline
		125 & 02.08.-08:55 & 4.1
		\\
		150 & 02.08.-19:53 & 6.5
		\\
		175 & 03.08.-13:20 & 2.5
		\\
		200 & 03.08.-20:03 & 5.2
		\\
		300 & 31.07.-19:46 & 4.7
		\\
		450 & 04.08.-10:06 & 2.5
		\\
		\hline
		\hline
	\end{tabular}

}

\end{document}
