\documentclass{beamer}
\usetheme{Madrid}
\usecolortheme{spruce}
\definecolor{green(pigment)}{rgb}{0.0, 0.65, 0.31}
\setbeamercolor*{item}{fg=green(pigment)}
\definecolor{Green}{rgb}{0.00, 1.00, 0.00}
\definecolor{Red}{rgb}{1.00, 0.00, 0.00}
\definecolor{Blue}{rgb}{0.00, 0.00, 1.00}
\usepackage[utf8]{inputenc}
\usepackage{amsmath}
\usepackage{amsfonts}
\usepackage{amssymb}
\usepackage[german]{babel}
\usepackage{graphicx}
\usepackage{rotating}
\usepackage{textcomp}
\usepackage{multirow,bigdelim,dcolumn,booktabs}
%\usepackage{beamerthemeshadow}
\usepackage{subfigure} 
\usepackage{siunitx}
\usepackage{appendixnumberbeamer}
\usepackage{hyperref}
%\usepackage{xmpmulti}
\usepackage{animate}
%\beamersetuncovermixins{\opaqueness<1>{25}}{\opaqueness<2->{15}}
\beamertemplatenavigationsymbolsempty

\usepackage{tikz}
\usetikzlibrary{decorations.text}
\usetikzlibrary{trees}
\usetikzlibrary{decorations.pathmorphing}
\usetikzlibrary{decorations.markings}
\usetikzlibrary{patterns}

\graphicspath{
	{pictures/}
	{pictures/resVScluTimeVSslope/}
%	{/home/m/Maximilian.Herrmann/Bilder/forLatex/plots/}
%	{/home/m/Maximilian.Herrmann/Bilder/forLatex/sketches/}
%	{/home/m/Maximilian.Herrmann/Bilder/forLatex/pictures/}
}

\begin{document}
\title[SM2 Module 1 in CRF]{Investigation of the \textmu TPC mode \\ with SM2 Module 1 in the Munich CRF}
\author[M. Herrmann]{speaker : Maximilian Herrmann}
\institute[LMU Munich]{Ludwig-Maximilians-Universit\"at M\"unchen - Lehrstuhl Schaile}
\date[04.09.2018]{04.09.2018, NSW Micromegas Review} 

\frame{
	\centering
	\vspace{2mm}

	\titlepage
	\vspace{-6mm}
	
	\includegraphics[width=0.35\textwidth]{LMUlogo.jpg}
	\hspace{4cm}
	\includegraphics[width=0.3\textwidth]{BMBFlogo.png}
} 

\frame{\frametitle{Cosmic Ray Facility LMU Munich}
		
	\begin{columns}
		\column{.50\textwidth}
			\centering
			\includegraphics[width=0.9\textwidth]{CRFprinciple3-crop.pdf}
			
			\tiny
			\color{red} track prediction accuracy given by multiple scattering
		\column{.50\textwidth}
			\centering
			\includegraphics[width=\textwidth]{CRF_wMicromegas.pdf}
	\end{columns}
	
	\vspace{0mm}
	\centering
	\footnotesize
	\begin{tabular}{ll}
	track reconstruction & 2 $\times$ Monitored Drift Tube chambers (MDTs)
%	\\
%	 & single tube resolution $\sim$ \SI{100}{\micro\m}
% 	\\
%	 & six layers $\sim$ \SI{41}{\micro\m}
	\\
	trigger & scintillator hodoscope
	\\
	active area & \SI{2.2}{m} $\times$ \SI{4}{m}
	\\
	angular acceptance & $\pm$ \SI{30}{\degree}
	\\
	readout & 12288 channels
	\\
	 & $\to$ 96 APVs (frontend electronics)
	\\
	 & $\to$ 6 FECs (scalable readout system)
	\\
	readout rate & 130 Hz (full muon rate)
	\\
	measurement period & 17.10. - 06.12. 2017
	\end{tabular}
}

\frame{\frametitle{Measurement Details (for considered run)}
	\small
	\begin{tabular}{ll}
		measurement period & 09:28 01.06.2018 - 02:23 02.06.2018 ($\sim$ 17 h)
		\\
		 &
		\\
		gas & Ar:CO$_{2}$ (93:7 vol\%)
		\\
		 &
		\\
		amplification voltage & 570 V
		\\
		 &
		\\
		drift voltage & -300 V
		\\
		 &
		\\
		atmospheric pressure & 955 mbar
		\\
		 &
		\\
		temperature & 15 - 25 $^{\circ}$C
		\\
		 &
		\\
		readout & online zerosuppressed (raw data NOT available)
		\\
		 &
		\\
		events & trigger 7.9 M
		\\
		 & $\Rightarrow$ MDT merged 4.2 M 
		\\
		 & $\Rightarrow$ in active area 1.7 M
		\\
		 &
		\\
		considered layer & eta in (mesh lying on anode)
	\end{tabular}
}

\frame{\frametitle{Signal Properties}

	\small
	\centering
	
	strip multiplicity
	
	\begin{columns}
		\column{.33\textwidth}
			\centering
			
			\includegraphics[width=0.85\textwidth]{m1_ei_b6_nStripsVsangle_20180601.pdf}
		\column{.33\textwidth}
			\centering
			
			\includegraphics[width=0.85\textwidth]{m1_ei_b7_nStripsVsangle_20180601.pdf}
		\column{.33\textwidth}
			\centering
			
			\includegraphics[width=0.85\textwidth]{m1_ei_b8_nStripsVsangle_20180601.pdf}
	\end{columns}
	
	time difference of last and first strip in cluster (in 25 ns)
	
	\begin{columns}
		\column{.33\textwidth}
			\centering
			
			\includegraphics[width=0.85\textwidth]{m1_ei_b6_timeDifVsangle_20180601.pdf}
		\column{.33\textwidth}
			\centering
			
			\includegraphics[width=0.85\textwidth]{m1_ei_b7_timeDifVsangle_20180601.pdf}
		\column{.33\textwidth}
			\centering
			
			\includegraphics[width=0.85\textwidth]{m1_ei_b8_timeDifVsangle_20180601.pdf}
	\end{columns}
	
	\begin{columns}
		\column{.33\textwidth}
			\centering
			
			board 6
		\column{.33\textwidth}
			\centering
						
			board 7
		\column{.33\textwidth}
			\centering
						
			board 8
	\end{columns}
	
}

\frame{\frametitle{\textmu TPC Analysis - Classic Approach}

	\begin{columns}
		\column{.5\textwidth}
			\centering
			
			NSW - TDR
						
			\includegraphics[width=1.1\textwidth]{uTPC_alaTDR.png}
		\column{.5\textwidth}
			\centering
			
			CERN-THESIS-2016-019 (K.Ntekas)
						
			\includegraphics[width=0.7\textwidth]{uTPC_alaKostas.png}
	\end{columns}
		
		\vspace{3mm}
		
		reconstruction of position in drift gap
		
		linear with strip time measurement : $z_{s} = v_{\mathrm{drift}} \cdot \underbrace{\left( t_{s} - t_{\mathrm{offset}} \right)}_{ = t_{s,\mathrm{drift}}}$
		
		\vspace{2mm}
		
		$\Rightarrow$ biased reconstruction if wrong drift velocity is assumed 
		
		\hspace{4.5mm} (e.g. water, air in gas mixture)
	
}

\frame{\frametitle{Drift Time Measurement}
	\centering

	Ar:CO$_{2}$ (93:7 vol\%) , $U_{\mathrm{drift}}$ -300 V @ 5 mm gap
	
	\vspace{5mm}
	
	\begin{columns}
		\column{.50\textwidth}
			\centering
						
			\includegraphics[width=\textwidth]{m1_ei_stripTime_firstNlast_20180601_wFWHM_wMaxDist.pdf}
		\column{.50\textwidth}
			\centering
			simulation
						
			\includegraphics[width=\textwidth]{driftTime_sim_loesel.png}
	\end{columns}
	\vspace{5mm}
				
	$\Rightarrow$ reconstructed drift time not in agreement with simulation
	
	\hspace{4.5mm} or expectation ( 5 mm / \SI[per-mode=fraction]{0.047}{\mm\per\ns} $\simeq$ 106 ns)
	
}

\frame{\frametitle{\textmu TPC Analysis - Drift Velocity Agnostic}

	\begin{columns}
		\column{.55\textwidth}
			\centering
						
			\includegraphics[width=1.1\textwidth]{timevsstrip_description.pdf}
		\column{.45\textwidth}
			\begin{itemize}
				\item
					angle reconstruction: 
						
					\vspace{3mm}
					
					$\theta = \arctan \left( \tfrac{\mathrm{pitch}}{m_{\mu \mathrm{TPC}} \;\cdot\; v_{\mathrm{drift}}} \right)$
						
					\vspace{3mm}
					
					$m_{\mu \mathrm{TPC}}$ : slope \textmu TPC fit
					
					\vspace{3mm}
				\item
					position reconstruction:
						
					\vspace{3mm}
								
					$
					\mathrm{pos}_{\mu\mathrm{TPC}}
					=
					\dfrac{t_{0} - t_{\mu \mathrm{TPC}}}{m_{\mu \mathrm{TPC}}}
					$
					
					\vspace{3mm}
					
					$t_{\mu \mathrm{TPC}}$ : intercept $\mu \mathrm{TPC}$ fit
			\end{itemize}
	\end{columns}
	
	\vspace{5mm}
	
	$\Rightarrow$ position reconstruction independent of drift velocity
	
	$\Rightarrow$ probe drift velocity for right angle reconstruction
	
}

\frame{\frametitle{\textmu TPC Angle Reconstruction}

	\small

	\begin{columns}
		\column{.6\textwidth}
			\centering
			angle reconstructed VS reference
			
			\includegraphics[width=0.9\textwidth]{m1_ei_uTPCangleVSangle_20180601.pdf}
		\column{.4\textwidth}
			\centering
			
			mean angle
			
			\includegraphics[width=0.85\textwidth]{m1_ei_uTPCmeanAngleVSangle_20180601.pdf}
			
			standard deviation
			
			\includegraphics[width=0.85\textwidth]{m1_ei_uTPCstdvAngleVSangle_20180601.pdf}
	\end{columns}
	
	\vspace{2mm}
	
	$\Rightarrow$ missing statistics for large angles ( $>$ 20$^{\circ}$ )
	
	$\Rightarrow$ drift velocity may be spoiled due to gas impurities (water, air)
	
}

\frame{\frametitle{\textmu TPC Angle Reconstruction}

	\small

	\begin{columns}
		\column{.6\textwidth}
			\centering
			angle reconstructed 
			
			for 20$^{\circ}$ to 22$^{\circ}$ reference angle
			
			\includegraphics[width=\textwidth]{m1_ei_uTPCangle_20-22degree_20180601.pdf}
		\column{.4\textwidth}
			\centering
			
			MPV angle
			
			\includegraphics[width=0.9\textwidth]{m1_ei_uTPCmpvAngleVSangle_20180601.pdf}
			
			gaus sigma
			
			\includegraphics[width=0.9\textwidth]{m1_ei_uTPCwidthAngleVSangle_20180601.pdf}
	\end{columns}
	
	fit angular distribution for each slice with gaussian (landau does not work)
	
	fit range: maximum bin $\pm$ 15$^{\circ}$
	
}

\frame{\frametitle{Determination of $t_{0}$}

%	\small
	
	\begin{columns}
		\column{.50\textwidth}
			\centering
						
			\includegraphics[width=0.9\textwidth]{m1_ei_uTPCresVSuTPCslope_20180601.pdf}
		\column{.50\textwidth}
			\centering
						
			\includegraphics[width=0.9\textwidth]{m1_ei_uTPCresVSuTPCslope_zoomed_20180601.pdf}
	\end{columns}
	\centering
	\vspace{5mm}
				
	$
		\mathrm{res} = 
		\underbrace{
			\tfrac{0 - \mathrm{intercept}_{\mu\mathrm{TPC}}}{\mathrm{slope}_{\mu\mathrm{TPC}}}
		}_{\mu \mathrm{TPC \;\; position} \;\; t_{0}=0} 
		 - \;\mathrm{pos}_\mathrm{ref}
		\;\;\;\Rightarrow\;\;\; t_{0} = - \mathrm{slope}_\mathrm{distribution} / \mathrm{pitch}
	$ 
	
}

\frame{\frametitle{\textmu TPC Position Reconstruction}

%	\small
	
	\begin{columns}
		\column{.50\textwidth}
			\centering
						
			\includegraphics[width=0.9\textwidth]{m1_ei_uTPCresVSangle_20180601.pdf}
		\column{.50\textwidth}
			\centering
						
			\includegraphics[width=\textwidth]{m1_ei_bothResolutionsVSangle_20180601.pdf}
	\end{columns}
	\vspace{5mm}
				
	$\Rightarrow$ \textmu TPC reconstruction much more sensitive 
	
	\hspace{4.5mm} to inhomogeneous clustering of ionizing particles
	
}

\appendix

\frame{\centering \Huge Backup}

%\begin{frame}
%	\transduration<0-29>{0}
%	\multiinclude[<+->][format=png,start=1,graphics={width=\textwidth}]{resVSclustertimeVSangleM1eiC100V}
%\end{frame}

\begin{frame}{Residual VS Clustertime VS Slope}
  \animategraphics[loop,controls,width=\linewidth]{1}{resVSclustertimeVSangleM1eiC100V-}{0}{29}
\end{frame}

\frame{\frametitle{Influence of Multiple Scattering}
	\footnotesize
	
	\begin{columns}
		\column{.50\textwidth}
			\centering
			MDT extrapolated residual VS angle
						
			\includegraphics[width=0.65\textwidth]{MDTintDifVSangle.pdf}
		\column{.50\textwidth}
			\centering
			MDT residual perpendicular incident
			\vspace{4mm}
						
			\includegraphics[width=0.8\textwidth]{MDTresidual_slopeDifCuts.pdf}
	\end{columns}
	\vspace{3mm}
		
	\begin{columns}
		\column{.50\textwidth}
			\centering
			Module 0 residual narrow width VS angle
			\vspace{2.5mm}
						
			\includegraphics[width=0.75\textwidth]{SM2-M0_CRFafterH8_eta-out_residualVSangle_slopeDifCuts.pdf}
		\column{.50\textwidth}
			\centering
			Module 0 residual perpendicular incident
						
			\includegraphics[width=0.8\textwidth]{SM2-M0_CRFafterH8_eta-out_residual_cutsMDTslopeDif.pdf}
	\end{columns}
}

\frame{\frametitle{Usage of Multiple Scattering - Muon Tomography (L1)}

	\begin{columns}
		\column{.50\textwidth}
			\centering
			\includegraphics[width=0.85\textwidth]{trackIntercept_yz_3e-2_1e-1_blackWhite.pdf}
		\column{.50\textwidth}
			\centering
			\includegraphics[width=0.7\textwidth]{CRFmounTomography_yz-crop.pdf}
	\end{columns}
	
	\begin{columns}
		\column{.50\textwidth}
			\centering
			\includegraphics[width=0.9\textwidth]{scatteredFraction_xy_blackWhite.pdf}
		\column{.50\textwidth}
			\centering
			\includegraphics[width=0.9\textwidth]{L1inCRF_muonTomoTop-crop.pdf}
	\end{columns}
	
}

\end{document}
