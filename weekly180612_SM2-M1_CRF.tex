\documentclass{beamer}
\usetheme{Madrid}
\usecolortheme{spruce}
\definecolor{green(pigment)}{rgb}{0.0, 0.65, 0.31}
\setbeamercolor*{item}{fg=green(pigment)}
\definecolor{Green}{rgb}{0.00, 1.00, 0.00}
\definecolor{Red}{rgb}{1.00, 0.00, 0.00}
\definecolor{Blue}{rgb}{0.00, 0.00, 1.00}
\usepackage[utf8]{inputenc}
\usepackage{amsmath}
\usepackage{amsfonts}
\usepackage{amssymb}
\usepackage[german]{babel}
\usepackage{graphicx}
\usepackage{rotating}
\usepackage{textcomp}
\usepackage{multirow,bigdelim,dcolumn,booktabs}
%\usepackage{beamerthemeshadow}
\usepackage{subfigure} 
\usepackage{siunitx}
\usepackage{appendixnumberbeamer}
\usepackage{hyperref}
%\beamersetuncovermixins{\opaqueness<1>{25}}{\opaqueness<2->{15}}
\beamertemplatenavigationsymbolsempty

\usepackage{tikz}
\usetikzlibrary{decorations.text}
\usetikzlibrary{trees}
\usetikzlibrary{decorations.pathmorphing}
\usetikzlibrary{decorations.markings}
\usetikzlibrary{patterns}

\graphicspath{
	{pictures/}
%	{/home/m/Maximilian.Herrmann/Bilder/forLatex/plots/}
%	{/home/m/Maximilian.Herrmann/Bilder/forLatex/sketches/}
%	{/home/m/Maximilian.Herrmann/Bilder/forLatex/pictures/}
}

\begin{document}
\title[SM2 Module 1 in CRF]{Performance and Calibration \\ of SM2 Module 1 in the Munich CRF}
\author[M. Herrmann]{speaker : Maximilian Herrmann}
\institute[LMU Munich]{Ludwig-Maximilians-Universit\"at M\"unchen - Lehrstuhl Schaile}
\date[12.06.2018]{12.06.2018, Weekly NSW Micromegas Meeting} 

\frame{
	\centering
	\vspace{2mm}

	\titlepage
	\vspace{-6mm}
	
	\includegraphics[width=0.35\textwidth]{LMUlogo.jpg}
	\hspace{4cm}
	\includegraphics[width=0.3\textwidth]{BMBFlogo.png}
} 

%\frame{\frametitle{Outline}
%	\tableofcontents
%} 

\section{LMU Cosmic Ray Facility}

\frame{\frametitle{LMU Cosmic Ray Facility}
		
	\begin{columns}
		\column{.50\textwidth}
			\centering
			\includegraphics[width=0.9\textwidth]{CRFprinciple3-crop.pdf}
		\column{.50\textwidth}
			\centering
			\includegraphics[width=0.9\textwidth]{CRF_small.JPG}
	\end{columns}
	
	\footnotesize
			
	\begin{itemize}
		\item
			2D track reconstruction with two Monitored Drift Tube (MDT) chambers
		\item
			trigger via scintillator hodoscope with $\approx$ \SI{10}{cm} resolution 
			
			in direction along the wires
		\item
			MDT chambers : \SI{2.2}{\m} $\times$ \SI{4}{\m} 
			
			$\Rightarrow$ active area : \SI{8}{\m\squared}, angular acceptance : $\pm30^{\circ}$
		\item
			readout of the full module (12288 channels) 
			
			with 96 APVs connected to 6 FECs @ full \SI{130}{Hz} \textmu -rate
			
			(tested up to \SI{500}{Hz} with random trigger)
	\end{itemize}
}

\frame{\frametitle{SM2 Module 1 in the Cosmic Ray Facility}
		
	\begin{columns}
		\column{.50\textwidth}
			\centering
			\includegraphics[width=\textwidth]{sm2_m1_inCRF_wL1_0.jpg}
		\column{.50\textwidth}
			\centering
			\includegraphics[width=\textwidth]{sm2_m1_inCRF_wL1_1.jpg}
	\end{columns}
	\vspace{5mm}
	
	\begin{tabular}{llcccc}
		stereo & out & : & gluing side 1 & (on top)
		\\
		stereo & in & :  & gluing side 2 & 
		\\
		\hline
		eta & in & :  & gluing side 1 & lost vacuum during gluing
		\\
		eta & out & :  & gluing side 2 & (on bottom)
	\end{tabular}
}

\frame{\frametitle{Measurement Overview : 11$\times 10^6$ muon trigger / day}
\scriptsize
	\begin{tabular}{ccccc}
		date & amplification [V] & drift [V] & trigger [$10^6$] & merged events [$10^6$]
		\\
		\hline
		28.05. & {\color{orange}550} & {\color{blue}300} & 6.4 & 3.4
		\\
		29.05. & 560 & {\color{blue}300} & 10.9 & 5.6
		\\
		30.05. & 570 & {\color{blue}300} & 3.7 & 2.0
		\\
		30.05. & 570 & {\color{blue}300} & 7.3 & 3.9
		\\
		31.05. & {\color{orange}580} & {\color{blue}300} & 2.7 & 1.4
		\\
		31.05. & 570 & {\color{blue}300} & 8.8 & 4.7
		\\
		01.06. & 570 & {\color{blue}300} & 7.9 & 4.2
		\\
		\hline
		02.06. & 550 & {\color{green}150} & 4.2 & 2.2
		\\
		02.06. & 560 & {\color{green}150} & 6.7 & 2.2
		\\
		03.06. & {\color{orange}540} & {\color{green}150} & 4.2 & 2.2
		\\
		03.06. & {\color{orange}570} & {\color{green}150} & 5.2 & 2.7
		\\
		\hline
		03.06. & 570 & {\color{blue}300} & 4.9 & 2.6
		\\
		04.06. & 570 & {\color{blue}300} & 11.5 & 5.3
		\\
		05.06. & 570 & {\color{blue}300} & 4.0 & 2.1
		\\
		05.06. & 570 & {\color{blue}300} & 7.9 & 4.2
		\\
		06.06. & 560 & {\color{blue}300} & 3.7 & 1.9
		\\
		06.06. & 570 & {\color{green}150} & 6.8 & 3.6
		\\
		07.06. & 570 & {\color{blue}300} & 4.1 & 2.2
		\\
		07.06. & 570 & {\color{blue}300} & 7.2 & 3.8
		\\
		08.06. & 570 & {\color{blue}300} & 12.1 & 2.8
		\\
		09.06. & 560 & {\color{blue}300} & 11.0 & 2.8
		\\
		10.06. & 570 & {\color{blue}300} & 5.4 & 
		\\
		10.06. & 570 & {\color{blue}300} & 5.4 & 
	\end{tabular}
}

\section{Full Area Pulse Height}

\frame{\frametitle{Anode : \SI{570}{V}, Cathode : \SI{300}{V}, Ar:CO$_2$ 93:7 vol\% \\ Mean Cluster Charge}
	\begin{columns}
		\column{.50\textwidth}
			\centering
			eta out
			
			\includegraphics[width=0.7\textwidth]{m1_etaout_A570V_C300V_clusterQmean_scaled.png}
		\column{.50\textwidth}
			\centering
			eta in
			
			\includegraphics[width=0.7\textwidth]{m1_etain_A570V_C300V_clusterQmean_scaled_APVnotworking.pdf}
	\end{columns}
	\begin{columns}
		\column{.50\textwidth}
			\centering
			stereo in
			
			\includegraphics[width=0.7\textwidth]{m1_stereoin_A570V_C300V_clusterQmean_scaled.png}
		\column{.50\textwidth}
			\centering
			stereo out
			
			\includegraphics[width=0.7\textwidth]{m1_stereoout_A570V_C300V_clusterQmean_scaled_badHV.pdf}
	\end{columns}
}

\frame{\frametitle{Most Probable Value of Cluster Charge by Landau Fit}
	\begin{columns}
		\column{.50\textwidth}
			\centering
			eta out
			
			\includegraphics[width=0.7\textwidth]{m1_etaout_A570V_C300V_clusterQmpv_scaled.png}
		\column{.50\textwidth}
			\centering
			eta in
			
			\includegraphics[width=0.7\textwidth]{m1_etain_A570V_C300V_clusterQmpv_scaled.png}
	\end{columns}
	\vspace{2.3mm}
	\begin{columns}
		\column{.50\textwidth}
			\centering
			stereo in
			
			\includegraphics[width=0.7\textwidth]{m1_stereoin_A570V_C300V_clusterQmpv_scaled.png}
		\column{.50\textwidth}
			\centering
			stereo out
			
			\includegraphics[width=0.7\textwidth]{m1_stereoout_A570V_C300V_clusterQmpv_scaled.png}
	\end{columns}
}

\frame{\frametitle{Cluster Analysis for Eta Out Board 6}
	\begin{columns}
		\column{.50\textwidth}
			\centering
			cluster charge ($U_{\mathrm{drift}} =$ \SI{150}{V})
			
			\includegraphics[width=0.8\textwidth]{m1_eo_board6_clusterQ_ampScan_C150V.pdf}
		\column{.50\textwidth}
			\centering
			cluster charge ($U_{\mathrm{drift}} =$ \SI{300}{V})
			
			\includegraphics[width=0.8\textwidth]{m1_eo_board6_clusterQ_ampScan_C300V.pdf}
	\end{columns}
	\vspace{2.3mm}
	\begin{columns}
		\column{.50\textwidth}
			\centering
			strips in leading cluster 
			
			{\scriptsize $U_{\mathrm{drift}} =$ \SI{150}{V}, $U_{\mathrm{amp}} =$ \SI{550}{V}}
			
			\includegraphics[width=0.6\textwidth]{m1_eo_board6_nStripsVSslope.pdf}
		\column{.50\textwidth}
			\centering
			mean \# of strips ($U_{\mathrm{drift}} =$ \SI{300}{V})
			
			\includegraphics[width=0.8\textwidth]{m1_eo_board6_meanNumberOfStripsVSslope.pdf}
	\end{columns}
}

\frame{\frametitle{Cluster Analysis for Stereo Out Board 7}
	\begin{columns}
		\column{.50\textwidth}
			\centering
			cluster charge ($U_{\mathrm{drift}} =$ \SI{150}{V})
			
			\includegraphics[width=0.8\textwidth]{m1_so_board7_clusterQ_ampScan_C150V.pdf}
		\column{.50\textwidth}
			\centering
			cluster charge ($U_{\mathrm{drift}} =$ \SI{300}{V})
			
			\includegraphics[width=0.8\textwidth]{m1_so_board7_clusterQ_ampScan_C300V.pdf}
	\end{columns}
	\vspace{2.3mm}
	\begin{columns}
		\column{.50\textwidth}
			\centering
			strips in leading cluster 
			
			{\scriptsize $U_{\mathrm{drift}} =$ \SI{150}{V}, $U_{\mathrm{amp}} =$ \SI{550}{V}}
			
			\includegraphics[width=0.6\textwidth]{m1_so_board7_nStripsVSslope.pdf}
		\column{.50\textwidth}
			\centering
			mean \# of strips ($U_{\mathrm{drift}} =$ \SI{300}{V})
			
			\includegraphics[width=0.8\textwidth]{m1_so_board7_meanNumberOfStripsVSslope.pdf}
	\end{columns}
}

\frame{\frametitle{Strip Signal for Eta Out Board 6}
	\begin{columns}
		\column{.50\textwidth}
			\centering
			\includegraphics[width=0.68\textwidth]{m1_etaout_board6_risetimeVScharge.png}
			
			\vspace{-1.3mm}
			signal maximum
			
			\includegraphics[width=0.9\textwidth]{m1_eo_board6_A570V_C300V_stripCharge.pdf}
		\column{.50\textwidth}
			\centering
			risetime
			
			\includegraphics[width=0.9\textwidth]{m1_etaout_board6_risetime.png}
			
			\begin{itemize}
				\item
					anode : \SI{570}{V}
					
					cathode : \SI{300}{V}
				\item
					only strips in cluster within \SI{5}{mm} to track prediction
				\item
					reasonable risetime distribution
				\item
					dynamic range up to \SI{1700}{ADC} channel
			\end{itemize}
	\end{columns}
}

\frame{\frametitle{Strip Signal for Stereo Out Board 7}
	\begin{columns}
		\column{.50\textwidth}
			\centering
			\includegraphics[width=0.68\textwidth]{m1_stereoout_board7_risetimeVScharge.png}
			
			\vspace{-1.2mm}
			signal maximum
						
			\includegraphics[width=0.9\textwidth]{m1_so_board7_A570V_C300V_stripCharge.pdf}
		\column{.50\textwidth}
			\centering
			risetime
			
			\includegraphics[width=0.9\textwidth]{m1_so_board7_A570V_C300V_risetime.pdf}
		
			\begin{itemize}
				\item
					anode : \SI{570}{V}
					
					cathode : \SI{300}{V}
				\item
					only strips in cluster within \SI{5}{mm} to track prediction
				\item
					slightly larger risetimes
				\item
					dynamic range up to {\color{red}\SI{1500}{ADC}} channel $\Rightarrow$ lower saturation
			\end{itemize}
	\end{columns}
}

\frame{\frametitle{Efficiency (Preliminary)}
	\begin{columns}
		\column{.50\textwidth}
			\centering
			amplification \SI{570}{V}, drift \SI{150}{V}
			
			cluster efficiency
			
			\includegraphics[width=0.6\textwidth]{m1_eo_clusterEfficiency.pdf}
			
			\vspace{-0.5mm}
			\SI{5}{mm} efficiency
			
			\includegraphics[width=0.6\textwidth]{m1_eo_5mmEfficiency.pdf}
		\column{.50\textwidth}
		\scriptsize
			\begin{itemize}
				\item
					preliminary results
					
					$\Rightarrow$ cuts have to be checked 
				\item
					cluster efficiency:
					
					number of cluster with at least 2 strips
					
					divided by number all of tracks going trough partition
				\item
					\SI{5}{mm} efficiency:
					
					number of cluster within \SI{5}{mm} to track prediction
										
					divided by number of all tracks going trough partition
				\item
					efficiency spoiled by connection of zebra connectors
			\end{itemize}
			
			\centering
			\includegraphics[width=0.8\textwidth]{m1_5mmEfficiency_C300V_selectedPartitions.pdf}
	\end{columns}
}

%\frame{\frametitle{HV Behavior - Voltages}
%	measurement duration : 06.06. 10:00 to 11.06. 10:00
%		
%	\vspace{5mm}
%
%	\hspace{35mm} board 6 \hspace{18mm} board 7 \hspace{18mm} board 8
%	\begin{columns}
%		\column{.20\textwidth}
%		stereo out (GS1)
%		\vspace{6mm}
%		
%		stereo in (GS2)
%		\vspace{6mm}
%		
%		eta in (GS1)
%		\vspace{6mm}
%		
%		eta out (GS2)
%		\column{.80\textwidth}
%			\centering
%			\includegraphics[width=\textwidth]{m1_voltages.pdf}
%	\end{columns}
%}
%
%\frame{\frametitle{HV Behavior - Currents}
%	measurement duration : 06.06. 10:00 to 11.06. 10:00
%	
%	\vspace{5mm}
%	
%	\hspace{35mm} board 6 \hspace{18mm} board 7 \hspace{18mm} board 8
%	\begin{columns}
%		\column{.20\textwidth}
%		stereo out (GS1)
%		\vspace{6mm}
%		
%		stereo in (GS2)
%		\vspace{6mm}
%		
%		eta in (GS1)
%		\vspace{6mm}
%		
%		eta out (GS2)
%		\column{.80\textwidth}
%			\centering
%			\includegraphics[width=\textwidth]{m1_currents.pdf}
%	\end{columns}
%}

\section{Calibration Using Reference Tracks over Whole Area}

\frame{\frametitle{Reconstructed Pitch Error and Board Alignment per Plane}
	\begin{columns}
		\column{.50\textwidth}
			\centering
			eta out
			
			\includegraphics[width=0.8\textwidth]{m1_eo_resVSstrip.pdf}
		\column{.50\textwidth}
			\centering
			eta in
			
			\includegraphics[width=0.8\textwidth]{m1_ei_resVSstrip.pdf}
	\end{columns}
	\vspace{2.3mm}
	\begin{columns}
		\column{.50\textwidth}
			\centering
			stereo in
			
			\includegraphics[width=0.8\textwidth]{m1_si_resVSstrip.pdf}
			
			\vspace{5mm}
		\column{.50\textwidth}
			\centering
			stereo out
			
			\includegraphics[width=0.8\textwidth]{m1_so_resVSstrip_pitchError.pdf}
			
			{\color{red} misaligned during gluing}
	\end{columns}
}

\frame{\frametitle{Strip Shape - Eta Out Board 6}
	\begin{columns}
		\column{.50\textwidth}
			\centering
			\includegraphics[width=0.8\textwidth]{m1_eo_board6_resVSscinX.pdf}
		\column{.50\textwidth}
			\centering
			\includegraphics[width=\textwidth]{m1_eo_board6_resMeanVSscinX.pdf}
	\end{columns}
	
	\begin{columns}
		\column{.50\textwidth}
			\scriptsize
			\begin{itemize}
				\item
					residual as function of the position along the strips given by the scintillator hodoscopes
				\item
					``banana shape''-like behavior similar to observation in Saclay
				\item
					different shapes for different boards
					
					$\Rightarrow$ has to be investigated further
			\end{itemize}
		\column{.50\textwidth}
			\centering
			corresponds to
			
			\includegraphics[width=\textwidth]{bananaStripSketch-crop.pdf}
	\end{columns}
}

\section{Position Reconstruction for Various Incident Angle}

\frame{\frametitle{Position Resolution as Function of the Incident Angle}
	\vspace{-1mm}
	
	\begin{columns}
		\column{.50\textwidth}
			\centering
			\footnotesize
			\textbf{eta out}
			\includegraphics[width=0.9\textwidth]{m1_eo_resolutionVSangle_wCluTimeCor.pdf}
			
			\vspace{-0.5mm}
			\textbf{eta in}
			\includegraphics[width=0.9\textwidth]{m1_ei_resolutionVSangle_wCluTimeCor.pdf}
		\column{.50\textwidth}
			\footnotesize
			\begin{itemize}
				\item
					residual distribution for each angle separately
				\item
					fit with double Gaussian
					
					analysis sigma narrow Gaussian only
					
					\hspace{3.5mm}(reject multiple scattering)
				\item
					consider track uncertainty of reference chambers: 
				
					$\sigma_{\mathrm{micromegas}} = \sqrt{\sigma_{\mathrm{res}}^2 - \sigma_{\mathrm{track}}^2}$
				\item
					resolution is for perpendicular incident close to expectation
				\item
					charge weighted clustertime correction improves residual distribution considerably
					
					analysis ongoing: 
					
					\SI{0.2}{mm} @ 20$^{\circ}$ expected
					
					(see B.Flierl PhD thesis - Particle Tracking with Micro-Pattern Gaseous Detectors)
			\end{itemize}
	\end{columns}
}

\frame{\frametitle{Summary}
	\footnotesize
	investigation of the SM2 module 1 in the Cosmic Ray Facility in Garching
	\begin{itemize}
		\footnotesize
		\item
			pulse height is not homogeneous over area and layers
			
			$\Rightarrow$ has to be further investigated
			
			lower APV saturation values for the type 7 boards
			
			$\Rightarrow$ closer look on APV/FEC interplay with the board capacity is needed
		\item
			efficiency $>$ 90\% observed, but regions with considerably lower efficiency exit
			
			$\Rightarrow$ closer look at cuts and zerosuppression
		\item
			efficiency spoiled by rare trips of a few HV sectors (P. Arruba $\Rightarrow$ HV Monitoring) and zebra connection
		\item
			calibration shows smaller defects due to humidity,
			
			misalignment during gluing reconstructible
		\item
			some boards show ``Saclay banana shape'' 
			
			(preliminary, systematics have to be excluded)
		\item
			resolution for perpendicular incident near expectation
		
			$\Rightarrow$ resolution for inclined incident has to be further investigated
		\item
			measurements and investigations ongoing
	\end{itemize}
}

\appendix

\frame{\centering \Huge Backup}

\frame{\frametitle{HV Behavior Eta Out Board 7}
	\vspace{-0.5mm}
	\begin{columns}
		\column{.50\textwidth}
			\centering
			voltage left sector
			
			\includegraphics[width=0.9\textwidth]{m1_voltage42.pdf}
		\column{.50\textwidth}
			\centering
			voltage right sector
			
			\includegraphics[width=0.9\textwidth]{m1_voltage43.pdf}
	\end{columns}
	\vspace{-1mm}
	\begin{columns}
		\column{.50\textwidth}
			\centering
			current left sector
			
			\includegraphics[width=0.9\textwidth]{m1_current42.pdf}
		\column{.50\textwidth}
			\centering
			current right sector
			
			\includegraphics[width=0.9\textwidth]{m1_current43.pdf}
	\end{columns}
}

\frame{\frametitle{Strip Charge Per Board}
	\begin{columns}
		\column{.50\textwidth}
			\centering
			eta out
			
			\includegraphics[width=0.9\textwidth]{m1_eo_stripCharge_perBoard.pdf}
		\column{.50\textwidth}
			\centering
			eta in
			
			\includegraphics[width=0.9\textwidth]{m1_ei_stripCharge_perBoard.pdf}
	\end{columns}
	\begin{columns}
		\column{.50\textwidth}
			\centering
			stereo in
			
			\includegraphics[width=0.9\textwidth]{m1_si_stripCharge_perBoard.pdf}
		\column{.50\textwidth}
			\centering
			stereo out
			
			\includegraphics[width=0.9\textwidth]{m1_so_stripCharge_perBoard.pdf}
	\end{columns}
}

\frame{\frametitle{Mean Cluster Charge Individual Scale}
	\begin{columns}
		\column{.50\textwidth}
			\centering
			eta out
			
			\includegraphics[width=0.7\textwidth]{m1_etaout_A570V_C300V_clusterQmean.png}
		\column{.50\textwidth}
			\centering
			eta in
			
			\includegraphics[width=0.7\textwidth]{m1_etain_A570V_C300V_clusterQmean.png}
	\end{columns}
	\vspace{2.3mm}
	\begin{columns}
		\column{.50\textwidth}
			\centering
			stereo in
			
			\includegraphics[width=0.7\textwidth]{m1_stereoin_A570V_C300V_clusterQmean.png}
		\column{.50\textwidth}
			\centering
			stereo out
			
			\includegraphics[width=0.7\textwidth]{m1_stereoout_A570V_C300V_clusterQmean.png}
	\end{columns}
}

\frame{\centering \Huge Module Zero Results}

\frame{\frametitle{Full Area Pulse Height and Efficiency for Module Zero}
	\begin{columns}
		\column{.50\textwidth}
			\centering
			eta in cluster charge
			
			\includegraphics[width=0.7\textwidth]{SM2-M0_CRFafterH8_eta-in_MPVclusterQ.pdf}
		\column{.50\textwidth}
			\centering
			eta in \SI{5}{mm} efficiency
			
			\includegraphics[width=0.7\textwidth]{SM2-M0_CRFafterH8_eta-in_5mmEfficiency600V.pdf}
	\end{columns}
	\vspace{2.3mm}
	\centering
	amplification scan for the central area of all layer
	\begin{columns}
		\column{.50\textwidth}
			\centering
			\includegraphics[width=0.8\textwidth]{SM2-M0_CRF_MPVclusterQvsAmplificationVoltage_errorWidth_halfLines.pdf}
		\column{.50\textwidth}
			\centering
			\includegraphics[width=0.8\textwidth]{SM2-M0_CRFafterH8_5mmEfficiency_allCluster_halfLines.pdf}
	\end{columns}
}

\frame{\frametitle{Alignment using Reference Tracks}
	\centering \textbf{Idea:}
	\begin{columns}
		\column{.50\textwidth}
			\centering
			\includegraphics[width=0.90\textwidth]{positionshift2-crop.pdf}
		\column{.50\textwidth}
			\centering
			\includegraphics[width=0.90\textwidth]{verticalshift-crop.pdf}
	\end{columns}
	\begin{columns}
		\column{.55\textwidth}
			\centering
			
			%{\rotatebox{90}{\textbf{\tiny \hspace{3mm} residual = measured - reference position}}}
			\includegraphics[width=0.93\textwidth]{resVSslope_blue.pdf}
		\column{.45\textwidth}
			\small
			\textbf{Implementation:}
			\begin{itemize}
				\item
					%residual = measured - reference position
					\textbf{residual} = $\mathrm{pos}_{\mathrm{measured}} - \mathrm{pos}_{\mathrm{reference}}$
				\item
					residual vs. slope 
					
					 of reference track 
					 
					$\Rightarrow$ {\color{red}linear fit}
				\item	
					$\mathrm{shift}_{\mathrm{horizontal}} = \mathrm{intercept}_{\mathrm{fit}}$
				\item						
					$\mathrm{shift}_{\mathrm{vertical}} \;\;\; = \mathrm{slope}_{\mathrm{fit}}$
			\end{itemize}
	\end{columns}
}

\frame{\frametitle{Reconstructed Deviations per Partition \\ Module Zero Eta In}
	\small

	$\Rightarrow$ subdivide the active area into smaller partitions
	
	$\Rightarrow$ reconstruct separate for each partition the horizontal and the vertical shift
	
	\vspace{3mm}
	
	\begin{columns}
		\column{.50\textwidth}
			\centering
			horizontal shift
			
			\includegraphics[width=0.90\textwidth]{sm2_m0_CRF_eta-out_deltaYperPartition.pdf}
			\scriptsize
			
			$\Rightarrow$ shifts between PCBs have to be investigated
			
			$\Rightarrow$ rotations on this scale are hardly recognizable
		\column{.50\textwidth}
			\centering
			vertical shift
			
			\includegraphics[width=0.90\textwidth]{sm2_m0_CRF_eta-out_deltaZperPartition.pdf}
			
			\scriptsize
			$\Rightarrow$ gravitational sag deforms all layers the same way
	\end{columns}
}

\frame{\frametitle{Reconstructed Pitch Error and Board Alignment per Plane \\ Module Zero Eta In}
	\begin{columns}
		\column{.50\textwidth}
			\centering
			{\color{blue}pitch error} and {\color{red}misaligned electronic adapter board}
			
			\includegraphics[width=0.90\textwidth]{sm2_m0_CRF_eta-in_color.pdf}
			
			\vspace{3mm}
			
			pitch error [10$^{-4}$]
			\scriptsize
			\begin{tabular}{cc|ccc}
			panel & side & small & middle & large
			\\
			\hline
			stereo & out & 0.8 & -2.4 & 2.7
			\\
			stereo & in &  & 1.4 & 
			\\
			\hline
			eta & in & 1.0 & 2.9 & 3.6
			\\
			eta & out & 2.3 & 4.4 & 4.7
			\end{tabular}
		\column{.50\textwidth}
			\centering
			corrected pitch and shifted adapter board
			
			\includegraphics[width=0.90\textwidth]{SM2-M0_CRFafterH8_eta-in_resMeanVSmdtY.pdf}
			
			\vspace{3mm}
						
			center shift [mm]
			\scriptsize
			\begin{tabular}{cc|cc}
			panel & side & small & large
			\\
			\hline
			stereo & out & 0.09 & -0.06
			\\
			stereo & in & -0.18 & 0.25
			\\
			\hline
			eta & in & -0.07 & 0.13
			\\
			eta & out & 0.00 & 0.24
			\end{tabular}
	\end{columns}
}

\frame{\frametitle{Position Resolution as Function of the Incident Angle}
	\vspace{-1mm}
	
	\begin{columns}
		\column{.50\textwidth}
			\centering
			\footnotesize
			\textbf{resolution eta layer}
			\includegraphics[width=0.9\textwidth]{sm2_m0_etas_resolutionVSangle.pdf}
			
			\vspace{-0.5mm}
			\textbf{eta out resolution {\color{red}timing corrected}}
			\includegraphics[width=0.9\textwidth]{SM2-M0_eta-out_CRFafterH8_resolution_wTimeCor.pdf}
		\column{.50\textwidth}
			\footnotesize
			\begin{itemize}
				\item
					residual distribution for each angle separate
				\item
					fit with double Gaussian
					
					$\Rightarrow$ sigma narrow Gaussian 
					
					\hspace{3.5mm}(reject multiple scattering)
				\item
					consider track uncertainty of reference chambers: 
				
					$\sigma_{\mathrm{micromegas}} = \sqrt{\sigma_{\mathrm{res}}^2 - \sigma_{\mathrm{track}}^2}$
				\item
					resolution reaches for perpendicular incident about \SI{110}{\micro\m} (eta in),
					
					close to expectation
				\item
					timing correction (see B.Flierl PhD thesis - Particle Tracking with Micro-Pattern Gaseous Detectors)
					improves resolution
			\end{itemize}
	\end{columns}
}

\end{document}
