\documentclass[usenamees,dvipsnames]{beamer}
\usetheme{Madrid}
\usecolortheme{spruce}
\definecolor{green(pigment)}{rgb}{0.0, 0.65, 0.31}
\setbeamercolor*{item}{fg=green(pigment)}
\definecolor{Green}{rgb}{0.00, 1.00, 0.00}
\definecolor{Red}{rgb}{1.00, 0.00, 0.00}
\definecolor{Blue}{rgb}{0.00, 0.00, 1.00}
\usepackage[utf8]{inputenc}
\usepackage{amsmath}
\usepackage{amsfonts}
\usepackage{amssymb}
\usepackage[german]{babel}
\usepackage{graphicx}
\usepackage{rotating}
\usepackage{textcomp}
\usepackage{multirow,bigdelim,dcolumn,booktabs}
%\usepackage{beamerthemeshadow}
\usepackage{subfigure} 
\usepackage{siunitx}
\usepackage{appendixnumberbeamer}
\usepackage{hyperref}
\usepackage{mdframed}
\usepackage{xcolor}
\usepackage[absolute,overlay]{textpos}
%\beamersetuncovermixins{\opaqueness<1>{25}}{\opaqueness<2->{15}}
\beamertemplatenavigationsymbolsempty

\usepackage{tikz}
\usetikzlibrary{decorations.text}
%\usetikzlibrary{trees}
\usetikzlibrary{decorations.pathmorphing}
\usetikzlibrary{decorations.markings}
\usetikzlibrary{patterns}

\graphicspath{
	{pictures/}
%	{/home/m/Maximilian.Herrmann/Bilder/forLatex/plots/}
%	{/home/m/Maximilian.Herrmann/Bilder/forLatex/sketches/}
%	{/home/m/Maximilian.Herrmann/Bilder/forLatex/pictures/}
}

\begin{document}
\title[SM2 in CRF]{Micromegas Modules in the Cosmic Ray Facility}  
\author[M. Herrmann]{Maximilian Herrmann}
\institute[LMU Munich]{Ludwig-Maximilians-Universit\"at M\"unchen - Lehrstuhl Schaile}
\date[08.05.2019]{Abteilungsseminar 08.05.2019} 

\frame{
	\titlepage
	
	\centering
	\includegraphics[width=0.5\textwidth]{hardwareGroupWithM5.png}
} 

%\frame{\frametitle{Outline}
%	\tableofcontents
%} 

\section{Micromegas Quadruplets for ATLAS}

\frame{\frametitle{Micromegas Quadruplets for ATLAS}

	\begin{columns}
		\column{.55\textwidth}
			\centering
			\includegraphics[width=\textwidth]{ATLAS_wSW.png}
			
			\vspace{2mm}
			\includegraphics[width=0.7\textwidth]{sectorsDimensions.png}
		\column{.45\textwidth}
			\centering
			\includegraphics[width=0.7\textwidth]{LHCunderground.png}
			
			\vspace{2mm}
			\hspace{3mm}\includegraphics[width=0.9\textwidth]{sectorsNsides_SM2marked.png}
	\end{columns}
}

\frame{\frametitle{Micromegas Concept and Realization}
	
	\begin{columns}
		\column{.5\textwidth}
			\centering
			\includegraphics[width=0.95\textwidth]{RSMM_principle8-crop.pdf}
			
			\vspace{3mm}
			\includegraphics[width=0.45\textwidth]{readoutNdriftPanelsInCleanroom.jpg}
		\column{.5\textwidth}
%			\small
			position reconstruction:
			\vspace{3mm}
			
			$
				{\bf \bf \mathrm{centroid}} 
			 = 
				\dfrac{
					\sum\limits_{\scriptscriptstyle\mathrm{strips}} \mathrm{strip} \cdot q_{\scriptscriptstyle\mathrm{strip}}
				}
				{\sum\limits_{\scriptscriptstyle\mathrm{strips}} q_{\scriptscriptstyle\mathrm{strip}} }
			$
			\vspace{5mm}
			
			drift time measurement
			
			$\Rightarrow$ incident angle reconstruction
			\vspace{6mm}
			
			five stable sandwich panels
			
			$\Rightarrow$ four active layers per module
			\vspace{1mm}
			\centering
			\includegraphics[width=\textwidth]{sandwichassemblyWarrows.png}
	\end{columns}
}

\frame{\frametitle{Design of Readout Anodes}
	\begin{columns}
		\column{.5\textwidth}
			\centering
			\includegraphics[width=\textwidth]{quadrupletStripOrientation3-crop.pdf}
		\column{.5\textwidth}
			\centering
			\includegraphics[width=\textwidth]{sandwichassembly.png}
	\end{columns}
	
	\vspace{3mm}
	\begin{columns}
		\column{.5\textwidth}
			\centering
			\includegraphics[width=\textwidth]{ROboardsAligned2-crop.png}
		\column{.5\textwidth}
			\centering
			\includegraphics[width=0.65\textwidth]{dryCleaningAtRuis.jpeg}
	\end{columns}
}

\section{Cosmic Ray Facility}

\frame{\frametitle{Cosmic Ray Facility}
		
	\begin{columns}
		\column{.50\textwidth}
			\centering
			\includegraphics[width=1.0\textwidth]{CRFprinciple3-crop.pdf}
		\column{.50\textwidth}
			\centering
			\includegraphics[width=0.9\textwidth]{CRF_small.JPG}
	\end{columns}
	
	\vspace{2mm}
	\centering
	\footnotesize
	\begin{tabular}{ll}
		trigger & 2 $\times$ scintillator hodoscopes
		\\
		track reconstruction & 2 $\times$ Monitored Drift Tube chambers (MDTs)
		\\
		active area & \SI{2.2}{m} $\times$ \SI{4}{m}
		\\
		angular acceptance & $\pm$ \SI{30}{\degree} to zenith
		\\
		energy cut (hardware) & iron absorber (\SI{34}{cm}) $\to E_{\mu} >$ \SI{600}{MeV} 
		\\
		readout & 12288 channels
		\\
		 & $\to$ 96 APVs (frontend electronics)
		\\
		 & $\to$ 6 FECs (scalable readout system)
		\\
		readout rate & 100 Hz (full muon rate)
	\end{tabular}
}

\section{Calibration for QA/QC}

\subsection{Efficiency}

\frame{\frametitle{Efficiency = good counts / all counts}

	\begin{columns}
		\column{.65\textwidth}
			\centering
			\includegraphics<1>[width=\textwidth]{m3_ei_ampScan_woLayer.png}
			\includegraphics<2>[width=\textwidth]{m3_ei_ampScan_woLayer_arrow.png}
		\column{.35\textwidth}
			\centering
			\includegraphics[width=0.65\textwidth]{RSMM_principle_bare-crop.pdf}
			\vspace{3mm}
			
			\includegraphics[width=0.65\textwidth]{ROboardsHVsectorNames-crop.pdf}
	\end{columns}
	\vspace{5mm}
	
	\begin{itemize}
		\item[] Why large spreads between HV sectors?
		\item[] Why not fully efficient? \only<2>{ \textcolor{red}{expectation 96\%} }
	\end{itemize}
	
}

\frame{\frametitle{Efficiency influenced by Pulse Height}

	\begin{columns}
		\column{.5\textwidth}
			\centering
			pulse height map
			
			\includegraphics[width=0.65\textwidth]{m3_eo_clusterQmpv_20180911.pdf}
		\column{.5\textwidth}
			\centering
			efficiency map
			
			\includegraphics[width=0.65\textwidth]{m3_eo_coinEffi_20180911.pdf}
	\end{columns}
	
	\vspace{2mm}
	
	\begin{columns}
		\column{.5\textwidth}
			\centering
			pulse height VS pillar height
			
			\includegraphics[width=\textwidth]{meanClusterQvsPillarHeight_etaNstereo_wFit.png}
		\column{.5\textwidth}
			\centering
			\includegraphics[width=0.6\textwidth]{RSMM_principle_bare-crop.pdf}
	\end{columns}
	
}

\frame{\frametitle{Efficiency spoiled by Electronic Noise (Preliminary)}

	\only<2>{
		\begin{textblock*}{5cm}(1.75cm,3.6cm) 
			\begin{tikzpicture}
				\draw[red,->, line width=1mm] (0,0) -- (0,3.6);
			\end{tikzpicture}
		\end{textblock*}
		\begin{textblock*}{5cm}(2.85cm,2.9cm) 
			\begin{tikzpicture}
				\draw[red,->, line width=1mm] (0,0) -- (0,4.6);
			\end{tikzpicture}
		\end{textblock*}
		\begin{textblock*}{5cm}(3.4cm,2.7cm) 
			\begin{tikzpicture}
				\draw[red,->, line width=1mm] (0,0) -- (0,4.9);
			\end{tikzpicture}
		\end{textblock*}
		\begin{textblock*}{5cm}(4.3cm,3.6cm) 
			\begin{tikzpicture}
				\draw[red,->, line width=1mm] (0,0) -- (0,3.5);
			\end{tikzpicture}
		\end{textblock*}
	}

	\begin{columns}
		\column{.5\textwidth}
			\centering
			efficiency per readout chip
			
			\includegraphics[width=0.85\textwidth]{m7_so_570V_5mmEffi_perAPV.pdf}
		\column{.5\textwidth}
			\centering
			\includegraphics[width=0.65\textwidth]{ROboardWithAPV-crop.pdf}
	\end{columns}
	
	\vspace{2mm}
	
	\begin{columns}
		\column{.5\textwidth}
			\centering
			electronic noise per readout chip
			
			\includegraphics[width=0.85\textwidth]{m7_so_sigmas_perAPV.pdf}
		\column{.5\textwidth}
			\centering
			efficiency VS electronic noise
			
			\includegraphics[width=0.85\textwidth]{m7_5mmEffiVSsigmaPerAPV_wFit.png}
	\end{columns}
	
}

\subsection{Strip Shape} 

\frame{\frametitle{Tracking in the Cosmic Ray Facility}
		
	\begin{columns}
		\column{.50\textwidth}
			\centering
			\includegraphics[width=1.0\textwidth]{CRFprinciple5-crop.pdf}
		\column{.50\textwidth}
			\centering
			\includegraphics[width=0.9\textwidth]{CRF_small.JPG}
	\end{columns}
	
	\vspace{7mm}
	\centering
	\begin{tabular}{ccccc}
		residual & = & {\color{blue}measured} & - & {\color{red}reference}
		\\
		 & & & & 
		\\
		 & = & centroid $\times$ pitch & - & track$_{\mathrm{MDTs}}$ @ MM
	\end{tabular}
}

\frame{\frametitle{Reconstruction of Readout Board Alignment}

	\begin{columns}
		\column{.6\textwidth}
			\centering
			\begin{columns}
				\column{.5\textwidth}
					\centering
					\includegraphics[width=0.7\textwidth]{m1_ei_residaul_nearZero.pdf}
				\column{.5\textwidth}
					\centering
					\hspace{-20mm} $\Rightarrow$ mean residual
			\end{columns}
			\vspace{2mm}
			\includegraphics[width=0.8\textwidth]{m3_eo_deltaYperPart_2019-09.png}
		\column{.4\textwidth}
			\centering
			\includegraphics[width=0.9\textwidth]{m3_eo_resMeanVSscinX_allBoards.png}
			\footnotesize
			
			\vspace{2mm}
			undesired strip shape
			
			$\Rightarrow$ calibration needed
			\vspace{2mm}
			
			\includegraphics[width=0.9\textwidth]{ROboardsBendStrips-crop.png}
			
			causes for deviation from design:						
			\begin{itemize}
				\footnotesize
				\item
					humidity $\Rightarrow$ known issue
					\vspace{-1mm}
				\item
					temperature ?
					\vspace{-1mm}
				\item
					PCB production ?
			\end{itemize}
	\end{columns}
	
}

\section{Track Reconstruction with Micromegas}

%\frame{\centering \Huge Track Reconstruction with Micromegas}

\subsection{Charge Weighted Position}

\frame{\frametitle{\large Resolution of Charge Weighted Reconstruction}
	
	\begin{columns}
		\column{.5\textwidth}
			\centering
			\only<1>{
				residual straight tracks
				
				\includegraphics[width=0.8\textwidth]{m1_ei_residaul_nearZero.pdf}
			}
			\only<2>{
				residual VS angle
				
				\includegraphics[width=0.65\textwidth]{m1_ei_resVSangle.pdf}
			}
		\column{.5\textwidth}
			\centering
			resolution VS angle
				
			\includegraphics[width=0.85\textwidth]{m1_resolutionVSangle_eoNei.pdf}
	\end{columns}
	
	\vspace{2mm}
		
	\begin{columns}
		\column{.5\textwidth}
			\centering
			simulation
			
			\includegraphics[width=0.65\textwidth]{inhomogeneousIonization_bE_cNm_centroidShift.pdf}
		\column{.5\textwidth}
			inhomogeneous ionization (MIP)
			
			\vspace{0.5mm}
			$\Rightarrow$ declining resolution 
			
			\hspace{4.5mm} for larger angles
			
			\vspace{3mm}
			
			solution: 
			
			use drift time information
	\end{columns}
	
}

\subsection{Drift Time Measurement}

\frame{\frametitle{Drift Time Measurement for Position Reconstruction}
	
	\begin{columns}
		\column{.5\textwidth}
			\centering
			drift time fit
			
			\includegraphics[width=0.9\textwidth]{timeVSstrip_meshNcathodeNstrips_fitNtimeNangle.png}
		\column{.5\textwidth}
			\centering
			angle reconstruction
			
			\includegraphics[width=0.65\textwidth]{m1_ei_uTPCangleVSangle_20180601_uptime_sameRange.pdf}
	\end{columns}
	
	\vspace{1mm}
		
	\begin{columns}
		\column{.5\textwidth}
			\centering
			simulation
			
			\includegraphics[width=0.7\textwidth]{inhomogeneousIonization_bE_cNm_timeNcentroidShift.pdf}
		\column{.5\textwidth}
			\centering
			resolution VS angle
			
			\includegraphics[width=0.8\textwidth]{m1_ei_allResolutionsVSangle_20180601.pdf}
	\end{columns}
	
}

\subsection{Stereo Reconstruction}

%\frame{\centering \Huge Stereo Reconstruction}

\frame{\frametitle{Position Reconstruction with Two Layer of Inclined Strips}
			
	\begin{columns}
		\column{.5\textwidth}
			\includegraphics[width=0.9\textwidth]{stereoBoards_angleNshift_centerNcoord-crop.pdf}
			\vspace{2mm}
			
			\textcolor{blue}{precision} 
			
			\hspace{1mm} = stereos mean / $\tan \alpha$ 
			\vspace{1.5mm}
						
			\textcolor{red}{non-precision}
			
			\hspace{1mm} = stereos difference / 2 $/ \tan \alpha$
			\vspace{1.5mm}
						
			$\Rightarrow$ \textcolor{Brown}{alignment} 
			
			\hspace{4.5mm} non-precision coordinate
			
			\vspace{5mm}
			$\Rightarrow$ non-precision residual spoiled 
			
			\hspace{4.5mm} by coarse scintillator resolution
		\column{.5\textwidth}
			\centering	
			\includegraphics[width=0.65\textwidth]{m1_stereo_posDifVSscinX_wCenter.pdf}
			
			\vspace{2mm}
			\includegraphics[width=0.95\textwidth]{m1_stereoResiduals_nonNprecision_BF.pdf}
	\end{columns}
	
}

\frame{\frametitle{Results with Final VMM Electronics from Max Rinnagel}
			
	\begin{columns}
		\column{.5\textwidth}
			\centering
			\includegraphics[width=0.9\textwidth]{m1_H8_2018_tiltedSetup.png}
			\vspace{2mm}
						
			\begin{itemize}
				\small
				\item
					testbeam at CERN (SPS) 
					
					with first series module (SM2-M1)
				\item
					final electronics 
					
					$\Rightarrow$ will be used in ATLAS
					
					(VMM 3a on MMFE8)
				\item
					promising results 
					
					also for other gas mixtures
			\end{itemize}
		\column{.5\textwidth}
			\centering	
			\includegraphics[width=\textwidth]{m1_vmm_resolutionVSampVoltage_3gases.pdf}
			
			\vspace{2mm}
			\includegraphics[width=\textwidth]{m1_vmm_efficiencyVSampVoltage_3gases.pdf}
	\end{columns}
	
}

\frame{\frametitle{Summary}
	\small
	\begin{itemize}
		\item
			Micromegas quadruplets for the ATLAS muon spectrometer inner end cap
		\item
			investigation in the Cosmic Ray Facility
		\item
			calibration
			
			\begin{itemize}
				\small
				\item
					efficiency spoiled by electronics and anode variation
				\item
					undesired strip shape and board alignment distort readout pattern
			\end{itemize}
		\item
			track reconstruction
						
			\begin{itemize}
				\small
				\item
					resolution depending on incident angle
				\item
					drift time measurement 
					
					$\Rightarrow$ resolution improvement for inclined incident
				\item
					stereo reconstruction enables more precise investigation
			\end{itemize}
		\item
			measurements with final electronics at CERN SPS
			
			$\Rightarrow$ good resolution \SI{120}{\micro m} at high efficiency 96\%
	\end{itemize}
}

\appendix

\frame{\centering \Huge Backup}

\frame{\frametitle{APV25 Frontend Electronics}

	\begin{columns}
		\column{.5\textwidth}
			\centering
			APV25
			
			\includegraphics[width=0.8\textwidth]{APV25.png}
		\column{.5\textwidth}
			\centering			
			\begin{itemize}
				\small
				\item
					128 input channel
					\vspace{2mm}
				\item
					charge-sensitive preamplifier
					\vspace{2mm}
				\item
					analog pipeline memory
					\vspace{2mm}
				\item
					\SI{40}{MHz} clock 
					
					$\Rightarrow$ \SI{25}{ns} sampling
			\end{itemize}
	\end{columns}
	
	\vspace{4mm}
	
	\begin{columns}
		\column{.5\textwidth}
			\centering
			eventdisplay
			
			\includegraphics[width=0.85\textwidth]{eventdisplay.png}
			\vspace{3mm}
		\column{.5\textwidth}
			\centering
			single strip signal
			
			\includegraphics[width=0.8\textwidth]{pulseheight_differentSignaltimes.pdf}
	\end{columns}
	
}

\frame{\frametitle{\large Reconstruction of Pitch Deviations and Readout Board Alignment}

	\begin{columns}
		\column{.5\textwidth}
			\centering
			without correction
			
			\includegraphics[width=0.8\textwidth]{eta_in_resMeanVSmdtY_woAdapterBoard_wCorPitch_wCenter.png}
		\column{.5\textwidth}
			\centering
			pitch deviation considered
			
			\vspace{-0.3mm}
			\includegraphics[width=0.8\textwidth]{eta_in_resMeanVSmdtY_wNewPitchCor_shifts.pdf}
	\end{columns}
	
	\vspace{4mm}
	
	\begin{columns}
		\column{.5\textwidth}
			\centering
			\includegraphics[angle=-90,width=0.55\textwidth]{ROboardsAligned1_wRingsNlines-crop.pdf}
			\vspace{3mm}
		\column{.5\textwidth}
			\centering
			\includegraphics[width=0.8\textwidth]{boardPitchDeviation_wLinesNdesNarrows-crop.pdf}
	\end{columns}
	
}

\frame{\frametitle{Recapitulation \textmu TPC Formulas}
	\centering
	\includegraphics[width=0.7\textwidth]{timeVSstrip_meshNcathodeNstrips_fitNtimeNangle.pdf}
			
	\begin{itemize}
		\item
			angle reconstruction: 
			{\boldmath$\textcolor{Plum}{\theta}$}$\; = \arctan \left( \tfrac{\mathrm{pitch}}{\textcolor{red}{\mathrm{slope}} \;\cdot\; v_{\mathrm{drift}}} \right)$
			
			\vspace{3mm}
		\item
			position reconstruction:
			$
			\mathrm{pos}_{\mu\mathrm{TPC}}
			=
			\dfrac{\textcolor{Cyan}{t_{\mathrm{mid}}} - \textcolor{red}{\mathrm{intercept}}}{\textcolor{red}{\mathrm{slope}}}
			\cdot \textcolor{brown}{\mathrm{pitch}}
			$
	\end{itemize}
	
}

\frame{\frametitle{Concept for Charge Weighted Timing Reconstruction}

	\centering
	\includegraphics<1,2>[width=0.5\textwidth]{inhomogeneousIonization_bE_cNm_timeNcentroidShift_deltaXnTnAngle.pdf}
	
	\only<1>{
		\vspace{4mm}
		\begin{tabular}{ccccc}
			$\bigtriangleup x$ & = & $\bigtriangleup z$ & $\cdot$ & $\tan \theta$
			\\
			 & & & &
			\\
			 & = & $( t_{\mathrm{c}} - t_{\mathrm{mid}} ) \cdot v_{\mathrm{drift}}$ &  $\cdot$ & $\mathrm{slope}_{\mathrm{ref}}$
		\end{tabular}
	}
	\only<2>{
		\vspace{4mm}
		\begin{tabular}{ccccc}
			$\mathrm{centroid}$ & = & $\sum\limits_{\scriptscriptstyle\mathrm{strips}} \mathrm{strip} \cdot q_{\scriptscriptstyle\mathrm{strip}}$ & $/$ & $\sum\limits_{\scriptscriptstyle\mathrm{strips}} q_{\scriptscriptstyle\mathrm{strip}}$
			\\
			 & & & &
			\\
			$t_{\mathrm{c}}$ & = & $\sum\limits_{\scriptscriptstyle\mathrm{strips}} t_{\scriptscriptstyle\mathrm{strip}} \cdot q_{\scriptscriptstyle\mathrm{strip}}$ &  $/$ & $\sum\limits_{\scriptscriptstyle\mathrm{strips}} q_{\scriptscriptstyle\mathrm{strip}}$
		\end{tabular}
	}
			
}

\end{document}
