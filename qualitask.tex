\documentclass{beamer}
\usetheme{Madrid}
\usecolortheme{spruce}
\definecolor{green(pigment)}{rgb}{0.0, 0.65, 0.31}
\setbeamercolor*{item}{fg=green(pigment)}
\definecolor{Green}{rgb}{0.00, 1.00, 0.00}
\definecolor{Red}{rgb}{1.00, 0.00, 0.00}
\definecolor{Blue}{rgb}{0.00, 0.00, 1.00}
\usepackage[utf8]{inputenc}
\usepackage{amsmath}
\usepackage{amsfonts}
\usepackage{amssymb}
\usepackage[german]{babel}
\usepackage{graphicx}
\usepackage{rotating}
\usepackage{multirow,bigdelim,dcolumn,booktabs}
%\usepackage{beamerthemeshadow}
\usepackage{subfigure} 
\usepackage{siunitx}
%\beamersetuncovermixins{\opaqueness<1>{25}}{\opaqueness<2->{15}}
\beamertemplatenavigationsymbolsempty

\usepackage{tikz}
\usetikzlibrary{decorations.text}
\usetikzlibrary{trees}
\usetikzlibrary{decorations.pathmorphing}
\usetikzlibrary{decorations.markings}
\usetikzlibrary{patterns}

\graphicspath{
	{/home/m/Maximilian.Herrmann/Bilder/forLatex/plots/}
	{/home/m/Maximilian.Herrmann/Bilder/forLatex/sketches/}
	{/home/m/Maximilian.Herrmann/Bilder/forLatex/pictures/}
}

\begin{document}
\title[CNC Calibration]{Calibration of CNC Machines for Readout Panel Drilling and Inspection}  
\author[M. Herrmann]{Maximilian Herrmann}
\institute[LMU Munich]{Ludwig-Maximilians-Universit\"at M\"unchen - Lehrstuhl Schaile}
\date{05.07.2017, Munich} 

\frame{\titlepage} 

\frame{\frametitle{Outline}\tableofcontents}

\section{Motivation: Tasks for the Munich Production Site}

\frame{\frametitle{Tasks for the Munich Production Site}
	Excerpt:

	\begin{itemize}
		\item
			SM2 Readout Panel Gluing 
			
			$\Rightarrow$ Granite Table (with Coordinate Measurement Machine = CMM)
		\item 
			Drilling of the Holes in the Drift and Readout Panels for SM2
			
			$\Rightarrow$ Computerized Numerical Controlled Drilling Machine ( = CNC)
	\end{itemize}
	
	Questions for Quality Assurance:
	
	\begin{itemize}
		\item
			Alignment Accuracy of Readout Boards during Gluing
		\item 
			Accuracy of Drilling Positions
	\end{itemize}
	
	Solutions for Quality Assurance:
		
	\begin{itemize}
		\item
			Measure Positions of Markers on Boards and Holes in Panels
		\item 
			Calibrate the CNC
	\end{itemize}
}

\frame{\frametitle{CNC Machines for Drilling and Quality Control}
	\begin{figure}
		\begin{tikzpicture}
			\node[anchor=south west,inner sep=0] (image) at (0,0) {\includegraphics[width=0.5\linewidth]{graniteWcmm.JPG}};
			\node[anchor=south west,inner sep=0] (image) at (6.3,0) {\includegraphics[width=0.5\linewidth]{drillingCNC.JPG}};
			
			\node[right,black] at (0.5,5.0) {\large Granite Table with CMM};
			\node[right,black] at (7.0,5.0) {\large CNC Drilling Machine};
			
			\node[right,red] at (3.9,3.5) {\large x};
			\draw[arrows=->,red,ultra thick](3.8,3.2)--(4.6,3.35);
			\node[right,red] at (2.2,1.0) {\large y};
			\draw[arrows=->,red,ultra thick](3.0,0.1)--(1.35,1.3);
			\node[right,red] at (5.0,2.6) {\large y};
			\draw[arrows=->,red,ultra thick](5.7,2.2)--(4.4,2.5);
			
			\node[right,red] at (10.6,2.9) {\large x};
			\draw[arrows=->,red,ultra thick](10.1,2.68)--(11.6,2.63);
			\node[right,red] at (7.7,1.8) {\large y};
			\draw[arrows=->,red,ultra thick](7.75,1.38)--(8.7,2.0);
			\node[right,red] at (10.7,1.5) {\large y};
			\draw[arrows=->,red,ultra thick](11.4,0.9)--(11.35,1.8);
		\end{tikzpicture}
	\end{figure}
}

\section{Scheme for Measurement of Board Alignment}

\frame{\frametitle{Scheme for Measurement of Board Alignment}
	\begin{columns}
		\column{.50\textwidth}
			\centering
			\includegraphics[width=0.9\textwidth]{ROpanelATpins_wLegend-crop.pdf}
		\column{.50\textwidth}
			\centering
			\begin{itemize}
				\item
					Pins on Table for Alignment of Panels
				\item 
					Precision Marker on each side of each Board on a Panel
				\item
					Measure Distances of Markers and Pins
					
					$\Rightarrow$ Investigation of Temperature and Humidity Effects 
			\end{itemize}
	\end{columns}
}

\frame{\frametitle{Realization of Measurement of Board Alignment}
	\begin{columns}
		\column{.03\textwidth}
		\column{.30\textwidth}
			\centering
			\includegraphics[angle=270, width=0.95\textwidth]{telecentricCameraAtArm2.JPG}
		\column{.67\textwidth}
			\centering
			\includegraphics[width=0.8\textwidth]{microscope_screenshot_20160420_marker_board7_awayPC.jpeg}
	\end{columns}
	
	\begin{itemize}
		\item
			Positioning of the CMM Measurement Arm above the Marker/Pin
		\item
			Taking a Picture of the Marker/Pin with a Telecentric Camera
		\item
			Analyzing the Position on the Picture and Adding these Coordinates to the Measurement Position
	\end{itemize}

	\centering
	$\Rightarrow$ Comparison of the measured Positions for each Marker/Pin
}

\frame{\frametitle{Approach}
	
	\begin{itemize}
		\item
			Automatized Identification of the Marker in the Picture and
			
			Calculation of the Coordinates in the CMM System
			
			$\Rightarrow$ OpenCV Program for Identification and Fitting of the Marker Shape
		\item
			Ensure Accuracy of the absolute CMM Position
			\begin{itemize}
				\item
					Repetitive Accuracy
				\item
					Reversing Play
			\end{itemize}
			
			$\Rightarrow$ Calibration of the CMM with a Laserinterferometer
		\item
			CMM and CNC are of the same Type and from the same Manufacturer
			
			$\Rightarrow$ Same Procedure for Calibration of CNC is used
	\end{itemize}
	
}

\section{Identification of the Markers and Calculation of their Positions Using OpenCV}

\frame{\frametitle{Marker Identification Using OpenCV}

	\begin{columns}
		\column{.50\textwidth}
			\centering
			
			Picture of the Marker with the Telecentric Optic
			
			\includegraphics[width=0.6\textwidth]{marker_wAxes.png}
			
			Pixels over Threshold
			
			\includegraphics[width=0.7\textwidth]{markerThresholdPixel.png}
		\column{.50\textwidth}
			\centering
			
			Brightness Distribution
			
			\includegraphics[width=0.8\textwidth]{brightnessDistribution.png}
			
			\begin{itemize}
				\item
					Fit Peak in Brightness Distribution
				\item
					Set Threshold to 
					
					Mean plus 2 $\times$ Sigma
				\item
					Weight Pixel over Threshold with 1, all other with 0
			\end{itemize}
	\end{columns}
	
}

\frame{\frametitle{Estimation of the Marker Position by the Projection}

	\begin{columns}
		\column{.50\textwidth}
			\centering
			
			Projection to the X axis
			
			\includegraphics[width=0.9\textwidth]{markerProjectionXaxis.png}
		\column{.50\textwidth}
			\centering
			
			Projection to the Y axis
			
			\includegraphics[width=0.9\textwidth]{markerProjectionYaxis.png}
	\end{columns}
	
	\vspace{5mm}
	\centering
	$\Rightarrow$ Problematic Identification of Marker due to surrounding Structures 
	
	(in this case neighboring test Marker)
}

\frame{\frametitle{Estimation of the Marker Position Using Template Matching}
			
	\begin{itemize}
		\item
			Make a Template of the Marker cropped to cover all necessary Parts
		\item
			Iterate over the Picture to be analyzed and calculate the Metric for each Pixel $(x,y)$:
			
			\small
			
			\begin{align*}
			R(x,y) = 
				\dfrac{
					\sum\limits_{x^{\prime} , y^{\prime}}
						T(x^{\prime} , y^{\prime} ) \cdot 
						I(x+x^{\prime} , y+y^{\prime} )
				}
				{	
					\sqrt{
						\sum\limits_{x^{\prime} , y^{\prime} }
							T^{2}(x^{\prime} , y^{\prime} ) 
						\sum\limits_{x^{\prime} , y^{\prime} }
							I^{2}(x+x^{\prime} , y+y^{\prime} )
					}
				}	
			\end{align*}
			
			with $(x^{\prime} , y^{\prime})$ the Pixels in the Template, $T$ brightness of the Template and $I$ brightness of the Picture.
			
			$\Rightarrow$ Maximum Value is given at the best Matching Position
		\item
			After Position is estimated the Circles of the Markers can be fitted to calculate the exact Position of the Marker
	\end{itemize}
	
}

\frame{\frametitle{Test of the Template Matching Method}

	\small

	\begin{columns}
		\column{.50\textwidth}
			\centering
			
			Template for Position Estimation
			
			\includegraphics[width=0.4\textwidth]{template.pdf}
		\column{.50\textwidth}
			\centering
			
			Separate Fits for each Circle
			
			\includegraphics[width=0.5\textwidth]{markerRingGraphs.png}
	\end{columns}
	
	\vspace{3mm}
	
	For Test of the Algorithm: Take Pictures at the same Position for different Heights above the Marker (in \SI{100}{\micro\m} Steps)
	
	\centering
	
	$\Rightarrow$ Average Chi$^2/$NDF of the three Fits VS Height above Marker:
	
	\begin{columns}
		\column{.50\textwidth}
			\centering
			
			Aperture 4{.}5
			
			\includegraphics[width=0.7\textwidth]{markerFit_averageChiSquareOverNDF_aperture4dot5.png}
		\column{.50\textwidth}
			\centering
			
			Aperture 22
			
			\includegraphics[width=0.7\textwidth]{markerFit_averageChiSquareOverNDF_aperture22.png}
	\end{columns}
	
}

\section{Calibration of the CMM and CNC Using a Laserinterferometer}

\frame{\frametitle{The CMM and the CNC in the SM2 Assembly Hall}
	\centering
	\includegraphics[width=1.0\textwidth]{cmmNcnc_atlasHall-crop.pdf}
}

\frame{\frametitle{Calibration with a Laserinterferometer}
	\begin{columns}
		\column{.60\textwidth}
			\centering
			Working Principle of a Laserinterferometer
			
			\includegraphics[width=1.0\textwidth]{laserinterferometerPrinciple.png}
			
			\vspace{2mm}
			
			Measurement Setup At the CMM 
			
			for Calibration
						
			\vspace{2mm}
			
			\includegraphics[width=0.65\textwidth]{laserinterferometerAtCMM-crop.pdf}
		\column{.40\textwidth}
			Laserinterferometer measures Distance Variations by Counting of Interference Maxima.
			
			\vspace{8mm}
			
			Difference of set Distance at CMM to measured Distance of Laserinterferometer allows Calibration of CMM Positions.
	\end{columns}
}

\frame{\frametitle{Calibration of the CMM - X Axis}

	\begin{columns}
		\column{.50\textwidth}
			\centering
			\SI{2}{mm} Steps over whole Range
			
			\includegraphics[width=0.8\textwidth]{cmmlaser_Xdeviation_shortNlong.pdf}
		\column{.50\textwidth}
			\centering
			Zoomed for Periodic Structure
			
			\includegraphics[width=0.8\textwidth]{cmmlaser_Xdeviation_shortNlong_zoomed.pdf}
	\end{columns}
	
	\begin{columns}
		\column{.50\textwidth}
			\centering
			Backward VS Forward
			
			\includegraphics[width=0.7\textwidth]{cmmlaser_Xdeviation_backVSforward.png}
		\column{.50\textwidth}
			\begin{itemize}
				\item
					no Y dependence of 
				\item
					Twofold Periodic Structure
					
					$\Rightarrow$ \SI{20}{mm} and \SI{100}{mm} Periods
				\item
					Backlash below \SI{5}{\micro\m}
					
					$\Rightarrow$ Negligible
			\end{itemize}
	\end{columns}
	
}

\frame{\frametitle{Calibration of the CMM - Y Axis}

	\begin{columns}
		\column{.50\textwidth}
			\centering
			\SI{2}{mm} Steps over whole Range
			
			\includegraphics[width=0.8\textwidth]{cmmlaser_Ydeviation_leftNright.pdf}
		\column{.50\textwidth}
			\centering
			Zoomed for Periodic Structure
			
			\includegraphics[width=0.8\textwidth]{cmmlaser_Ydeviation_leftNright_zoomed.pdf}
	\end{columns}
	
	\begin{columns}
		\column{.50\textwidth}
			\centering
			Backward VS Forward
			
			\includegraphics[width=0.7\textwidth]{cmmlaser_Ydeviation_backVSforward.png}
		\column{.50\textwidth}
			\begin{itemize}
				\item
					X position dependent Deviation
					
					$\Rightarrow$ no simple Calibration possible
				\item
					only \SI{20}{mm} Periodicity
				\item
					Backlash about \SI{40}{\micro\m}
					
					$\Rightarrow$ has to be considered
			\end{itemize}
	\end{columns}
	
}

\frame{\frametitle{Calibration of the CNC - Y Axis}

	\begin{columns}
		\column{.50\textwidth}
			\centering
			Long Side of Trapezoid
			
			\includegraphics[width=0.8\textwidth]{cncLaser_Y_Xat-16_linearFit.pdf}
		\column{.50\textwidth}
			\centering
			Short Side of Trapezoid
			
			\includegraphics[width=0.8\textwidth]{cncLaser_Y_Xat1352_linearFit.pdf}
	\end{columns}
	
	\begin{columns}
		\column{.50\textwidth}
			\centering
			\SI{2}{mm} Steps
			
			\includegraphics[width=0.8\textwidth]{cncLaser_Y_Xat-16_2mmSteps.pdf}
		\column{.50\textwidth}
			\begin{itemize}
				\item
					Linear Correction of the Global Slope Sufficient
				\item
					Substructure at 2 mm Steps below accuracy 
					
					$\Rightarrow$ Negligible
			\end{itemize}
	\end{columns}
	
}

\frame{\frametitle{Calibration of the CNC - X Axis}

	\begin{columns}
		\column{.50\textwidth}
			\centering
			Perpendicular to Parallel Sides of Trapezoid
			
			\includegraphics[width=0.8\textwidth]{cncLaser_X_Yat772dot5_twoStepSizes.pdf}
			
			\SI{2}{mm} Steps
			
			\includegraphics[width=0.8\textwidth]{cncLaser_X_Yat772dot5_2mmSteps.pdf}
		\column{.50\textwidth}
			\begin{itemize}
				\item
					Differences between Measurements of different Step Sizes indicate Substructure
				\item
					Substructure at 2 mm Steps visible 
					
					$\Rightarrow$ \SI{10}{mm} Periodicity
				\item
					to avoid Calibration of the whole Axis, only calibrate the Drilling Positions
			\end{itemize}
	\end{columns}
	
}

\end{document}
