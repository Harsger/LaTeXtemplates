\documentclass{beamer}
\usetheme{Madrid}
\usecolortheme{spruce}
\definecolor{green(pigment)}{rgb}{0.0, 0.65, 0.31}
\setbeamercolor*{item}{fg=green(pigment)}
\definecolor{Green}{rgb}{0.00, 1.00, 0.00}
\definecolor{Red}{rgb}{1.00, 0.00, 0.00}
\definecolor{Blue}{rgb}{0.00, 0.00, 1.00}
\usepackage[utf8]{inputenc}
\usepackage{amsmath}
\usepackage{amsfonts}
\usepackage{amssymb}
\usepackage[german]{babel}
\usepackage{graphicx}
\usepackage{rotating}
\usepackage{multirow,bigdelim,dcolumn,booktabs}
%\usepackage{beamerthemeshadow}
\usepackage{subfigure} 
\usepackage{siunitx}
%\beamersetuncovermixins{\opaqueness<1>{25}}{\opaqueness<2->{15}}
\beamertemplatenavigationsymbolsempty

\usepackage{tikz}
\usetikzlibrary{decorations.text}
\usetikzlibrary{trees}
\usetikzlibrary{decorations.pathmorphing}
\usetikzlibrary{decorations.markings}
\usetikzlibrary{patterns}

\graphicspath{
	{pictures/}
%	{/home/m/Maximilian.Herrmann/Bilder/forLatex/plots/}
%	{/home/m/Maximilian.Herrmann/Bilder/forLatex/sketches/}
%	{/home/m/Maximilian.Herrmann/Bilder/forLatex/pictures/}
}
\begin{document}
\title[\SI{2}{\square\m}-sized Micromegas Quadruplets]{Investigation of Square Meter Sized Micromegas Quadruplets with Cosmic Muons}  
\author[M. Herrmann]{Maximilian Herrmann}
\institute[LMU Munich]{Ludwig-Maximilians-Universit\"at M\"unchen - Lehrstuhl Schaile}
\date{21.03.2018, W\"urzburg} 

\frame{\titlepage
	\centering
	Fr\"uhjahrstagung der Deutschen Physikalischen Gesellschaft
	
	\includegraphics[width=0.5\textwidth]{LMUlogo.jpg}
	\hspace{2cm}
	\includegraphics[width=0.3\textwidth]{BMBFlogo.png}
} 

%\frame{\frametitle{Outline}\tableofcontents}

\section{Motivation}

\subsection{Micromegas Quadruplets for Track Reconstruction}

\frame{\frametitle{Micromegas Quadruplets for Track Reconstruction}
	\textbf{MICROMEGAS - MICROMEesh GAseous Structure}
	\vspace{2mm}
	
	\begin{columns}
		\column{.5\textwidth}
			\centering
			\includegraphics[width=1.1\textwidth]{RSMM_principle7-crop.pdf}
		\column{.5\textwidth}
			\footnotesize
			\begin{itemize}
				\item
					drift region for electron and ion transport
				\item	
					high field in amplification region to create electron avalanches
				\item
					position reconstruction using charge weighted mean of hit strips
				\item
					reconstruction of incident angle using drift time measurements
			\end{itemize}
	\end{columns}
	\vspace{3mm}
	
	\begin{columns}
		\column{.5\textwidth}
			\textbf{Micromegas Quadruplet}
			
			\centering
			\includegraphics[width=\textwidth]{sandwichassembly.png}
		\column{.5\textwidth}
			\centering
			\includegraphics[width=\textwidth]{quadrupletStripOrientation2-crop.pdf}
	\end{columns}
}

\subsection{Determination of Construction Inaccuracies}

\frame{\frametitle{Design of \SI{2}{\square\m}-sized Anode Panels}
	\hspace{12mm}
	\includegraphics[width=\textwidth]{ROboardMechanicalAlignment-crop.pdf}
	
	\vspace{2mm}
%	\footnotesize
	segmented readout anode (3 PCBs for technical reasons)
	\begin{itemize}
		\item[$\Rightarrow$]
			sophisticated construction, requirements:
			\begin{itemize}
%				\footnotesize
				\item
					shifts $<$ \SI{50}{\micro\m}
				\item
					rotation $<$ $10^{-4}$ rad
			\end{itemize}
		\item[$\Rightarrow$]				
			analysis: reconstruction and calibration of inaccuracies
	\end{itemize}
}

\section{Investigation with Cosmic Muons}

\subsection{Cosmic Ray Facility in Garching}

\frame{\frametitle{Cosmic Ray Facility in Garching}
	\centering
	
	\includegraphics[width=0.82\textwidth]{CRF_small.JPG}
	
	\vspace{1mm}
	MDT chambers : \SI{2.2}{\m} $\times$ \SI{4}{\m} $\Rightarrow$ active area : \SI{8}{\m\squared}
	
	angular acceptance : $\pm30^{\circ}$
}

\frame{\frametitle{Cosmic Ray Facility}
	\centering
	%scheme of test facility in Garching near Munich
	%\vspace{1mm}
	
	\includegraphics[width=\textwidth]{CRF_sketch4-crop.pdf}
	
	\begin{itemize}
		\item
			precision prediction along y axis 
			
			by Monitored Drift Tube chambers (MDT)
			
			$\Rightarrow$ Micromegas aligned for position reconstruction
			
			\vspace{3mm}
		\item
			\SI{10}{\cm} wide scintillator hodoscope
			
			$\Rightarrow$ coarse segmentation along x axis 
	\end{itemize}
}

\frame{\frametitle{Detector Alignment}
	\centering \textbf{Idea:}
	\begin{columns}
		\column{.50\textwidth}
			\centering
			\includegraphics[width=0.90\textwidth]{positionshift2-crop.pdf}
		\column{.50\textwidth}
			\centering
			\includegraphics[width=0.90\textwidth]{verticalshift-crop.pdf}
	\end{columns}
	\begin{columns}
		\column{.55\textwidth}
			\centering
			
			%{\rotatebox{90}{\textbf{\tiny \hspace{3mm} residual = measured - reference position}}}
			\includegraphics[width=0.93\textwidth]{resVSslope_blue.pdf}
		\column{.45\textwidth}
			\small
			\textbf{Implementation:}
			\begin{itemize}
				\item
					%residual = measured - reference position
					\textbf{residual} = $\mathrm{pos}_{\mathrm{measured}} - \mathrm{pos}_{\mathrm{reference}}$
				\item
					residual vs. slope 
					
					 of reference track 
					 
					$\Rightarrow$ {\color{red}linear fit}
				\item	
					$\mathrm{shift}_{\mathrm{horizontal}} = \mathrm{intercept}_{\mathrm{fit}}$
				\item						
					$\mathrm{shift}_{\mathrm{vertical}} \;\;\; = \mathrm{slope}_{\mathrm{fit}}$
			\end{itemize}
	\end{columns}
}

\subsection{Reconstruction of Geometrical Properties}

\frame{\frametitle{Reconstruction of the Gravitational Sag}
	\small
	partitioning of the active area into smaller subdivisions 
	
	$\Rightarrow$ detailed reconstruction of local properties for each layer separate

	\vspace{1mm}
	\begin{columns}
		\column{.7\textwidth}
			\centering
			\textbf{eta in}
			
			\includegraphics[width=0.72\textwidth]{SM2-M0_CRFafterH8_eta-in_deltaZ.pdf}
		\column{.1\textwidth}
			\centering
			\textbf{stereo in}
			\vspace{10mm}
			
			\textbf{stereo out}
			\vspace{10mm}	
					
			\textbf{eta out}
		\column{.2\textwidth}
			\centering
			
			\includegraphics[width=0.85\textwidth]{SM2-M0_CRFafterH8_stereo-in_deltaZ.pdf}
			
			\includegraphics[width=0.85\textwidth]{SM2-M0_CRFafterH8_stereo-out_deltaZ.pdf}
			
			\includegraphics[width=0.85\textwidth]{SM2-M0_CRFafterH8_eta-out_deltaZ.pdf}
	\end{columns}

	\vspace{1mm}	
	$\Rightarrow$ all layer show gravitational sag due to insufficient support structure 
	
	actually detector should be used vertical
}

\frame{\frametitle{Reconstruction of Strip Alignment}
	\begin{columns}
		\column{.50\textwidth}
			\centering
			{\color{red}mismatch by one strip} 
			
			and {\color{blue}pitch error}
			
			\includegraphics[width=0.90\textwidth]{SM2-M0_CRFafterH8_eta-in_resMeanVSmdtY_pitchErrorNshift.pdf}
		\column{.50\textwidth}
			\centering
			corrected mismatch 
			
			and corrected pitch error
			
			\includegraphics[width=0.90\textwidth]{SM2-M0_CRFafterH8_eta-in_resMeanVSmdtY.pdf}
	\end{columns}
	
	mean of residual distribution as function of the position enables reconstruction of 
	\begin{itemize}
		\item
			pitch errors: feasible up to an accuracy of $10^{-4}$ 
			
			pitch = \SI{0.425}{\micro\m} $\Rightarrow$ resolution = \SI{40}{n\m}
		\item
			board alignment on \SI{}{\micro\m} level
	\end{itemize}
}

\subsection{Variation of the Amplification Voltage}

\subsection{Homogeneity of Pulse Hight and Efficiency}

\frame{\frametitle{Investigation of the Pulse Height}
	\begin{columns}
		\column{.50\textwidth}
			\centering
			\textbf{eta in MPV cluster charge}
			
			\includegraphics[width=0.8\textwidth]{SM2-M0_CRFafterH8_eta-in_MPVclusterQ.pdf}
			
			\includegraphics[width=0.8\textwidth]{SM2-M0_CRF_MPVclusterQvsAmplificationVoltage_errorWidth_halfLines.pdf}
		\column{.50\textwidth}
			\footnotesize
			\begin{itemize}
				\item
					for each partition:
					
					landau fit of cluster charge distribution
				\item
					Most Probable Values (MPV) of landau fit shows differences between readout boards
				\item
					high amplification in whole detector
				\item
					inhomogeneities due to master-slave readout (horizontal slices)
					
					\vspace{5mm}
				\item
					variation of amplification voltage shows exponential rise of cluster charge as expected
				\item	
					differences between layer have to be investigated further
			\end{itemize}
	\end{columns}
}

\frame{\frametitle{Investigation of the Efficiency}
	\begin{columns}
		\column{.50\textwidth}
			\centering
			\textbf{eta in hit efficiency}
			\includegraphics[width=0.8\textwidth]{SM2-M0_CRFafterH8_eta-in_hitEfficiency_1123.pdf}
			
			\vspace{4mm}
			\includegraphics[width=0.8\textwidth]{SM2-M0_CRFafterH8_5mmEfficiency_allCluster_halfLines.pdf}
		\column{.50\textwidth}
			\centering
			\textbf{ \footnotesize hit within \SI{5}{mm} to reference track}
			
			\vspace{1mm}
			\includegraphics[width=0.8\textwidth]{SM2-M0_CRFafterH8_5mmEfficiency_allCluster_1123.pdf}
			
			\vspace{2mm}
			\footnotesize
			\begin{itemize}
				\item
					large areas with homogeneous efficiency
				\item
					94\% \SI{5}{mm}-efficiency 
					
					for all layers at \SI{600}{V}
				\item
					minor differences between layer
			\end{itemize}
			\vspace{6mm}
	\end{columns}
}

\subsection{Position Resolution and Track Reconstruction}

\frame{\frametitle{Raw Position Reconstruction}
	\begin{columns}
		\column{.50\textwidth}
			\centering
			\footnotesize
			\textbf{centroid residual width}
			\includegraphics[width=0.85\textwidth]{SM2-M0_CRFafterH8_eta-out_residualVSangle_woStereoX.pdf}
			
			\textbf{X position by scintillators}
			\includegraphics[width=0.85\textwidth]{SM2-M0_CRFafterH8_eta-out_resMeanVSstereoX_best.pdf}
		\column{.50\textwidth}
			\footnotesize
			\begin{itemize}
				\item
					for each angle separate residual distribution
					
					$\Rightarrow$ fit with double Gaussian
					
					$\Rightarrow$ sigma of narrow Gaussian 
				\item
					bad mechanical alignment of the module before measurement
					
					(angle between micromegas strips and MDT wires 0.0126 rad)
					
					$\Rightarrow$ spatial resolution limited by coarse segmentation in non-precision direction 
				\item
					stereo reconstruction enables a better resolution in non-precision direction
					
					$\Rightarrow$ diminish dependence of residual mean
			\end{itemize}
	\end{columns}
}

\frame{\frametitle{Spatial Resolution}
	\vspace{-1mm}
	
	\begin{columns}
		\column{.50\textwidth}
			\centering
			\footnotesize
			\textbf{reference track resolution}
			\includegraphics[width=0.9\textwidth]{MDTresidualVSangle.pdf}
			
			\vspace{-0.5mm}
			\textbf{centroid resolution eta out}
			\includegraphics[width=0.9\textwidth]{SM2-M0_CRFafterH8_eta-out_resolutionVSangle.pdf}
		\column{.50\textwidth}
			\footnotesize
			\begin{itemize}
				\item
					$\sigma_{\mathrm{micromegas}} = \sqrt{\sigma_{\mathrm{res}}^2 - \sigma_{\mathrm{track}}^2}$
				\item
					consider spatial resolution of MDTs
					
					$\Rightarrow$ subtract width of residual 
					
					\hspace{3.5mm} distribution quadratically
				\item
					resolution reaches for perpendicular incident about \SI{110}{\micro\m} (eta in),
					
					close to expectation
			\end{itemize}
			
			\vspace{6mm}
			\centering
			\textbf{centroid resolution eta in}
			\includegraphics[width=0.9\textwidth]{SM2-M0_CRFafterH8_eta-in_resolutionVSangle.pdf}
	\end{columns}
}

\section{Summary}

\frame{\frametitle{Summary}
	\footnotesize
	\begin{itemize}
		\item
			\SI{2}{\square\m}-sized micromegas quadruplets for tracking of minimal ionizing particles ($\mu$)
		\item
			segmented readout structure due to technical limitations
			
			$\Rightarrow$ alignment and calibration is needed after construction
		\item
			Cosmic Ray Facility in Garching
			
			$\Rightarrow$ investigation of a \SI{2}{\square\m}-sized micromegas quadruplet prototype
			\begin{itemize}
				\footnotesize
				\item
					reconstruction of a gravitational sag of about \SI{1.5}{mm}
				\item
					rotations between readout boards better than 0.0001 rad,
					
					shifts between readout boards under investigation
				\item
					deviation to nominal pitch on a sub permille level
				\item
					large area homogeneous pulse hight and efficiency, 
					
					despite differences between readout boards
				\item
					\SI{5}{mm}-efficiency of 94 \% at \SI{600}{V}
					
					$\Rightarrow$ working point for prototype module
				\item
					\SI{110}{\micro\m} for perpendicular incident
					
					close to expected value
			\end{itemize}
		\item
			TODO: timing analysis for inclined incident
	\end{itemize}
}

\appendix

\frame[noframenumbering]{
	\Huge
	\centering
	Backup
}

\frame[noframenumbering]{\frametitle{\large LHC upgrade and Efficiency for the ATLAS Muon Spectrometer}
	\centering
	\includegraphics[width=0.8\textwidth]{HL_LHC_PlanUpdateJuly2015}
	\begin{columns}
		\column{.55\textwidth}
			\centering
			\includegraphics[width=0.95\textwidth]{ATLASnewer.jpg}
		\column{.45\textwidth}
			\centering
			\includegraphics[width=0.9\textwidth]{MDTefficiencyPerHitRate_NSW-TDR_edited.png}
	\end{columns}
}

\frame[noframenumbering]{\frametitle{Upgrade of the Inner Barrel End Caps}

	replacement of the current detector systems by small-strip Thin Gap Chamber (sTGC) and Micromegas quadruplets
	
	\vspace{3mm}
	\begin{columns}
		\column{.55\textwidth}
			\centering
			\textbf{New Small Wheel Sectors}
			\vspace{7mm}
			
			\includegraphics[width=1\textwidth]{sectorsDimensions.png}
		\column{.45\textwidth}
			\centering
			\textbf{Micromegas quadruplets}
			
			\includegraphics[width=1\textwidth]{sectorsNsides.png}
	\end{columns}
}

\frame[noframenumbering]{\frametitle{Hit Distribution}
	
	\begin{columns}
		\column{.5\textwidth}
			\centering
			\textbf{eta out}
			
			\includegraphics[width=0.65\textwidth]{SM2-M0_CRFafterH8_eta-out_hits.pdf}
			
			\vspace{-2mm}
			\textbf{eta in}
			
			\includegraphics[width=0.65\textwidth]{SM2-M0_CRFafterH8_eta-in_hits.pdf}
		\column{.5\textwidth}
			\centering
			\textbf{stereo out}
			
			\includegraphics[width=0.65\textwidth]{SM2-M0_CRFafterH8_stereo-out_hits.pdf}
			
			\vspace{-2mm}
			\textbf{stereo in}
						
			\includegraphics[width=0.65\textwidth]{SM2-M0_CRFafterH8_stereo-in_hits.pdf}
	\end{columns}
}

\frame[noframenumbering]{\frametitle{Pulse Hight}
	
	\begin{columns}
		\column{.5\textwidth}
			\centering
			\textbf{eta out}
			
			\includegraphics[width=0.65\textwidth]{SM2-M0_CRFafterH8_eta-out_meanCluQ.pdf}
			
			\vspace{-2mm}
			\textbf{eta in}
			
			\includegraphics[width=0.65\textwidth]{SM2-M0_CRFafterH8_eta-in_meanCluQ.pdf}
		\column{.5\textwidth}
			\centering
			\textbf{stereo out}
			
			\includegraphics[width=0.65\textwidth]{SM2-M0_CRFafterH8_stereo-out_meanCluQ.pdf}
			
			\vspace{-2mm}
			\textbf{stereo in}
						
			\includegraphics[width=0.65\textwidth]{SM2-M0_CRFafterH8_stereo-in_meanCluQ.pdf}
	\end{columns}
}

\frame[noframenumbering]{\frametitle{Noise and Pulse Hight}
	
	\begin{columns}
		\column{.5\textwidth}
			\centering
			\textbf{2 strips required}
			
			\includegraphics[width=0.85\textwidth]{SM2-M0_CRFafterH8_eta-out_board7_clusterQ_min2strips.pdf}
			
			\textbf{nearest cluster is leading}
			
			\includegraphics[width=0.6\textwidth]{SM2-M0_CRFafterH8_eta-out_ratioNearLeading.pdf}
		\column{.5\textwidth}
			\centering
			\textbf{3 strips required}
			
			\includegraphics[width=0.85\textwidth]{SM2-M0_CRFafterH8_eta-out_board7_clusterQ_min3strips.pdf}
			
			\textbf{nearest cluster has one strip}
						
			\includegraphics[width=0.6\textwidth]{SM2-M0_CRFafterH8_eta-out_oneStripCluster.pdf}
	\end{columns}
}

\frame[noframenumbering]{\frametitle{Multiple Scattering Rejection}
	
	\begin{columns}
		\column{.5\textwidth}
			\centering
			\textbf{MDTs}
			
			\includegraphics[width=0.9\textwidth]{MDTresidualVSangle_slopeDifCuts.pdf}
			
			\includegraphics[width=0.9\textwidth]{MDTresidual_slopeDifCuts.pdf}
		\column{.5\textwidth}
			\centering
			\textbf{micromegas quadruplet (1 layer)}
			
			\includegraphics[width=0.9\textwidth]{SM2-M0_CRFafterH8_eta-out_residualVSangle_slopeDifCuts.pdf}
			
			\includegraphics[width=0.9\textwidth]{SM2-M0_CRFafterH8_eta-out_residual_cutsMDTslopeDif.pdf}
	\end{columns}
}

\frame[noframenumbering]{\frametitle{Stereo Reconstruction}
	\centering
	
	\includegraphics[width=0.7\textwidth]{SM2-M0_CRFafterH8_eta-out_resVSstereoX.pdf}
}

\end{document}
