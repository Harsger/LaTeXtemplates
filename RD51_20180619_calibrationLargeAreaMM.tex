\documentclass{beamer}
\usetheme{Madrid}
\usecolortheme{spruce}
\definecolor{green(pigment)}{rgb}{0.0, 0.65, 0.31}
\setbeamercolor*{item}{fg=green(pigment)}
\definecolor{Green}{rgb}{0.00, 1.00, 0.00}
\definecolor{Red}{rgb}{1.00, 0.00, 0.00}
\definecolor{Blue}{rgb}{0.00, 0.00, 1.00}
\usepackage[utf8]{inputenc}
\usepackage{amsmath}
\usepackage{amsfonts}
\usepackage{amssymb}
\usepackage[german]{babel}
\usepackage{graphicx}
\usepackage{rotating}
\usepackage{textcomp}
\usepackage{multirow,bigdelim,dcolumn,booktabs}
%\usepackage{beamerthemeshadow}
\usepackage{subfigure} 
\usepackage{siunitx}
\usepackage{appendixnumberbeamer}
\usepackage{hyperref}
%\beamersetuncovermixins{\opaqueness<1>{25}}{\opaqueness<2->{15}}
\beamertemplatenavigationsymbolsempty

\usepackage{tikz}
\usetikzlibrary{decorations.text}
\usetikzlibrary{trees}
\usetikzlibrary{decorations.pathmorphing}
\usetikzlibrary{decorations.markings}
\usetikzlibrary{patterns}

\graphicspath{
	{pictures/}
%	{/home/m/Maximilian.Herrmann/Bilder/forLatex/plots/}
%	{/home/m/Maximilian.Herrmann/Bilder/forLatex/sketches/}
%	{/home/m/Maximilian.Herrmann/Bilder/forLatex/pictures/}
}

\begin{document}
\title[Calibration of 2 m$^2$ 4-layered Micromegas]{Calibration of Detector Parameters and Detector Geometry of Large Area Micromegas using Cosmic Rays}
\author[M. Herrmann]{speaker : Maximilian Herrmann}
\institute[LMU Munich]{Ludwig-Maximilians-Universit\"at M\"unchen - Lehrstuhl Schaile}
\date[19.06.2018, RD51]{19.06.2018, RD51 Collaboration Meeting} 

\frame{
	\centering
	\vspace{2mm}

	\titlepage
	\vspace{-6mm}
	
	\includegraphics[width=0.35\textwidth]{LMUlogo.jpg}
	\hspace{4cm}
	\includegraphics[width=0.3\textwidth]{BMBFlogo.png}
} 

%\frame{\frametitle{Outline}\tableofcontents} 

\section{\SI{2}{\square\m}-sized 4-layered Micromegas \newline for the NSW upgrade of the ATLAS Experiment}

\frame{\frametitle{LHC Upgrade Status}
	\centering
	\includegraphics[width=0.8\textwidth]{HL_LHC_PlanUpdateJuly2015_wLumi.pdf}
	\begin{columns}
		\column{.55\textwidth}
			\centering
			\includegraphics[width=0.95\textwidth]{ATLASnewer.jpg}
		\column{.45\textwidth}
			\centering
			\includegraphics[width=0.9\textwidth]{MDTefficiencyPerHitRate_NSW-TDR_edited.png}
	\end{columns}
}

\frame{\frametitle{Upgrade of the Muon Small Wheels}

	replacement of the current end caps of the muon spectrometer by small-strip Thin Gap Chamber (sTGC) and Micromegas quadruplets
	
	\vspace{3mm}
	\begin{columns}
		\column{.55\textwidth}
			\centering
			\textbf{New Small Wheel Sectors}
			\vspace{7mm}
			
			\includegraphics[width=1\textwidth]{sectorsDimensions.png}
		\column{.45\textwidth}
			\centering
			\textbf{Micromegas quadruplets}
			
			\includegraphics[width=1\textwidth]{sectorsNsides.png}
	\end{columns}
}

\frame{\frametitle{\normalsize Micromegas Quadruplets for track reconstruction in the NSW}
	\textbf{MICROMEGAS - MICROMEesh GAseous Structure}
	\vspace{2mm}
	
	\begin{columns}
		\column{.5\textwidth}
			\centering
			\includegraphics[width=1.1\textwidth]{RSMM_principle7-crop.pdf}
		\column{.5\textwidth}
			\footnotesize
			\begin{itemize}
				\item
					drift region for electron and ion transport
				\item	
					high field in amplification region to create electron avalanches
				\item
					position reconstruction using charge weighted mean of hit strips
				\item
					reconstruction of incident angle using drift time measurements
			\end{itemize}
	\end{columns}
	\vspace{3mm}
	
	\textbf{Micromegas Quadruplet}
	\begin{columns}
		\column{.55\textwidth}
			\centering
			\includegraphics[width=1\textwidth]{sandwichassembly.png}
		\column{.45\textwidth}
			\footnotesize
			4 active layers of Micromegas
			
			\vspace{5mm}
			
			$\Rightarrow$ 2 $\times$ back-to-back
	\end{columns}
}

\frame{\frametitle{Design of Readout Anodes}
	\begin{columns}
		\column{.5\textwidth}
			\centering
			\includegraphics[width=\textwidth]{quadrupletStripOrientation3-crop.pdf}
		\column{.5\textwidth}
			\footnotesize
			\begin{itemize}
				\item
					{\color{red}eta planes} for precision reconstruction in pseudorapidity direction perpendicular to anode strips
				\item	
					{\color{blue}stereo planes} for additional coarse position information along the anode strips
			\end{itemize}
	\end{columns}
	\vspace{5mm}
	
	\begin{columns}
		\column{.5\textwidth}
			\centering
			\includegraphics[width=\textwidth]{ROboardMechanicalAlignment2-crop.pdf}
		\column{.5\textwidth}
			\footnotesize
			limitations by industry
			
			micropattern readout anode: width $\le$ \SI{50}{cm}
			\begin{itemize}
				\item[$\Rightarrow$] 
					
					3 printed circuit boards (PCB) 
					
					per active layer
				\item[$\Rightarrow$] 	
					reconstruction and calibration of alignment errors (during production) required
			\end{itemize}
	\end{columns}
}

\frame{\frametitle{Precision Reconstruction of Geometrical Properties}
	Boards of the Readout Anode can have:

	\hspace{20mm} $\Rightarrow$ {\color{red}rotations} and {\color{orange}shifts} w.r.t. each other
	
	\hspace{20mm} $\Rightarrow$ {\color{blue}non-straight strip shape} and {\color{green}pitch deviation}
	
	\vspace{3mm}
	
	\begin{columns}
		\column{.50\textwidth}
			{\centering
			\textbf{residual mean VS \\ non-precision direction}
			
			\includegraphics[width=0.80\textwidth]{stripShapeNrotation-crop.pdf}
			}
			
			\scriptsize
			$\Rightarrow$ slope indicates rotations
			
			$\Rightarrow$ strip shape is given by 
			
			\hspace{3mm} deviation from straight line
		\column{.50\textwidth}
			{\centering
			\textbf{residual mean VS \\ precision direction}
			
			\vspace{7mm}
			
			\includegraphics[width=0.9\textwidth]{boardAlignmentNpitchError-crop.pdf}
			}
			
			\scriptsize
			$\Rightarrow$ shift between boards is given by 
			
			\hspace{3mm} difference of the centers
			
			$\Rightarrow$ slope indicates deviation to nominal pitch 
	\end{columns}
}

\section{LMU Cosmic Ray Facility}

\frame{\frametitle{LMU Cosmic Ray Facility in Garching}
		
	\begin{columns}
		\column{.50\textwidth}
			\centering
			\includegraphics[width=0.9\textwidth]{CRFprinciple3-crop.pdf}
		\column{.50\textwidth}
			\centering
			\includegraphics[width=0.9\textwidth]{CRF_small.JPG}
	\end{columns}
	
	\footnotesize
			
	\begin{itemize}
		\item
			2D track reconstruction with two Monitored Drift Tube (MDT) chambers
		\item
			trigger via scintillator hodoscope with $\approx$ \SI{10}{cm} resolution 
			
			in direction along the wires
		\item
			MDT chambers : \SI{2.2}{\m} $\times$ \SI{4}{\m} 
			
			$\Rightarrow$ active area : \SI{8}{\m\squared}, angular acceptance : $\pm30^{\circ}$
		\item
			readout of the full module (12288 channels) 
			
			with 96 APVs connected to 6 FECs @ full \SI{130}{Hz} \textmu -rate
			
			(tested up to \SI{500}{Hz} with random trigger)
	\end{itemize}
}

\frame{\frametitle{Alignment using Reference Tracks}
	\centering 
	\textbf{Idea:}
	\begin{columns}
		\column{.50\textwidth}
			\centering
			\includegraphics[width=0.90\textwidth]{positionshift2-crop.pdf}
		\column{.50\textwidth}
			\centering
			\includegraphics[width=0.90\textwidth]{verticalshift-crop.pdf}
	\end{columns}
	\vspace{2mm}
	\textbf{Implementation:}
	\begin{columns}
		\column{.55\textwidth}
			\centering
			
			\includegraphics[width=0.8\textwidth]{resVSslope_blue.pdf}
		\column{.45\textwidth}
			\small
			\textbf{residual} = pos\textsubscript{measured} - pos\textsubscript{reference}
%					\textbf{residual} = $\mathrm{pos}_{\mathrm{measured}} - \mathrm{pos}_{\mathrm{reference}}$

			\vspace{3mm}
			$\Rightarrow$ residual vs. slope {\scriptsize(reference track)}
	 
	 		\vspace{3mm}
			$\Rightarrow$ {\color{red}linear fit}
			
	 		\vspace{3mm}
			shift\textsubscript{horizontal} = intercept\textsubscript{fit}
%					$\mathrm{shift}_{\mathrm{horizontal}} = \mathrm{intercept}_{\mathrm{fit}}$

			\vspace{3mm}
			shift\textsubscript{vertical} = slope\textsubscript{fit}	
%					$\mathrm{shift}_{\mathrm{vertical}} \;\;\; = \mathrm{slope}_{\mathrm{fit}}$
	\end{columns}
}

\frame{\frametitle{Measurement Overview}
	\begin{itemize}
		\item
			\SI{1}{\square\m}-sized prototype detector called L1
			
			$\Rightarrow$ proof of principles
			
			\vspace{5mm}
		\item
			\SI{2}{\square\m}-sized 4-layer prototype chamber called Module 0
			
			$\Rightarrow$ first working quadruplet
			
			\vspace{5mm}
		\item
			\SI{2}{\square\m}-sized 4-layer series chamber for NSW called Module 1
			
			$\Rightarrow$ first series module
	\end{itemize}
}

\section{Calibration Using Reference Tracks over Whole Active Area}

\subsection{\SI{1}{\square\m}-sized prototype detector - L1 \newline $\Rightarrow$ Proof of Principles}

\frame{
	\LARGE
	\centering
	
	L1
	\vspace{10mm}
	
	\SI{1}{\square\m}-sized Micromegas
}

\frame{\frametitle{Deformation of the Drift Region due to Overpressure (L1)}
	\begin{columns}
		\column{.50\textwidth}
			\centering
			\includegraphics[width=0.9\textwidth]{L1_blowUp_reconstructed.png}
			
			\includegraphics[width=0.7\textwidth]{L1_simDif-crop.pdf}
		\column{.50\textwidth}
			\centering
			\includegraphics[width=\textwidth]{L1blowUpSketch.pdf}
			
			\scriptsize
			\begin{itemize}
				\item
					drift gap deformation due to 
					
					small overpressure (no interconnections)
				\item
					maximum deviation of \SI{0.8}{\mm}
					from central plane
					
					$\Rightarrow$ \SI{1.6}{\mm} at cathode 
					
					\hspace{3mm} (stiff base plate support)
				\item
					deformation in agreement with 
					
					finite element simulation (ANSYS)
				\item
					for NSW Micromegas interconnections will reduce deformation to about \SI{50}{\micro\m}
			\end{itemize}
			
	\end{columns}
}

\frame{\frametitle{Rotations and Shifts between Anode Boards (L1)}
	\begin{columns}
		\column{.50\textwidth}
			\centering
			\textbf{residual mean VS \\ non-precision direction}
			
			\scriptsize
			board 1
			
			\includegraphics[width=0.8\textwidth]{L1_board1_resMeanVSscinX.pdf}
			
			\vspace{-2mm}
			board 2 (\textbf{not aligned during assembly})
			
			\includegraphics[width=0.8\textwidth]{L1_board2_resMeanVSscinX_wFit.pdf}
		\column{.50\textwidth}
			\centering
			\textbf{residual mean VS \\ precision direction}
			
			\includegraphics[width=0.8\textwidth]{L1_resMeanVSmdtY_wFit.pdf}
			
			\scriptsize
			\begin{itemize}
				\item
					board 1 shows deformed strip shape of about \SI{50}{\micro\m}
				\item
					board 2 is rotated w.r.t. board 1 of about \SI{470}{\micro\m/\m} ($\mathrel{\hat=}$ \SI{0.027}{\degree})
				\item
					board 2 is shifted w.r.t. board 1 of about \SI{320}{\micro\m}
				\item
					board 2 has a pitch deviation 
					
					of about 0.2\textperthousand 
					
					($\mathrel{\hat=}$ \SI{80}{\nano\m} @ nominal pitch of \SI{450}{\micro\m})
			\end{itemize}
			
	\end{columns}
}

\subsection{\SI{2}{\square\m}-sized 4-layered prototype chamber - Module 0 \newline $\Rightarrow$ Pulse Height and Efficiency over whole Area}

\frame{
	\LARGE
	\centering
	
	Module 0 
	\vspace{10mm}
	
	\SI{2}{\square\m}-sized Micromegas Quadruplet Prototype
}

\frame{\frametitle{Reconstruction of the Gravitational Sag (Module 0)}

	\begin{columns}
		\column{.7\textwidth}
			\centering
			\textbf{eta in}
			
			\includegraphics[width=0.72\textwidth]{SM2-M0_CRFafterH8_eta-in_deltaZ.pdf}
		\column{.1\textwidth}
			\centering
			\textbf{stereo in}
			\vspace{10mm}
			
			\textbf{stereo out}
			\vspace{10mm}	
					
			\textbf{eta out}
		\column{.2\textwidth}
			\centering
			
			\includegraphics[width=0.85\textwidth]{SM2-M0_CRFafterH8_stereo-in_deltaZ.pdf}
			
			\includegraphics[width=0.85\textwidth]{SM2-M0_CRFafterH8_stereo-out_deltaZ.pdf}
			
			\includegraphics[width=0.85\textwidth]{SM2-M0_CRFafterH8_eta-out_deltaZ.pdf}
	\end{columns}

	\vspace{1mm}	
	$\Rightarrow$ all layers show gravitational sag due to insufficient support structure, 
	
	\hspace{3.5mm} detectors at the experiment will be used vertical
}

\frame{\frametitle{Full Area Pulse Height and Efficiency (Module 0)}
	\begin{columns}
		\column{.50\textwidth}
			\centering
			eta in cluster charge
			
			\includegraphics[width=0.7\textwidth]{SM2-M0_CRFafterH8_eta-in_MPVclusterQ.pdf}
		\column{.50\textwidth}
			\centering
			eta in \SI{5}{mm} efficiency
			
			\includegraphics[width=0.7\textwidth]{SM2-M0_CRFafterH8_eta-in_5mmEfficiency600V.pdf}
	\end{columns}
	\vspace{2.3mm}
	\centering
	amplification scan for the central area of all layers
	\begin{columns}
		\column{.50\textwidth}
			\centering
			\includegraphics[width=0.8\textwidth]{SM2-M0_CRF_MPVclusterQvsAmplificationVoltage_errorWidth_halfLines.pdf}
		\column{.50\textwidth}
			\centering
			\includegraphics[width=0.8\textwidth]{SM2-M0_CRFafterH8_5mmEfficiency_allCluster_halfLines.pdf}
	\end{columns}
}

\frame{\frametitle{\normalsize Reconstructed Pitch Deviation and Board Alignment per Plane (Module 0)}
	\begin{columns}
		\column{.50\textwidth}
			\small
			\centering
			{\color{green}pitch deviation}
			
			\includegraphics[width=0.90\textwidth]{eta_in_resMeanVSmdtY_woAdapterBoard_wPitchLine.pdf}
			
			\vspace{1.5mm}
			
			pitch deviation [10$^{-4}$]
			\scriptsize
			\begin{tabular}{cc|ccc}
			panel & side & small & middle & large
			\\
			\hline
			stereo & out &  & 1.4 & 0.7
			\\
			stereo & in & 5.2 & 4.7 & 5.4
			\\
			\hline
			eta & in & 0.7 & 2.8 & 3.3
			\\
			eta & out & 1.4 & 3.8 & 4.9
			\end{tabular}
		\column{.50\textwidth}
			\small
			\centering
			corrected pitch
			
			\includegraphics[width=0.90\textwidth]{eta_in_resMeanVSmdtY_wNewPitchCor_boards.pdf}
			
			\vspace{1.5mm}
						
			shift [mm]
			
			\scriptsize
			\begin{tabular}{cc|cc}
			panel & side & small & large
			\\
			\hline
			stereo & out & 0.10 & -0.06
			\\
			stereo & in & -0.16 & 0.22
			\\
			\hline
			eta & in & -0.08 & 0.02
			\\
			eta & out & -0.17 & 0.11
			\end{tabular}
	\end{columns}
	\vspace{1mm}
	\small
	
	$\Rightarrow$ better construction equipment will be used for series detectors to avoid shifts
	
	$\Rightarrow$ humidity control of readout boards to avoid pitch deviation
}

\frame{
	\LARGE
	\centering
	
	Module 1 
	\vspace{10mm}
	
	first Series \SI{2}{\square\m}-sized Micromegas Quadruplet
}

\frame{\frametitle{\normalsize Reconstructed Pitch Deviation and Board Alignment per Plane (Module 1)}
	\begin{columns}
		\column{.50\textwidth}
			\centering
			eta out
			
			\includegraphics[width=0.8\textwidth]{m1_eo_resVSstrip.pdf}
		\column{.50\textwidth}
			\centering
			eta in
			
			\includegraphics[width=0.8\textwidth]{m1_ei_resVSstrip.pdf}
	\end{columns}
	\begin{columns}
		\column{.50\textwidth}
			\centering
			stereo in
			
			\includegraphics[width=0.8\textwidth]{m1_si_resVSstrip.pdf}
			
			\vspace{5mm}
		\column{.50\textwidth}
			\centering
			stereo out
			
			\includegraphics[width=0.8\textwidth]{m1_so_resVSstrip_pitchError.pdf}
			
			{\color{red} misaligned during gluing}
	\end{columns}
}

\subsection{\SI{2}{\square\m}-sized 4-layered series chamber - Module 1 \newline $\Rightarrow$ Position Reconstruction}

\frame{\frametitle{Strip Shape (Module 1)}
	\begin{columns}
		\column{.50\textwidth}
			\centering
			\includegraphics[width=0.8\textwidth]{m1_eo_board6_resVSscinX.pdf}
		\column{.50\textwidth}
			\centering
			\includegraphics[width=\textwidth]{m1_eo_board6_resMeanVSscinX.pdf}
	\end{columns}
	
	\begin{columns}
		\column{.50\textwidth}
			\footnotesize
			\begin{itemize}
				\item
					residual VS position along strips 
					
					(by scintillator hodoscopes)
				\item
					deformed strip shape 
					
					also measured by optical inspection (CMM Saclay)
				\item
					further investigation required
			\end{itemize}
		\column{.50\textwidth}
			\centering
			corresponds to
			
			\includegraphics[width=\textwidth]{bananaStripSketch-crop.pdf}
	\end{columns}
}

\frame{\frametitle{\normalsize Most Probable Value of Cluster Charge by Landau Fit (Module 1)}
	\begin{columns}
		\column{.50\textwidth}
			\centering
			eta out
			
			\includegraphics[width=0.7\textwidth]{m1_etaout_A570V_C300V_clusterQmpv_scaled.png}
		\column{.50\textwidth}
			\centering
			eta in
			
			\includegraphics[width=0.7\textwidth]{m1_etain_A570V_C300V_clusterQmpv_scaled.png}
	\end{columns}
	\vspace{2.3mm}
	\begin{columns}
		\column{.50\textwidth}
			\centering
			stereo in
			
			\includegraphics[width=0.7\textwidth]{m1_stereoin_A570V_C300V_clusterQmpv_scaled.png}
		\column{.50\textwidth}
			\centering
			stereo out
			
			\includegraphics[width=0.7\textwidth]{m1_stereoout_A570V_C300V_clusterQmpv_scaled.png}
	\end{columns}
}

\frame{\frametitle{\normalsize Position Resolution as Function of the Incident Angle (Module 1)}
	\vspace{-1mm}
	
	\begin{columns}
		\column{.50\textwidth}
			\centering
			\footnotesize
			\textbf{eta out}
			\includegraphics[width=0.9\textwidth]{m1_eo_resolutionVSangle_wCluTimeCor.pdf}
			
			\vspace{-0.5mm}
			\textbf{eta in}
			\includegraphics[width=0.9\textwidth]{m1_ei_resolutionVSangle_wCluTimeCor.pdf}
		\column{.50\textwidth}
			\footnotesize
			\begin{itemize}
				\item
					residual distribution for each angle separately
				\item
					fit with double Gaussian
					
					analysis sigma narrow Gaussian only
					
					(reject multiple scattering)
				\item
					consider track uncertainty of reference chambers: 
				
					$\sigma_{\mathrm{micromegas}} = \sqrt{\sigma_{\mathrm{res}}^2 - \sigma_{\mathrm{track}}^2}$
				\item
					resolution is for perpendicular incident close to expectation
				\item
					charge weighted clustertime correction improves residual distribution considerably
			\end{itemize}
	\end{columns}
}

\frame{\frametitle{Summary}
	\scriptsize
	\begin{itemize}
		\scriptsize
		\item
			replacement of the ATLAS muon spectrometer inner end cap
			\begin{itemize}
				\scriptsize
				\item
					sTGC and Micromegas quadruplets
				\item
					Micromegas SM2 Modules will be built by a German collaboration
				\item
					segmented readout structure due to limitations by industry
					
					$\Rightarrow$ reconstruction and calibration is required after construction
			\end{itemize}
		\item
			Investigation at the Cosmic Ray Facility in Garching
			\begin{itemize}
				\scriptsize
				\item
					\SI{1}{\square\m}-size Micromegas (L1) for proof of principles
					\begin{itemize}
						\scriptsize
						\item
							reconstruction of deformation due to overpressure
						\item
							rotation of about \SI{0.027}{\degree} and shift of about \SI{320}{\micro\m} reconstructed
						\item
							deviation to nominal pitch of about 0.2\textperthousand
					\end{itemize}
				\item
					\SI{2}{\square\m}-size 4-layered Micromegas Prototype (Module 0)
					\begin{itemize}
						\scriptsize
						\item
							gravitational sag due to insufficient support
							
							$\Rightarrow$ modules used vertical  in ATLAS
						\item
							pulse height and efficiency behave as expected
							
							$\Rightarrow$ \SI{5}{mm} efficiency reaches more than 90\% for all layers 
					\end{itemize}
				\item
					first \SI{2}{\square\m}-size 4-layered Micromegas series module (Module 1)
					\begin{itemize}
						\scriptsize
						\item
							alignment better than for prototype module
						\item
							measured strip shape in agreement with optical inspection
							
							$\Rightarrow$ further investigation required
						\item
							homogeneous pulse height over large area
						\item
							position reconstruction for perpendicular incident below \SI{100}{\micro\m}
							
							$\Rightarrow$ resolution for inclined incident behave as expected, 
							
							\hspace{3mm} but can be improved further
					\end{itemize}
			\end{itemize}
	\end{itemize}
}

\appendix

\frame{\centering \Huge Backup}

\frame{\frametitle{Anode : \SI{570}{V}, Cathode : \SI{300}{V}, Ar:CO$_2$ 93:7 vol\% \\ Mean Cluster Charge}
	\begin{columns}
		\column{.50\textwidth}
			\centering
			eta out
			
			\includegraphics[width=0.7\textwidth]{m1_etaout_A570V_C300V_clusterQmean_scaled.png}
		\column{.50\textwidth}
			\centering
			eta in
			
			\includegraphics[width=0.7\textwidth]{m1_etain_A570V_C300V_clusterQmean_scaled_APVnotworking.pdf}
	\end{columns}
	\begin{columns}
		\column{.50\textwidth}
			\centering
			stereo in
			
			\includegraphics[width=0.7\textwidth]{m1_stereoin_A570V_C300V_clusterQmean_scaled.png}
		\column{.50\textwidth}
			\centering
			stereo out
			
			\includegraphics[width=0.7\textwidth]{m1_stereoout_A570V_C300V_clusterQmean_scaled_badHV.pdf}
	\end{columns}
}

\end{document}
