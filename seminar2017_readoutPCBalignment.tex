\documentclass{beamer}
\usetheme{Madrid}
\usecolortheme{spruce}
\definecolor{green(pigment)}{rgb}{0.0, 0.65, 0.31}
\setbeamercolor*{item}{fg=green(pigment)}
\definecolor{Green}{rgb}{0.00, 1.00, 0.00}
\definecolor{Red}{rgb}{1.00, 0.00, 0.00}
\definecolor{Blue}{rgb}{0.00, 0.00, 1.00}
\usepackage[utf8]{inputenc}
\usepackage{amsmath}
\usepackage{amsfonts}
\usepackage{amssymb}
\usepackage[german]{babel}
\usepackage{graphicx}
\usepackage{rotating}
\usepackage{textcomp}
\usepackage{multirow,bigdelim,dcolumn,booktabs}
%\usepackage{beamerthemeshadow}
\usepackage{subfigure} 
\usepackage{siunitx}
\usepackage{appendixnumberbeamer}
\usepackage{hyperref}
%\beamersetuncovermixins{\opaqueness<1>{25}}{\opaqueness<2->{15}}
\beamertemplatenavigationsymbolsempty

\usepackage{tikz}
\usetikzlibrary{decorations.text}
\usetikzlibrary{trees}
\usetikzlibrary{decorations.pathmorphing}
\usetikzlibrary{decorations.markings}
\usetikzlibrary{patterns}

\graphicspath{
	{pictures/}
%	{/home/m/Maximilian.Herrmann/Bilder/forLatex/plots/}
%	{/home/m/Maximilian.Herrmann/Bilder/forLatex/sketches/}
%	{/home/m/Maximilian.Herrmann/Bilder/forLatex/pictures/}
}

\begin{document}
\title[Readout PCB Alignment]{Investigation of Readout PCB Alignment \\ for SM2 Modules}  
\author[M. Herrmann]{ Maximilian Herrmann}
\institute[LMU Munich]{Ludwig-Maximilians-Universit\"at M\"unchen - Lehrstuhl Schaile}
\date[20.12.2017]{Abteilungsseminar 20.12.2017} 

\frame{\titlepage} 

\frame{\frametitle{Outline}
	\tableofcontents
} 

\section{Motivation}

\subsection{New Small Wheels for the ATLAS Experiment}

\frame{\frametitle{\large LHC upgrade and Efficiency for the ATLAS Muon Spectrometer}
	\centering
	\includegraphics[width=0.8\textwidth]{HL_LHC_PlanUpdateJuly2015}
	\begin{columns}
		\column{.55\textwidth}
			\centering
			\includegraphics[width=0.95\textwidth]{ATLASnewer.jpg}
		\column{.45\textwidth}
			\centering
			\includegraphics[width=0.9\textwidth]{MDTefficiencyPerHitRate_NSW-TDR_edited.png}
	\end{columns}
}

\frame{\frametitle{Upgrade of the Inner Barrel End Caps}

	replacement of the current detector systems by small-strip Thin Gap Chamber (sTGC) and Micromegas quadruplets
	
	\vspace{3mm}
	\begin{columns}
		\column{.55\textwidth}
			\centering
			\textbf{New Small Wheel Sectors}
			\vspace{7mm}
			
			\includegraphics[width=1\textwidth]{sectorsDimensions.png}
		\column{.45\textwidth}
			\centering
			\textbf{Micromegas quadruplets}
			
			\includegraphics[width=1\textwidth]{sectorsNsides.png}
	\end{columns}
}

\subsection{Micromegas Quadruplets}

\frame{\frametitle{\normalsize Micromegas Quadruplets for track reconstruction in the NSW}
	\textbf{MICROMEGAS - MICROMEesh GAseous Structure}
	\vspace{2mm}
	
	\begin{columns}
		\column{.5\textwidth}
			\centering
			\includegraphics[width=1.1\textwidth]{RSMM_principle7-crop.pdf}
		\column{.5\textwidth}
			\footnotesize
			\begin{itemize}
				\item
					drift region for electron and ion transport
				\item	
					high field in amplification region to create electron avalanches
				\item
					position reconstruction using charge weighted mean of hit strips
				\item
					reconstruction of incident angle using drift time measurements
			\end{itemize}
	\end{columns}
	\vspace{3mm}
	
	\textbf{Micromegas Quadruplet}
	\begin{columns}
		\column{.55\textwidth}
			\centering
			\includegraphics[width=1\textwidth]{sandwichassembly.png}
		\column{.45\textwidth}
			\footnotesize	
			alternating orientation of drift and amplification region
			
			$\Rightarrow$ construction of dedicated 
			
			readout and drift panels
	\end{columns}
}

\frame{\frametitle{Construction of Readout Panels}
	\begin{columns}
		\column{.5\textwidth}
			\centering
			\includegraphics[width=1\textwidth]{quadrupletStripOrientation-crop.pdf}
		\column{.5\textwidth}
			\footnotesize
			\begin{itemize}
				\item
					eta planes for precision reconstruction in transversal direction
				\item	
					stereo planes for additional coarse information about orthogonal direction
			\end{itemize}
	\end{columns}
	
	\begin{columns}
		\column{.5\textwidth}
			\centering
			\includegraphics[width=0.9\textwidth]{alignmentFrameOnBoards-crop.pdf}
		\column{.5\textwidth}
			\footnotesize
			\begin{itemize}
				\item
					two step gluing process for one panel
				\item	
					limitations by industry
				
					$\Rightarrow$ segmented readout anode: 
					
					3 Printed Circuit Boards (PCB) per plane
				\item
					alignment using a frame and cylinders
								
					attached via V-shaped and line spacers 
			\end{itemize}
	\end{columns}
}

\frame{\frametitle{PCB Alignment before Gluing}
	\centering
	\begin{tikzpicture}
		\node[anchor=south west,inner sep=0] (image) at (0,0) {\includegraphics[width=0.7\linewidth]{alignmentFrame.jpeg}};
		\node[right,red] at (3.4,2.8) {\large alignment};
		\node[right,red] at (3.7,2.4) {\large frame};
		\draw[arrows=->,red,ultra thick](3.4,2.6)--(2.9,2.3);
		\draw[arrows=->,red,ultra thick](3.4,2.8)--(2.8,3.3);
		\draw[arrows=->,red,ultra thick](5.0,3.0)--(5.7,3.1);
		\draw[arrows=->,red,ultra thick](5.0,2.6)--(5.5,2.2);
		\node[right,red] at (7.0,0.8) {\large pins in};
		\node[right,red] at (7.0,0.4) {\large holes};
		\draw[arrows=->,red,ultra thick](7.5,1.3)--(7.45,2.5);
		\draw[arrows=->,red,ultra thick](7.5,1.3)--(6.4,2.0);
		\draw[arrows=->,red,ultra thick](7.5,1.3)--(5.0,1.5);
		\node[right,red] at (0.0,1.0) {\large attached to};
		\node[right,red] at (0.0,0.6) {\large precision};
		\node[right,red] at (0.0,0.2) {\large cylinder};
		\draw[arrows=->,red,ultra thick](1.0,1.3)--(1.15,3.1);
		\draw[arrows=->,red,ultra thick](2.0,0.5)--(3.6,1.15);
	\end{tikzpicture}
	
	\begin{columns}
		\column{.50\textwidth}
			\centering
			washer with pin
			
			\includegraphics[width=0.7\textwidth]{washerWithPin.jpeg}
		\column{.50\textwidth}
			\centering
			V-shaped spacer at pin
			
			\includegraphics[width=0.7\textwidth]{VwasherAtCyl2.png}
	\end{columns}
}

\frame{\frametitle{Measurement Overview}
	naming convention
					
	\vspace{3mm}
	\centering
	\includegraphics[width=0.9\textwidth]{panelStripOrientation-crop.pdf}
	
	\vspace{3mm}
	\begin{itemize}
		\item
			Freiburg CMM: Module 0 panels
		\item	
			Munich CMM: calibration and first series panels
		\item
			Saclay CMM: first series panels
		\item
			H8 beamtime: Module 0 quadruplet
		\item
			CRF: Module 0 quadruplet
	\end{itemize}
}

\section{Optical Surveying of the Board Alignment}

\subsection{Investigation with Coordinate Measurement Machines}

\frame{\frametitle{Alignment Measurements \\ with a Camera attached to a CMM}
	\begin{columns}
		\column{.5\textwidth}
			\centering
			\begin{tikzpicture}
				\node[anchor=south west,inner sep=0] (image) at (0,0) {\includegraphics[width=0.9\textwidth]{granittableleftresized.JPG}};
				\begin{scope}[x={(image.south east)},y={(image.north west)}]
					\draw[arrows=->,red,ultra thick](0.4,0.16)--(0.8,0.408);
					\node[below,red] at (0.68,0.31) {X};
					\draw[arrows=->,red,ultra thick](0.35,0.12)--(0.12,0.5);
					\node[right,red] at (0.22,0.36) {Y};
					\draw[arrows=->,red,ultra thick](0.1,0.5)--(0.08,0.8);
					\node[right,red] at (0.1,0.65) {Z};
					\node[above,red] at (0.1,0.8) {center};
					\node[above,red] at (0.8,0.7) {camera};
				\end{scope}
			\end{tikzpicture}
		\column{.5\textwidth}
			\centering
			\includegraphics[width=0.9\textwidth]{microscope_screenshot_20160420_marker_board7_awayPC.jpeg}
	\end{columns}
	
	\begin{itemize}
		\item
			usage of 
			%Computerized Numerical Controlled 
			three axes machines 
			%(CNC) 
			as Coordinate Measurement Machine (CMM)
		\item
			axes for movement of measurement head
			
			$\Rightarrow$ calibration of movement inaccuracies
		\item	
			image evaluation for reconstruction of exact position
	\end{itemize}
}

\frame{\frametitle{Alignment Measurement of Module 0 in Freiburg}
	\begin{columns}
		\column{.5\textwidth}
			\centering
			deviations to nominal values [mm]
			
			\includegraphics[width=0.9\textwidth]{Module0_Ydeviation_FreiburgOptical.pdf}
		\column{.5\textwidth}
			\footnotesize
			\begin{itemize}
				\item
					goal: $\le$ \SI{0.030}{mm}
				\item
					large deviations to nominal values
					
					$\Rightarrow$ exchanged old alignment frame
				\item	
					shifts and rotations w.r.t. central board can be reconstructed
			\end{itemize}
	\end{columns}
	
	\begin{columns}
		\column{.70\textwidth}
			\tiny
			\begin{tabular}{ccc|ccc}
			panel & side & board & rotation & corner shift & center shift
			\\
			 &  &  & [\SI{10}{\micro\m/\m}] & [mm] & [mm]
			\\
			\hline
			stereo & 1 & small & 4 & 0.07 & 0.10
			\\
			 &  & large & 6 & -0.17 & -0.13
			\\
 			\hline
 			stereo & 2 & small & -16 & 0.10 & 0.20
 			\\
			 &  & large & 7 & -0.02 & 0.04
 			\\
 			\hline
 			eta & 1 & small & 2 & 0.07 & 0.08
 			\\
		 	 &  & large & 3 & -0.03 & -0.01
 			\\
 			\hline
 			eta & 2 & small & 2 & 0.05 & 0.07
 			\\
			 &  & large & 6 & -0.01 & 0.04
			\end{tabular}
		\column{.30\textwidth}
			\centering
			\includegraphics[width=1\textwidth]{shiftDefinitions-crop.pdf}
	\end{columns}
}

\frame{\frametitle{Alignment w.r.t. opposite side of the panel}
	\centering
	deviations to other side of the panel [mm]
	\vspace{3mm}
	
	\begin{columns}
		\column{.5\textwidth}
			\centering
			stereo
			\vspace{2mm}
			
			\includegraphics[width=0.7\textwidth]{Module0_deviationGluingSides_stereo.pdf}
		\column{.5\textwidth}
			\centering
			eta
			\vspace{2mm}
			
			\includegraphics[width=0.7\textwidth]{Module0_deviationGluingSides_eta.pdf}
	\end{columns}
	\vspace{3mm}
	
	\scriptsize
	\begin{tabular}{cc|cc}
		panel & board & rotation [\SI{10}{\micro\m/\m}] & center shift [mm]
		\\
		\hline
		stereo & small & -11 & -0.06
		\\
	 	 & middle & 1 & 0.04
		\\
	 	 & large & -12 & 0.21
		\\
		\hline
		eta & small & 11 & -0.04
		\\
		 & middle & 6 & -0.06
		\\
		 & large & -4 & -0.01
	\end{tabular}
			
}

\subsection{Calibration of a Coordinate Measurement Machine}

\frame{\frametitle{Mapping with a Laser-Interferometer}
	\begin{columns}
		\column{.5\textwidth}
			\centering
			\includegraphics[width=0.8\textwidth]{laserinterferometerAtCMM_withLegs-crop.pdf}
		\column{.5\textwidth}
			\footnotesize
			\begin{itemize}
				\item
					laserinterferometer specification: 
					
					error below \SI{1}{\micro\m/\m}
				\item
					\SI{20}{mm} periodicity along X and Y
					
					due to rotating spindle
				\item
					\SI{100}{mm} periodicity along X additionally,
					
					screwing of the rotating spindel
			\end{itemize}
	\end{columns}
	
	\begin{columns}
		\column{.5\textwidth}
			\centering
			\includegraphics[width=0.9\textwidth]{cmmlaser_Xdeviation_zoomed.pdf}
		\column{.5\textwidth}
			\centering
			Fourier Analysis
			
			\includegraphics[width=0.9\textwidth]{CMMfequencyAlongX.png}
	\end{columns}
}

\frame{\frametitle{CMM Calibration}
	\begin{columns}
		\column{.45\textwidth}
			\centering
			\includegraphics[width=0.9\textwidth]{measuredNreconstructedVdeviation.pdf}
		\column{.55\textwidth}
			\scriptsize
			\begin{itemize}
				\item	
					interpolation of leg movement 
					
					$\Rightarrow$ reconstruction of movement of the measurement arm along Y axis between the legs
				\item
					at small scales reconstruction looks correct
				\item
					reconstruction fails at large scale
					
					\begin{itemize}
						\scriptsize
						\item
							depending on X position
						\item
							can be explained by bending of the measurement arm
					\end{itemize}
					
					$\Rightarrow$ more measurements required
			\end{itemize}
	\end{columns}
	
	\begin{columns}
		\column{.33\textwidth}
			\centering
			\includegraphics[width=\textwidth]{CMMdiffRecoMeas_severalX.png}
		\column{.33\textwidth}
			\centering
			\includegraphics[width=\textwidth]{CMMdiffRecoMeas_hightVSx.png}
		\column{.33\textwidth}
			\centering
			\includegraphics[width=\textwidth]{diffRecoMeasNangle.pdf}
	\end{columns}
}

\frame{\frametitle{Workaround instead of a full Calibration}
	\begin{columns}
		\column{.5\textwidth}
			\centering
			\includegraphics[width=\textwidth]{CMMworkaroundTry1-crop.pdf}
		\column{.5\textwidth}
			\footnotesize
			\begin{itemize}
				\item
					once:
				
					measure distances of specific locations to pins on granite table with the laserinterferometer
				\item	
					for each alignment measurement:
					\begin{itemize}
						\footnotesize
						\item
							take a picture of the marker at the locations
						\item
							evaluate pictures to reconstruct center of the marker
						\item
							add evaluated position to distance of the laserinterferometer at that location
					\end{itemize}
			\end{itemize}
	\end{columns}
}

\frame{\frametitle{Comparison of Measurements of the new Alignment}
	\begin{columns}
		\column{.5\textwidth}
			\centering
			Freiburg
			
			\includegraphics[width=0.7\textwidth]{newAlignment_Freiburg-crop.pdf}
		\column{.5\textwidth}
			\centering
			Munich
			
			\includegraphics[width=0.7\textwidth]{newAlignment_Munich-crop.pdf}
	\end{columns}
	\vspace{3mm}
	
	measurement from Freiburg shows more reasonable results
	
	$\Rightarrow$ method do not work as expected
	
	error sources:
	\begin{itemize}
		\item
			parallax error due to bending of measurement arm not considered
		\item
			combination of CMM and picture coordinates
		\item
			X and Y axis are \textbf{not} exactly perpendicular 
	\end{itemize}
}

\frame{\frametitle{A more sophisticated Approach}
	\begin{columns}
		\column{.5\textwidth}
			\centering
			\includegraphics[width=0.8\textwidth]{CMMworkaroundTry2-crop.pdf}
		\column{.5\textwidth}
			\footnotesize
			\begin{itemize}
				\item
					same procedure as for the other method
				\item	
					different measurement scheme for the laserinterferometer
					\begin{itemize}
						\footnotesize
						\item
							measure distances between the pins and one marker
						\item
							calculate the distance ({\color{blue}blue}) to the baseline between the pins
						\item
							repeat this for each marker
					\end{itemize}
			\end{itemize}
	\end{columns}
	\vspace{5mm}
	
	\begin{align*}
		h_{a} = \sqrt{2 \left(a^2 b^2 + b^2 c^2 +c^2 a^2 \right) - \left( a^4 + b^4 + c^4 \right)} / (2a)
	\end{align*}
	
	\footnotesize
	measurements not yet done (granite table with CMM needed for series production)
	
}

\frame{\frametitle{Alignment Measurements w.r.t. the Alignment Frame}
	\begin{columns}
		\column{.5\textwidth}
			\centering
			alignment frame hole
			
			\includegraphics[width=0.75\textwidth]{alignmentFrameHole.pdf}
		\column{.5\textwidth}
			\centering
			precision marker
			
			\includegraphics[width=0.75\textwidth]{markerFitted.pdf}
	\end{columns}
	\vspace{2mm}
	
	\footnotesize
	compare position of hole in alignment frame with marker on panel via a circular fit
	
	\begin{columns}
		\column{.5\textwidth}
			\centering
			\includegraphics[width=0.9\textwidth]{aliNfirstPanels_pixelpositionGood_wPixelSize.pdf}
		\column{.5\textwidth}
			\centering
			\includegraphics[width=0.9\textwidth]{aliNfirstPanels_pixelpositionBad_wError.pdf}
	\end{columns}
	\centering
	
	$\Rightarrow$ reconstruction of alignment error feasible
}

\frame{\frametitle{Measurements done in Saclay}
	\begin{itemize}
		\item
			CMM scans for planarity
		\item
			CMM scans with camera for alignment in single plane
		\item 
			Raskfork for alignment w.r.t. other side of panel
	\end{itemize}
	\vspace{3mm}
	
	\begin{columns}
		\column{.50\textwidth}
			\centering
			CMM prepared for measurement with the laserinterferometer
			\vspace{1mm}
			
			\includegraphics[width=\textwidth]{sacleyNov2017_gantryWithLaserinterferometer.JPG}
		\column{.50\textwidth}
			\centering
			Raskfork attached to a readout panel
			\vspace{2.5mm}
			
			\includegraphics[width=\textwidth]{sacleyNov2017_raskfork.JPG}
	\end{columns}
}

\frame{\frametitle{Saclay Results for the first Series Panels}
		alignment in a single plane $\Rightarrow$ w.r.t. middle board [$\mu$m]
		\vspace{3mm}
		
		\scriptsize
		\begin{tabular}{cc|cccc}
			panel & side &  large left & large right & small left & small right
			\\
			\hline
			stereo1 & 1 & -11 & -16 & -27 & -38
			\\
		 	 & 2 & 4 & -15 & -1 & -14
			\\
			\hline
			stereo2 & 1 & -13 & {\color{red}-73} & -18 & -7
			\\
			 & 2 & -8 & -24 & -19 & -32
			\\
			\hline
			eta2 & 1 & -4 & -39 & {\color{red}-54} & -20
			\\
			 & 2 & 32 & 19 & -10 & {\color{red}-60}
			\\
			\hline
			frame & Freiburg & 35 & 39 & 28 & 24
		\end{tabular}
		\vspace{3mm}
		
		\normalsize
		alignment w.r.t. the other side of the panel [$\mu$m]
		\vspace{3mm}
		
		\scriptsize
		\begin{tabular}{c|cccccc}
			panel & large left & large right & middle left & middle right & small left & small right
			\\
			\hline
			stereo1 & {\color{blue}-266} & -5 & {\color{blue}-238} & -27 & {\color{blue}-187} & -24
			\\
			stereo2 & 34 & -33 & {\color{red}95} & -36 & {\color{red}81} & {\color{red}44}
			\\
			eta2 & -28 & -11 & -38 & 15 & {\color{red}56} & 4
		\end{tabular}
}

\section{Reconstruction of the Board Alignment}

\subsection{Beamtime in August 2017 at the H8 Beamline}

\frame{\frametitle{H8 Testbeam Setup for Tracking}
	\begin{columns}
		\column{.50\textwidth}
			\centering
			\includegraphics[width=0.9\textwidth]{setup-crop.pdf}
		\column{.50\textwidth}
			\centering
			\includegraphics[width=0.9\textwidth]{SM2-M0_H8aug17_setup.png}
	\end{columns}
	
	\footnotesize
	
	\begin{itemize}
		\item
			module fully equipped with 96 APVs
		\item 
			1024 strips for each layer were read out by two FEC cards
		\item
			two further FEC cards for 28 APVs of tracking telescope:
			
			3 twodimensional GEMs and 2 twodimensional TMMs
			
			$\Rightarrow$ 4 FECs read out @ \SI{220}{Hz} %(during spills)
		\item
			acquisition of time jitter via TDC (VME)
	\end{itemize}
}

\frame{\frametitle{Measurement Program for Module 0}
	view from telescope against beam direction
	\vspace{3mm}
	
	\begin{columns}
		\column{.50\textwidth}
			\centering
			\includegraphics[width=0.9\textwidth]{SM2-M0_pointsAtH8_stereofront-crop.pdf}
		\column{.50\textwidth}
			\centering
			\includegraphics[width=0.9\textwidth]{SM2-M0_pointsAtH8_etafront-crop.pdf}
	\end{columns}
	\vspace{3mm}
	
	\begin{itemize}
		\item
			scan of a large part of the active area to investigate the efficiency  
			
			$\Rightarrow$ dead or noisy areas (for example between the PCBs)
		\item
			determination of resolution as function of amplification and drift voltage, as well as incidence angle
		\item
			\textbf{reconstruction of PCB alignment}
	\end{itemize}
}

\frame{\frametitle{Results for Pulse Height and Spatial Resolution}
	\begin{itemize}
		\item
			pulse height behaves as expected
		\item
			for both eta layers the same resolution is achieved
			
			$\Rightarrow$ below \SI{90}{\micro\m}
		\item
			resolution independent of amplification voltage
		\item
			resolution independent of drift voltage
		\item
			angular dependence very similar to 10$\times$10 cm$^2$ chambers
	\end{itemize}
	
	\begin{columns}
		\column{.50\textwidth}
			\centering
			\includegraphics[width=0.9\textwidth]{SM2-M0_H8aug17_ampScan_nice.pdf}
		\column{.50\textwidth}
			\centering
			\includegraphics[width=0.9\textwidth]{SM2-M0_H8aug17_resolutionVSampVolt-crop.pdf}
	\end{columns}
	
	$\Rightarrow$ module showed a good performance despite bad PCB quality
}

\frame{\frametitle{Reconstruction of the PCB Rotation}
	residual between the layers as function of position along the strips
	
	\begin{columns}
		\column{.50\textwidth}
			\centering
			\includegraphics[width=0.9\textwidth]{H8_run193_residualEtaLayers.pdf}
			
			\scriptsize
			rotation [\SI{10}{\micro\m/\m}]
			
			\begin{tabular}{cc|cc}
				panel & board & CMM & H8 
				\\
				\hline
				stereo & small & -11 & 
				\\
			 	 & middle & 1 & 13
				\\
			 	 & large & -12 & -14
				\\
				\hline
				eta & small & 11 & 10 
				\\
				 & middle & 6 & -4
				\\
				 & large & -4 & 6
			\end{tabular}
		\column{.50\textwidth}
			\centering
			\includegraphics[width=0.8\textwidth]{sm2_m0_h8_board7_etaDifVSalongStrips.png}
			\vspace{3mm}
			\scriptsize
			
			problems:
			
			\begin{itemize}
				\item
					only small areas investigated
				\item
					no exact measurement of the position along the strips
				\item
					noisy and dead areas on board 6
				\item
					systematic error due to inclination w.r.t. beam
			\end{itemize}
	\end{columns}
}

\subsection{Investigation in the Cosmic Ray Facility}

\frame{\frametitle{Cosmic Ray Test Facility}
		
	\begin{columns}
		\column{.50\textwidth}
			\centering
			\includegraphics[width=1.0\textwidth]{CRFprinciple-crop.pdf}
		\column{.50\textwidth}
			\centering
			\includegraphics[width=0.9\textwidth]{CRF_small.JPG}
	\end{columns}
	
	\footnotesize
			
	\begin{itemize}
		\item
			2D track reconstruction with two Monitored Drift Tube (MDT) chambers
		\item
			trigger via scintillator hodoscope with $\approx$ \SI{10}{cm} resolution in orthogonal direction
		\item
			MDT chambers : \SI{2.2}{\m} $\times$ \SI{4}{\m} 
			
			$\Rightarrow$ active area : \SI{9}{\m\squared}, angular acceptance : $\pm30^{\circ}$
		\item
			readout of the full module with six FEC cards @ full \SI{100}{Hz} \textmu -rate
			
			(tested up to \SI{500}{Hz} with random trigger)
	\end{itemize}
}

\frame{\frametitle{Alignment using Reference Tracks}
	\centering \textbf{Idea:}
	\begin{columns}
		\column{.50\textwidth}
			\centering
			\includegraphics[width=0.90\textwidth]{positionshift2-crop.pdf}
		\column{.50\textwidth}
			\centering
			\includegraphics[width=0.90\textwidth]{verticalshift-crop.pdf}
	\end{columns}
	\begin{columns}
		\column{.55\textwidth}
			\centering
			
			%{\rotatebox{90}{\textbf{\tiny \hspace{3mm} residual = measured - reference position}}}
			\includegraphics[width=0.93\textwidth]{resVSslope_blue.pdf}
		\column{.45\textwidth}
			\small
			\textbf{Implementation:}
			\begin{itemize}
				\item
					%residual = measured - reference position
					\textbf{residual} = $\mathrm{pos}_{\mathrm{measured}} - \mathrm{pos}_{\mathrm{reference}}$
				\item
					residual vs. slope 
					
					 of reference track 
					 
					$\Rightarrow$ {\color{red}linear fit}
				\item	
					$\mathrm{shift}_{\mathrm{horizontal}} = \mathrm{intercept}_{\mathrm{fit}}$
				\item						
					$\mathrm{shift}_{\mathrm{vertical}} \;\;\; = \mathrm{slope}_{\mathrm{fit}}$
			\end{itemize}
	\end{columns}
}

\frame{\frametitle{Reconstructed Deviations per Partition}
	\small

	$\Rightarrow$ subdivide the active area into smaller partitions
	
	$\Rightarrow$ reconstruct separate for each partition the vertical and the horizontal shift
	
	\vspace{3mm}
	
	\begin{columns}
		\column{.50\textwidth}
			\centering
			vertical shift
			
			\includegraphics[width=0.90\textwidth]{sm2_m0_CRF_eta-out_deltaZperPartition.pdf}
			
			\scriptsize
			$\Rightarrow$ gravitational sag deforms all layers the same way
		\column{.50\textwidth}
			\centering
			horizontal shift
			
			\includegraphics[width=0.90\textwidth]{sm2_m0_CRF_eta-out_deltaYperPartition.pdf}
			\scriptsize
			
			$\Rightarrow$ shifts between PCBs have to be investigated
			
			$\Rightarrow$ rotations on this scale are hardly recognizable
	\end{columns}
}

\frame{\frametitle{Reconstructed Pitch Error and Board Alignment per Plane}
	\begin{columns}
		\column{.50\textwidth}
			\centering
			{\color{blue}pitch error} and {\color{red}misaligned electronic adapter board}
			
			\includegraphics[width=0.90\textwidth]{sm2_m0_CRF_eta-in_color.pdf}
			
			\vspace{3mm}
			
			pitch error [10$^{-4}$]
			\scriptsize
			\begin{tabular}{cc|ccc}
			panel & side & small & middle & large
			\\
			\hline
			stereo & out & 0.8 & -2.4 & 2.7
			\\
			stereo & in &  & 1.4 & 
			\\
			\hline
			eta & in & 1.0 & 2.9 & 3.6
			\\
			eta & out & 2.3 & 4.4 & 4.7
			\end{tabular}
		\column{.50\textwidth}
			\centering
			corrected pitch and shifted adapter board
			
			\includegraphics[width=0.90\textwidth]{sm2_m0_CRF_eta-in_shiftNpitchCor.pdf}
			
			\vspace{3mm}
						
			center shift [mm]
			\scriptsize
			\begin{tabular}{cc|cc}
			panel & side & small & large
			\\
			\hline
			stereo & out & 0.09 & -0.06
			\\
			stereo & in & -0.18 & 0.25
			\\
			\hline
			eta & in & -0.07 & 0.13
			\\
			eta & out & 0.00 & 0.24
			\end{tabular}
	\end{columns}
}

\frame{\frametitle{Rotations and Shifts per PCB}
	\begin{columns}
		\column{.50\textwidth}
			\centering
			stereo in middle
			
			\includegraphics[width=\textwidth]{resVSscinX.pdf}
		\column{.50\textwidth}
			\begin{itemize}
				\item
					reconstruction of the rotation: 
						
					residual as function along the strips
					
					$\Rightarrow$ separately for each PCB
					
					$\Rightarrow$ calculate relative to middle PCB
				\item
					reconstruction of the shift:
					
					residual distribution of each PCB separately
					
					$\Rightarrow$ calculate relative to middle board
			\end{itemize}
	\end{columns}
}

\frame{\frametitle{Comparison of CMM and CRF measurements}
	alignment in single plane relative to middle board
	\vspace{2mm}
	\scriptsize
	\begin{tabular}{ccc|cc|cc}
		 &  &  & \multicolumn{2}{c}{CMM Freiburg} &  \multicolumn{2}{c}{CRF}
		\\
		panel & side & board & rotation & center & rotation & center
		\\
		 &  &  &  [\SI{10}{\micro\m/\m}] & [mm] &  [\SI{10}{\micro\m/\m}] & [mm]
		\\
		\hline
		stereo & out & small & 4 & 0.07 & -2 & 0.10
		\\
		 &  & large & 6 & -0.17 & 20 & -0.04
		\\
		\hline
		stereo & in & small & -16 & 0.10 & -11 & -0.17
		\\
		 &  & large & 7 & -0.02 & -8 & 0.26
		\\
		\hline
		eta & in & small & 2 & 0.07 & 7 & -0.08
		\\
		 &  & large & 3 & -0.03 & 4 & 0.23
		\\
		\hline
		eta & out & small & 2 & 0.05 & 8 & 0.01
		\\
		 &  & large & 6 & -0.01 & -5 & 0.27
	\end{tabular}
	\vspace{2mm}
	
	
		
	\scriptsize
	\begin{tabular}{cc|cc|cc}
		&  & \multicolumn{2}{c}{CMM Freiburg} &  \multicolumn{2}{c}{CRF}
		\\
		panel & board & rotation & center & rotation & center
		\\
		 &  &  [\SI{10}{\micro\m/\m}] & [mm] &  [\SI{10}{\micro\m/\m}] & [mm]
		\\
		\hline
		stereo & small & -11 & -0.06 & 73 &
		\\
		& middle & 1 & 0.04 & 81 & 
		\\
		& large & -12 & 0.21 & 53 & 
		\\
		\hline
		eta & small & 11 & -0.04 & 11 & -0.07
		\\
		& middle & 6 & -0.06 & 10 & 0.01
		\\
		& large & -4 & -0.01 & 1 & 0.06
	\end{tabular}
}

\frame{\frametitle{Conclusion of CMM and CRF measurements}
	
	$\Rightarrow$ no obvious correlation visible
	\vspace{2mm}
	\scriptsize
	
	error sources:
	\begin{itemize}
		\item
			method: 
			\begin{itemize}
				\scriptsize
				\item	
					comparison with middle board may be problematic
				\item						
					errors during measurement in Freiburg can be reconstructed hardly
			\end{itemize}
		\item
			bad PCB quality: noisy and dead areas distort residual distributions
		\item
			imperfect alignment during fabrication of PC in industry
		\item
			deformation of the glued panel during transport (Munich - Freiburg and vice versa), due to temperature, humidity or mechanical stress
	\end{itemize}
	\vspace{5mm}
	
	\normalsize
	
	\centering
	
	$\Rightarrow$ work in progress

}

\frame{\frametitle{Summary}
	\footnotesize
	\begin{itemize}
		\item
			replacement of the ATLAS inner end cap with new detector systems
			
			$\Rightarrow$ sTGC and Micromegas quadruplets
		\item 
			Micromegas quadruplets for SM2 Modules will be built by the German collaboration
			
			$\Rightarrow$ huge part of work is and will be done in Munich
		\item
			segmented readout structure due to limitations by industry
			
			$\Rightarrow$ surveying and calibration is needed during and after construction
		\item
			CMM in combination with cameras could be used for surveying
			
			\begin{itemize}
				\footnotesize
				\item
					calibration of the local CMM is sophisticated and needs more measurements
				\item
					effects of surveying with the camera have to be understood better
			\end{itemize}
			
			$\Rightarrow$ measurements with calibrated CMMs show good results for first panels
		\item
			measurement of readout PCB alignment with the CRF should be feasible
						
			\begin{itemize}
				\footnotesize
				\item
					distortions of the reconstruction with the residuals due to imperfect board quality should decline with series PCB
				\item
					much better alignment than for the L1 chamber already reconstructed
				\item
					for comparison with optical measurements better analysis methods have to be considered
			\end{itemize}
	\end{itemize}
}

\frame{\centering \Huge Merry Christmas \\ and a \\ Happy New Year}

\appendix

\frame{\centering \Huge Backup}

\frame{\frametitle{Nominal Distances between the Precision Marker}
	\centering
	\includegraphics[width=0.9\textwidth]{nominalDistances.pdf}
}

\end{document}
