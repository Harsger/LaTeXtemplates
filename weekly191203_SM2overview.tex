\documentclass[usenamees,dvipsnames]{beamer}
\usetheme{Madrid}
\usecolortheme{spruce}
\definecolor{green(pigment)}{rgb}{0.0, 0.65, 0.31}
\setbeamercolor*{item}{fg=green(pigment)}
\definecolor{Green}{rgb}{0.00, 1.00, 0.00}
\definecolor{Red}{rgb}{1.00, 0.00, 0.00}
\definecolor{Blue}{rgb}{0.00, 0.00, 1.00}
\usepackage[utf8]{inputenc}
\usepackage{amsmath}
\usepackage{amsfonts}
\usepackage{amssymb}
\usepackage[german]{babel}
\usepackage{graphicx}
\usepackage{rotating}
\usepackage{textcomp}
\usepackage{multirow,bigdelim,dcolumn,booktabs}
%\usepackage{beamerthemeshadow}
\usepackage{subfigure} 
\usepackage{siunitx}
\usepackage{appendixnumberbeamer}
\usepackage{hyperref}
\usepackage{mdframed}
\usepackage{xcolor}
\usepackage[absolute,overlay]{textpos}
%\beamersetuncovermixins{\opaqueness<1>{25}}{\opaqueness<2->{15}}
\beamertemplatenavigationsymbolsempty

\usepackage{tikz}
\usetikzlibrary{decorations.text}
%\usetikzlibrary{trees}
\usetikzlibrary{decorations.pathmorphing}
\usetikzlibrary{decorations.markings}
\usetikzlibrary{patterns}

\graphicspath{
	{pictures/}
%	{/home/m/Maximilian.Herrmann/Bilder/forLatex/plots/}
%	{/home/m/Maximilian.Herrmann/Bilder/forLatex/sketches/}
%	{/home/m/Maximilian.Herrmann/Bilder/forLatex/pictures/}
}

\title[SM2 in CRF]{SM2 Micromegas Modules in the Cosmic Ray Facility}  
\author[M. Herrmann]{Maximilian Herrmann}
\institute[LMU Munich]{Ludwig-Maximilians-Universit\"at M\"unchen - Lehrstuhl Schaile}
\date[03.12.2019]{Micromegas Weekly Meeting 03.12.2019} 

\begin{document}

\frame{
	\titlepage
}

\frame{\frametitle{First Wheel}
	\Large
%	\centering
	16 modules validated and delivered
    
    \vspace{10mm} 
     
	\begin{tabular}{lcl}
		{\color{blue} M5 - M8} & , & {\color{red} M9} , {{\color{red} M11 - M21}}
		\\
		& &
		\\
		{\color{blue} 4 no passivation} & , & {\color{red} 12 incl. passivation}
	\end{tabular}
	    
	\vspace{10mm} 
       
    \large
    QAQC-plots see Ralf's H. talk at Muon Week:
    \vspace{2mm}
    
    \scriptsize
    \url{https://indico.cern.ch/event/856527/contributions/3605312/attachments/1939339/3214964/muon_week_sm2_nov_2019.pdf}
}

\frame{\frametitle{Second Wheel}
	\large
	     
	\begin{tabular}{lcl}
		M4 & : & former eltx. module
		\\
		& & passivated, assembled
		\\
		& & CRF passed
	\end{tabular}
    
    \vspace{10mm} 
     
	\begin{tabular}{cccll}
		module & eta & stereo & status & plan
		\\
		\hline
		22 & 22 & 22 & assembled $\Rightarrow$ HV-test & CRF this week
		\\
		23 & 23 & 24 & RO-panel finalisation & 
		\\
		24 & 24 & 25 & RO-panel finalisation & assembly next week
		\\
		25 & 25 & 26 & RO-panel finalisation &
	\end{tabular}
	    
	\vspace{5mm} 
       
    $\Rightarrow$ M22 - M25 will be finished this year
     	    
   	\vspace{5mm} 
     
    RO-boards for panels of two more quadruplets available
    
    $\Rightarrow$ will be glued until end of year
}

\frame{\frametitle{Efficiency Investigation (example M19, eta-out)}
	
	\centering
	maps @ $U_{\mathrm{amp}} = $ \SI{580}{V}
	
	\begin{columns}
		\column{.5\textwidth}
			\centering
			\includegraphics[width=0.8\textwidth]{MMS200019L1efficiency.png}
		\column{.5\textwidth}
			\centering
			\includegraphics[width=0.8\textwidth]{MMS200019L1gain.png}
	\end{columns}
	\vspace{2mm}

	\begin{columns}
		\column{.5\textwidth}
			\centering
			turn on curve
			
			\includegraphics[width=0.8\textwidth]{MMS200019L1ampScan.png}
		\column{.5\textwidth}
			\centering
			H8 testbeam 2018
			
			\includegraphics[width=0.8\textwidth]{m1_H8_2018_board7efficiency.png}
	\end{columns}
	
}

\frame{\frametitle{Effects on Efficiency}
	\centering
	\small
	
	\begin{columns}
		\column{.5\textwidth}
			\centering
			efficiency VS noise level (M7)
			
			\includegraphics[width=0.8\textwidth]{m7_5mmEffiVSsigmaPerAPV_wFit.png}
		\column{.5\textwidth}
			\centering
			pulse height VS pillar height (M1,M3,M6,M7)
			
			\includegraphics[width=0.9\textwidth]{meanClusterQvsPillarHeight_etaNstereo_wFit.png}
	\end{columns}
	\vspace{2mm}

	\begin{columns}
		\column{.5\textwidth}
			\centering
			efficiency per APV 
			
			(M7, $U_{\mathrm{amp}} = $ \SI{570}{V})
			
			\includegraphics[width=0.8\textwidth]{m7_5mmEffiVSnormedClusterQperAPV.pdf}
		\column{.5\textwidth}
%		\normalsize
			\begin{itemize}
				\item
					noise level individual 
					
					for front-end eltx.
				\vspace{4mm}
				\item
					pillar height variation
				
					$\Rightarrow$ pulse height inhomogeneities 
				\vspace{4mm}
			\end{itemize}
			
			$\Rightarrow$ new master student (Sebastian Trost)
	\end{columns}
	
}

\frame{\frametitle{Tracking in the Cosmic Ray Facility}
		
	\begin{columns}
		\column{.50\textwidth}
			\centering
			\includegraphics[width=1.0\textwidth]{CRFprinciple5-crop.pdf}
		\column{.50\textwidth}
			\centering
			\includegraphics[width=0.9\textwidth]{CRF_small.JPG}
	\end{columns}
	
	\vspace{7mm}
	\begin{columns}
		\column{.70\textwidth}
			\centering
%			\setlength{\tabcolsep}{0pt}
			\begin{tabular}[]{ccccc}
				residual & = & {\color{blue}measured} & - & {\color{red}reference}
				\\
				 & & & & 
				\\
				 & = & centroid $\times$ pitch & - & track$_{\mathrm{MDTs}}$ @ MM
			\end{tabular}
		\column{.30\textwidth}
			\centering
			$\Rightarrow$ mean residual 
			
			VS position
			
			\includegraphics[width=0.7\textwidth]{m1_ei_residaul_nearZero.pdf}
	\end{columns}
}

\frame{\frametitle{\large Alignment Investigation Eta-panel 3 $\Rightarrow$ Consistent Reconstruction}
	\centering
		
	\begin{columns}
		\column{.05\textwidth}
		\column{.3\textwidth}
			\centering
			September 2018, 
			
			in M3
		\column{.3\textwidth}
			\centering
			January 2019, 
			
			as doublet
		\column{.3\textwidth}
			\centering
			May 2019, 
			
			as doublet
	\end{columns}
	\vspace{2mm}
	
	\begin{columns}
		\column{.05\textwidth}
			\rotatebox{90}{board 8}
		\column{.3\textwidth}
			\centering
			\includegraphics[width=0.9\textwidth]{eta3inM3_b8_resMeanVsscinX.pdf}
		\column{.3\textwidth}
			\centering
			\includegraphics[width=0.9\textwidth]{eta3withM6_b8_resMeanVsscinX.pdf}
		\column{.3\textwidth}
			\centering
			\includegraphics[width=0.9\textwidth]{eta3withM8_b8_resMeanVsscinX.pdf}
	\end{columns}
	\vspace{2mm}

	\begin{columns}
		\column{.05\textwidth}
			\rotatebox{90}{board 7}
		\column{.3\textwidth}
			\centering
			\includegraphics[width=0.9\textwidth]{eta3inM3_b7_resMeanVsscinX.pdf}
		\column{.3\textwidth}
			\centering
			\includegraphics[width=0.9\textwidth]{eta3withM6_b7_resMeanVsscinX.pdf}
		\column{.3\textwidth}
			\centering
			\includegraphics[width=0.9\textwidth]{eta3withM8_b7_resMeanVsscinX.pdf}
	\end{columns}
	\vspace{2mm}

	\begin{columns}
		\column{.05\textwidth}
			\rotatebox{90}{board 6}
		\column{.3\textwidth}
			\centering
			\includegraphics[width=0.9\textwidth]{eta3inM3_b6_resMeanVsscinX.pdf}
		\column{.3\textwidth}
			\centering
			\includegraphics[width=0.9\textwidth]{eta3withM6_b6_resMeanVsscinX.pdf}
		\column{.3\textwidth}
			\centering
			\includegraphics[width=0.9\textwidth]{eta3withM8_b6_resMeanVsscinX.pdf}
	\end{columns}
	
}

\frame{\frametitle{\large Layer-to-Layer Alignment Investigation Eta-panel 3 \\ (direct Comparison to Rasfork-measurement)}
	\begin{columns}
		\column{.5\textwidth}
			\centering
			mean residual
			
			back-to-back
			
			\includegraphics[width=0.95\textwidth]{m3_201809_etaDifZcor.pdf}
		\column{.5\textwidth}
			\centering
			Rasfork
			
			\includegraphics[width=0.8\textwidth]{RS2E00003_GluingSide1_plot.pdf}
	\end{columns}
	\vspace{2mm}
	
	comparison aggravated by 
	
	\begin{itemize}
		\item
			strip shape
		\item
			mechanical placement in the CRF (deformation and illumination)
	\end{itemize}
	
}

\frame{\frametitle{\large Layer-to-Layer Alignment Investigation Eta-panels 8 and 9}
	
	\begin{columns}
		\column{.05\textwidth}
			\rotatebox{90}{Eta 8}
		\column{.5\textwidth}
			\centering
			\includegraphics[width=0.9\textwidth]{m8_8020_etaDifZcor.pdf}
		\column{.45\textwidth}
			\centering
			\includegraphics[width=0.65\textwidth]{RS2E00008_GluingSide2_plot.pdf}
	\end{columns}
%	\vspace{2mm}

	\begin{columns}
		\column{.05\textwidth}
			\rotatebox{90}{Eta 9}
		\column{.5\textwidth}
			\centering
			\includegraphics[width=0.9\textwidth]{m9_etaDifZcor.pdf}
		\column{.45\textwidth}
			\centering
			\includegraphics[width=0.65\textwidth]{RS2E00009_GluingSide2_plot.pdf}
	\end{columns}
	
}

\frame{\frametitle{\large Layer-to-Layer Alignment Investigation last Eta-panels}
	
	\centering
	\includegraphics[width=0.9\textwidth]{rotationEtaLayersPerBoard.png}
	
	$\Rightarrow$ new master student (Fabian Vogel), also working on Jig-Calibration
	
}

\frame{\frametitle{Summary}
	\small
	\begin{itemize}
		\item
			all 16 SM2 modules for the first wheel investigated in the CRF
		\item
			reference-tracking
			\begin{itemize}
				\item
					efficiency influenced by 
					
					\begin{itemize}
						\item
							noise level (preliminary eltx.)
						\item
							pulse height $\leftrightarrow$ pillar height
						\item
							dead strips ($\Rightarrow$ QAQC)
					\end{itemize}
				\item
					alignment-reconstruction aggravated by
										
					\begin{itemize}
						\item
							strip shape
						\item
							illumination
						\item
							placement in CRF
					\end{itemize}
					
					despite this:
					
					correlation between Rasfork and Cosmics observed
			\end{itemize}
	\end{itemize}
		
	\vspace{5mm}
	
	\tiny
	
	last talk : 
	
	\url{https://indico.cern.ch/event/843298/contributions/3552706/attachments/1902301/3140652/SM2inCRF_20190904_mherrmann.pdf} 
	
	\vspace{2mm}
	
	thesis : 
	
	\url{https://www.etp.physik.uni-muenchen.de/publications/theses/download/phd_mherrmann.pdf}
}

\appendix

\frame{\centering \Huge Backup}

\frame{\frametitle{In-Plane Alignment Investigation Eta-panels 8 and 9}
	
	\begin{columns}
		\column{.05\textwidth}
		\column{.475\textwidth}
			\centering
			gluing side 1
		\column{.475\textwidth}
			\centering
			gluing side 2
	\end{columns}
	
	\begin{columns}
		\column{.05\textwidth}
			\rotatebox{90}{Eta 8}
		\column{.475\textwidth}
			\centering
			\includegraphics[width=0.9\textwidth]{m8_ei_8020_meanResidualMap.pdf}
		\column{.475\textwidth}
			\centering
			\includegraphics[width=0.9\textwidth]{m8_eo_8020_meanResidualMap.pdf}
	\end{columns}
%	\vspace{2mm}

	\begin{columns}
		\column{.05\textwidth}
			\rotatebox{90}{Eta 9}
		\column{.475\textwidth}
			\centering
			\includegraphics[width=0.9\textwidth]{m9_ei_alignedEI_meanResidualMap.pdf}
		\column{.475\textwidth}
			\centering
			\includegraphics[width=0.9\textwidth]{m9_eo_meanResidualMap.pdf}
	\end{columns}
	
}

\frame{\frametitle{Cosmic Ray Facility}
		
	\begin{columns}
		\column{.50\textwidth}
			\centering
			\includegraphics[width=1.0\textwidth]{CRFprinciple7-crop.pdf}
		\column{.50\textwidth}
			\centering
			\includegraphics[width=0.8\textwidth]{CRFwModuleNdoublet.jpg}
	\end{columns}
	
	\vspace{2mm}
	\centering
	\footnotesize
	\begin{tabular}{ll}
		trigger & 2 $\times$ scintillator hodoscopes
		\\
		track reconstruction & 2 $\times$ Monitored Drift Tube chambers (MDTs)
		\\
		active area & \SI{2}{m} $\times$ \SI{4}{m}
		\\
		angular acceptance & $\pm$ \SI{30}{\degree} to zenith
		\\
		energy cut (hardware) & iron absorber (\SI{34}{cm}) $\to E_{\mu} >$ \SI{600}{MeV} 
		\\
		readout & 12288 channels
		\\
		 & $\to$ 96 APVs (frontend electronics)
		\\
		 & $\to$ 6 FECs (scalable readout system)
		\\
		readout rate & 100 Hz (muon rate)
	\end{tabular}
}

\frame{\frametitle{\large Charge and Time Resolved Readout using APV25 Electronics}
	\scriptsize
	
	\begin{columns}
		\column{.32\textwidth}
			\centering
			96 $\times$ APV25-chips
			
			\includegraphics[width=0.95\textwidth]{APV25.jpg}
		\column{.32\textwidth}
			\centering
			6 $\times$ FEC-cards
			
			\includegraphics[width=0.82\textwidth]{FECcabling.jpg}
		\column{.32\textwidth}
			\centering
			strips parallel to MDT-wires
			
			\includegraphics[width=0.95\textwidth]{moduleInCRF.jpg}
	\end{columns}
		
	\vspace{3mm}
	\begin{columns}
		\column{.32\textwidth}
			128 channel 
			
			charge sensitive preamplifier
			
			\SI{40}{MHz} signal sampling
		\column{.32\textwidth}
			jitter recorded individually 
			
			$\Rightarrow$ unbiased time-evaluation
		\column{.32\textwidth}
			reference track
			
			$\Rightarrow$ efficiency and resolution
	\end{columns}
	
	\vspace{3mm}
	\begin{columns}
		\column{.32\textwidth}
			\centering
			\includegraphics[width=\textwidth]{eventdisplay_m5_570V.pdf}
		\column{.32\textwidth}
			\centering
			\includegraphics[width=\textwidth]{FECjitter.pdf}
		\column{.32\textwidth}
			\centering
			\includegraphics[width=\textwidth]{trackFitFourLayers.png}
	\end{columns}
} 

\frame{\frametitle{Measurement Overview}

	\begin{columns}
		\column{.5\textwidth}
			\centering
			\includegraphics[width=\textwidth]{CRFintegratedCountsActiveArea.pdf}
			
			\includegraphics[width=\textwidth]{temperatureNpressure_Garching2018to19_edited.png}
		\column{.5\textwidth}
			\small
					
			\begin{itemize}
				\footnotesize
				\item
					central part of active area
					
					$\Rightarrow$ \SI[product-units = repeat]{96 x 78}{cm}
					
					overall more than $\tfrac{1}{2}$ billion trigger
				\item
					counts integrated over all amplification voltages
					
					$\Rightarrow$ also low gain measurements
				\item
					CRF average {\color{blue}pressure} :  \SI{960}{\hecto\pascal} 
					
					ATLAS-cavern : \SI{980}{\hecto\pascal} 
					
					$\Rightarrow$ different gain
				\item
					CRF {\color{red}temperature} : \SI{21}{\degreeCelsius} 
					
					controlled to about $\pm$\SI{2}{\degreeCelsius}
				\item
					modules measured: 0, 1, 3 (spare) 
					
					5, 6, 7, 8, 9, 11, 12, 13, 14, 15, 16, 17, 18, 19, 21 (first wheel)
					
					4 (for second wheel)
			\end{itemize}
	\end{columns}
	
}

\frame{\frametitle{\large Reconstruction of Pitch Deviations and Readout Board Alignment}

	\begin{columns}
		\column{.5\textwidth}
			\centering
			without correction
			
			\includegraphics[width=0.8\textwidth]{eta_in_resMeanVSmdtY_woAdapterBoard_wCorPitch_wCenter.png}
		\column{.5\textwidth}
			\centering
			pitch deviation considered
			
			\vspace{-0.3mm}
			\includegraphics[width=0.8\textwidth]{eta_in_resMeanVSmdtY_wNewPitchCor_shifts.pdf}
	\end{columns}
	
	\vspace{4mm}
	
	\begin{columns}
		\column{.5\textwidth}
			\centering
			\includegraphics[angle=-90,width=0.55\textwidth]{ROboardsAligned1_wRingsNlines-crop.pdf}
			\vspace{3mm}
		\column{.5\textwidth}
			\centering
			\includegraphics[width=0.8\textwidth]{boardPitchDeviation_wLinesNdesNarrows-crop.pdf}
	\end{columns}
	
}

\frame{\frametitle{\large Gain Dependencies}

	\begin{align*}
		G &= \exp\left( \alpha \cdot d \right)
		\\[10pt]
		&= \exp\left[ \;\; \dfrac{A p d}{T} \; \cdot \; \exp\!\left(\; -\dfrac{B p d}{T U} \; \right) \;\; \right].
		\label{gainParametrization}
	\end{align*}
	
	Taylor-series-expansion up to first order:
	
	\begin{align*}
		G(d) &= 
				\exp\left( a \cdot d_0 \cdot e^{- b \cdot d_0} \right) 
				\nonumber
				\\
				&\qquad + a \cdot ( 1 - b \cdot d_0 ) \cdot \exp\left( - b \cdot d_0 + a \cdot d_0 \cdot e^{- b \cdot d_0} \right) \cdot ( d - d_0 )
				\nonumber
				\\
				&\qquad + \mathcal{O}\left( ( d - d_0 )^2\right) \;\; ,
	\end{align*}
	
	where 
	
	\begin{align*}
		a = \dfrac{A \cdot p}{T} \;\;\; \text{and} \;\;\; b = \dfrac{B \cdot p}{T \cdot U} \;\; .
	\end{align*}
	
}

\frame{\frametitle{\normalsize Position Reconstruction using Charge and Drift-Time Measurements (M1)}
	
	\begin{columns}
		\column{.5\textwidth}
			\centering
			residual normal tracks
			
			\includegraphics[width=0.8\textwidth]{m1_ei_residaul_nearZero.pdf}
		\column{.5\textwidth}
			\centering
			angle reconstruction
			
			\includegraphics[width=0.65\textwidth]{m1_ei_uTPCangleVSangle_20180601_uptime_sameRange.pdf}
	\end{columns}
	
	\vspace{1mm}
		
	\begin{columns}
		\column{.5\textwidth}
			\centering
			simulation
			
			\includegraphics[width=0.7\textwidth]{inhomogeneousIonization_bE_cNm_timeNcentroidShift.pdf}
		\column{.5\textwidth}
			\centering
			resolution VS angle
			
			\includegraphics[width=0.8\textwidth]{m1_ei_allResolutionsVSangle_20180601.pdf}
	\end{columns}
	
}

\frame{\frametitle{Position Reconstruction using two Layers of Inclined Strips}
			
	\begin{columns}
		\column{.5\textwidth}
			\includegraphics[width=0.9\textwidth]{stereoBoards_angleNshift_centerNcoord-crop.pdf}
			\vspace{2mm}
			
			\textcolor{blue}{precision} 
			
			\hspace{1mm} = stereos mean / $\cos \alpha$ 
			\vspace{1.5mm}
						
			\textcolor{red}{non-precision}
			
			\hspace{1mm} = stereos difference / 2 $/ \sin \alpha$
			\vspace{1.5mm}
						
			$\Rightarrow$ \textcolor{Brown}{alignment} 
			
			\hspace{4.5mm} non-precision coordinate
			
			\vspace{5mm}
			$\Rightarrow$ non-precision residual spoiled 
			
			\hspace{4.5mm} by coarse scintillator resolution
		\column{.5\textwidth}
			\centering	
			\includegraphics[width=0.65\textwidth]{m1_stereo_posDifVSscinX_wCenter.pdf}
			
			\vspace{2mm}
			\includegraphics[width=0.95\textwidth]{m1_stereoResiduals_nonNprecision_BF.pdf}
	\end{columns}
	
}

\frame{\frametitle{\large Stereo-Reconstruction Dependencies on Track-Inclination}

	\begin{columns}
		\column{.5\textwidth}
			\centering
			\includegraphics[width=0.6\textwidth]{m1_stereo_resXvsPhi_woAddTerm_rebin.pdf}
		\column{.5\textwidth}
			\centering
			\vspace{-0.3mm}
			\includegraphics[width=0.6\textwidth]{m1_stereo_resXvsTheta_woAddTerm_rebin.pdf}
	\end{columns}
	
	\vspace{2mm}
	
	\begin{columns}
		\column{.5\textwidth}
			\centering
			\includegraphics[width=0.65\textwidth]{stereoBoards_angleNshift_centerNcoord-crop.pdf}
			\vspace{2mm}
			
			\includegraphics[width=0.43\textwidth]{Kugelkoord-def.pdf}
		\column{.5\textwidth}
			\small
			\textcolor{blue}{precision}  = mean / $\cos \alpha$ 
			\vspace{2mm}
					
			\textcolor{red}{non-precision} = difference / 2 $/ \sin \alpha$
			\vspace{2mm}
			
			track-inclination-correction :
			
			$-\dfrac{1}{2 \tan\!\alpha} \cdot \bigtriangleup\!z \cdot \tan\!\Theta \cdot \sin\!\Phi$
	\end{columns}
	
}

\frame{\frametitle{Timing Resolution }
	\footnotesize
	\centering

	\begin{columns}
		\column{.5\textwidth}
			\centering
			M11, eta-in
			
			$U_{\mathrm{amp}} =$ \SI{570}{V} , $U_{\mathrm{drift}} =$ \SI{300}{V}
			
			\includegraphics[width=0.6\textwidth]{m11_ei_meanFirstTimeMap.pdf}
		\column{.5\textwidth}
			\centering
			M1, eta-in
			
			$U_{\mathrm{amp}} =$ \SI{570}{V} , $U_{\mathrm{drift}} =$ \SI{300}{V}
			
			\includegraphics[width=0.8\textwidth]{m1_ei_stripTime_firstNlast_20180601_turntime_wDifferences.pdf}
	\end{columns}
	
	\vspace{1mm}
	
	module 3 , board 7
	
	\begin{columns}
		\column{.5\textwidth}
			\centering
			single layer first-strip time-width (eta-in)
			
			\includegraphics[width=0.78\textwidth]{m3_ei_b7_driftScan_firstTimeWidths.pdf}
		\column{.5\textwidth}
			\centering
			eta-layers first-strips time-difference
			
			\includegraphics[width=0.78\textwidth]{m3_etas_b7_driftScan_firstTimeDifferenceWidths.pdf}
	\end{columns}
	
	$\Rightarrow$ resistivity effects not yet considered
	
}

\end{document}
