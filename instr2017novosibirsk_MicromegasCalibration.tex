\documentclass{beamer}
\usetheme{Madrid}
\usecolortheme{spruce}
\definecolor{green(pigment)}{rgb}{0.0, 0.65, 0.31}
\setbeamercolor*{item}{fg=green(pigment)}
\definecolor{Green}{rgb}{0.00, 1.00, 0.00}
\definecolor{Red}{rgb}{1.00, 0.00, 0.00}
\definecolor{Blue}{rgb}{0.00, 0.00, 1.00}
\usepackage[utf8]{inputenc}
\usepackage{amsmath}
\usepackage{amsfonts}
\usepackage{amssymb}
\usepackage[german]{babel}
\usepackage{graphicx}
\usepackage{rotating}
\usepackage{multirow,bigdelim,dcolumn,booktabs}
%\usepackage{beamerthemeshadow}
\usepackage{subfigure} 
\usepackage{siunitx}
%\beamersetuncovermixins{\opaqueness<1>{25}}{\opaqueness<2->{15}}
\beamertemplatenavigationsymbolsempty

\usepackage{tikz}
\usetikzlibrary{decorations.text}
\usetikzlibrary{trees}
\usetikzlibrary{decorations.pathmorphing}
\usetikzlibrary{decorations.markings}
\usetikzlibrary{patterns}

\graphicspath{
	%{pictures/}
	{/home/m/Maximilian.Herrmann/Bilder/forLatex/plots/}
	{/home/m/Maximilian.Herrmann/Bilder/forLatex/sketches/}
	{/home/m/Maximilian.Herrmann/Bilder/forLatex/pictures/}
}
\begin{document}
\title[Calibration of Micromegas]{Calibration of Precise Large Area Micromegas Detectors Using Cosmic Rays}  
\author[M. Herrmann]{Maximilian Herrmann}
\institute[LMU Munich]{Ludwig-Maximilians-Universit\"at M\"unchen - Lehrstuhl Schaile}
\date{02.03.2017, Novosibirsk} 

\frame{\titlepage
	\centering
	Instrumentation for Colliding Beam Physics
	
	\includegraphics[width=0.5\textwidth]{LMUlogo.jpg}
	\hspace{2cm}
	\includegraphics[width=0.3\textwidth]{BMBFlogo.png}
} 

\frame{\frametitle{Outline}
	\tableofcontents
}

\section{Micromegas Principle}

\frame{\frametitle{MICROMEsh GAseous Structure}
	\centering
	
	\begin{columns}
		\column{.6\textwidth}
			\centering
			\only<1>{\includegraphics[width=\textwidth]{RSMM_principle5-crop.pdf}}
			%\only<2>{\includegraphics[width=\textwidth]{RSMM_principle5_withtimes-crop.pdf}}
		\column{.4\textwidth}
			\centering
			\begin{itemize}
				{\small
				\item
					ionized electrons drift between cathode and grounded mesh 
				\item
					gas amplification between mesh and anode
				\item
					charge collection on resistive strips
				\item
					charge detection on readout strips
				\item
					positioning of strips with high accuracy mandatory
				}
			\end{itemize}
	\end{columns}
%	\vspace{0.5cm}
%	\centering
%	\begin{tabular}{lcl}
%		$\Rightarrow$ position reconstruction & : & centroid method
%		\\
%			& & charge average over cluster of strips
%		\\
%		$\Rightarrow$ track reconstruction & : & TPC-like analysis
%		\\
%			& & with $t_{\mathrm{drift}} = \mathrm{f}(\mathrm{strip})$
%	\end{tabular}
	\vspace{7mm}
	
	\textbf{calibration $\Rightarrow$ determine position of strips using cosmic muons}
}

\frame{\frametitle{Construction of a \SI{1}{\m\squared} Prototype Micromegas}
	\begin{columns}
		\column{.50\textwidth}
			\centering
			\includegraphics[width=0.6\textwidth]{L1_PCB_sketch-crop.pdf}
			\begin{itemize}
				\item
					two readout anode boards
					
					(due to photolithographic limitations)
					%
					%(maximal width $\sim$ \SI{50}{\cm})
				\item
					no alignment tooling used during gluing on Al plate
				\item 
					active area: 0.92 $\times$ \SI{1.02}{\m\squared}
			\end{itemize}
		\column{.50\textwidth}
			\centering
			\includegraphics[width=\textwidth]{mmconstruction2-crop.pdf}
			
			\begin{itemize}
				\item
					stiffening panels as support structure for anode and cathode
				\item
					mesh mounted on drift panel
				\item
					gas volume enclosed by anode and cathode
				\item
					potential deformation due to overpressure of Ar:CO$_{2}$
			\end{itemize}
	\end{columns}
}

%\subsection{Signal Properties}

\frame{\frametitle{Time Evolution of the Signal on a Single Strip}
	beginning of the signal : fit by an inverse Fermi function
	\vspace{5mm}
	\begin{columns}
		\column{.50\textwidth}
			\centering
			\includegraphics[width=\textwidth]{pulseheight_withExtrapolationLine_wPoint3.pdf}
		\column{.50\textwidth}
			{\small
			\centering
			\hspace{3mm} $ \mathrm{f}_{\mathrm{Fermi}} = \dfrac{p_{0}}{1 + \exp[(p_{1} - x)/p_{2}]} + p_{3}$
			
			
			%\hspace{3mm} $ \dfrac{p_{0}}{1 + e^{(p_{1} - x)/p_{2}}} + p_{3}$
			\begin{itemize}
				\item
					$p_{0}$ : maximal pulse height 
					
					\hspace{8mm} $\Rightarrow$ charge of signal
				\item
					$p_{1}$ : time of 50\% 
					
					\hspace{5.5mm} maximal pulse height
				\item
					$p_{2}$ : $\propto$ rise time
				\item
					$p_{3}$ : pedestal
			\end{itemize}
			}
	\end{columns}
	\vspace{5mm}
	\centering
	$\Rightarrow$ {\color{green}3 values} of $\mathrm{f}_{\mathrm{Fermi}}$ at 10\% , 50\% and 90\% define 
	
	{\color{teal}start time} of signal by {\color{blue}extrapolation}
}

\subsection{Position and Track Reconstruction}

\frame{\frametitle{Position and Track Reconstruction}
	\begin{columns}
			\column{.50\textwidth}
				\centering
				\includegraphics[width=1.1\textwidth]{eventdisplay.png}
				
				drift time measurement
				
				\includegraphics[width=0.9\textwidth]{timevsstrip2.pdf}
			\column{.50\textwidth}
				\small
				\begin{itemize}
					\item
						centroid method
						
						$\Rightarrow$ charge average over strips
						
						\vspace{3mm}$
							 {\bf \color{violet}x_{\bf \mathrm{centroid}}} 
								 = 
							\dfrac{
								\sum\limits_{\scriptscriptstyle\mathrm{strips}} x_{\scriptscriptstyle\mathrm{strip}} \cdot q_{\scriptscriptstyle\mathrm{strip}}
							}
							{\sum\limits_{\scriptscriptstyle\mathrm{strips}} q_{\scriptscriptstyle\mathrm{strip}} }
						$\vspace{4mm}
					\item
						TPC-like method
						
						angle reconstruction by drift time measurement
						
						${\color{blue}\alpha} = \arctan\left(\dfrac{\mathrm{pitch}}{{\color{red}\mathrm{slope}_{\mathrm{fit}}} \cdot v_{\mathrm{drift}}}\right)$
				\end{itemize}
	\end{columns}
}

\section{Cosmic Ray Facility}

\frame{\frametitle{Cosmic Ray Facility: Calibration}
	
	\centering
	\includegraphics[width=0.8\textwidth]{CRFprinciple-crop.pdf}
%	\begin{columns}
%		\column{.60\textwidth}
%			\centering
%			\includegraphics[width=\textwidth]{CRF_sketch2-crop.pdf}
%		\column{.40\textwidth}
%			\centering
%			\includegraphics[width=\textwidth]{CRFpictureLoesel.png}
%	\end{columns}
	
	
	\begin{itemize}
		\item
			2D track reconstruction with two Monitored Drift Tube (MDT) chambers
		\item
			trigger via Scintillator hodoscope with coarse resolution ($\approx$ \SI{10}{cm}) in orthogonal direction
	\end{itemize}
}

\frame{\frametitle{Cosmic Ray Facility}
	\centering
	facility to calibrate detectors in Garching near Munich
	\vspace{1mm}
	
	\includegraphics[width=0.8\textwidth]{CRF_small.JPG}
	
	\vspace{1mm}
	MDT chambers : \SI{2.2}{\m} $\times$ \SI{4}{\m} $\Rightarrow$ active area : \SI{9}{\m\squared}
	
	angular acceptance : $\pm30^{\circ}$
}

\subsection{Calibration and Potential Alignment \newline by use of $\mu$ reference tracks and \newline by partitioning of the detector area}

\frame{\frametitle{Alignment by Use of Reference Tracks}
	\centering \textbf{Idea:}
	\begin{columns}
		\column{.50\textwidth}
			\centering
			\includegraphics[width=0.90\textwidth]{positionshift2-crop.pdf}
		\column{.50\textwidth}
			\centering
			\includegraphics[width=0.90\textwidth]{verticalshift-crop.pdf}
	\end{columns}
	\begin{columns}
		\column{.55\textwidth}
			\centering
			
			%{\rotatebox{90}{\textbf{\tiny \hspace{3mm} residual = measured - reference position}}}
			\includegraphics[width=0.93\textwidth]{residualVSslope_newest.pdf}
		\column{.45\textwidth}
			\small
			\textbf{Implementation:}
			\begin{itemize}
				\item
					%residual = measured - reference position
					\textbf{residual} = $\mathrm{pos}_{\mathrm{measured}} - \mathrm{pos}_{\mathrm{reference}}$
				\item
					residual vs. slope 
					
					 of reference track 
					 
					$\Rightarrow$ {\color{red}linear fit}
				\item	
					$\mathrm{shift}_{\mathrm{horizontal}} = \mathrm{intercept}_{\mathrm{fit}}$
				\item						
					$\mathrm{shift}_{\mathrm{vertical}} \;\;\; = \mathrm{slope}_{\mathrm{fit}}$
			\end{itemize}
	\end{columns}
}

\frame{\frametitle{Partitioning of the Detector Area}
	\begin{columns}
		\column{.50\textwidth}
			\centering
			\includegraphics[width=\textwidth]{L1_partitions4-crop.pdf}
		\column{.50\textwidth}
			\small
			\begin{itemize}
				\item
					two readout boards 
					
					with in 2048 strips
				\item
					16 APV25 frontend boards 
					
					$\Rightarrow$ 16 segments in y direction
				\item
					\SI{10}{\cm} resolution of scintillator hodoscope
										
					$\Rightarrow$ 10 segments  in x direction
			\end{itemize}
			\centering
			{\large$\Rightarrow$ 160 partitions}
			
			\hspace{2.1mm} \`{a} 100 $\times$ \SI{57.6}{\mm\squared} 
	\end{columns}
	\vspace{12mm}
	$\Rightarrow$ calibration and alignment for each of the 160 partitions individually
}

\section{Calibration Results}
\section{Homogeneity and Efficiency of Large Area Micromegas}

\frame{\frametitle{Deformation of the Drift Region due to Overpressure}
	\begin{columns}
		\column{.50\textwidth}
			\centering
			\includegraphics[width=0.9\textwidth]{deltaZ__apv_scin_font-crop.pdf}
			%\includegraphics[width=\textwidth]{L1_deformation_znew_richtigKlein.pdf}
			
			\includegraphics[width=0.7\textwidth]{L1_simDif-crop.pdf}
		\column{.50\textwidth}
			\centering
			\includegraphics[width=\textwidth]{iglu_10mbar_new.pdf}
			
			\begin{itemize}
				\item
					drift gap deformation due to small overpressure
				\item
					maximum deviation of \SI{0.8}{\mm}
					from central plane
					
					$\Rightarrow$ \SI{1.6}{\mm} at cathode 
					
					(stiff base plate support)
				\item
					deformation in agreement with finite element simulation (ANSYS)
			\end{itemize}
			
	\end{columns}
}

\frame{\frametitle{Result of Calibration: \\ Shift and Rotation between Readout Boards}
	\begin{columns}
		\column{.50\textwidth}
			\centering
			\includegraphics[width=0.9\textwidth]{deltaY__apv_scin_font-crop.pdf}
			\vspace{1mm}
			
			\includegraphics[width=0.8\textwidth]{L1BoardTilt_sketch.png}
		\column{.50\textwidth}
			
			\begin{itemize}
				\item
					analysis of all 160 partitions individually
				\item
					alignment of the right half of the detector
					
					$\Rightarrow$ misalignment between PCBs 
					
					\hspace{4mm} becomes visible
				\item
					\textbf{shift : } \SI{0.1}{\mm}
				\item
					\textbf{rotation : } \SI{0.35}{\mm/\m}
				\item
					\SI{50}{\micro\m} effects are clearly observable
			\end{itemize}
			
	\end{columns}
}

\frame{\frametitle{Impact of Calibration on Spatial Resolution}
	\begin{columns}
		\column{.50\textwidth}
			\centering
			\includegraphics[width=0.9\textwidth]{doppelGausCut-crop_bunt.pdf}

			\includegraphics[width=0.9\textwidth]{residualVSangle_calibration_narrow_woOutliers.pdf}
		\column{.50\textwidth}
			\begin{itemize}
				\item
					centroid position reconstruction 
%					show residual distribution for each angle
				\item
					%fit with double Gaussian
					
					fit residual distribution for each angle with double Gaussian
				\item
					plot narrow Gaussian width as function of angle
			\end{itemize}
			$\Rightarrow$ calibration improves resolution
			
			\vspace{3mm}
			
			improvement @ $0^{\circ}$ :
			
			\centering
			$\approx$ \SI{100}{\micro\m}
	\end{columns}
}

\frame{\frametitle{Investigation of Multiple Scattering}
	\begin{columns}
		\column{.50\textwidth}
			\centering
			weighted width
			
			\includegraphics[width=0.9\textwidth]{residualVSangle_cutMDTslopeNewest.pdf}
			
			\includegraphics[width=0.9\textwidth]{doppelgaus_scat.pdf}
		\column{.50\textwidth}
			assumption: 
			
			multiple scattering of muons
			
			$\Rightarrow$ broadening of residual 
			
			\hspace{4mm} distributions
			
			\begin{itemize}
				\item
					centroid residual distribution
				\item
					fit with double Gaussian
					
					$\Rightarrow$ weighted sigma
				\item
					cut on slope difference of reference tracks decreases residual width by \SI{350}{\micro\m}
					\vspace{3mm}
			\end{itemize}
	\end{columns}
}

\frame{\frametitle{Homogeneity of Pulse Height and Efficiency}
	\begin{columns}
		\column{.50\textwidth}
			\centering
			\small MPV of charge distribution 
			
			[ADC counts]
			
			\includegraphics[width=0.9\textwidth]{MPVclusterChargeL1.pdf}
			\vspace{-2mm}
			
			efficiency [\%]
			
			\includegraphics[width=0.9\textwidth]{efficiencyL1.pdf}
		\column{.50\textwidth}
			\begin{itemize}
				\item
					charge average : 
					
					1440 $\pm$ 310 ADC channel
				\item
					efficiency average : 
					
					91.5 $\pm$ 0.9 \%
				\item
					no significant difference between readout boards
					\vspace{3mm}
				\item
					master-slave differences of APV25 frontend boards
					\vspace{3mm}
				\item
					homogeneous 3$\sigma$ efficiency (deviations due to border effects)
					
					91.5 \% limited by multiple scattering
			\end{itemize}
	\end{columns}
}

\section{Summary}

\frame{\frametitle{Summary}
	\begin{itemize}
		\item
			Cosmic Ray Facility
		\item
			position and track reconstruction
		\item
			investigation of \SI{1}{\m\squared} Micromegas detector
			\begin{itemize}
				\item
					offline calibration by partitioning of detector plane
				\item
					deformation due to overpressure (\SI{1.6}{\mm} @ \SI{10}{mbar})
				\item
					misalignment of the readout PCBs during assembly (100 - \SI{450}{\micro\m})
					
					(no alignment tool available)
				\item
					broadening of the residual distribution due to multiple scattering of muons
				\item
					homogeneous pulse height and high efficiency over large area
			\end{itemize}
		\item
			results of calibration:
			\begin{itemize}
				\item
					deviation of micro-strips detectable with sensitivity $<$ \SI{50}{\micro\m}
				\item
					deformations of the active volume perpendicular to the readout area are measurable with sensitivity $<$ \SI{100}{\micro\m}
			\end{itemize}
	\end{itemize}
}

\appendix

\frame[noframenumbering]{
	\Huge
	\centering
	Backup
}

\frame[noframenumbering]{\frametitle{APV25 Frontend Readout Chips}
	\begin{columns}
		\column{.50\textwidth}
			\centering
			\includegraphics[width=0.8\textwidth]{APV25_chip_bluebackground.png}
			
			\includegraphics[width=0.9\textwidth]{APV25_circuit_sketch.png}
		\column{.50\textwidth}
			\begin{itemize}
				\item
					charge integration over \SI{25}{\ns} 
				\item
					pairwise connection as master-slave to reduce output channels 
			\end{itemize}
	\end{columns}
}

\frame[noframenumbering]{\frametitle{Angle Reconstruction by Single Plane TPC Analysis}
	\begin{columns}
		\column{.50\textwidth}
			\centering
			\includegraphics[width=0.9\textwidth]{reconstructedVSrefAngle_wNwoQcor.pdf}
			
			\includegraphics[width=0.7\textwidth]{capacitiveCoupling2-crop.pdf}
		\column{.50\textwidth}
			\begin{itemize}
				\item
					angular reconstruction via TPC like method [ $t_{\mathrm{Drift}} = \mathrm{f}(\mathrm{strip})$ ]
				\item
					reference track angle by MDT chambers
				\item
					larger angles reconstructed due to capacitive coupling between \SI{1}{\m} long strips
				\item
					 $\Rightarrow$ correction improves angular resolution  
			\end{itemize}
	\end{columns}
}


\end{document}
