\documentclass[usenamees,dvipsnames]{beamer}
\usetheme{Madrid}
\usecolortheme{spruce}
\definecolor{green(pigment)}{rgb}{0.0, 0.65, 0.31}
\setbeamercolor*{item}{fg=green(pigment)}
\definecolor{Green}{rgb}{0.00, 1.00, 0.00}
\definecolor{Red}{rgb}{1.00, 0.00, 0.00}
\definecolor{Blue}{rgb}{0.00, 0.00, 1.00}
\usepackage[utf8]{inputenc}
\usepackage{amsmath}
\usepackage{amsfonts}
\usepackage{amssymb}
\usepackage[german]{babel}
\usepackage{graphicx}
\usepackage{rotating}
\usepackage{textcomp}
\usepackage{multirow,bigdelim,dcolumn,booktabs}
%\usepackage{beamerthemeshadow}
\usepackage{subfigure} 
\usepackage{siunitx}
\usepackage{appendixnumberbeamer}
\usepackage{hyperref}
\usepackage{mdframed}
\usepackage{xcolor}
\usepackage[absolute,overlay]{textpos}
%\beamersetuncovermixins{\opaqueness<1>{25}}{\opaqueness<2->{15}}
\beamertemplatenavigationsymbolsempty

\usepackage{tikz}
\usetikzlibrary{decorations.text}
%\usetikzlibrary{trees}
\usetikzlibrary{decorations.pathmorphing}
\usetikzlibrary{decorations.markings}
\usetikzlibrary{patterns}

\graphicspath{
	{pictures/}
%	{/home/m/Maximilian.Herrmann/Bilder/forLatex/plots/}
%	{/home/m/Maximilian.Herrmann/Bilder/forLatex/sketches/}
%	{/home/m/Maximilian.Herrmann/Bilder/forLatex/pictures/}
}

\begin{document}
\title[SM2 in CRF]{SM2 Micromegas Modules in the Cosmic Ray Facility}  
\author[M. Herrmann]{Maximilian Herrmann}
\institute[LMU Munich]{Ludwig-Maximilians-Universit\"at M\"unchen - Lehrstuhl Schaile}
\date[04.09.2019]{NSW Software and Detector Integration 04.09.2019} 

\frame{
	\titlepage
	
	\centering
	\includegraphics[width=0.5\textwidth]{hardwareGroupWithM5.png}
} 

%\frame{\frametitle{Outline}
%	\tableofcontents
%}

\frame{\frametitle{SM2 Micromegas Modules for the NSW}
	\begin{columns}
		\column{.5\textwidth}
			\centering
			\includegraphics[width=0.78\textwidth]{sectorsNsides_SM2marked.png}
		\column{.5\textwidth}
			\centering
			\includegraphics[width=0.9\textwidth]{ROboardsAligned2-crop.png}
	\end{columns}
	
	\vspace{3mm}
	\begin{columns}
		\column{.5\textwidth}
			\centering
			\includegraphics[width=\textwidth]{moduleLayers.jpg}
		\column{.5\textwidth}
			\centering
			\includegraphics[width=\textwidth]{quadrupletStripOrientation3-crop.pdf}
	\end{columns}
} 

\section{Cosmic Ray Facility}

\frame{\frametitle{Cosmic Ray Facility}
		
	\begin{columns}
		\column{.50\textwidth}
			\centering
			\includegraphics[width=1.0\textwidth]{CRFprinciple7-crop.pdf}
		\column{.50\textwidth}
			\centering
			\includegraphics[width=0.9\textwidth]{CRF_small.JPG}
	\end{columns}
	
	\vspace{2mm}
	\centering
	\footnotesize
	\begin{tabular}{ll}
		trigger & 2 $\times$ scintillator hodoscopes
		\\
		track reconstruction & 2 $\times$ Monitored Drift Tube chambers (MDTs)
		\\
		active area & \SI{2}{m} $\times$ \SI{4}{m}
		\\
		angular acceptance & $\pm$ \SI{30}{\degree} to zenith
		\\
		energy cut (hardware) & iron absorber (\SI{34}{cm}) $\to E_{\mu} >$ \SI{600}{MeV} 
		\\
		readout & 12288 channels
		\\
		 & $\to$ 96 APVs (frontend electronics)
		\\
		 & $\to$ 6 FECs (scalable readout system)
		\\
		readout rate & 100 Hz (muon rate)
	\end{tabular}
}

\frame{\frametitle{\large Charge and Time Resolved Readout using APV25 Electronics}
	\scriptsize
	
	\begin{columns}
		\column{.32\textwidth}
			\centering
			96 $\times$ APV25-chips
			
			\includegraphics[width=0.95\textwidth]{APV25.jpg}
		\column{.32\textwidth}
			\centering
			6 $\times$ FEC-cards
			
			\includegraphics[width=0.82\textwidth]{FECcabling.jpg}
		\column{.32\textwidth}
			\centering
			strips parallel to MDT-wires
			
			\includegraphics[width=0.95\textwidth]{moduleInCRF.jpg}
	\end{columns}
		
	\vspace{3mm}
	\begin{columns}
		\column{.32\textwidth}
			128 channel 
			
			charge sensitive preamplifier
			
			\SI{40}{MHz} signal sampling
		\column{.32\textwidth}
			jitter recorded individually 
			
			$\Rightarrow$ unbiased time-evaluation
		\column{.32\textwidth}
			reference track
			
			$\Rightarrow$ efficiency and resolution
	\end{columns}
	
	\vspace{3mm}
	\begin{columns}
		\column{.32\textwidth}
			\centering
			\includegraphics[width=\textwidth]{eventdisplay_m5_570V.pdf}
		\column{.32\textwidth}
			\centering
			\includegraphics[width=\textwidth]{FECjitter.pdf}
		\column{.32\textwidth}
			\centering
			\includegraphics[width=\textwidth]{trackFitFourLayers.png}
	\end{columns}
} 

\frame{\frametitle{Measurement Overview}

	\begin{columns}
		\column{.5\textwidth}
			\centering
			\includegraphics[width=\textwidth]{CRFintegratedCountsActiveArea.pdf}
			
			\includegraphics[width=\textwidth]{temperatureNpressure_Garching2018to19_edited.png}
		\column{.5\textwidth}
			\small
					
			\begin{itemize}
				\small
				\item
					central part of active area
					
					$\Rightarrow$ \SI[product-units = repeat]{96 x 78}{cm}
					
					overall more than $\tfrac{1}{2}$ billion trigger
				\item
					counts integrated over all amplification voltages
					
					$\Rightarrow$ also low gain measurements
				\item
					CRF average {\color{blue}pressure} :  \SI{960}{\hecto\pascal} 
					
					ATLAS-cavern : \SI{980}{\hecto\pascal} 
					
					$\Rightarrow$ different gain
				\item
					CRF {\color{red}temperature} : \SI{21}{\degreeCelsius} 
					
					controlled to about $\pm$\SI{2}{\degreeCelsius}
				\item
					modules measured:
					
%					0, 1, 3, 6, 7, 5, 8, 12, 11, 9, 16, 15, 14, 13, 12
					0, 1, 3, 5, 6, 7, 8, 9, 11, 12, 13, 14, 15, 16
			\end{itemize}
	\end{columns}
	
}

\frame{\frametitle{Efficiency Turn On Curves (Module 1)}
	\scriptsize

	\begin{columns}
		\column{.5\textwidth}
			\centering
			H8 testbeam 2018
			
			\SI[product-units = repeat]{9 x 9}{cm}
		\column{.5\textwidth}
			\centering
			CRF
			
			\SI[product-units = repeat]{32 x 18.6}{\cm}
	\end{columns}
	\vspace{2mm}
	
	\begin{columns}
		\column{.5\textwidth}
			\centering
			\includegraphics[width=0.8\textwidth]{m1_H8_2018_board7efficiency.png}
		\column{.5\textwidth}
			\centering
			\includegraphics[width=0.8\textwidth]{m1_board7right_coinEffi.pdf}
	\end{columns}
	\vspace{2mm}

	\begin{columns}
		\column{.5\textwidth}
			\centering
			\includegraphics[width=0.8\textwidth]{m1_H8_2018_board8efficiency.png}
		\column{.5\textwidth}
			\centering
			\includegraphics[width=0.8\textwidth]{m1_board8left_coinEffi.pdf}
	\end{columns}
	
}

\frame{\frametitle{\small Pulse Height and Efficiency Maps (Module 1, eta-layers @ $U_{\mathrm{amp}} = $ \SI{580}{V})}

	\begin{columns}
		\column{.5\textwidth}
			\centering
			pulse height map
		\column{.5\textwidth}
			\centering
			efficiency map
	\end{columns}
	\vspace{2mm}
	
	\begin{columns}
		\column{.5\textwidth}
			\centering
			\includegraphics[width=0.8\textwidth]{MMS200001L1gain.png}
		\column{.5\textwidth}
			\centering
			\includegraphics[width=0.8\textwidth]{MMS200001L1efficiency.png}
	\end{columns}
	\vspace{2mm}

	\begin{columns}
		\column{.5\textwidth}
			\centering
			\includegraphics[width=0.8\textwidth]{MMS200001L2gain.png}
		\column{.5\textwidth}
			\centering
			\includegraphics[width=0.8\textwidth]{MMS200001L2efficiency.png}
	\end{columns}
	
}

\frame{\frametitle{Efficiency Turn On Curve (Module 3, eta-in)}

	\begin{columns}
		\column{.65\textwidth}
			\centering
			\includegraphics[width=\textwidth]{m3_ei_ampScan_woLayer.png}
		\column{.35\textwidth}
			\centering
			\includegraphics[width=0.65\textwidth]{RSMM_principle_bare-crop.pdf}
			\vspace{3mm}
			
			\includegraphics[width=0.65\textwidth]{ROboardsHVsectorNamesRectangle.pdf}
	\end{columns}
	\vspace{5mm}
	
	central parts of active area for single High-Voltage Sectors
	
	$\Rightarrow$ \SI[product-units = repeat]{32 x 18.6}{\cm}
	
}

\frame{\frametitle{\normalsize Efficiency spoiled by Noise of Preliminary Electronics (Module 7)}

	\only<4>{
		\begin{textblock*}{5cm}(1.75cm,3.6cm) 
			\begin{tikzpicture}
				\draw[red,->, line width=1mm] (0,0) -- (0,3.6);
			\end{tikzpicture}
		\end{textblock*}
		\begin{textblock*}{5cm}(2.85cm,2.9cm) 
			\begin{tikzpicture}
				\draw[red,->, line width=1mm] (0,0) -- (0,4.6);
			\end{tikzpicture}
		\end{textblock*}
		\begin{textblock*}{5cm}(3.4cm,2.7cm) 
			\begin{tikzpicture}
				\draw[red,->, line width=1mm] (0,0) -- (0,4.9);
			\end{tikzpicture}
		\end{textblock*}
		\begin{textblock*}{5cm}(4.3cm,3.6cm) 
			\begin{tikzpicture}
				\draw[red,->, line width=1mm] (0,0) -- (0,3.5);
			\end{tikzpicture}
		\end{textblock*}
	}
	
	\begin{columns}
		\column{.5\textwidth}
			\centering
			\only<1>{efficiency map - stereo-in}
			\only<2>{efficiency map - stereo-out}
			\only<1,2>{\vspace{-4mm}}
			\only<3->{efficiency per readout chip}
			
			\includegraphics<1>[angle=-90,width=0.7\textwidth]{m7_si_570V_5mmEffi.pdf}
			\includegraphics<2>[angle=-90,width=0.7\textwidth]{m7_so_570V_5mmEffi.pdf}
			\includegraphics<3->[width=0.85\textwidth]{m7_so_570V_5mmEffi_perAPV.pdf}
			
			\vspace{2mm}
						
			electronic noise per readout chip
			
			\includegraphics<1>[width=0.7\textwidth]{m7_si_sigmas_perAPV.pdf}
			\includegraphics<2>[width=0.7\textwidth]{m7_so_sigmas_perAPV.pdf}
			\includegraphics<3->[width=0.85\textwidth]{m7_so_sigmas_perAPV.pdf}
		\column{.5\textwidth}
			\centering
			\includegraphics[angle=-90,width=0.5\textwidth]{ROboardWithAPV-crop.pdf}
				
			\vspace{4mm}
				
			efficiency VS electronic noise
			
			\includegraphics[width=0.9\textwidth]{m7_5mmEffiVSsigmaPerAPV_wFit.png}
	\end{columns}
	
}

\frame{\frametitle{Efficiency influenced by Pulse Height}

	\begin{columns}
		\column{.5\textwidth}
			\centering
			pulse height map -
			\only<1>{m3, eta out}
			\only<2>{m12, eta in}
			
			\includegraphics<1>[width=0.6\textwidth]{m3_eo_clusterQmpv_20180911.pdf}
			\includegraphics<2>[width=0.6\textwidth]{m12_ei_570V_MPVclusterQmap.pdf}
		\column{.5\textwidth}
			\centering
			efficiency map -
			\only<1>{m3, eta out}
			\only<2>{m12, eta in}
			
			\includegraphics<1>[width=0.6\textwidth]{m3_eo_coinEffi_20180911.pdf}
			\includegraphics<2>[width=0.6\textwidth]{m12_ei_570V_coincidenceEfficiency.pdf}
	\end{columns}
	
	\vspace{2mm}
	
	\begin{columns}
		\column{.5\textwidth}
			\centering
			mean pulse height VS pillar height
			
			\includegraphics[width=\textwidth]{meanClusterQvsPillarHeight_etaNstereo_wFit.png}
		\column{.5\textwidth}
			\centering
			\includegraphics[width=0.6\textwidth]{RSMM_principle_bare-crop.pdf}
	\end{columns}
	
}

\frame{\frametitle{Cluster Charge Evaluation - M8 eta-out (=L1) board 8}
	\footnotesize

	\only<2>{
		\begin{textblock*}{5cm}(8.7cm,4.47cm) 
			\begin{tikzpicture}
				\draw[green,-, line width=0.4mm] (0,1) -- (2.7,1);
			\end{tikzpicture}
		\end{textblock*}
		\begin{textblock*}{5cm}(8.7cm,3.24cm) 
			\begin{tikzpicture}
				\draw[green,-, line width=0.4mm] (0,1) -- (2.9,1);
			\end{tikzpicture}
		\end{textblock*}
		\begin{textblock*}{5cm}(8.7cm,3.24cm) 
			\begin{tikzpicture}
				\draw[orange,-, line width=0.4mm] (0,0) -- (0,2.3);
			\end{tikzpicture}
		\end{textblock*}
		\begin{textblock*}{5cm}(10.45cm,3.24cm) 
			\begin{tikzpicture}
				\draw[orange,-, line width=0.4mm] (0,0) -- (0,2.3);
			\end{tikzpicture}
		\end{textblock*}
		\begin{textblock*}{5cm}(10.28cm,4.47cm) 
			\begin{tikzpicture}
				\draw[orange,-, line width=0.4mm] (0,0) -- (0,1.1);
			\end{tikzpicture}
		\end{textblock*}
		\begin{textblock*}{5cm}(11.60cm,3.24cm) 
			\begin{tikzpicture}
				\draw[orange,-, line width=0.4mm] (0,0) -- (0,2.3);
			\end{tikzpicture}
		\end{textblock*}
		\begin{textblock*}{5cm}(11.4cm,4.47cm) 
			\begin{tikzpicture}
				\draw[orange,-, line width=0.4mm] (0,0) -- (0,1.1);
			\end{tikzpicture}
		\end{textblock*}
	}
		
	\begin{columns}
		\column{.45\textwidth}
			\centering
			Ar:CO$_{2}$ 93:7 vol\% 
			
			$U_{\mathrm{amp}}$ = \SI{570}{V} , $U_{\mathrm{drift}}$ = \SI{300}{V}
			
			\includegraphics[width=0.85\textwidth]{m8_eo_b8_A570V_C300V_20190528_1223_clusterQ_perStrip.png}
		\column{.55\textwidth}
			\centering
			\textbf{93:7} , \hspace{8mm} {\color{red}\textbf{85:15}} , \hspace{3mm} {\color{blue}\textbf{80:20}}
			
			\includegraphics[width=\textwidth]{m8_eo_b8_gasNampScan_clusterQposition.pdf}
	\end{columns}
	
	\vspace{4mm}
	
	$\Rightarrow$ for higher fractions of CO$_{2}$, larger gains ($\sim$ efficiency) can be reached
		
	\vspace{2mm}
	
	dependence on board quality (e.g. pillar height)?
	
	$\Rightarrow$ compare \only<1>{voltages} \only<2>{{\color{orange} voltages}} 
	at \only<1>{same pulse height} \only<2>{{\color{green} same pulse height}}
			
	\vspace{1mm}
	
	\centering
	
	\begin{tabular}{cccc}
		\hline
		\hline
		[V] & 93:7 & {\color{red}85:15} & {\color{blue}80:20}
		\\
		\hline
		mean & 570 & 615 & 645
		\\
		MPV & 570 & 610 & 640
		\\
		\hline
		\hline
	\end{tabular}
	
}

\frame{\frametitle{Pulse Height Dependence on Pillar Height (M8 , eta3)}
		
	\begin{columns}
		\column{.32\textwidth}
			\centering
			Ar:CO$_{2}$ \textbf{93:7} vol\% 
			
			$U_{\mathrm{amp}}$ = \SI{570}{V}
			
			$U_{\mathrm{drift}}$ = \SI{300}{V}
		\column{.32\textwidth}
			\centering
			Ar:CO$_{2}$ {\color{red}\textbf{85:15}} vol\% 
			
			$U_{\mathrm{amp}}$ = \SI{615}{V}
			
			$U_{\mathrm{drift}}$ = \SI{300}{V}
		\column{.32\textwidth}
			\centering
			Ar:CO$_{2}$ {\color{blue}\textbf{80:20}} vol\% 
			
			$U_{\mathrm{amp}}$ = \SI{645}{V}
			
			$U_{\mathrm{drift}}$ = \SI{475}{V}
	\end{columns}
	
	\vspace{3mm}
			
	\begin{columns}
		\column{.32\textwidth}
			\centering
			\includegraphics[width=\textwidth]{m8_9307_A570V_C300V_clusterQvsPillarHeight.pdf}
		\column{.32\textwidth}
			\centering
			\includegraphics[width=\textwidth]{m8_8515_A615V_C300V_clusterQvsPillarHeight.pdf}
		\column{.32\textwidth}
			\centering
			\includegraphics[width=\textwidth]{m8_8020_A645V_C475V_clusterQvsPillarHeight.pdf}
	\end{columns}
			
	\vspace{3mm}
	\centering
	\begin{tabular}{cccc}
		\hline
		\hline
		Ar:CO$_{2}$ vol\% & \textbf{93:7} & {\color{red}\textbf{85:15}} & {\color{blue}\textbf{80:20}}
		\\
		\hline
		mean \%/\textmu m & -(6.0$\pm$0.6) & -(5.7$\pm$0.6) & -(6.7$\pm$0.8)
		\\
		MPV \%/\textmu m & -(9.8$\pm$1.0) & -(10.4$\pm$1.1) & -(10.7$\pm$1.1)
		\\
		\hline
		\hline
	\end{tabular}
	
}

\frame{\frametitle{\large Cluster Properties (M8, eta-out, board 8)}
		
	\begin{columns}
		\column{.32\textwidth}
			\centering
			Ar:CO$_{2}$ \textbf{93:7} vol\% 
			
			$U_{\mathrm{drift}}$ = \SI{300}{V}
		\column{.32\textwidth}
			\centering
			Ar:CO$_{2}$ {\color{red}\textbf{85:15}} vol\% 
			
			$U_{\mathrm{drift}}$ = \SI{300}{V}
		\column{.32\textwidth}
			\centering
			Ar:CO$_{2}$ {\color{blue}\textbf{80:20}} vol\% 
			
			$U_{\mathrm{drift}}$ = \SI{475}{V}
	\end{columns}
	
	\vspace{4mm}

	\begin{columns}
		\column{.32\textwidth}
			\centering
			\includegraphics[width=0.95\textwidth]{m8_eo_b8_meanNstripsVSslope_woCCC_ampScan9307_C300V.pdf}
		\column{.32\textwidth}
			\centering
			\includegraphics[width=0.95\textwidth]{m8_eo_b8_meanNstripsVSslope_woCCC_ampScan8515_C300V.pdf}
		\column{.32\textwidth}
			\centering
			\includegraphics[width=0.95\textwidth]{m8_eo_b8_meanNstripsVSslope_woCCC_ampScan8020_C475V.pdf}
	\end{columns}
	
	\vspace{4mm}
	
	\begin{columns}
		\column{.32\textwidth}
			\centering
			\includegraphics[width=0.95\textwidth]{m8_eo_b8_9307_C300V_ampScan_clusterQ_logy_scale.pdf}
		\column{.32\textwidth}
			\centering
			\includegraphics[width=0.95\textwidth]{m8_eo_b8_8515_C300V_ampScan_clusterQ_logy_scale.pdf}
		\column{.32\textwidth}
			\centering
			\includegraphics[width=0.95\textwidth]{m8_eo_b8_8020_C475V_ampScan_clusterQ_logy_scale.pdf}
	\end{columns}
	
}

\frame{\frametitle{\normalsize Efficiency Dependence on Cluster Charge (M8, eta-out = L1)}
	\scriptsize
	\centering
	amplification scan (HV sector L7)
		
	\begin{columns}
		\column{.5\textwidth}
			\centering
			MPV cluster charge
			
			\includegraphics[width=0.9\textwidth]{m8_L1L7_ampScanMPV_gasScan.pdf}
		\column{.5\textwidth}
			\centering
			coincidence efficiency
						
			\includegraphics[width=0.9\textwidth]{m8_L1L7_coincidenceTurnOnCurve_gasScan.pdf}
	\end{columns}
	
	\vspace{2mm}
			
	\begin{columns}
		\column{.32\textwidth}
			\centering
			Ar:CO$_{2}$ \textbf{93:7} vol\% 
			
			$U_{\mathrm{amp}}$ = \SI{570}{V}
			
			$U_{\mathrm{drift}}$ = \SI{300}{V}
		\column{.32\textwidth}
			\centering
			Ar:CO$_{2}$ {\color{red}\textbf{85:15}} vol\% 
			
			$U_{\mathrm{amp}}$ = \SI{610}{V}
			
			$U_{\mathrm{drift}}$ = \SI{300}{V}
		\column{.32\textwidth}
			\centering
			Ar:CO$_{2}$ {\color{blue}\textbf{80:20}} vol\% 
			
			$U_{\mathrm{amp}}$ = \SI{640}{V}
			
			$U_{\mathrm{drift}}$ = \SI{475}{V}
	\end{columns}
					
	\begin{columns}
		\column{.32\textwidth}
			\centering
			\includegraphics<1>[width=\textwidth]{m8_eo_9307_A570V_C300V_gainMap.pdf}
			\includegraphics<2->[width=\textwidth]{m8_eo_9307_A570V_C300V_5mmEfficiencyMap.pdf}
		\column{.32\textwidth}
			\centering
			\includegraphics<1>[width=\textwidth]{m8_eo_8515_A610V_C300V_gainMap.pdf}
			\includegraphics<2->[width=\textwidth]{m8_eo_8515_A610V_C300V_5mmEfficiencyMap.pdf}
		\column{.32\textwidth}
			\centering
			\includegraphics<1>[width=\textwidth]{m8_eo_8020_A640V_C475V_gainMap.pdf}
			\includegraphics<2->[width=\textwidth]{m8_eo_8020_A640V_C475V_5mmEfficiencyMap.pdf}
	\end{columns}
	
}

\frame{\frametitle{Efficiency Dependence on Pulse Height and Noise}
	\scriptsize
	\centering
				
	\begin{columns}
		\column{.5\textwidth}
			\centering
			Gas Study with Module 8
			
			\includegraphics[width=0.9\textwidth]{m8_coincidenceEfficiencyVSmpvClusterQ_gasScan.pdf}
		\column{.5\textwidth}
			\centering
			Noise Study with Module 7
						
			\includegraphics[width=0.9\textwidth]{m7_5mmEffiVSnormedClusterQperAPV.pdf}
	\end{columns}
					
	\begin{columns}
		\column{.5\textwidth}
			\centering						
			\begin{itemize}
				\scriptsize
				\item
					all gas-mixtures show a similar behavior
				\item
					for 90\%-efficiency $\Rightarrow$
				
					MPV of pulse-height has to exceed about 40 times the noise-level
				\item
					vice versa:
					
					noise should be kept below a 2\%-level of pulse-height to reach 90\%-efficiency
			\end{itemize}
		\column{.5\textwidth}
			\centering
			\includegraphics[width=0.9\textwidth]{m7_5mmEffiVSnormedSigmaPerAPV.pdf}
	\end{columns}
		
}

\subsection{Strip Shape} 

\frame{\frametitle{Tracking in the Cosmic Ray Facility}
		
	\begin{columns}
		\column{.50\textwidth}
			\centering
			\includegraphics[width=1.0\textwidth]{CRFprinciple5-crop.pdf}
		\column{.50\textwidth}
			\centering
			\includegraphics[width=0.9\textwidth]{CRF_small.JPG}
	\end{columns}
	
	\vspace{7mm}
	\centering
	\begin{tabular}{ccccc}
		residual & = & {\color{blue}measured} & - & {\color{red}reference}
		\\
		 & & & & 
		\\
		 & = & centroid $\times$ pitch & - & track$_{\mathrm{MDTs}}$ @ MM
	\end{tabular}
}

\frame{\frametitle{\normalsize Reconstruction of Readout Board Alignment (Module 1, eta-in)}

	\begin{columns}
		\column{.6\textwidth}
			\centering
			\begin{columns}
				\column{.5\textwidth}
					\centering
					\includegraphics[width=0.7\textwidth]{m1_ei_residaul_nearZero.pdf}
				\column{.5\textwidth}
					\centering
					\hspace{-20mm} $\Rightarrow$ mean residual
			\end{columns}
			\vspace{2mm}
			\includegraphics[width=0.8\textwidth]{m3_eo_deltaYperPart_2019-09.png}
		\column{.4\textwidth}
%			\centering
			\includegraphics[width=0.9\textwidth]{m3_eo_resMeanVSscinX_allBoards.png}
			\footnotesize
			
			\vspace{2mm}
			undesired strip shape
			
			$\Rightarrow$ calibration needed
			\vspace{2mm}
			
			\includegraphics[width=0.9\textwidth]{ROboardsBendStrips-crop.png}
			\vspace{2mm}
			
			humidity causes 
			
			deviation from design:	
			
			$\Rightarrow$ known issue
	\end{columns}
	
}

\frame{\frametitle{Alignment Reconstruction Comparison (eta-panel 9)}
%	\scriptsize
	\centering
				
	\begin{columns}
		\column{.5\textwidth}
			\centering
			Rasfork measurement
		\column{.5\textwidth}
			\centering
			mean hit-position-difference between eta-layers
	\end{columns}
					
	\begin{columns}
		\column{.5\textwidth}
			\centering
			\includegraphics[width=0.75\textwidth]{RS2E00009_GluingSide2_plot.pdf}
		\column{.5\textwidth}
			\centering
			\includegraphics[width=0.95\textwidth]{m9_etaDifZcor.pdf}
	\end{columns}
		
	$\Rightarrow$ rotation between layers reconstructable in both measurements
}

\section{Track Reconstruction with Micromegas}

\frame{\frametitle{\normalsize Position Reconstruction using Charge and Drift-Time Measurements (M1)}
	
	\begin{columns}
		\column{.5\textwidth}
			\centering
			residual normal tracks
			
			\includegraphics[width=0.8\textwidth]{m1_ei_residaul_nearZero.pdf}
		\column{.5\textwidth}
			\centering
			angle reconstruction
			
			\includegraphics[width=0.65\textwidth]{m1_ei_uTPCangleVSangle_20180601_uptime_sameRange.pdf}
	\end{columns}
	
	\vspace{1mm}
		
	\begin{columns}
		\column{.5\textwidth}
			\centering
			simulation
			
			\includegraphics[width=0.7\textwidth]{inhomogeneousIonization_bE_cNm_timeNcentroidShift.pdf}
		\column{.5\textwidth}
			\centering
			resolution VS angle
			
			\includegraphics[width=0.8\textwidth]{m1_ei_allResolutionsVSangle_20180601.pdf}
	\end{columns}
	
}

\subsection{Stereo Reconstruction}

%\frame{\centering \Huge Stereo Reconstruction}

\frame{\frametitle{Position Reconstruction using two Layers of Inclined Strips}
			
	\begin{columns}
		\column{.5\textwidth}
			\includegraphics[width=0.9\textwidth]{stereoBoards_angleNshift_centerNcoord-crop.pdf}
			\vspace{2mm}
			
			\textcolor{blue}{precision} 
			
			\hspace{1mm} = stereos mean / $\cos \alpha$ 
			\vspace{1.5mm}
						
			\textcolor{red}{non-precision}
			
			\hspace{1mm} = stereos difference / 2 $/ \sin \alpha$
			\vspace{1.5mm}
						
			$\Rightarrow$ \textcolor{Brown}{alignment} 
			
			\hspace{4.5mm} non-precision coordinate
			
			\vspace{5mm}
			$\Rightarrow$ non-precision residual spoiled 
			
			\hspace{4.5mm} by coarse scintillator resolution
		\column{.5\textwidth}
			\centering	
			\includegraphics[width=0.65\textwidth]{m1_stereo_posDifVSscinX_wCenter.pdf}
			
			\vspace{2mm}
			\includegraphics[width=0.95\textwidth]{m1_stereoResiduals_nonNprecision_BF.pdf}
	\end{columns}
	
}

\frame{\frametitle{\large Stereo-Reconstruction Dependencies on Track-Inclination}

	\begin{columns}
		\column{.5\textwidth}
			\centering
			\includegraphics[width=0.6\textwidth]{m1_stereo_resXvsPhi_woAddTerm_rebin.pdf}
		\column{.5\textwidth}
			\centering
			\vspace{-0.3mm}
			\includegraphics[width=0.6\textwidth]{m1_stereo_resXvsTheta_woAddTerm_rebin.pdf}
	\end{columns}
	
	\vspace{2mm}
	
	\begin{columns}
		\column{.5\textwidth}
			\centering
			\includegraphics[width=0.65\textwidth]{stereoBoards_angleNshift_centerNcoord-crop.pdf}
			\vspace{2mm}
			
			\includegraphics[width=0.43\textwidth]{Kugelkoord-def.pdf}
		\column{.5\textwidth}
			\small
			\textcolor{blue}{precision}  = mean / $\cos \alpha$ 
			\vspace{2mm}
					
			\textcolor{red}{non-precision} = difference / 2 $/ \sin \alpha$
			\vspace{2mm}
			
			track-inclination-correction :
			
			$-\dfrac{1}{2 \tan\!\alpha} \cdot \bigtriangleup\!z \cdot \tan\!\Theta \cdot \sin\!\Phi$
	\end{columns}
	
}

\frame{\frametitle{Timing Resolution }
	\footnotesize
	\centering

	\begin{columns}
		\column{.5\textwidth}
			\centering
			M11, eta-in
			
			$U_{\mathrm{amp}} =$ \SI{570}{V} , $U_{\mathrm{drift}} =$ \SI{300}{V}
			
			\includegraphics[width=0.6\textwidth]{m11_ei_meanFirstTimeMap.pdf}
		\column{.5\textwidth}
			\centering
			M1, eta-in
			
			$U_{\mathrm{amp}} =$ \SI{570}{V} , $U_{\mathrm{drift}} =$ \SI{300}{V}
			
			\includegraphics[width=0.8\textwidth]{m1_ei_stripTime_firstNlast_20180601_turntime_wDifferences.pdf}
	\end{columns}
	
	\vspace{1mm}
	
	module 3 , board 7
	
	\begin{columns}
		\column{.5\textwidth}
			\centering
			single layer first-strip time-width (eta-in)
			
			\includegraphics[width=0.78\textwidth]{m3_ei_b7_driftScan_firstTimeWidths.pdf}
		\column{.5\textwidth}
			\centering
			eta-layers first-strips time-difference
			
			\includegraphics[width=0.78\textwidth]{m3_etas_b7_driftScan_firstTimeDifferenceWidths.pdf}
	\end{columns}
	
	$\Rightarrow$ resistivity effects not yet considered
	
}

\frame{\frametitle{Summary}
	\small
	\begin{itemize}
		\item
			about 13 SM2-Micromegas investigated in the Cosmic Ray Facility
		\item
			efficiency influenced by
			
			\begin{itemize}
				\small
				\item
					noise behavior of preliminary APV25-electronics 
					
					local setup : 30 - 40 ADC channel
				\item
					pulse-height coupled to amplification-gap-height
					
					MPV : -10\%/\si{\micro\m}
			\end{itemize}
			
			$\Rightarrow$ MPV of pulse-height has to exceed 
			
			\hspace{4mm} 40 times noise-level to reach 
			
			\hspace{4mm} 90\%-efficiency
		\item
			reference-tracking enables
						
			\begin{itemize}
				\small
				\item
					alignment-reconstruction
					
					$\Rightarrow$ calibration of deviations
				\item
					resolution estimation
					
					$\Rightarrow$ centroid similar results as testbeam
				\item
					position-reconstruction using inclined strip-layers works as intended
			\end{itemize}
	\end{itemize}
}

\appendix

\frame{\centering \Huge Backup}

\frame{\frametitle{\large Reconstruction of Pitch Deviations and Readout Board Alignment}

	\begin{columns}
		\column{.5\textwidth}
			\centering
			without correction
			
			\includegraphics[width=0.8\textwidth]{eta_in_resMeanVSmdtY_woAdapterBoard_wCorPitch_wCenter.png}
		\column{.5\textwidth}
			\centering
			pitch deviation considered
			
			\vspace{-0.3mm}
			\includegraphics[width=0.8\textwidth]{eta_in_resMeanVSmdtY_wNewPitchCor_shifts.pdf}
	\end{columns}
	
	\vspace{4mm}
	
	\begin{columns}
		\column{.5\textwidth}
			\centering
			\includegraphics[angle=-90,width=0.55\textwidth]{ROboardsAligned1_wRingsNlines-crop.pdf}
			\vspace{3mm}
		\column{.5\textwidth}
			\centering
			\includegraphics[width=0.8\textwidth]{boardPitchDeviation_wLinesNdesNarrows-crop.pdf}
	\end{columns}
	
}

\frame{\frametitle{\large Gain Dependencies}

	\begin{align*}
		G &= \exp\left( \alpha \cdot d \right)
		\\[10pt]
		&= \exp\left[ \;\; \dfrac{A p d}{T} \; \cdot \; \exp\!\left(\; -\dfrac{B p d}{T U} \; \right) \;\; \right].
		\label{gainParametrization}
	\end{align*}
	
	Taylor-series-expansion up to first order:
	
	\begin{align*}
		G(d) &= 
				\exp\left( a \cdot d_0 \cdot e^{- b \cdot d_0} \right) 
				\nonumber
				\\
				&\qquad + a \cdot ( 1 - b \cdot d_0 ) \cdot \exp\left( - b \cdot d_0 + a \cdot d_0 \cdot e^{- b \cdot d_0} \right) \cdot ( d - d_0 )
				\nonumber
				\\
				&\qquad + \mathcal{O}\left( ( d - d_0 )^2\right) \;\; ,
	\end{align*}
	
	where 
	
	\begin{align*}
		a = \dfrac{A \cdot p}{T} \;\;\; \text{and} \;\;\; b = \dfrac{B \cdot p}{T \cdot U} \;\; .
	\end{align*}
	
}

\end{document}
