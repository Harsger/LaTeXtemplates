\documentclass{beamer}
\usetheme{Montpellier}
\usecolortheme{spruce}
\definecolor{green(pigment)}{rgb}{0.0, 0.65, 0.31}
\setbeamercolor*{item}{fg=green(pigment)}
\definecolor{Green}{rgb}{0.00, 1.00, 0.00}
\definecolor{Red}{rgb}{1.00, 0.00, 0.00}
\definecolor{Blue}{rgb}{0.00, 0.00, 1.00}
\usepackage[utf8]{inputenc}
\usepackage{amsmath}
\usepackage{amsfonts}
\usepackage{amssymb}
\usepackage[german]{babel}
\usepackage{graphicx}
\usepackage{rotating}
\usepackage{multirow,bigdelim,dcolumn,booktabs}
%\usepackage{beamerthemeshadow}
\usepackage{subfigure} 
\usepackage{siunitx}
%\beamersetuncovermixins{\opaqueness<1>{25}}{\opaqueness<2->{15}}
\beamertemplatenavigationsymbolsempty

\usepackage{tikz}
\usetikzlibrary{decorations.text}
\usetikzlibrary{trees}
\usetikzlibrary{decorations.pathmorphing}
\usetikzlibrary{decorations.markings}
\usetikzlibrary{patterns}

\graphicspath{
	{LatexPics/}
}
\begin{document}
\title{Calibration of a \SI{1}{\m\squared} sized Resistive Strip Micromegas in a Test Facility with Cosmic Mouns}  
\author{ Maximilian Herrmann}
\date{21.10.2016} 

\frame{\titlepage} 

\frame{\frametitle{Outline}\tableofcontents}

\section{Test Facility in Graching}

\frame{\frametitle{Cosmic Ray Facility (CRF)}

	Two MDT chambers for tracking in precision direction (y).
	
	Scintillator hodoskop for triggering and a coarse information in x direction. 
	
	\begin{columns}
		\column{.50\textwidth}
			\centering
			\includegraphics[width=\textwidth]{CRFpictureLoesel.png}
		\column{.50\textwidth}
			\centering
			\includegraphics[width=\textwidth]{CRFsketchLoesel.png}						
	\end{columns}
	
}

\section{Large Area Micromegas}

\frame{\frametitle{A \SI{1}{\m\squared} sized Resistive Strip Micromegas}

	\begin{columns}
		\column{.50\textwidth}
			\centering
			\includegraphics[width=\textwidth]{RS_MM_principle.png}
			
			\tiny
			\begin{itemize}
				\item
					MICROMEsh GAseous Structure
				\item
					high electric fields leads to an avalanche of electrons
					after primary ionization by incident charged particles
				\item
					resistive strips collect electrons and prevent discharges
			\end{itemize}
		\column{.50\textwidth}
			\centering
			\includegraphics[width=\textwidth]{L1_schematic.png}	
				
			\tiny			
			\begin{itemize}
				\item
					128 readout strips (pitch \SI{450}{\micro\m}) are connected to an APV
				\item
					16 APV are needed for all strips (2048), connected in Master-Slave pairs 
				\item
					in orthogonal direction a coarse position information is given 
					by the trigger scintillators
					
					$\rightarrow$ 10 divisions
			\end{itemize}					
	\end{columns}

}

\section{Quality of the Readout Structure}

\frame{\frametitle{Dead Strips}

	\centering
	\includegraphics[width=0.6\textwidth]{deadStripInCluster-crop.pdf}
	
	\vspace{8mm}

	\begin{columns}
		\column{.50\textwidth}
			\centering
			\includegraphics[width=\textwidth]{deadstrips.pdf}
		\column{.50\textwidth}
			\centering
			\includegraphics[width=\textwidth]{deadcounts.pdf}						
	\end{columns}

}

\frame{\frametitle{Noisy Strips}

	\begin{columns}
		\column{.50\textwidth}
		
			\begin{itemize}
				\item
					$ 100 < q_{\mathrm{max}} < 2000 $
				\item
					$ \bigtriangleup t > 0.1 $ timebins
				\item
					$ 3 < t_{0} < 15 $ timebins
				\item
					$ \chi^{2}/\mathrm{NDF} < 100 $
			\end{itemize}
		
			\centering
			\includegraphics[width=\textwidth]{noisystrips.pdf}
		\column{.50\textwidth}
			\vspace{6mm}
					
			\centering		
			$\rightarrow$ if one condition is not fulfilled, the signal is counted as noise
			
			\vspace{15mm}
		
			\centering
			\includegraphics[width=\textwidth]{noisecounts.pdf}					
	\end{columns}

}

\section{Properties of the Readout Electronics}

\frame{\frametitle{Maximal Charge During a Signal on a Strip}

	\centering
	\includegraphics[width=0.6\textwidth]{maxQ_vs_strip.pdf}
	
	\begin{columns}
		\column{.50\textwidth}
			\centering
			\includegraphics[width=0.9\textwidth]{maxQ_APV12.pdf}
		\column{.50\textwidth}
			\centering
			\includegraphics[width=0.9\textwidth]{maxQ_APV7.pdf}					
	\end{columns}
	
}

\frame{\frametitle{Crosstalk between succesive Channels}
	
	\begin{columns}
		\column{.50\textwidth}
			\centering
			Master APV
			
			\includegraphics[width=\textwidth]{chargeVSchargeNext_APV0.pdf}
		\column{.50\textwidth}
			\centering
			Slave APV
			
			\includegraphics[width=\textwidth]{chargeVSchargeNext_APV1.pdf}					
	\end{columns}
	
	\vspace{8mm}
	
	\centering
	$\rightarrow$ correction of this has no visible effect on timing
	
}

\section{Signal Properties}

\frame{\frametitle{Signal Propagation}

	\centering
	\includegraphics[width=0.7\textwidth]{timingCut25ns_vs_x_scin20_5cm_fit.pdf}
	
	signal propagation time on strips : $ t / s = \mathrm{fitted\;slope} \cdot 25 \mathrm{ns} / 5 \mathrm{cm} \simeq 2 \tfrac{\mathrm{ns}}{\mathrm{m}}  $
	
}

\section{Position Reconstruction}

\subsection{Centroid Method}

\frame{\frametitle{Positiondependent Residual}
	
	\centering
	10 divisions in x direction
	
	\includegraphics[width=0.4\textwidth]{res_y_vs_x_scin10_fitPol5.pdf}
	
	\begin{columns}
		\column{.50\textwidth}
			\centering
			20 divisions in x direction
				
			\includegraphics[width=0.8\textwidth]{res_y_vs_x_scin20.pdf}
		\column{.50\textwidth}
			\centering
			40 divisions in x direction
			
			\includegraphics[width=0.8\textwidth]{res_y_vs_x_scin40.pdf}					
	\end{columns}
	
}

\frame{\frametitle{Correction Schemes}

	\centering
	\includegraphics[width=0.7\textwidth]{snakeLineCorrectionsComparison.pdf}

}

\frame{\frametitle{Systematics of the MDT Referenze Chambers}

	\centering
	\includegraphics[width=0.7\textwidth]{MDTinterceptDif_vs_x_scin_10to40divisions.pdf}
	
	\vspace{5mm}
	
	\centering
	$\rightarrow$ unexpected behavior for 40 Divisions

}

\subsection{$\mu$TPC Method}

\frame{\frametitle{Position Dependent Timing}

	\begin{columns}
		\column{.50\textwidth}
			\centering
			fastest strip in cluster
				
			\includegraphics[width=\textwidth]{fastest_time_APV0_scin0.pdf}
		\column{.50\textwidth}
			\centering
			slowest strip in cluster
			
			\includegraphics[width=\textwidth]{slowest_time_APV0_scin0.pdf}					
	\end{columns}
	
	\vspace{5mm}
	
	\centering
	$\rightarrow \;\;\; t_{\mathrm{mean}} = \tfrac{1}{2} \cdot \Big[ \big( \mathrm{Mean}_{\mathrm{fast}} - \mathrm{Sigma}_{\mathrm{fast}} \big) + \big( \mathrm{Mean}_{\mathrm{slow}} + \mathrm{Sigma}_{\mathrm{slow}} \big) \Big]$ 

}

\frame{\frametitle{Mean Timing for 20$\times$16 Partitions}

	\begin{columns}
		\column{.55\textwidth}
			\centering
			\includegraphics[width=\textwidth]{mean_timing_for_det_scin20.pdf}
		\column{.45\textwidth}
			\small
			$\mu$TPC position :
			
			\vspace{5mm}
			
			$\left( t_{\mathrm{mean}} - \mathrm{intercept}_{\mu\mathrm{TPC}} \right) / \mathrm{slope}_{\mu\mathrm{TPC}} $	
						
			\vspace{5mm}
			
			$\rightarrow$ use position dependent $t_{\mathrm{mean}}$
	\end{columns}
		
}

\frame{\frametitle{Improvement of Resolution}

	\begin{columns}
		\column{.50\textwidth}
			\centering
			constant mean timing
				
			\includegraphics[width=\textwidth]{residual_y_vs_angle_uTPC.pdf}
		\column{.50\textwidth}
			\centering
			position dependent timing
			
			\includegraphics[width=\textwidth]{residual_y_vs_angle_uTPC_positionTime.pdf}					
	\end{columns}	

}

\end{document}
