\documentclass{beamer}
\usetheme{Madrid}
\usecolortheme{spruce}
\definecolor{green(pigment)}{rgb}{0.0, 0.65, 0.31}
\setbeamercolor*{item}{fg=green(pigment)}
\definecolor{Green}{rgb}{0.00, 1.00, 0.00}
\definecolor{Red}{rgb}{1.00, 0.00, 0.00}
\definecolor{Blue}{rgb}{0.00, 0.00, 1.00}
\usepackage[utf8]{inputenc}
\usepackage{amsmath}
\usepackage{amsfonts}
\usepackage{amssymb}
\usepackage[german]{babel}
\usepackage{graphicx}
\usepackage{rotating}
\usepackage{textcomp}
\usepackage{multirow,bigdelim,dcolumn,booktabs}
%\usepackage{beamerthemeshadow}
\usepackage{subfigure} 
\usepackage{siunitx}
\usepackage{appendixnumberbeamer}
\usepackage{hyperref}
%\beamersetuncovermixins{\opaqueness<1>{25}}{\opaqueness<2->{15}}
\beamertemplatenavigationsymbolsempty
\def\checkmark{\tikz\fill[scale=0.4](0,.35) -- (.25,0) -- (1,.7) -- (.25,.15) -- cycle;}

\usepackage{tikz}
\usetikzlibrary{decorations.text}
\usetikzlibrary{trees}
\usetikzlibrary{decorations.pathmorphing}
\usetikzlibrary{decorations.markings}
\usetikzlibrary{patterns}

\graphicspath{
	{pictures/}
%	{/home/m/Maximilian.Herrmann/Bilder/forLatex/plots/}
%	{/home/m/Maximilian.Herrmann/Bilder/forLatex/sketches/}
%	{/home/m/Maximilian.Herrmann/Bilder/forLatex/pictures/}
}

\begin{document}
%\title[SM2 Module 0 @ H8 2017]{MM results from the H8C beam test with SM2, including micro-TPC mode and First Doublet results}
\title[NSW - SM2 Module 0 @ H8 2017]{Investigation of the SM2 Module 0 at H8C Testbeam and in the Cosmic Ray Test Facility}  
\author[M. Herrmann]{ Maximilian Herrmann}
\institute[LMU Munich]{Ludwig-Maximilians-Universit\"at M\"unchen - Lehrstuhl Schaile}
\date[10.11.2017]{Muon \& NSW Week \\ 10.11.2017, NSW Performance \& Testbeam} 

\frame{\titlepage} 

%\frame{\frametitle{Outline}\tableofcontents}

\section{H8 Testbeam Setup}

\frame{\frametitle{H8 Testbeam Setup for Tracking}
	\begin{columns}
		\column{.50\textwidth}
			\centering
			\includegraphics[width=0.9\textwidth]{setup-crop.pdf}
		\column{.50\textwidth}
			\centering
			\includegraphics[width=0.9\textwidth]{SM2-M0_H8aug17_setup.png}
	\end{columns}
	
	\footnotesize
	
	\begin{itemize}
		\item
			module fully equipped with 96 APVs
		\item 
			1024 strips for each layer were read out by two FEC cards
		\item
			two further FEC cards for 28 APVs of tracking telescope:
			
			3 twodimensional GEMs and 2 twodimensional TMMs
			
			$\Rightarrow$ 4 FECs read out @ \SI{220}{Hz} %(during spills)
		\item
			acquisition of \SI{25}{ns} time jitter via TDC (VME)
	\end{itemize}
}

\subsection{Centroid Spatial Resolution}

\frame{\frametitle{ Centroid Spatial Resolution for Perpendicular Incident}
	\begin{columns}
		\column{.50\textwidth}
			\centering
			%\textbf{tracking}
			
			\includegraphics[width=0.9\textwidth]{H8tracking-crop.pdf}
			
			\textbf{centroid distribution}
			
			\includegraphics[width=0.9\textwidth]{SM2-M0_H8aug17_residualDistribution-woTrack.pdf}
		\column{.50\textwidth}
			\footnotesize
			\begin{itemize}
				\item
					5 reference detectors: $\sigma_{\mathrm{track}} = $ \SI{70}{\micro\m} 
					
					track uncertainty @ SM2
				\item
					residual distribution fitted by double Gaussian
						\begin{itemize}
							\footnotesize
							\item
								each Gaussian corrected by the track uncertainty 
							\item
								sigmas weighted by integral of the respective Gaussian ($I_{i}$)
						\end{itemize}
						
%					$\sigma_{\mathrm{centroid}} = 
%					\dfrac{
%						I_{narrow} \cdot \sqrt{ \sigma_{\mathrm{narrow}} - \sigma_{\mathrm{track}}}
%						+
%						I_{broad} \cdot \sqrt{ \sigma_{\mathrm{broad}} - \sigma_{\mathrm{track}}}
%					}
%					{I_{narrow} + I_{broad}}$
											
					$\sigma_{\mathrm{centroid}} =$ 
					\vspace{3mm}
					
					$\dfrac{
						I_{1} \cdot \sqrt{ \sigma_{\mathrm{1}}^2 - \sigma_{\mathrm{track}}^2}
						+
						I_{2} \cdot \sqrt{ \sigma_{\mathrm{2}}^2 - \sigma_{\mathrm{track}}^2}
					}
					{I_{1} + I_{2}}$
				\item
					resolution : \textbf{86$\pm$\SI{5}{\mu\m}}
					
					for eta out @  
					
					$U_\mathrm{amplification} = $ \SI{580}{V}
					
					$U_\mathrm{drift} = $ \SI{-300}{V}
			\end{itemize}
	\end{columns}	
	\vspace{3mm}
	\centering
	
	\textbf{conclusion : tracking works }
}

\frame{\frametitle{Investigation of Centroid Resolution}
	\begin{columns}
		\column{.50\textwidth}
			\centering
			
			\includegraphics[width=0.8\textwidth]{SM2-M0_H8aug17_resolutionVSampVolt-crop.pdf}
			
			\includegraphics[width=0.8\textwidth]{SM2-M0_H8aug17_resolutionVSdriftVolt_threeAngles-crop.pdf}
		\column{.50\textwidth}
			\footnotesize
			\begin{itemize}
				\item
					for both eta layers the same resolution is achieved
					
					$\Rightarrow$ below \SI{90}{\micro\m}
				\item
					resolution independent of amplification voltage
				\item
					resolution independent of drift voltage
				\item
					angular dependence very similar to {\color{orange}K.Ntekas} (10$\times$10 cm$^2$ chambers)
			\end{itemize}
			\centering
			
			\includegraphics[width=0.9\textwidth]{SM2-M0_H8_resolutionVSangle_wKostas.pdf}
	\end{columns}
	\centering
	
	\textbf{conclusion : centroid resolution is excellent and behaves as expected }	
}

\frame{\frametitle{Cluster Reconstruction}
	\begin{columns}
		\column{.50\textwidth}
			\centering
			\textbf{number of clusters}
			
			\includegraphics[width=0.8\textwidth]{SM2-M0_H8run109angle20_numberOfClu_logscale.pdf}
			
			\textbf{number of strips \\ in leading cluster}
			
			\includegraphics[width=0.8 \textwidth]{SM2-M0_H8_numberOfStripsPerAngle.pdf}
		\column{.50\textwidth}
			\footnotesize
			\begin{itemize}
				\item
					only the cluster with the highest charge is considered (leading cluster, 90\% single cluster events)
					
					clusters created by noise are avoided
				\item
					number of strips in leading cluster increases with angle as expected
				\item
					readout PCB differences of Module 0 boards influence heavily the amplification, but not the resolution
			\end{itemize}
			
			\normalsize
			\centering
						
			\textbf{charge of leading cluster}
			
			\includegraphics[width=0.8 \textwidth]{SM2-M0_H8run109angle20_cluQ.pdf}
	\end{columns}
}

\section{Results}

\subsection{Pulse Height}

\frame{\frametitle{Pulse Height}
	\centering
	\includegraphics[width=0.75\textwidth]{SM2-M0_H8aug17_ampScan_asexpected.pdf}
	
	\begin{itemize}
		\item
			gas amplification depended on layer 
		\item
			gas amplification for stereo layers lower
	\end{itemize}
}

\section{\textmu TPC Analysis}

\frame{\frametitle{APV Readout and Signal Reconstruction for \textmu TPC Analysis}
	\begin{columns}
		\column{.50\textwidth}
			\centering
			\textbf{APV principle}
			
			\includegraphics[width=0.9\textwidth]{APV_principle.png}
			
			\textbf{signal fit}
			
			\includegraphics[width=\textwidth]{pulseheight_withExtrapolationLine_wPoint3.pdf}
		\column{.50\textwidth}
			\footnotesize
			\begin{itemize}
				\item
					single strip APV readout
				\item
					APV is sampling charge in \SI{25}{ns} steps
				\item
					charge on strip corresponds to maximum value of APV signal
				\item
					ionization time in drift region corresponds to start time of signal
					
					$\Rightarrow$ fit with inverse Fermi function:
					
					$q(t) = \dfrac{p_{0}}{1+\exp[(t-p_{1})/p_{2}]} + p_{3}$
					
					\begin{itemize}
						\item
							$p_{0}$ : maximum charge
						\item
							$p_{1}$ : turn time ($\mathrel{\hat=}$ 50\%)
						\item
							$p_{2}$ : rise time
						\item
							$p_{3}$ : offset ($\approx$ 0)   
					\end{itemize}
				\item
					\SI{25}{ns} time jitter due to \SI{40}{MHz} sampling recorded via TDC
			\end{itemize}
	\end{columns}
}

\frame{\frametitle{Single Channel Signal Properties @ \SI{20}{\degree} , \SI{300}{V} $U_{\mathrm{drift}}$}
	\begin{columns}
		\column{.50\textwidth}
			\centering
			\textbf{rise time}
			
			\includegraphics[width=0.8\textwidth]{SM2-M0_H8run109angle20_risetimes.pdf}
			
			\textbf{turn time}
			
			\includegraphics[width=0.8\textwidth]{SM2-M0_H8run109angle20_turntimes.pdf}
		\column{.50\textwidth}
			\footnotesize
			\begin{itemize}
				\item
					noise signals: 
					
					rise times smaller than \SI{2.5}{ns}
				\item
					very similar signal shapes for all layers 
				\item
					adjust amplification voltage to avoid saturation of readout channels
				\item
					cluster reconstruction:
					
					combine neighboring strips with at max 2 missing strips 
			\end{itemize}
						
			\normalsize
			\centering
			
			\textbf{maximum charge}
			
			\includegraphics[width=0.8\textwidth]{SM2-M0_H8run109angle20_stripQ.pdf}
	\end{columns}
}

\frame{\frametitle{\textmu TPC Analysis}
	\begin{columns}
		\column{.55\textwidth}
			%\centering
			\footnotesize
			\includegraphics[width=0.9\textwidth]{timevsstrip_descripted.pdf}
			
			\vspace{1.5mm}
			\normalsize
			
			linear fit: 
			
			signal time = $m_{\mathrm{fit}} \cdot$ strip number + $t_{\mathrm{fit}}$
		\column{.45\textwidth}
		\footnotesize
			\begin{itemize}
				\item
					angle reconstruction: 
											
					\vspace{2mm}
					
					$m_{\mathrm{fit}}$ : slope \textmu TPC fit
						
					\vspace{2mm}
					
					$\theta = \arctan\left( \tfrac{\mathrm{pitch}}{m_{\mathrm{fit}} \cdot v_{\mathrm{drift}}} \right)$
				\item
					position reconstruction:
										
					$t_{\mathrm{fit}}$ : intercept \textmu TPC fit
					
					$t_{\mathrm{mid}} = z_{\mathrm{half}} / v_{\mathrm{drift}} + t_{\mathrm{start}}$
%					$
%					\mathrm{pos}_{\mu\mathrm{TPC}}
%					=
%					\dfrac{t_{0} - t_{\mu \mathrm{TPC}}}{m_{\mu \mathrm{TPC}}}
%					$
					\begin{align*}
						&\mathrm{pos} = 
						\\
						&=
						\dfrac{z_{\mathrm{half}} - v_{\mathrm{drift}} \cdot \left( t_{\mathrm{fit}} - t_{\mathrm{start}}\right)}
						{v_{\mathrm{drift}} \cdot m_{\mathrm{fit}} / \mathrm{pitch}}
						\\
						&=
						\dfrac{t_{\mathrm{mid}} - t_{\mathrm{fit}}}{m_{\mathrm{fit}}} \cdot \mathrm{pitch}
					\end{align*}
			\end{itemize}
	\end{columns}
	
	\vspace{3mm}
	
	\begin{itemize}
		\item
			determination of $t_{\mathrm{mid}}$ with new method
		\item
			time jitter correction (\SI{40}{MHz} clock of APV/FEC) works \checkmark
		\item
			correction of capacitive coupling between strips $\approx$ 30\% \checkmark
			
			(LT Spice simulation)
	\end{itemize}
}

\frame{\frametitle{Preliminary \textmu TPC Results at \SI{20}{\degree}}
	\begin{columns}
		\column{.50\textwidth}
			\centering
			\textbf{angle reconstruction}
			
			\includegraphics[width=0.85\textwidth]{SM2-M0_H8run109angle20_angle_wNwoCCC.pdf}
						
			\textbf{residual distribution}
			
			\includegraphics[width=0.85\textwidth]{SM2-M0_H8run109-angle20_uTPCresidual_correction.pdf}
		\column{.50\textwidth}
			\small
			\begin{itemize}
				\item
					run parameter:
					\begin{itemize}	
						\item
							\SI{590}{V} amplification voltage
						\item
							\SI{300}{V} drift voltage
						\item
							muons
						\item
							\SI{20}{\degree} incident angle
					\end{itemize}
				\item
					corrections improve angular and residual distributions by decreasing the width
					
					optimum \@ 30\% capacitive coupling
				\item
					reconstructed angular distribution peaks at \SI{20}{\degree} when corrections applied
				\item
					very similar results using turn time of Fermi fit, as start time
			\end{itemize}
	\end{columns}	
}

\frame{\frametitle{Time Jitter Correction and Signal Time}
	
	\begin{columns}
		\column{.34\textwidth}
			\centering
			\textbf{jitter recording}
						
			\includegraphics[width=0.8\textwidth]{FECjitterRecording.pdf}
		\column{.33\textwidth}
			\centering
			\textbf{measured jitter}
									
			\includegraphics[width=\textwidth]{jitter_25ns.pdf}
		\column{.33\textwidth}
			\centering
			\textbf{signal fit}
			
			\includegraphics[width=\textwidth]{pulseheight_withExtrapolationLine_wPoint3.pdf}
	\end{columns}
		
	\begin{columns}
		\column{.34\textwidth}
			\footnotesize
			\begin{itemize}
				\item
					time jitter due to APV sampling in \SI{25}{ns} steps
				\item
					recorded via TDC
				\item
					slope of 
					
					\textmu TPC residual 
					
					VS time jitter 
					
					distribution 
					
					drift time dependent
					
			\end{itemize}
		\column{.33\textwidth}
			\centering
			
			\includegraphics[width=1.0\textwidth]{SM2-M0_H8run109-angle20_uTPCresVSjitter_otherColor.pdf}
		\column{.33\textwidth}
			\centering
			
			\includegraphics[width=1.0\textwidth]{timing_vergleich.pdf}
	\end{columns}
	\vspace{3mm}
	
	\footnotesize
	\hspace{45mm} extrapolated time = turn time - $\log (81) / 1.6\;\cdot$ rise time
}

\section{Cosmic Ray Facility}

\frame{\frametitle{Cosmic Ray Test Facility for Module Calibration}
		
	\begin{columns}
		\column{.50\textwidth}
			\centering
			\includegraphics[width=1.0\textwidth]{CRFprinciple-crop.pdf}
		\column{.50\textwidth}
			\centering
			\includegraphics[width=0.9\textwidth]{CRF_small.JPG}
	\end{columns}
	
	\footnotesize
			
	\begin{itemize}
		\item
			2D track reconstruction with two Monitored Drift Tube (MDT) chambers
		\item
			trigger via scintillator hodoscope with $\approx$ \SI{10}{cm} resolution in orthogonal direction
		\item
			MDT chambers : \SI{2.2}{\m} $\times$ \SI{4}{\m} 
			
			$\Rightarrow$ active area : \SI{9}{\m\squared}, angular acceptance : $\pm30^{\circ}$
		\item
			readout of the full module with six FEC cards @ full \SI{100}{Hz} \textmu -rate
			
			(tested up to \SI{500}{Hz} with random trigger)
	\end{itemize}
}

\frame{\frametitle{Average Pulse Height Distribution}
	\begin{columns}
		\column{.33\textwidth}
			\small
			\begin{itemize}
				\item
					MDT chambers 
					
					$\Rightarrow$ \textbf{\color{red}Y} precision coordinate 
				\item	
					scintillators
					
					$\Rightarrow$ \textbf{\color{red}X} coordinate
					
					coarse segmentation 
				\item
					HV: \SI{600}{V}, one sector at \SI{560}{V} (stereo out)
				\item	
					homogeneous pulse height (some inefficient spots)
				\item
					inefficient regions under investigation
			\end{itemize}
		\column{.33\textwidth}
			\centering
			\textbf{eta out}
						
			\includegraphics[width=0.9\textwidth]{SM2-M0_CRF_eta-out_meanCluQ.pdf}
						
			\textbf{eta in}
			
			\includegraphics[width=0.9\textwidth]{SM2-M0_CRF_eta-in_meanCluQ.pdf}
		\column{.33\textwidth}
			\centering
			\textbf{stereo in}
			
			\includegraphics[width=0.9\textwidth]{SM2-M0_CRF_stereo-in_meanCluQ.pdf}
			
			\textbf{stereo out}
						
			\includegraphics[width=0.9\textwidth]{SM2-M0_CRF_stereo-out_meanCluQ.pdf}
	\end{columns}
}

\frame{\frametitle{Readout PCB Alignment}
	\begin{columns}
		\column{.50\textwidth}
			\centering
			\textbf{eta in}
			
			\includegraphics[width=0.8\textwidth]{SM2-M0_CRF_eta-in_resMeanVSmdtY_zebraMisaligned.pdf}
			
			\textbf{stereo out}
			
			\includegraphics[width=0.8\textwidth]{SM2-M0_CRF_stereo-out_resMeanVSmdtY_APVadapterMisaligned.pdf}
		\column{.50\textwidth}
			\begin{itemize}
				\item
					mean of residual to reconstructed track intersection by MDT as function of precision coordinate of the MDT 
				\item
					reconstruction of:
						
					$\Rightarrow$ misalignment of APV adapter boards
						
					$\Rightarrow$ shift of readout PCB with respect ot each other
					
					$\Rightarrow$ pitch error $\mathcal{O}$(\SI{100}{nm})
			\end{itemize}
	\end{columns}
}

\frame{\frametitle{First Data of SM2 Module 1}
	
	\begin{columns}
		\column{.50\textwidth}
			\centering
			
			\textbf{cluster charge}
			
			\includegraphics[width=0.9\textwidth]{M1Charge_Neu.pdf}
			
			\textbf{cluster position}
			
			\includegraphics[width=0.9\textwidth]{M1Pos_Neuer.pdf}
		\column{.50\textwidth}
			\centering
			\textbf{setup}
			\vspace{3mm}
			
			\includegraphics[width=0.8\textwidth]{cosmics308_setup.JPG}
			
			\footnotesize
			\begin{itemize}
				\item
					series eta panel 1 doublet
				\item
					board 7 of one side read out
				\item
					drift voltage : \SI{-300}{V}
					
					amplification voltage : \SI{600}{V}
				\item
					coincidence of scintillator trigger from above and below
			\end{itemize}
		
	\end{columns}	
}

\frame{\frametitle{Conclusions}
	\begin{itemize}
		\item
			centroid analysis shows:
			\begin{itemize}
				\item
					resolution for perpendicular tracks is below \SI{90}{\micro\m} $\Rightarrow$ tracking works \checkmark
				\item
					resolution behaves as expected $\Rightarrow$ Module works \checkmark
			\end{itemize}
		\item
			\textmu TPC analysis:
			\begin{itemize}
				\item
					signal properties as expected \checkmark
				\item
					cluster properties as expected \checkmark
				\item
					corrections (jitter, capacitive coupling) work \checkmark
				\item
					incident angle is reconstructed correctly \checkmark
				\item
					resolution will be further investigated
			\end{itemize}
		\item
			 investigation in the cosmic ray test facility:
 			\begin{itemize}
				\item
					homogeneous pulse height distribution for the full SM2 Module 0 measured \checkmark
				\item
					board alignment will be further investigated
			\end{itemize}
	\end{itemize}
}

\appendix

\frame{\centering \Huge Backup}

\frame{\frametitle{Measurement Program}
	view from telescope against beam direction
	\vspace{3mm}
	
	\begin{columns}
		\column{.50\textwidth}
			\centering
			\includegraphics[width=0.9\textwidth]{SM2-M0_pointsAtH8_stereofront-crop.pdf}
		\column{.50\textwidth}
			\centering
			\includegraphics[width=0.9\textwidth]{SM2-M0_pointsAtH8_etafront-crop.pdf}
	\end{columns}
	\vspace{3mm}
	
	\begin{itemize}
		\item
			scan of a large part of the active area to investigate the efficiency  
			
			$\Rightarrow$ dead or noisy areas (for example between the PCBs)
		\item
			resolution as function of amplification and drift voltage, as well as incidence angle
		\item
			PCB alignment
	\end{itemize}
}

\frame{\frametitle{Readout with Jitter Correction Possibility}
			\centering
			
			\includegraphics[width=\textwidth]{SM2-M0_H8aug17_triggerNreadout-crop.pdf}
}

\frame{\frametitle{Correction due to Capacitive Coupling \\ of Neighboring Strips}

	charge spread due to capacitive coupling between resistive/readout strips
	
	\vspace{3mm}
	
	\begin{columns}
		\column{.50\textwidth}
			\centering
			\textbf{concept} \hspace{8mm}
			
			\includegraphics[width=0.9\textwidth]{capacitiveCoupling_withResStrips_smallQ.png}
		\column{.50\textwidth}
			\centering
			\textbf{simulation} \hspace{9mm}
			\vspace{3mm}
			
			\includegraphics[width=0.9\textwidth]{capacitveCoupling_circuit_simulation.png}
	\end{columns}
	
	\vspace{1mm}
	
	\textbf{implementation}
	
	\footnotesize
		
	\textbf{loop} strips in cluster
	
	\hspace{3mm} \textbf{loop} timebins from signalstart to maximum
	
	\hspace{3mm} \hspace{3mm} \textbf{loop} neighbors from 1 to 3
	
	\hspace{3mm} \hspace{3mm} \hspace{3mm} neighbor charge \hspace{5mm} - 0.29$^n$ central strip charge
	
	\hspace{3mm} \hspace{3mm} \hspace{3mm} central strip charge + 0.29$^n$ central strip charge
	
}

\frame{\frametitle{Readout PCB Alignment}
	\begin{columns}
		\column{.5\textwidth}
			\centering
			\textbf{eta out}
						
			\includegraphics[width=0.8\textwidth]{SM2-M0_CRF_eta-out_resMeanVSmdtY_wShift.pdf}
						
			\textbf{eta in}
			
			\includegraphics[width=0.8\textwidth]{SM2-M0_CRF_eta-in_resMeanVSmdtY_wShift.pdf}
		\column{.5\textwidth}
			\centering
			\textbf{stereo in}
			
			\includegraphics[width=0.8\textwidth]{SM2-M0_CRF_stereo-in_resMeanVSmdtY_wShift.pdf}
			
			\textbf{stereo out}
						
			\includegraphics[width=0.8\textwidth]{SM2-M0_CRF_stereo-out_resMeanVSmdtY_wShift.pdf}
	\end{columns}
}

\end{document}
