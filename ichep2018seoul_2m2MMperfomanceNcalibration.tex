\documentclass{beamer}
\usetheme{Madrid}
\usecolortheme{spruce}
\definecolor{green(pigment)}{rgb}{0.0, 0.65, 0.31}
\setbeamercolor*{item}{fg=green(pigment)}
\definecolor{Green}{rgb}{0.00, 1.00, 0.00}
\definecolor{Red}{rgb}{1.00, 0.00, 0.00}
\definecolor{Blue}{rgb}{0.00, 0.00, 1.00}
\usepackage[utf8]{inputenc}
\usepackage{amsmath}
\usepackage{amsfonts}
\usepackage{amssymb}
\usepackage[german]{babel}
\usepackage{graphicx}
\usepackage{rotating}
\usepackage{textcomp}
\usepackage{multirow,bigdelim,dcolumn,booktabs}
%\usepackage{beamerthemeshadow}
\usepackage{subfigure} 
\usepackage{siunitx}
\usepackage{appendixnumberbeamer}
\usepackage{hyperref}
%\beamersetuncovermixins{\opaqueness<1>{25}}{\opaqueness<2->{15}}
\beamertemplatenavigationsymbolsempty
%\usepackage[absolute,overlay]{textpos}

\usepackage{tikz}
\usetikzlibrary{decorations.text}
\usetikzlibrary{trees}
\usetikzlibrary{decorations.pathmorphing}
\usetikzlibrary{decorations.markings}
\usetikzlibrary{patterns}

\graphicspath{
	{pictures/}
%	{/home/m/Maximilian.Herrmann/Bilder/forLatex/plots/}
%	{/home/m/Maximilian.Herrmann/Bilder/forLatex/sketches/}
%	{/home/m/Maximilian.Herrmann/Bilder/forLatex/pictures/}
}

\begin{document}
\title[\SI{2}{\square\m} \& 4-layered Micromegas for ATLAS]{Performance and Calibration of \SI{2}{\square\m}-size 4-layered Micromegas Detectors for the ATLAS Upgrade}
\author[M. Herrmann]{ Maximilian Herrmann, \\ on behalf of the ATLAS muon collaboration}
\institute[LMU Munich]{Ludwig-Maximilians-Universit\"at M\"unchen - Lehrstuhl Schaile}
\date[05.07.2018]{05.07.2018, Seoul} 

\frame{
	\centering
	\vspace{2mm}
	{\Large International Conference on High Energy Physics 2018}

	\titlepage
	\vspace{-6mm}
	
	\includegraphics[width=0.35\textwidth]{LMUlogo.jpg}
	\hspace{2.5cm}
	\includegraphics[width=0.4\textwidth]{ATLAS-Logo.png}
}

\section{Motivation}

\subsection{New Small Wheels for the ATLAS Experiment}

\frame{\frametitle{LHC Upgrade Status}
	\centering
	\includegraphics[width=0.8\textwidth]{HL_LHC_PlanUpdateJuly2015_wLumi.pdf}
	\begin{columns}
		\column{.55\textwidth}
			\centering
			\includegraphics<1>[width=0.95\textwidth]{ATLASnewer.jpg}
			\includegraphics<2>[width=0.95\textwidth]{ATLAS_wSW.png}
		\column{.45\textwidth}
			\centering
			\includegraphics[width=0.9\textwidth]{MDTefficiencyPerHitRate_NSW-TDR_edited.png}
	\end{columns}
}

\frame{\frametitle{Upgrade of the Muon Small Wheels}
%	\begin{textblock*}{5cm}(5.3cm,7.8cm)
%	3/5 PCBs
%	
%	per active 
%	
%	layer
%	\end{textblock*}


	replacement of the current end caps of the muon spectrometer by small-strip Thin Gap Chambers (sTGC) and Micromegas quadruplets
	
	\vspace{3mm}
	\begin{columns}
		\column{.55\textwidth}
			\centering
			\textbf{New Small Wheel Sectors}
			\vspace{7mm}
			
			\includegraphics[width=1\textwidth]{sectorsDimensions.png}
		\column{.45\textwidth}
			\centering
			\textbf{Micromegas quadruplets}
			
			\includegraphics<1>[width=1\textwidth]{sectorsNsides.png}
			\includegraphics<2>[width=1\textwidth]{sectorsNsides_SM2marked.png}
	\end{columns}
}

\subsection{Micromegas Quadruplets}

\frame{\frametitle{\normalsize Micromegas Quadruplets for track reconstruction in the NSW}
	\textbf{MICROMEGAS - MICROMEesh GAseous Structure}
	\vspace{2mm}
	
	\begin{columns}
		\column{.55\textwidth}
			\centering
			\includegraphics[width=1.05\textwidth]{RSMM_principle7-crop.pdf}
		\column{.45\textwidth}
			\small
			\hspace{5mm}
			$
				{\bf \bf \mathrm{centroid}} 
				 = 
				\dfrac{
					\sum\limits_{\scriptscriptstyle\mathrm{strips}} \mathrm{strip} \cdot q_{\scriptscriptstyle\mathrm{strip}}
				}
				{\sum\limits_{\scriptscriptstyle\mathrm{strips}} q_{\scriptscriptstyle\mathrm{strip}} }
			$
			
			\vspace{3mm}
			\hspace{4mm}
			$\Rightarrow$ position = $\mathrm{centroid} \; \times$ pitch
			
			\vspace{3mm}
			\hspace{4mm}
			$t_{\mathrm{drift}} = f(\mathrm{strip})$
			
			\vspace{1mm}
			\hspace{4mm}
			$\Rightarrow$ incident angle reconstruction
	\end{columns}
	\vspace{3mm}
	
	\textbf{Micromegas Quadruplet}
	\begin{columns}
		\column{.6\textwidth}
			\centering
			\includegraphics[width=1\textwidth]{sandwichassembly.png}
		\column{.4\textwidth}
			\footnotesize	
			4 active layers of Micromegas
			
			\vspace{5mm}
			
			$\Rightarrow$ 2 $\times$ back-to-back
	\end{columns}
}

\frame{\frametitle{Design of Readout Anodes}
	\begin{columns}
		\column{.5\textwidth}
			\centering
			\includegraphics[width=\textwidth]{quadrupletStripOrientation3-crop.pdf}
		\column{.5\textwidth}
			\footnotesize
			\begin{itemize}
				\item
					{\color{red}eta planes} for precision reconstruction in pseudorapidity direction perpendicular to anode strips
				\item	
					{\color{blue}stereo planes} for additional coarse position information along the anode strips
			\end{itemize}
	\end{columns}
	\vspace{5mm}
	
	\begin{columns}
		\column{.5\textwidth}
			\centering
			SM2 anode
			
			\vspace{2mm}
			\includegraphics[width=\textwidth]{ROboardMechanicalAlignment2-crop.pdf}
		\column{.5\textwidth}
			\footnotesize
			technical limitations
			
			micropattern readout anode: width $\le$ \SI{50}{cm}
			\begin{itemize}
				\item[$\Rightarrow$] 
					
					3(/5) printed circuit boards (PCB) 
					
					per active layer
				\item[$\Rightarrow$] 	
					reconstruction and calibration of alignment errors (during production) required
			\end{itemize}
	\end{columns}
}

%\frame{\frametitle{Cosmic Ray Facility LMU Munich}
%		
%	\begin{columns}
%		\column{.50\textwidth}
%			\centering
%			\includegraphics[width=0.9\textwidth]{CRFprinciple3-crop.pdf}
%		\column{.50\textwidth}
%			\centering
%			\includegraphics[width=0.9\textwidth]{CRF_small.JPG}
%	\end{columns}
%	
%	\footnotesize
%			
%	\begin{itemize}
%		\item
%			2D track reconstruction with two Monitored Drift Tube (MDT) chambers
%		\item
%			trigger via scintillator hodoscope with $\approx$ \SI{10}{cm} resolution in orthogonal direction
%		\item
%			MDT chambers : \SI{2.2}{\m} $\times$ \SI{4}{\m} 
%			
%			$\Rightarrow$ active area : \SI{9}{\m\squared}, angular acceptance : $\pm30^{\circ}$
%		\item
%			readout of the full module with six FEC cards @ full \SI{100}{Hz} \textmu -rate
%			
%			(tested up to \SI{500}{Hz} with random trigger)
%	\end{itemize}
%}


\frame{
	\begin{center} 
		\Huge SM2 Prototype - M0
	\end{center}
	\vspace{10mm}
	
	\normalsize
	Measurement Campaign
	
	\begin{itemize}
		\item
			Cosmic Ray Facility in Munich
		\item
			testbeam at H8 beamline at the SPS
	\end{itemize}
}

\frame{\frametitle{Cosmic Ray Facility LMU Munich}
		
	\begin{columns}
		\column{.50\textwidth}
			\centering
			\includegraphics[width=0.9\textwidth]{CRFprinciple4-crop.pdf}
			
			\tiny
			\color{red} track prediction accuracy given by multiple scattering
		\column{.50\textwidth}
			\centering
			\includegraphics[width=\textwidth]{CRF_wMicromegas.pdf}
	\end{columns}
	
	\vspace{0mm}
	\centering
	\footnotesize
	\begin{tabular}{ll}
	track reconstruction & 2 $\times$ Monitored Drift Tube chambers (MDTs)
%	\\
%	 & single tube resolution $\sim$ \SI{100}{\micro\m}
% 	\\
%	 & six layers $\sim$ \SI{41}{\micro\m}
	\\
	trigger & scintillator hodoscope
	\\
	active area & \SI{2.2}{m} $\times$ \SI{4}{m}
	\\
	angular acceptance & $\pm$ \SI{30}{\degree}
	\\
	readout & 12288 channels
	\\
	 & $\to$ 96 APVs (frontend electronics)
	\\
	 & $\to$ 6 FECs (scalable readout system)
	\\
	readout rate & 130 Hz (full muon rate)
	\\
	measurement period & 17.10. - 06.12. 2017
	\end{tabular}
}

\frame{\frametitle{M0 : Full Area Pulse Height}
	\begin{columns}
		\column{.45\textwidth}
			\centering
			eta in: cluster charge
			
			\includegraphics[width=0.8\textwidth]{SM2-M0_CRFafterH8_eta-in_MPVclusterQ_wTrigLim.png}
			
			\vspace{-1mm}
			amplification scan
			
			\includegraphics[width=0.9\textwidth]{SM2-M0_CRF_MPVclusterQvsAmplificationVoltage_errorWidth_halfLines_fatBalls.png}
		\column{.65\textwidth}
			\footnotesize
			\vspace{-5mm}
			\begin{itemize}
				\item
					$U_{\mathrm{amp}} = $ \SI{600}{V}
					
					$U_{\mathrm{drift}} = $ \SI{-300}{V}
										
					Ar:CO$_{2}$ 93:7 vol\%
				\item
					for each bin (\SI{54.4}{mm}$\times$\SI{100}{mm}):
					
					cluster charge distribution fitted with Landau
					
					$\Rightarrow$ Most Probable Value (MPV)
				\item
					differences between 
					
					readout boards clearly visible
					
					$\Rightarrow$ higher amplification for central board
					
					$\Rightarrow$ homogeneity spoiled by 
					
					\hspace{3.5mm} prototype PCB quality
				\item
					smaller features due to trigger acceptance
					
					\vspace{5mm}
				\item		
					exponential rise as function of the amplification voltage (Townsend)
				\item
					differences between readout layers 
					
					due to variation in prototype PCB quality
			\end{itemize}
	\end{columns}
}

\frame{\frametitle{M0 : Full Area Efficiency}
	\begin{columns}
		\column{.45\textwidth}
			\centering
			eta in: \SI{5}{mm} efficiency
			
			\includegraphics[width=0.8\textwidth]{SM2-M0_CRFafterH8_eta-in_5mmEfficiency600V.pdf}
			
			\vspace{-1mm}
			efficiency turn on curve
			
			\includegraphics[width=0.9\textwidth]{SM2-M0_CRFafterH8_5mmEfficiency_allCluster_halfLines_fatBalls.png}
		\column{.65\textwidth}
			\footnotesize
			\vspace{-5mm}
			\begin{itemize}
				\item
					$U_{\mathrm{amp}} = $ \SI{600}{V}
					
					$U_{\mathrm{drift}} = $ \SI{-300}{V}
					
					Ar:CO$_{2}$ 93:7 vol\%
				\item
					\SI{5}{mm} efficiency:
					
					number of cluster found within $\pm$ \SI{5}{mm} to reference track
					
					divided by
					
					number all tracks going through partition
					
					$\Rightarrow$ calculated for each bin separately
				\item
					higher amplification of central board 
					
					leads to higher efficiency
				\item
					efficiency at boarders spoiled due to 
					
					tapered edges (rectangular partitions)
					
					\vspace{2mm}
				\item
					efficiency turn on curve reaches more than 90\% 
					
					at \SI{590}{V} for all layer
				\item
					differences between layer due to problematic prototype PCB material
			\end{itemize}
	\end{columns}
}

\subsection{Calibration Using Reference Tracks over Whole Area}

\frame{\frametitle{Alignment using Reference Tracks}
	\centering 
	\textbf{Concept:}
	\begin{columns}
		\column{.50\textwidth}
			\centering
			\includegraphics[width=0.90\textwidth]{positionshift2-crop.pdf}
		\column{.50\textwidth}
			\centering
			\includegraphics[width=0.90\textwidth]{verticalshift-crop.pdf}
	\end{columns}
	\vspace{2mm}
	\textbf{Implementation:}
	\begin{columns}
		\column{.55\textwidth}
			\centering
			\footnotesize before alignment
			
			\includegraphics[width=0.75\textwidth]{resVSslope_blue.pdf}
		\column{.45\textwidth}
			\small
			\textbf{residual} = pos\textsubscript{measured} - pos\textsubscript{reference}
%					\textbf{residual} = $\mathrm{pos}_{\mathrm{measured}} - \mathrm{pos}_{\mathrm{reference}}$

			\vspace{3mm}
			$\Rightarrow$ residual vs. slope {\scriptsize(reference track)}
	 
	 		\vspace{3mm}
			$\Rightarrow$ {\color{red}linear fit}
			
	 		\vspace{3mm}
			shift\textsubscript{horizontal} = intercept\textsubscript{fit}
%					$\mathrm{shift}_{\mathrm{horizontal}} = \mathrm{intercept}_{\mathrm{fit}}$

			\vspace{3mm}
			shift\textsubscript{vertical} = slope\textsubscript{fit}	
%					$\mathrm{shift}_{\mathrm{vertical}} \;\;\; = \mathrm{slope}_{\mathrm{fit}}$
	\end{columns}
}

\frame{\frametitle{Reconstruction of the Gravitational Sag of M0}

	\begin{columns}
		\column{.7\textwidth}
			\centering
			\textbf{eta in plane}
			
			\includegraphics[width=0.72\textwidth]{SM2-M0_CRFafterH8_eta-in_deltaZ.pdf}
		\column{.1\textwidth}
			\centering
			\textbf{stereo in}
			\vspace{10mm}
			
			\textbf{stereo out}
			\vspace{10mm}	
					
			\textbf{eta out}
		\column{.2\textwidth}
			\centering
			
			\includegraphics[width=0.85\textwidth]{SM2-M0_CRFafterH8_stereo-in_deltaZ.pdf}
			
			\includegraphics[width=0.85\textwidth]{SM2-M0_CRFafterH8_stereo-out_deltaZ.pdf}
			
			\includegraphics[width=0.85\textwidth]{SM2-M0_CRFafterH8_eta-out_deltaZ.pdf}
	\end{columns}

	\vspace{1mm}	
	$\Rightarrow$ all layers show gravitational sag 
	
	$\Rightarrow$ irrelevant for ATLAS, as detectors will be used vertically in NSW
}

\frame{\frametitle{M0 : Board Alignment  and Reconstructed Pitch Deviation}
	\vspace{-3mm}
	\begin{center}
		\Large
		position = centroid $\times$ pitch
	\end{center}
	\vspace{-5mm}
	\begin{columns}
		\column{.50\textwidth}
			\small
			\begin{center}
				{\color{red}pitch deviation}
				
				\includegraphics[width=\textwidth]{eta_in_resMeanVSmdtY_woAdapterBoard_wCorPitch.png}
				\vspace{1.5mm}
				
				\centering\includegraphics[width=0.45\textwidth]{ROboardsAligned-crop.pdf}
			\end{center}
		\column{.50\textwidth}
			\small
			\vspace{-2.7mm}
			\begin{center}
				corrected pitch
				
				\includegraphics[width=\textwidth]{eta_in_resMeanVSmdtY_wNewPitchCor_boards.pdf}
			\end{center}
			\vspace{1.5mm}
				
			\hspace{4mm}		
			construction equipment optimized 

			\hspace{4mm}			
			$\Rightarrow$ avoid shifts
			\vspace{2mm}
			
			\hspace{4mm}
			humidity control of readout boards 

			\hspace{4mm}			
			$\Rightarrow$ avoid pitch deviation
	\end{columns}
}

\subsection{Setup for Tracking}

%\frame{\frametitle{H8 Testbeam Setup for Tracking}
%	\begin{columns}
%		\column{.50\textwidth}
%			\centering
%			\includegraphics[width=0.9\textwidth]{setup-crop.pdf}
%		\column{.50\textwidth}
%			\centering
%			\includegraphics[width=0.9\textwidth]{SM2-M0_H8aug17_setup.png}
%	\end{columns}
%	
%	\footnotesize
%	
%	\begin{itemize}
%		\item
%			module fully equipped with 96 APVs
%		\item 
%			1024 strips for each layer were read out by two FEC cards
%		\item
%			two further FEC cards for 28 APVs of tracking telescope:
%			
%			3 twodimensional GEMs and 2 twodimensional TMMs
%			
%			$\Rightarrow$ 4 FECs read out @ \SI{220}{Hz} %(during spills)
%		\item
%			TDC for acquisition of time jitter between APC clock and trigger
%	\end{itemize}
%}

\frame{\frametitle{H8 Testbeam for SM2 M0 in August 2017}
	\begin{columns}
		\column{.50\textwidth}
			\centering
			\includegraphics[width=0.9\textwidth]{H8_SM2-M0_setup-crop.pdf}
		\column{.50\textwidth}
			\centering
			\includegraphics[width=0.9\textwidth]{SM2-M0_H8aug17_wDescription.png}
	\end{columns}
		
	\vspace{1mm}
	\centering
	\footnotesize{\tiny 
	\begin{tabular}{lll}
		tracking telescope & 3 $\times$ 2D GEM &
		\\
		 & 2 $\times$ 2D TMM &
		\\
		track accuracy & \SI{65}{\micro\m} & (extrapolated at module) 
		\\
		channels & 4 $\times$ 1024 & module
		\\
		 & 2976 & telescope
		\\
		readout & 32 APVs $\to$ 2 FECs  & module 
		\\
		 & 24 APVs $\to$ 2 FECs & telescope
		\\
		trigger rate & $\sim$ \SI{1}{kHz} (muons) & for \SI{9}{cm} $\times$ \SI{9}{cm}
		\\
		readout rate & 220 Hz & (limited by bandwidth)
	\end{tabular}}
}

\subsection{Behavior of Pulse Height and Efficiency}

\frame{\frametitle{M0 : Behavior of Pulse Height and Efficiency}
	\begin{itemize}
		\item
			pulse height:
			
			exponential rise as function of the amplification voltage (Townsend)
		\item
			differences between layers due to variation in prototype PCB quality
		\item
			efficiency plateau starting at \SI{590}{V}
		\item
			lower efficiency due to unconnected strips in measurement region
			
			(e.g. eta out plane)
	\end{itemize}
	
	\begin{columns}
		\column{.50\textwidth}
			\centering
			\includegraphics[width=0.9\textwidth]{H8_M0_clusterQvsAmpVolt_bigDots.pdf}
		\column{.50\textwidth}
			\centering
			\includegraphics[width=0.9\textwidth]{SM2-M0_H8aug17_efficiencyVSampVoltage_unconStrips.png}
	\end{columns}
}

\frame{\frametitle{M0 : Results for Charge Weighted Position Reconstruction}
	\begin{columns}
		\column{.50\textwidth}
			\centering
			\includegraphics[width=\textwidth]{SM2-M0_H8aug17_residualDistribution_res.pdf}
			
			\includegraphics[width=\textwidth]{H8_M0_eoNei_resoltionVSampVolt.pdf}
		\column{.50\textwidth}
			\footnotesize
			\begin{itemize}
				\item
					residual distribution (difference of measured position and track prediction) fitted with double Gaussian
					
					\vspace{3mm}
					
					$\Rightarrow$ weighted sigma:
					
					\scriptsize
					$\sigma_{\mathrm{w}}\,= \dfrac{I_{\mathrm{narrow}} \cdot \sigma_{\mathrm{narrow}} + I_{\mathrm{broad}} \cdot \sigma_{\mathrm{broad}}}{I_{\mathrm{narrow}} + I_{\mathrm{broad}}}$
					
					\vspace{2mm}
					
					\footnotesize
					$\Rightarrow$ consider track uncertainty:
					
					$\sigma_{\mathrm{res}} = \sqrt{\sigma_{\mathrm{w}}^2 - \sigma_{\mathrm{Track}}^2}$
					
					\vspace{5mm}
				\item
					resolution for perpendicular incident for both eta layers similar
					
					$\Rightarrow$ \SI{80}{\micro\m}
				\item
					resolution independent of amplification and drift voltage
			\end{itemize}
	\end{columns}
}

\frame{\frametitle{M0 : Drift Time Measurement for Track Reconstruction}
	\begin{columns}
		\column{.50\textwidth}
			
			\centering
			\includegraphics[width=0.9\textwidth]{timevsstrip_wDrift.pdf}
			
			\includegraphics<1>[width=0.7\textwidth]{SM2-M0_H8_resolutionVSangle_onlyEta.png}
			\includegraphics<2>[width=0.7\textwidth]{SM2-M0_H8_resolutionVSangle_wStereo.png}
			\includegraphics<3>[width=0.7\textwidth]{SM2-M0_H8_resolutionVSangle_timingCor.png}
		\column{.50\textwidth}
			\footnotesize
			\vspace{-2mm}
			\begin{itemize}
				\item
					drift time measurement enables reconstruction of inclined tracks: 
					
					time-projection-chamber like 
				\item
					inhomogeneous ionization leads to a timing dependence of the residual
			\end{itemize}
			\vspace{4mm}
			\begin{itemize}
				\item
					degrading resolution for inclined incident using charge weighted reconstruction only
					
					$\Rightarrow$ similar behavior as 
					
					\hspace{3.5mm} small size chambers 
				\item
					charge weighted timing correction improves resolution considerably
					
					$\Rightarrow$ almost constant for angles $\le$ \SI{30}{\degree}
				\item
					resolution limited by signal to noise ratio of APV readout
			\end{itemize}
			\vspace{7mm}
	\end{columns}
}

\frame{\frametitle{Summary}
	\footnotesize
	\begin{itemize}
		\item
			upgrade of the ATLAS muon spectrometer inner end cap
			\begin{itemize}
				\footnotesize
				\item
					sTGC and Micromegas quadruplets (16 active layers in total)
				\item
					threepart (/fivepart) readout structure 
					
					$\Rightarrow$ reconstruction and calibration is required after construction
			\end{itemize}
		\item
			Investigation of the SM2 Prototype (M0) at the Cosmic Ray Facility in Munich
			\begin{itemize}
				\footnotesize
				\item
					full active area responsive, despite problematic prototype PCB quality
				\item
					calibration of the full active area of SM2 demonstrated
			\end{itemize}
		\item
			Measurement at H8 Beamline of the SPS with the SM2 M0
			\begin{itemize}
				\footnotesize
				\item
					reasonable pulse height and efficiency behavior
				\item
					charge weighted position reconstruction for perpendicular tracks 
					
					$\Rightarrow$ \SI{80}{\micro\m} resolution
					
					$\Rightarrow$ same for both eta layers
					
					$\Rightarrow$ independent of drift and amplification voltage
				\item
					drift time measurement for tracks with $\le$ \SI{30}{\degree} inclination
					
					$\Rightarrow$ similar resolution
					
					$\Rightarrow$ limitation by signal to noise ratio of APV electronics 
					
					\hspace{3.5mm} (final electronics currently under test with first series SM2 module at H8) 
			\end{itemize}
	\end{itemize}
}

\appendix

\frame{\centering \Huge Backup}

\frame{\frametitle{Time Evolution of the Signal on a Single Strip}
	beginning of the signal : fit by an inverse Fermi function
	\vspace{5mm}
	\begin{columns}
		\column{.50\textwidth}
			\centering
			\includegraphics[width=\textwidth]{pulseheight_withExtrapolationLine_wPoint3.pdf}
		\column{.50\textwidth}
			{\small
			\centering
			\hspace{3mm} $ \mathrm{f}_{\mathrm{Fermi}} = \dfrac{p_{0}}{1 + \exp[(p_{1} - x)/p_{2}]} + p_{3}$
			
			
			%\hspace{3mm} $ \dfrac{p_{0}}{1 + e^{(p_{1} - x)/p_{2}}} + p_{3}$
			\begin{itemize}
				\item
					$p_{0}$ : maximal pulse height 
					
					\hspace{8mm} $\Rightarrow$ charge of signal
				\item
					$p_{1}$ : time of 50\% 
					
					\hspace{5.5mm} maximal pulse height
				\item
					$p_{2}$ : $\propto$ rise time
				\item
					$p_{3}$ : pedestal
			\end{itemize}
			}
	\end{columns}
	\vspace{5mm}
	\centering
	$\Rightarrow$ {\color{green}3 values} of $\mathrm{f}_{\mathrm{Fermi}}$ at 10\% , 50\% and 90\% define 
	
	{\color{teal}start time} of signal by {\color{blue}extrapolation}
}

\frame{\frametitle{Position and Track Reconstruction}
	\begin{columns}
			\column{.50\textwidth}
				\centering
				\includegraphics[width=1.1\textwidth]{eventdisplay.png}
				
				drift time measurement
				
				\includegraphics[width=0.9\textwidth]{timevsstrip2.pdf}
			\column{.50\textwidth}
				\small
				\begin{itemize}
					\item
						centroid method
						
						$\Rightarrow$ charge average over strips
						
						\vspace{3mm}$
							 {\bf \color{violet}x_{\bf \mathrm{centroid}}} 
								 = 
							\dfrac{
								\sum\limits_{\scriptscriptstyle\mathrm{strips}} x_{\scriptscriptstyle\mathrm{strip}} \cdot q_{\scriptscriptstyle\mathrm{strip}}
							}
							{\sum\limits_{\scriptscriptstyle\mathrm{strips}} q_{\scriptscriptstyle\mathrm{strip}} }
						$\vspace{4mm}
					\item
						TPC-like method
						
						angle reconstruction by drift time measurement
						
						${\color{blue}\alpha} = \arctan\left(\dfrac{\mathrm{pitch}}{{\color{red}\mathrm{slope}_{\mathrm{fit}}} \cdot v_{\mathrm{drift}}}\right)$
				\end{itemize}
	\end{columns}
}

\frame{\frametitle{\normalsize Position Reconstruction Using Charge Weighted Clustertime}
	\begin{columns}
		\column{.50\textwidth}
			\centering
			\includegraphics[width=0.75\textwidth]{inhomogeneousIonization.png}
			
			\vspace{-1mm}
			\includegraphics[width=0.75\textwidth]{H8_M0_resVScluTime.png}
		\column{.50\textwidth}
			\scriptsize
			\begin{itemize}
				\item
					charge weighted timing:
					
					$t_{\mathrm{q}} = \dfrac{\sum t_{\mathrm{strip}} q_{\mathrm{strip}}}{\sum q_{\mathrm{strip}}}$
					
					$\Rightarrow$ vertical position in drift gap
				\item
					for inclined incident:
					
					centroid residual 
					
					VS charge weighted timing 
					
					$\Rightarrow$ linear dependence 
				\item
					slope given by 
					
					drift time and incident angle
										
					$\Rightarrow$ drift time is given by gas mixture, 
					
					\hspace{3mm} cathode voltage and drift gap
										
					$\Rightarrow$ for NSW Micromegas incident angle is 
					
					\hspace{3mm} almost fixed 
					
					\hspace{3mm} (direction to interaction point)
				\item
					new position is given by:
					
					$ x = x_{\mathrm{cen}} + \bigtriangleup t \cdot v_{\mathrm{drift}} \cdot \tan \theta$
					
					with:
					
					\begin{tabular}{ll}
					$x_{\mathrm{cen}}$ & centroid position
					\\
					$\bigtriangleup t$ & $t_q - t_{\mathrm{mean}}$
					\\
					$v_{\mathrm{drift}}$ & drift velocity
					\\
					$\theta$ & incident angle
					\end{tabular}
			\end{itemize}
	\end{columns}
}

\frame{\frametitle{SM2-M0 at H8 Testbeam August 2017 Measurement Overview}
	\begin{columns}
		\column{.50\textwidth}
			\centering
			\includegraphics[width=0.9\textwidth]{SM2-M0_pointsAtH8_stereofront-crop.pdf}
		\column{.50\textwidth}
			\centering
			\includegraphics[width=0.9\textwidth]{SM2-M0_pointsAtH8_etafront-crop.pdf}
	\end{columns}
	\vspace{3mm}
	\begin{itemize}
		\item
		for second configuration the order of layers was inverted
		\item
		at Points 12 and 13 module was tilted for inclined measurements
		\item
		several amplification scans made at different Points
	\end{itemize}
}

\frame{\frametitle{Jitter Correction for \SI{40}{MHz} Readout of APVs}
	\begin{columns}
		\column{.50\textwidth}
		\centering
		\includegraphics[width=\textwidth]{SM2-M0_H8aug17_triggerNreadout-crop.pdf}
		\column{.50\textwidth}
		\centering
		\includegraphics[width=\textwidth]{jitter_25ns.pdf}
	\end{columns}
	\vspace{3mm}
	\begin{itemize}
		\item
			expected time jitter of signal due to \SI{40}{MHz} sampling of APVs
		\item
			record time difference of trigger signal and FEC cards (connected to APVs) via TDC
		\item
			merge at DAQ PC
			
			$\Rightarrow$ offline offsetcorrection and zerosuppression
	\end{itemize}
}

%\frame{\frametitle{Position Resolution as Function of the Incident Angle}
%	\vspace{-1mm}
%	
%	\begin{columns}
%		\column{.50\textwidth}
%			\centering
%			\footnotesize
%			\textbf{eta out}
%			\includegraphics[width=0.9\textwidth]{m1_eo_resolutionVSangle_wCluTimeCor.pdf}
%			
%			\vspace{-0.5mm}
%			\textbf{eta in}
%			\includegraphics[width=0.9\textwidth]{m1_ei_resolutionVSangle_wCluTimeCor.pdf}
%		\column{.50\textwidth}
%			\footnotesize
%			\begin{itemize}
%				\item
%					first series SM2 Module in Cosmic Ray Facility
%				\item
%					residual distribution for each angle separately
%				\item
%					fit with double Gaussian
%					
%					analysis sigma narrow Gaussian only
%					
%					(reject multiple scattering)
%				\item
%					consider track uncertainty of reference chambers: 
%				
%					$\sigma_{\mathrm{micromegas}} = \sqrt{\sigma_{\mathrm{res}}^2 - \sigma_{\mathrm{track}}^2}$
%				\item
%					resolution is for perpendicular incident close to expectation
%				\item
%					charge weighted clustertime correction improves residual distribution considerably
%			\end{itemize}
%	\end{columns}
%}

\end{document}
