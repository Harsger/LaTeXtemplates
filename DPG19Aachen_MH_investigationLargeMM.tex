\documentclass{beamer}
\usetheme{Madrid}
\usecolortheme{spruce}
\definecolor{green(pigment)}{rgb}{0.0, 0.65, 0.31}
\setbeamercolor*{item}{fg=green(pigment)}
\definecolor{Green}{rgb}{0.00, 1.00, 0.00}
\definecolor{Red}{rgb}{1.00, 0.00, 0.00}
\definecolor{Blue}{rgb}{0.00, 0.00, 1.00}
\usepackage[utf8]{inputenc}
\usepackage{amsmath}
\usepackage{amsfonts}
\usepackage{amssymb}
\usepackage[german]{babel}
\usepackage{graphicx}
\usepackage{rotating}
\usepackage{textcomp}
\usepackage{multirow,bigdelim,dcolumn,booktabs}
%\usepackage{beamerthemeshadow}
\usepackage{subfigure} 
\usepackage{siunitx}
\usepackage{appendixnumberbeamer}
\usepackage{hyperref}
\usepackage{mdframed}
\usepackage{xcolor}
\usepackage[absolute,overlay]{textpos}
%\beamersetuncovermixins{\opaqueness<1>{25}}{\opaqueness<2->{15}}
\beamertemplatenavigationsymbolsempty

\usepackage{tikz}
%\usetikzlibrary{decorations.text}
%\usetikzlibrary{trees}
%\usetikzlibrary{decorations.pathmorphing}
%\usetikzlibrary{decorations.markings}
%\usetikzlibrary{patterns}

\graphicspath{
	{pictures/}
%	{/home/m/Maximilian.Herrmann/Bilder/forLatex/plots/}
%	{/home/m/Maximilian.Herrmann/Bilder/forLatex/sketches/}
%	{/home/m/Maximilian.Herrmann/Bilder/forLatex/pictures/}
}
\begin{document}
\title[\SI{2}{\square\m}-sized Micromegas Quadruplets]{Characterisation of 2 m$^2$ sized 4 layered Micromegas Modules with Cosmic Muons}  
\author[M. Herrmann]{Maximilian Herrmann}
\institute[LMU Munich]{Ludwig-Maximilians-Universit\"at M\"unchen - Lehrstuhl Schaile}
\date[25.03.19]{25.03.2019, Aachen} 

\frame{\titlepage
	\centering
	Fr\"uhjahrstagung der Deutschen Physikalischen Gesellschaft
	
	\hspace{-1.1cm}
	\includegraphics[width=0.5\textwidth]{LMUlogo.jpg}
	\hspace{2cm}
	\includegraphics[width=0.3\textwidth]{BMBFlogo.png}
	\vspace{1mm}
} 

%\frame{\frametitle{Outline}\tableofcontents}

\section{Multilayered Squaremeter Sized Micromegas Modules}

\subsection{Micromegas Quadruplets for Track Reconstruction}

\frame{\frametitle{Micromegas Principle}
	
	\begin{columns}
		\column{.6\textwidth}
			\centering
			\includegraphics[width=0.95\textwidth]{RSMM_principle8-crop.pdf}
		\column{.4\textwidth}
			\small
			position reconstruction:
			\vspace{3mm}
			
			$
				{\bf \bf \mathrm{centroid}} 
			 = 
				\dfrac{
					\sum\limits_{\scriptscriptstyle\mathrm{strips}} \mathrm{strip} \cdot q_{\scriptscriptstyle\mathrm{strip}}
				}
				{\sum\limits_{\scriptscriptstyle\mathrm{strips}} q_{\scriptscriptstyle\mathrm{strip}} }
			$
			\vspace{5mm}
			
			angle reconstruction:
			\vspace{5mm}
			
			\footnotesize
			$
					{\bf \bf \mathrm{angle}}
				=
					\arctan
						\left(
							\dfrac{\mathrm{pitch} \cdot \bigtriangleup \mathrm{strips}}
							{v_\mathrm{drift} \cdot \bigtriangleup t_\mathrm{drift}}
						\right)
			$
			\vspace{6mm}
	\end{columns}
}

\frame{\frametitle{Detector Design (\SI{2}{\m\squared})}
	\begin{columns}
		\column{.5\textwidth}
			\centering
			4 active layers per module
			\vspace{1mm}
			
			\includegraphics[width=\textwidth]{sandwichassembly.png}
			\vspace{5mm}
					
			\includegraphics[width=\textwidth]{quadrupletStripOrientation3-crop.pdf}
		\column{.5\textwidth}
			\centering
			3 PCBs per active layer
			\vspace{5.5mm}
			
			\includegraphics[width=\textwidth]{ROboardsAligned2-crop.pdf}
			\vspace{3mm}
	\end{columns}
	\vspace{4mm}
	
	$\Rightarrow$ investigation of homogeneity: pulse height and efficiency

	$\Rightarrow$ reconstruction of alignment: shape of readout structure
}

\section{Investigation with Cosmic Muons}

\subsection{Cosmic Ray Facility in Garching}

\section{Cosmic Ray Facility}

\frame{\frametitle{Cosmic Ray Facility}
		
	\begin{columns}
		\column{.50\textwidth}
			\centering
			\includegraphics<1>[width=0.9\textwidth]{CRFprinciple3-crop.pdf}
			
			\includegraphics<2>[width=0.9\textwidth]{CRFprinciple5-crop.pdf}
		\column{.50\textwidth}
			\centering
			\includegraphics[width=0.9\textwidth]{CRF_small.JPG}
	\end{columns}
	
	\vspace{4mm}
	\footnotesize
	\begin{tabular}{ll}
		trigger & scintillator hodoscope
		\\
		 & non-precision coordinate, resolution $\sim$ \SI{10}{cm}
		\\
		track reconstruction & 2 $\times$ Monitored Drift Tube chambers (MDTs)
		\\
		 & precision coordinate, resolution $\sim$ \SI{0.2}{mm} (multiple scattering)
		\\
		active area & \SI{2.2}{m} $\times$ \SI{4}{m}
		\\
		angular acceptance & $\pm$ \SI{30}{\degree}
		\\
		energy cut & iron absorber (\SI{34}{cm}) $\to E_{\mu} > $ \SI{600}{MeV} 
		\\
		readout (full module) & 12288 channels $\to$ APVs (frontend electronics)
		\\
		readout rate & 100 Hz (online zerosuppression)
	\end{tabular}
}

\frame{\frametitle{\large Pulse Height and Efficiency (Ar:CO$_2$ 93:7 vol\%, $U_{\mathrm{amp}} =$ \SI{570}{V})}

	\begin{columns}
		\column{.5\textwidth}
			\centering
			pulse height
			\vspace{1mm}
			
			\includegraphics[width=0.65\textwidth]{m3_eo_clusterQmpv_20180911.pdf}
			\vspace{0.5mm}
		\column{.5\textwidth}
			\centering
			efficiency
			\vspace{1mm}
			
			\includegraphics[width=0.65\textwidth]{m3_eo_coinEffi_20180911.pdf}
	\end{columns}
	\vspace{2mm}
	
	\begin{columns}
		\column{.7\textwidth}
			\centering
			pulse height VS pillar height
			\vspace{1mm}
			
			\includegraphics[width=0.65\textwidth]{meanClusterQvsPillarHeight_etaNstereo.png}
			\vspace{1mm}
		\column{.3\textwidth}
			pillar height variation
			
			\vspace{2mm}
			$\Rightarrow$ pulse height
			
			\vspace{2mm}
			$\Rightarrow$ efficiency
	\end{columns}
	
}

\frame{\frametitle{Reconstruction of Board Alignment}
	\begin{columns}
		\column{.5\textwidth}
			\centering
			mean residual per partition
		\column{.5\textwidth}
			\centering
			\small
			mean residual VS position along strips
	\end{columns}
	
	\begin{columns}
		\column{.5\textwidth}
			\centering
			\includegraphics[width=0.95\textwidth]{m3_eo_deltaYperPart_2019-09.pdf}
		\column{.5\textwidth}
			\centering
			\includegraphics[width=0.95\textwidth]{m3_eo_resMeanVSscinX_allBoards.pdf}
	\end{columns}
			
	\centering	
	$\Rightarrow$ bend strip shape
				
	\includegraphics[width=0.6\textwidth]{ROboardsBendStrips-crop.pdf}
}

\frame{\frametitle{\large Resolution of Charge Weighted Reconstruction}
	
	\begin{columns}
		\column{.5\textwidth}
			\centering
			\includegraphics[width=0.75\textwidth]{m1_ei_resVSangle.pdf}
			
			\only<1>{
				centroid resolution
						
				\includegraphics[width=0.75\textwidth]{m1_resolutionVSangle_eoNei.pdf}
			}
			\includegraphics<2>[width=0.7\textwidth]{inhomogeneousIonization_bE_cNm_centroidShift.pdf}
		\column{.5\textwidth}
			\hspace{3mm} residual, angle $\in$ [ \SI{0}{\degree} , \SI{2}{\degree} ]
			\vspace{1mm}
			
			\begin{centering}
				\includegraphics[width=0.85\textwidth]{m1_ei_residaul_nearZero.pdf}
			\end{centering}
			\footnotesize
		
			$\sigma$ of narrow Gaussian $\Rightarrow$
			
			\vspace{2mm}
			
			resolution $ = \sqrt{\sigma_{\mathrm{MM}}^{2} - \left(\tfrac{1}{\sqrt{2}}\cdot\sigma_{\mathrm{MDT}}\right)^{2}}$
			
			\vspace{4mm}
						
			inhomogeneous ionization 
			
			\vspace{0.5mm}
			$\Rightarrow$ declining resolution 
			
			\hspace{4.5mm} for larger angles
			
			\vspace{3mm}
			
			$\Rightarrow$ use drift time information
	\end{columns}
	
}

\frame{\frametitle{Drifttime Measurements for Track Reconstruction}
	\small
	\begin{columns}
		\column{.50\textwidth}
			\centering
			\includegraphics[width=0.9\textwidth]{timeVSstrip_meshNcathodeNstrips_fitNtime.pdf}
		\column{.50\textwidth}
			\centering
			angle reconstruction
			
			\includegraphics[width=0.65\textwidth]{m1_ei_uTPCangleVSangle_20180601_turntime.pdf}
	\end{columns}
	\vspace{1.5mm}

	\begin{columns}
		\column{.5\textwidth}
			\centering
			\includegraphics<1>[width=0.7\textwidth]{inhomogeneousIonization_bE_cNm_timeNcentroidShift_deltaXnTnAngle.pdf}
			\only<2>{
				fix track slope
				
			}
			\includegraphics<2>[width=0.6\textwidth]{m3_ei_C150V_resVSclutime_slope-dot46.pdf}
		\column{.5\textwidth}
			\centering
			\includegraphics[width=0.9\textwidth]{m1_ei_allResolutionsVSangle_20180601.pdf}
	\end{columns}
}

\subsection{Stereo Reconstruction}

\frame{\frametitle{Position Reconstruction with Stereo Readout}
	\begin{columns}
		\column{.65\textwidth}
			\centering
			\includegraphics[width=0.9\textwidth]{stereoBoards_angleNshift_centerNcoord-crop.pdf}
		\column{.35\textwidth}
			\centering
			\includegraphics[width=0.75\textwidth]{m1_stereo_posDifVSscinX_wCenter.pdf}
	\end{columns}
	
	\vspace{3mm}
	improve position information along strips (non-precision coordinate)
	\vspace{2mm}

	\begin{columns}
		\column{.5\textwidth}
			\centering
			\includegraphics[width=0.75\textwidth]{m1_si_resVSposAlongStripsByStereos_slopeX1e-2.pdf}
		\column{.5\textwidth}
			\centering
			\includegraphics[width=0.95\textwidth]{m1_si_resdiual_wNwoNewXtrack.pdf}
	\end{columns}
}

\section{Summary}

\frame{\frametitle{Summary}
	\footnotesize
	\begin{itemize}
		\item
			investigation of four-layered and segmented Micromegas Modules
			
			at the Munich Cosmic Ray Facility
			\vspace{3mm}
		\item
			correlation of pillar height and pulse height (accordingly efficiency)
			\vspace{3mm}
		\item
			alignment and strip shape reconstructed at a precision of $\sim$ \SI{20}{\micro\m}
			\vspace{3mm}
		\item
			resolution for perpendicular tracks $\sim$ \SI{90}{\micro\m}
			\vspace{3mm}
		\item
			improvement of position resolution for inclined tracks via drift time measurement
			\vspace{3mm}
		\item
			combination of the two layers with inclined strips 
			
			$\Rightarrow$ reconstruction of position perpendicular to precision direction
			\vspace{3mm}
	\end{itemize}
}

\appendix

\frame[noframenumbering]{
	\Huge
	\centering
	Backup
}

\frame{\frametitle{CRF systematics}

	systematics due to different placement in CRF
			
	$\Rightarrow$ investigate effect of CRF tracks
	\vspace{5mm}
	
	\centering
	\includegraphics[width=0.7\textwidth]{CRF_MDTdiffVSscinX_perBoard_2019-09.pdf}
}

\frame[noframenumbering]{\frametitle{\large LHC upgrade and Efficiency for the ATLAS Muon Spectrometer}
	\centering
	\includegraphics[width=0.8\textwidth]{HL_LHC_PlanUpdateJuly2015}
	\begin{columns}
		\column{.55\textwidth}
			\centering
			\includegraphics[width=0.95\textwidth]{ATLASnewer.jpg}
		\column{.45\textwidth}
			\centering
			\includegraphics[width=0.9\textwidth]{MDTefficiencyPerHitRate_NSW-TDR_edited.png}
	\end{columns}
}

\frame[noframenumbering]{\frametitle{Upgrade of the Inner Barrel End Caps}

	replacement of the current detector systems by small-strip Thin Gap Chamber (sTGC) and Micromegas quadruplets
	
	\vspace{3mm}
	\begin{columns}
		\column{.55\textwidth}
			\centering
			\textbf{New Small Wheel Sectors}
			\vspace{7mm}
			
			\includegraphics[width=1\textwidth]{sectorsDimensions.png}
		\column{.45\textwidth}
			\centering
			\textbf{Micromegas quadruplets}
			
			\includegraphics[width=1\textwidth]{sectorsNsides.png}
	\end{columns}
}

\frame[noframenumbering]{\frametitle{Influence of Muon Multiple Scattering}
	
	\begin{columns}
		\column{.5\textwidth}
			\centering
			\textbf{MDTs}
			
			\includegraphics[width=0.9\textwidth]{MDTresidualVSangle_slopeDifCuts.pdf}
			
			\includegraphics[width=0.9\textwidth]{MDTresidual_slopeDifCuts.pdf}
		\column{.5\textwidth}
			\centering
			\textbf{micromegas quadruplet (1 layer)}
			
			\includegraphics[width=0.9\textwidth]{SM2-M0_CRFafterH8_eta-out_residualVSangle_slopeDifCuts.pdf}
			
			\includegraphics[width=0.9\textwidth]{SM2-M0_CRFafterH8_eta-out_residual_cutsMDTslopeDif.pdf}
	\end{columns}
}

\end{document}
