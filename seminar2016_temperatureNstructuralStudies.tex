\documentclass{beamer}
%\usetheme{Montpellier}
\usetheme{CambridgeUS}
\usecolortheme{spruce}
\definecolor{green(pigment)}{rgb}{0.0, 0.65, 0.31}
\setbeamercolor*{item}{fg=green(pigment)}
\definecolor{Green}{rgb}{0.00, 1.00, 0.00}
\definecolor{Red}{rgb}{1.00, 0.00, 0.00}
\definecolor{Blue}{rgb}{0.00, 0.00, 1.00}
\usepackage[utf8]{inputenc}
\usepackage{amsmath}
\usepackage{amsfonts}
\usepackage{amssymb}
\usepackage[german]{babel}
\usepackage{graphicx}
\usepackage{rotating}
\usepackage{multirow,bigdelim,dcolumn,booktabs}
%\usepackage{beamerthemeshadow}
\usepackage{subfigure} 
%\beamersetuncovermixins{\opaqueness<1>{25}}{\opaqueness<2->{15}}
\beamertemplatenavigationsymbolsempty

\usepackage{tikz}
\usetikzlibrary{decorations.text}
\usetikzlibrary{trees}
\usetikzlibrary{decorations.pathmorphing}
\usetikzlibrary{decorations.markings}
\usetikzlibrary{patterns}

\graphicspath{
	{LatexBilder/}
	{LatexBilder/pics/}
	{LatexBilder/sketches/}
	{LatexBilder/Temp/}
	{LatexBilder/Temp/TempCalibration/}
	{LatexBilder/Temp/TempTest_20151030_1800/}
	{LatexBilder/Temp/TempTest_20151109_1550/}
	{LatexBilder/Temp/TempTest_20151113_1740/}
	{LatexBilder/Temp/bbrm_correction/}
	{LatexBilder/Temp/repeatedMeasurements/}
	{LatexBilder/ANSYS/deform_mesh/}
	{LatexBilder/ANSYS/L1/}
	{LatexBilder/ANSYS/MicroMeGas/}
	{LatexBilder/ANSYS/SM2_6int/}
	{LatexBilder/ANSYS/stiff/}
}

\begin{document}
\title{Construction of Large Area Micromegas}  
\author{ Maximilian Herrmann}
\date{22.06.2016} 

\part{1}

\frame{\titlepage
	\includegraphics[width=0.5\textwidth]{logo.jpg}
} 

\frame{\frametitle{Outline}\tableofcontents[part=1]}

%\section{Calibration of the Temperature Sensors} 

\section{Getting started with Temperature Measurements} 

\frame{\frametitle{Camparison of the Temperature Sensors} 
	\begin{columns}
		\column{.50\textwidth}
			\centering
			ATLAS Hall (front room)
			
			temperature vs time			
			%\vspace{3mm}
			\begin{columns}
				\column{.05\textwidth}
					{\scriptsize $ 23 ^\circ $C}
						
					\vspace{23mm}	
					%\begin{turn}{90}
					%	{\tiny
					%		temperature [$ ^\circ $C]
					%	}
					%\end{turn}
						
					{\scriptsize $ 20 ^\circ $C}
				\column{.95\textwidth}
					\includegraphics[width=\textwidth]{temp_calib_20151005_1225_tempvstime_SensAll_better.pdf}
					
					{\scriptsize 14h - 05.10. \hspace{16mm} 14h - 08.10.}
			\end{columns}
			 
			{\small
				\vspace{3mm}
				measurement accuracy : 
								
				$ \pm $0.5 $^{\circ}$C (from datasheet)
				
				$ \pm $0.06 $^{\circ}$C (from digitization)
			}
		\column{.50\textwidth}
			\centering
			temperature differences to mean
			
			\begin{columns}
				\column{.50\textwidth}
					\includegraphics[width=\textwidth]{temp_calib_20151005_1225_tempDiffToMean_Sens0.pdf}
				\column{.50\textwidth}
					\includegraphics[width=\textwidth]{temp_calib_20151005_1225_tempDiffToMean_Sens1.pdf}
			\end{columns}
			
			\begin{columns}
				\column{.50\textwidth}
					\includegraphics[width=\textwidth]{temp_calib_20151005_1225_tempDiffToMean_Sens2.pdf}
				\column{.50\textwidth}
					\includegraphics[width=\textwidth]{temp_calib_20151005_1225_tempDiffToMean_Sens3.pdf}
			\end{columns}
						
			\begin{columns}
				\column{.50\textwidth}
					\includegraphics[width=\textwidth]{temp_calib_20151005_1225_tempDiffToMean_Sens4.pdf}
				\column{.50\textwidth}
					\includegraphics[width=\textwidth]{temp_calib_20151005_1225_tempDiffToMean_Sens5.pdf}
			\end{columns}
			
			{\small
				\vspace{3mm}
				means and sigmas around 0.04 $^{\circ}$C 
				
				$\rightarrow$ below measurement accuracy
				\vspace{6mm}
			}
	\end{columns}
}

\frame{\frametitle{ATLAS Hall : granit table with CMM}
	\centering
	\begin{tikzpicture}
		\node[anchor=south west,inner sep=0] (image) at (0,0) {\includegraphics[width=\textwidth]{granittableleft.JPG}};
		\begin{scope}[x={(image.south east)},y={(image.north west)}]
			\draw[arrows=->,red,ultra thick](0.6,0)--(0.45,0.05);
			\draw[arrows=->,red,ultra thick](1,0.3)--(0.92,0.39);
			\draw[arrows=->,red,ultra thick](0.8,0.9)--(0.65,0.7);
			\draw[arrows=->,red,ultra thick](0.3,0.71)--(0.13,0.68);
			\draw[arrows=->,red,ultra thick](0.4,0.72)--(0.58,0.75);
			\draw[arrows=->,red,ultra thick](0.3,0.65)--(0.17,0.55);
			\draw[arrows=->,red,ultra thick](0.4,0.65)--(0.52,0.64);
		\end{scope}
	\end{tikzpicture}
}

\frame{\frametitle{Measurement Conditions} 
	\begin{columns}
		\column{.50\textwidth}
			\centering
			\includegraphics[angle=90, width=\textwidth]{granittable_CNC.pdf}
		\column{.50\textwidth}
			\centering
			\includegraphics[angle=90, width=\textwidth]{granittable_CNC_sideview.pdf}
	\end{columns}
	{\small
	\begin{tabular}{ccl}
		$ h_{g } \simeq 30 $cm &  & 
		\\
		$ h_{lh} \simeq 37 $cm & : & movable mounted for different height measurements $ \rightarrow h_{m} $ 
		\\
		$ h_{al} \simeq 68 $cm & : & the bridge stands on a linear bearing unit 
		\\
		$ h_{m } \simeq 2.5 $cm & : & measureing distance to laserhead (dynamic range 2mm) 
	\end{tabular}
	}
}

\frame{\frametitle{Calculated Thermal Expansion for our Setup} 
	\begin{columns}
		\column{.50\textwidth}
			\centering
			\includegraphics[angle=90, width=\textwidth]{granittable_CNC.pdf}
			
			linear thermal expansion coefficient for 
			\begin{tabular}{lcr}
				aluminum & : & 23.1 $\mu$m/m K
				\\	
				granite & : & 3\,-\,7 $\mu$m/m K
			\end{tabular}
		\column{.50\textwidth}
			\vspace{5mm}
			\begin{align*}
				 \bigtriangleup\!h_{m} &= {\color{green}+} \bigtriangleup\!h_{al} {\color{red}-} \bigtriangleup\!h_{lh} {\color{red}- \tfrac{1}{2}} \bigtriangleup\!h_{g} 
				 \\
				 \Rightarrow \; \bigtriangleup h_{m} / \mathrm{K} &= \left( h_{al} {\color{red}-} h_{lh} \right)  \alpha_{alu} {\color{red}- \tfrac{1}{2}} h_{g} \alpha_{granite}
			\end{align*}
			\begin{align*}
				\;\;\;\;\dfrac{\bigtriangleup\!h_{m}}{\mathrm{K}} & = 6 - 7\;\dfrac{\mu \mathrm{m}}{\mathrm{K}}
			\end{align*}	
			$\boxed{h_{corrected} = h_{measured} - \dfrac{\bigtriangleup h_{m}}{\mathrm{K}} \left( T - T_{ref} \right)}$	
	\end{columns}
}

\section{Reference Measurement for Temperature Correction} 

\frame{\frametitle{Measuring the Height at One Point} 
	\hspace{5.5cm}
	temperature vs time
	\vspace{1mm}
	\begin{columns}
		\column{.33\textwidth}
			\vspace{10mm}
			
			\centering	
			height vs time
			
			\includegraphics[width=\textwidth]{outerPCB_TempTest_20151030_1800_tempAtOnePoint_heightvstime.pdf}
						
			{\small
				$ \bigtriangleup h_{max}\;\simeq\;20 \mu $m 
				\vspace{13mm}
				
				slope : $ \bigtriangleup h / \bigtriangleup\!T $
			}
		\column{.33\textwidth}
			\centering
			sensor at the alu leg
			
			\includegraphics[width=\textwidth]{outerPCB_TempTest_20151030_1800_tempAtOnePoint_tempvstime_Sens0.pdf}
			
			{\small
				$ T \; : \; 19.6 ^{\circ}$C - $ 21.4 ^{\circ}$C
			}
			\vspace{2mm}
			
			\includegraphics[width=\textwidth]{outerPCB_TempTest_20151030_1800_tempAtOnePoint_heightandtempvstime_sens0.pdf}
			
			{\small
				11 $ \mu $m/ K
			}
		\column{.33\textwidth}
			\centering	
			sensor in the laser
			
			\includegraphics[width=\textwidth]{outerPCB_TempTest_20151030_1800_tempAtOnePoint_tempvstime_SensLH.pdf}
		
			{\small
				$ T \; : \; 26.25 ^{\circ}$C - $ 29 ^{\circ}$C
			}
			\vspace{2mm}
			
			\includegraphics[width=\textwidth]{outerPCB_TempTest_20151030_1800_tempAtOnePoint_heightandtempvstime_sensLH.pdf}
						
			{\small
				7.3 $ \mu $m/ K
			}
	\end{columns}
}

\frame{\frametitle{Height against Temperature} 
	\begin{columns}
		\column{.50\textwidth}
			\centering
			sensor at aluminum leg
		\column{.50\textwidth}
			\centering
			sensor in the laser head
	\end{columns}
	\centering
	height vs temperature
	\begin{columns}
		\column{.50\textwidth}
			\centering
			\includegraphics[width=0.7\textwidth]{outerPCB_TempTest_20151030_1800_tempAtOnePoint_heightvstemp_sens0_graph_fit.pdf}
		\column{.50\textwidth}
			\centering
			\includegraphics[width=0.7\textwidth]{outerPCB_TempTest_20151030_1800_tempAtOnePoint_heightvstemp_sensLH_graph_fit.pdf}
	\end{columns}
	\centering
	temperature projected along fitted function
	\begin{columns}
		\column{.50\textwidth}
			\centering
			\includegraphics[width=0.7\textwidth]{outerPCB_TempTest_20151030_1800_tempAtOnePoint_tempHeightCorr_Sens0.pdf}
		\column{.50\textwidth}
			\centering
			\includegraphics[width=0.7\textwidth]{outerPCB_TempTest_20151030_1800_tempAtOnePoint_tempHeightCorr_SensLH.pdf}
	\end{columns}
	\centering
	height increase per temperature
	\begin{columns}
		\column{.50\textwidth}
			\centering
			7.0 $ \mu $m / K
		\column{.50\textwidth}
			\centering
			5.9 $ \mu $m / K
	\end{columns}
}

\frame{\frametitle{Compairing Height Measurements (at One Point)} 
	\centering
	{\large height vs time}
	\vspace{5mm}
	
	\begin{columns}
		\column{.33\textwidth}
			\centering
			servo motors on
			\vspace{6mm}
		
			\includegraphics[width=\textwidth]{outerPCB_TempTest_20151030_1800_tempAtOnePoint_heightvstime_font.pdf}
			
			{\small
				$ \bigtriangleup h_{max}\;\simeq\;20 \mu $m 
				
				at $ \bigtriangleup T_{max}\;\simeq\;1.8 $K
			}
					
			\includegraphics[width=\textwidth]{outerPCB_TempTest_20151030_1800_tempAtOnePoint_heighttempcorrvstime_0.pdf}
		\column{.33\textwidth}
			\centering
			servo motors off
			\vspace{6mm}
			
			\includegraphics[width=\textwidth]{outerPCB_TempTest_20151109_1550_tempAtOnePoint_heightvstime_font.pdf}
		
			{\small
				$ \bigtriangleup h_{max}\;\simeq\;17 \mu $m 
				
				at $ \bigtriangleup T_{max}\;\simeq\;1.7 $K
			}
						
			\includegraphics[width=\textwidth]{outerPCB_TempTest_20151109_1550_tempAtOnePoint_heighttempcorrvstime_0.pdf}
		\column{.33\textwidth}
			\centering
			servo motors off and position prepared
		
			\includegraphics[width=\textwidth]{outerPCB_TempTest_20151113_1740_tempAtOnePoint_heightvstime_font.pdf}
					
			{\small							
				$ \bigtriangleup h_{max}\;\simeq\;12 \mu $m 
				
				at $ \bigtriangleup T_{max}\;\simeq\;1.3 $K
			}
					
			\includegraphics[width=\textwidth]{outerPCB_TempTest_20151113_1740_tempAtOnePoint_heighttempcorrvstime_0.pdf}
	\end{columns}
}

\frame{\frametitle{Compairing Height Measurements (at One Point)} 
	\begin{tabular}{l|ccc} 
		 & \multicolumn{3}{c}{slope from height-temperature fit [$\mu$m/K] }
		\\
		\hline
		 measurement & 18:00 30.10. & 15:50 09.11. & 17:40 13.11.
		\\
		\cline{1-1}
		sensor at &  & engine off & engine off, tape
		\\
		\hline
		granite (upper left) & 11 & 11 & 11
		\\
		granite (upper right) & 9 & 10 & 10
		\\
		granite (lower left) & 10 & 11 & 10
		\\
		granite (lower right) & 9 & 7 & 8
		\\
		alu (left leg) & 4 & 7 & 7
		\\
		alu (right leg) & 7 & 8 & 9
		\\
		alu (plate at LH) & 6 & 7 & 8
		\\
		laser head (LH) & 6 & 6 & 6
	\end{tabular}
}

\section{Temperature Correction and Comparison} 

\frame{\frametitle{Measurement of the Granite Table (from Ralph M\"uller)} 
	\begin{columns}
		\column{.50\textwidth}
			\centering
			Measurement 03.08.15
		\column{.50\textwidth}
			\centering
			Measurement 01.12.15
	\end{columns}
	\centering
	topology
	\begin{columns}
		\column{.50\textwidth}
			\centering
			\includegraphics[width=0.7\textwidth]{bbrm1_onPlane.pdf}
					
			$ \bigtriangleup\!h_{max} \simeq 400 \mu $m
		\column{.50\textwidth}
			\centering
			\includegraphics[width=0.7\textwidth]{bbrm2_onPlane.pdf}
					
			$ \bigtriangleup\!h_{max} \simeq 400 \mu $m
	\end{columns}
	\centering
	temperature vs time
	\begin{columns}
		\column{.50\textwidth}
			\centering
			\includegraphics[width=0.7\textwidth]{bbrm_tempvstime.pdf}
			
			$ \bigtriangleup\!T_{max} = 4 $K
		\column{.50\textwidth}
			\centering
			\includegraphics[width=0.7\textwidth]{bbrm2_tempvstime.pdf}
			
			$ \bigtriangleup\!T_{max} = 2.25 $K
	\end{columns}
}

\frame{\frametitle{Before and After Temperature Correction}
	\centering
	{\Large height difference} 
	\vspace{2mm}
	
	\begin{columns}
		\column{.50\textwidth}
			\centering
			before temperature correction
			
			$ h_{1,i} - h_{2,i} $
			%\vspace{15mm}

			\includegraphics[width=\textwidth]{heightDiff_fit.pdf}
							
			$ \sigma = 6.9 \mu $m
		\column{.50\textwidth}
			\centering
			after temperature correction 
			
			$ h_{1,i}^{corr} - h_{2,i}^{corr} $
			
			\includegraphics[width=\textwidth]{heightDiffTempCorr_fit.pdf}
											
			$ \sigma = 5.6 \mu $m
	\end{columns}
	\vspace{2.5mm}
	
	\centering
	with :  $ \;\; h_{j,i}^{corr} = h_{j,i} - \alpha \cdot \left( T_{j,i} - T_{j,ref} \right) $
						
	\hspace{17mm} \footnotesize ( $ \alpha : $ slope from height-temperature fit at one point )
}

\frame{\frametitle{Means of Repeated Height Measurements} 
	50 surface scans of the granite table with 1.5 h per measurement
	\vspace{3mm}
	
	\begin{columns}
		\column{.50\textwidth}
			\centering
			\includegraphics[width=0.7\textwidth]{repmeas_bbrm_1030_heightOnPlane_0_besser.pdf}
		\column{.50\textwidth}
			$ \rightarrow $ compare with reference measurement $ \left( h_{ref,i} - h_{j,i} \right) $ 
			and get mean and sigma of the difference (histogram fitted with Gaussian)
	\end{columns}
	
	\begin{columns}
		\column{.33\textwidth}
			%\vspace{3mm}
		
			\centering			
			{\small
				height mean vs repetition
			}
							
			\includegraphics[width=\textwidth]{repmeas_bbrm_1030_heightmeanvsrepetition_BIG_line_label.pdf}
		\column{.66\textwidth}
		%\begin{columns}
		%	\column{.50\textwidth}			
		%		\centering
		%	\column{.50\textwidth}		
		%		\centering	
		%\end{columns}
		\vspace{1mm}
		
		\centering
		{\small
			temperature mean vs repetition
		}
		\vspace{1mm}
		\begin{columns}
			\column{.50\textwidth}
				\centering
				{\small
					sensor at alu plate
				}
				\includegraphics[width=\textwidth]{repmeas_bbrm_1030_tempmeanvsrep_Alu_BIG_line_label.pdf}
			\column{.50\textwidth}
				\centering
				{\small
					sensor in the laser head
				}
				\includegraphics[width=\textwidth]{repmeas_bbrm_1030_tempmeanvsrep_LH_BIG_line_label.pdf}
		\end{columns}
	\end{columns}
}

\frame{\frametitle{Means and Sigmas after Temperature Correction} 
	\begin{columns}
		\column{.50\textwidth}	
			\centering			
			{\small
				sensor at aluminum plate
			}
		\column{.50\textwidth}	
			\centering			
			{\small
				sensor in the laser head
			}
	\end{columns}
	\vspace{3mm}
	\centering
	height mean temperature corrected vs repetition
	\begin{columns}
		\column{.50\textwidth}
			\centering
			\includegraphics[width=0.7\textwidth]{repmeas_bbrm_1030_heightmeanvsrepetition_Alu_BIG_line_label.pdf}
		\column{.50\textwidth}
			\centering
			\includegraphics[width=0.7\textwidth]{repmeas_bbrm_1030_heightmeanvsrepetition_LH_BIG_line_label.pdf}
	\end{columns}
	\vspace{3mm}
	\centering
	height sigma difference : $ \sigma_{j} - \sigma_{j}^{corr} $
	\begin{columns}
		\column{.50\textwidth}
			\centering
			\includegraphics[width=0.7\textwidth]{repmeas_bbrm_1030_heightsigmadiff_Alu_marked.pdf}
		\column{.50\textwidth}
			\centering
			\includegraphics[width=0.7\textwidth]{repmeas_bbrm_1030_heightsigmadiff_LH_marked.pdf}
	\end{columns}
}

\section{Summery and Outlook} 

\frame{\frametitle{Summery and Outlook} 
	summary : 
	\begin{itemize}
				\item
					height difference at the border of our measurement accuracy (5-15$\mu$m) for observed temperature changes (2$^\circ$C/day)
				\item
					temperature correction possible
				\item
					only useful for measurements over extended times (and with higher temperature differences)
	\end{itemize}
	outlook : 
	\begin{itemize}
				\item
					measure at different reference points and compare their temperature correction possibilities
				\item
					repeat temperature sensor calibration measurement with all used sensors
				\item
					 calculate errors
	\end{itemize}
}

\part{2}

\setcounter{section}{0}

\title{Structural and Electric Simulations of Detector Parts with ANSYS}

\frame{\titlepage} 

\frame{\frametitle{Outline}\tableofcontents[part=2]}

\section{Introduction to ANSYS} 

\frame{\frametitle{Making Simulations with ANSYS} 
	\begin{columns}
		\column{.50\textwidth}
			simulations in 3 steps : 
			\begin{enumerate}[I.]
				\item
					preprocessing
				\item
					solving
				\item
					postprocessing
			\end{enumerate}
		\column{.50\textwidth}
			$\rightarrow$ preprocessing
			\begin{enumerate}[a.]
				{\small
					\item
						choose elements specified for your problem
					\item
						define material constants
					\item
						setup geometry $\rightarrow$ assign element specifications and material constants to volumes, areas etc.
					\item
						meshing $\rightarrow$ get a finite element model (FEM)
					\item
						define constraints / forces ... on nodes of the FEM
				}
			\end{enumerate}
	\end{columns}
}

\section{Stiffback Deformation due to Weight} 

\frame{\frametitle{Stiffback}
	\begin{columns}
		\column{.50\textwidth}
			\centering
			\includegraphics[angle=90, width=\textwidth]{stiffback_sizes.pdf}
			
			\includegraphics[height=3cm]{stiffback_beamprofile-crop.pdf}
		\column{.50\textwidth}
			\begin{itemize}
			\item
				layered structure 
				
				$ \rightarrow $ SHELL element
				\vspace{5mm}
				
				\begin{tabular}{lcr}
					aluminum plate & : & 1.12mm
					\\
					glue & : & 0.1mm
					\\
					honeycomb & : & 80mm
					\\
					glue & : & 0.1mm
					\\
					aluminum plate & : & 1.12mm
				\end{tabular}
				\vspace{5mm}
				
			\item
				borders aluminum frame 
				
				$ \rightarrow $ BEAM element
								
				approximate with a hollow rectangle profile
			\end{itemize}
	\end{columns}
}

\frame{\frametitle{Stiffback - on One Distance Piece} 
	\begin{columns}
		\column{.50\textwidth}
			\centering
			\includegraphics[width=\textwidth]{stiff4_centroiddisall_grav_beamfullrect_bigfont.png}
					
			\centering			
			{\small
				stiffback - simulated deformation
				
				$ \bigtriangleup h_{max}\;=\;117 \mu $m
			}
		\column{.50\textwidth}
			\centering
			\includegraphics[width=\textwidth]{stiffback_9distpieces.pdf}
						
			\centering			
			{\small
				(from Ralph M\"uller)
				\vspace{2mm}
				
				stiffback - measured deformation
								
				$ \bigtriangleup h_{max}\;\simeq\;170 \mu $m
			}
	\end{columns}
}

\frame{\frametitle{Stiffback - on Several Distance Pieces} 
	\begin{columns}
		\column{.33\textwidth}
			\centering
			\includegraphics[width=\textwidth]{stiff4_8holdingsdisZ_centroiddisXYRZ_grav_bigfont.png}
					
			\centering			
			{\small
				stiffback - simulated on eight distance pieces
				\vspace{4mm}
				
				$ \bigtriangleup h_{max}\;=\;12 \mu $m
			}
		\column{.33\textwidth}
			\centering
			\includegraphics[width=\textwidth]{stiff4_cornersdisZ_centroiddisXYRZ_grav_hrec_bigfont.png}
						
			\centering			
			{\small
				stiffback - simulated on four distance pieces (at the corners)
								
				$ \bigtriangleup h_{max}\;=\;88 \mu $m
			}
		\column{.33\textwidth}
			\centering
			\includegraphics[width=\textwidth]{stiff4_4holdingsdisZ_centroiddisXYRZ_grav_hrec_bigfont.png}
								
			\centering			
			{\small			
				stiffback - simulated on four distance pieces (at the borders)
											
				$ \bigtriangleup h_{max}\;=\;19 \mu $m
			}
	\end{columns}
}

\section{SM2 with 6 Interconnections Deformation due to Pressure} 

\frame{\frametitle{SM2}
	\begin{columns}
		\column{.50\textwidth}
			\centering
			\includegraphics[angle=90, width=1.2\textwidth]{SM2_6int_sizes-crop.pdf}
			
			\vspace{3mm}
			\includegraphics[angle=90, width=\textwidth]{SM2_6int_beamprofile-crop.pdf}
		\column{.50\textwidth}
			\begin{itemize}
			\item
				layered structure 
				
				$ \rightarrow $ SHELL element
				\vspace{5mm}
				
				\begin{tabular}{lcr}
					FR4 & : & 0.5mm
					\\
					glue & : & 0.1mm
					\\
					honeycomb & : & 10mm
					\\
					glue & : & 0.1mm
					\\
					FR4 & : & 0.5mm
				\end{tabular}
				\vspace{5mm}
				
			\item
				frame and diagonal struts aluminum bars 
				
				$ \rightarrow $ BEAM element
				
				approximate with a hollow rectangle profile
			\end{itemize}
	\end{columns}
}

\frame{\frametitle{SM2 with 6 Interconnections - after Meshing and Deformed} 
	\begin{columns}
		\column{.50\textwidth}
			\centering
			\includegraphics[width=\textwidth]{SM2_6int_3_mesh.png}
					
			\centering			
			{\small
				SM2 with 6 interconnections - meshed
				\vspace{4mm}
			}
		\column{.50\textwidth}
			\centering
			\includegraphics[width=\textwidth]{SM2_6int_3_nodesdisall_bordersdisall_press200_beam3.png}
						
			\centering			
			{\small
				SM2 with 6 interconnections - deformed due to pressure
								
				$ \bigtriangleup h_{max}\;=\;69 \mu $m
			}
	\end{columns}
}

\section{Panel Deformation due to Mesh Tension} 

\frame{\frametitle{Panel - model for Mesh Tension}
	\begin{columns}
		\column{.50\textwidth}
			\centering
			\includegraphics[width=\textwidth]{panel_bending_parts.pdf}
		\column{.50\textwidth}
			mesh tension : $ 10 \tfrac{\mathrm{N}}{\mathrm{cm}} $
			
			$ \rightarrow $ bending moment per strip : 
			\begin{align*}
				M &= F \cdot d
				\\
				&= 10 \mathrm{N} \cdot 1 \mathrm{cm} = 0.1 \mathrm{Nm}
			\end{align*}
			
	\end{columns}
}

\frame{\frametitle{Panel - Simplified Model} 
	\begin{columns}
		\column{.50\textwidth}
			\centering
			{\small simplifying the model of the panel}
			\vspace{3mm}
			
			\includegraphics[angle=90, width=\textwidth]{deform_mesh_simplemodel-crop.pdf}
			
			\vspace{3mm}	
			{\small
				bending moment acts as force over lever arm on the inner of the panel
			}
		\column{.50\textwidth}
			\vspace{4mm}
			\centering
			\includegraphics[width=\textwidth]{deform_mesh1_linesdisZ_cornernodedisXYRZ_forcelength100divnodes_grav_bigfont.png}
						
			\centering			
			{\small
				panel - simulated deformation
								
				$ \bigtriangleup h_{max}\;=\;1.55 $mm
			}
	\end{columns}
}

\frame{\frametitle{Panel - Better Model, Position Dependend Force} 
	\begin{columns}
		\column{.50\textwidth}
			\centering
			\includegraphics[width=\textwidth]{deform_mesh2_MESH.png}
					
			\centering			
			{\small
				panel - meshed, with boundary conditions
			}
		\column{.50\textwidth}
			\centering
			\includegraphics[width=\textwidth]{deform_mesh2_linesdisXYZ_tableforcelength_grav_bigfont.png}
						
			\centering			
			{\small
				panel - simulated deformation
								
				$ \bigtriangleup h_{max}\;=\;1.90 $mm
			}
	\end{columns}
}

\frame{\frametitle{Panel - Evolved Model, Bending Moment instead of Force} 
	\begin{columns}
		\column{.50\textwidth}
			\centering
			\includegraphics[width=\textwidth]{deform_mesh6_MESH.png}
					
			\centering			
			{\small
				panel - meshed, with boundary conditions
			}
		\column{.50\textwidth}
			\centering
			\includegraphics[width=\textwidth]{deform_mesh6_linsdisZ_centroiddisXYRZ_moment_grav_bigfont.png}
						
			\centering			
			{\small
				panel - simulated deformation
								
				$ \bigtriangleup h_{max}\;=\;1.63 $mm
			}
	\end{columns}
}

\frame{\frametitle{Panel - Measured Deformation} 
	\begin{columns}
		\column{.50\textwidth}
		\column{.50\textwidth}
			\centering
			\includegraphics[width=\textwidth]{paneldeformation_meshtension_measured.pdf}
						
			\centering			
			{\small
				(from Ralph M\"uller)
				\par\smallskip
				panel - measured deformation
								
				$ \bigtriangleup h_{max}\;=\;1.8 $mm
			}
	\end{columns}
}

\section{L1 Deformation due to Pressure} 

\frame{\frametitle{L1 - Setup}
	\begin{columns}
		\column{.50\textwidth}
			\centering
			L1 detector supported by two beams
			
			\includegraphics[width=\textwidth]{L1_sizes.pdf}
		\column{.50\textwidth}
			layered structure 
					
			$ \rightarrow $ SHELL element : 
			\vspace{3mm}
					
			\begin{tabular}{lcrc}
				FR4 & : & 1mm & \rdelim\}{3}{17.5mm}[\parbox{12.5mm}{top\\side}]
				\\
				honeycomb & : & 10mm & 
				\\
				FR4 & : & 1mm
				\\
				vacuum & : & 5mm & 
				\\
				FR4 & : & 2mm & \rdelim\}{4}{17.5mm}[\parbox{12.5mm}{bottom\\side}]
				\\
				aluminum & : & 2mm & 
				\\
				honeycomb & : & 10mm & 
				\\
				FR4 & : & 1mm & 
			\end{tabular}
	\end{columns}
}

\frame{\frametitle{L1 - bottom and top side} 
	\begin{columns}
		\column{.50\textwidth}
			\centering
			{\small 
				L1 - top
			}
			
			\includegraphics[width=0.7\textwidth]{L1_deformup_linesdisall_press1000_grav_bigfont.png}
						
			\centering			
			{\small				
				$ \bigtriangleup h_{simulated}\;=\;1.5 $mm
			}
			\vspace{6mm}
			
			\includegraphics[width=0.7\textwidth]{L1_deformation_znew_richtig.pdf}
		\column{.50\textwidth}
			\centering
			{\small 
				L1 - bottom
			}
			
			\includegraphics[width=0.7\textwidth]{L1_deformdown_linesdisall_press1000_grav_bigfont.png}
					
			\centering			
			{\small	
				$ \bigtriangleup h_{simulated}\;=\;311 \mu $m
			}
			\vspace{6mm}
						
			{\small
				L1 - deformation measured by data in cosmic ray facility 
				
				(from Philipp L\"osel)
															
				$ \bigtriangleup h_{measured}\;\simeq\;2 \cdot 0.7 $mm
			}
			\vspace{5mm}
	\end{columns}
}

\section{MicroMeGas : Strip Structure with Voltages and Charges} 

\frame{\frametitle{MicroMeGas - Unit Cell of a 2D Strip Structure with Height} 
	\begin{columns}
		\column{.50\textwidth}
			\centering
			sizes of a unit cell
			
			\includegraphics[width=\textwidth]{MicroMeGas_topview.pdf}
			
			{\small
			\begin{tabular}{lcr}
				gas (neon C0$ _2 $) & : & 125 $ \mu $m
				\\
				floating strip (and gas) & : & 35 $ \mu $m
				\\
				FR4 & : & 25 $ \mu $m
				\\
				X strip (and FR4) & : & 35 $ \mu $m
				\\
				FR4 & : & 25 $ \mu $m
				\\
				Y strip (and gas) & : & 35 $ \mu $m
			\end{tabular}
			}
		\column{.50\textwidth}
			\centering
			volume model
			
			\includegraphics[width=\textwidth]{MicroMeGas_com91_volumesColor.png}
							
			{\small
				MicroMeGas - unitcell : 
				\vspace{2mm}
				
				\begin{tabular}{c@{ : }c@{ $ \rightarrow $ }c}
					strips & copper & orange   
					\\
					FR4 & kapton & yellow  
					\\
					gas & neon CO$ _2$ & green  
				\end{tabular}			
			}
	\end{columns}
}

\frame{\frametitle{MicroMeGas - Calculating the Capacities}
	\begin{columns}
		\column{.50\textwidth}
			\centering
			\includegraphics[width=\textwidth]{MicroMeGas_topview_areas_letters.pdf}
			
			%{\small model for calculating the capacity between the floating- and the Y-strip}	
			\includegraphics[width=\textwidth]{MicroMeGas_capacities2.pdf}
		\column{.50\textwidth}
			formulas for capacities : 
			\begin{align}
				C & = \dfrac{Q}{U}
				\label{chargeOverVolt}
				\\
				C & = \epsilon_{0} \epsilon_{r} \dfrac{A}{d}
				\label{areaOverDist}
			\end{align}
			
			in our case : $ \epsilon_{r} = \epsilon_{kapton} = 4 $ 
			\vspace{3mm}
			
			{\small
			\begin{tabular}{lcc}
				combination & areas & distances
				\\
				\hline
				float - X & A, C1, C2 & 25 $ \mu $m
				\\
				X - Y & A & 25 $ \mu $m
				\\
				float - Y & B1, B2 & 85 $ \mu $m
				\\
				 & 2 $\times$ A & 25 $ \mu $m
			\end{tabular}
			}
	\end{columns}
}

\frame{\frametitle{MicroMeGas - Calculated and Simulated Capacities} 
	\begin{columns}
		\column{.33\textwidth}
			\centering
			\includegraphics[width=\textwidth]{MicroMeGas_com91_charge_NegBotFloat_PosTopX_NegBotX_PosTopY_voltages.png}
									
			\centering			
			{\small
				MicroMeGas - Voltages 
				\vspace{3mm} 
							
				$ 10^6 $ negative charges on floating strip and positive on Y strip
			}
		\column{.66\textwidth}
		\begin{columns}
			\column{.50\textwidth}	
				{\small
					calculated capacities (\ref{areaOverDist}) 
					%\vspace{1mm}
					\begin{align*}		
						C_{fx} &= 5.7 \cdot 10^{-14} \mathrm{F} 
						\\			
						C_{xy} &= 4.5 \cdot 10^{-14} \mathrm{F}  
						\\			
						C_{fy} &= 6.0 \cdot 10^{-14} \mathrm{F} 
					\end{align*}	
				}
			\column{.50\textwidth}	
				{\small
					simulated capacities (\ref{chargeOverVolt}) 
					%\vspace{1mm}
					\begin{align*}		
						C_{fx} &= 6.2 \cdot 10^{-14} \mathrm{F} 
						\\			
						C_{xy} &= 5.1 \cdot 10^{-14} \mathrm{F}  
						\\			
						C_{fy} &= 4.7 \cdot 10^{-14} \mathrm{F} 
					\end{align*}	
				}
		\end{columns}
		different capacities ratios as expected 
		
		$ \rightarrow $ hint for observed signal behavior
	\end{columns}
}

\section{Outlook} 

\frame{\frametitle{Outlook} 
	\begin{itemize}
				\item
					 calculate the current / fluxes in the strips 
				\item
					 expand the model to several unit cells 
				\item
					 build MicroMeGas
	\end{itemize}
	\begin{figure}
	{\Large
		Thank you 
		\par\bigskip
		for your attention!
	}
	\end{figure}
}

\frame{\frametitle{Additional : Simultaneously Measureing Temperature and Height} 
	\begin{columns}
		\column{.33\textwidth}
			\includegraphics[width=\textwidth]{outerPCB_TempTest_20151021_1823_bbrm_tempvstime.pdf}
	
			\centering					
			{\small
				temperature vs time
			}
		\column{.33\textwidth}
			\includegraphics[width=\textwidth]{outerPCB_TempTest_20151021_1823_bbrm_tempOnPlane.pdf}

			\centering			
			{\small
				temperature on plane
			}
		\column{.33\textwidth}
			\includegraphics[width=\textwidth]{outerPCB_TempTest_20151021_1823_bbrm_topologyHeightCorr.pdf}

			\centering			
			{\small
				height on plane
			}
	\end{columns}
}

\frame{\frametitle{Additional : Measureing the Height at one Point (Engine turned off)} 
	\begin{columns}
		\column{.33\textwidth}
			\includegraphics[width=\textwidth]{outerPCB_TempTest_20151109_1550_tempAtOnePoint_tempvstime_Sens0.pdf}

			\centering			
			{\tiny
				height and temperature vs time
				\smallskip
				(sensor 0)
			}
		\column{.33\textwidth}
			\includegraphics[width=\textwidth]{outerPCB_TempTest_20151109_1550_tempAtOnePoint_tempvstime_SensLH.pdf}

			\centering			
			{\tiny
				height and temperature vs time
				\smallskip
				(sensor at laser head)
			}
		\column{.33\textwidth}
			\includegraphics[width=\textwidth]{outerPCB_TempTest_20151109_1550_tempAtOnePoint_heightvstime.pdf}
			
			\centering			
			{\tiny
				height vs time
				\vspace{6mm}
			}
	\end{columns}
	\begin{columns}
		\column{.50\textwidth}
			\centering
			\includegraphics[width=.7\textwidth]{outerPCB_TempTest_20151109_1550_tempAtOnePoint_heightandtempvstime_sens0.pdf}

			\centering			
			{\tiny
				height and temperature vs time
				\smallskip
				(sensor 0)
			}
		\column{.50\textwidth}
			\centering
			\includegraphics[width=.7\textwidth]{outerPCB_TempTest_20151109_1550_tempAtOnePoint_heightandtempvstime_sensLH.pdf}

			\centering			
			{\tiny
				height and temperature vs time
				\smallskip		
				(sensor at laser head)
			}
	\end{columns}
}

\frame{\frametitle{Additional : ploting Height against Temperature (Engine turned off)} 
	\begin{columns}
		\column{.50\textwidth}
			\centering
			\includegraphics[width=0.7\textwidth]{outerPCB_TempTest_20151109_1550_tempAtOnePoint_heightvstemp_sens0_graph_fit.pdf}

			\centering
			\includegraphics[width=0.7\textwidth]{outerPCB_TempTest_20151109_1550_tempAtOnePoint_heightvstemp_sens0_hist.pdf}
			
			\centering			
			{\small
				height vs temperature
				
				(sensor 0)
			}
		\column{.50\textwidth}
			\centering
			\includegraphics[width=0.7\textwidth]{outerPCB_TempTest_20151109_1550_tempAtOnePoint_heightvstemp_sensLH_graph_fit.pdf}

			\centering
			\includegraphics[width=0.7\textwidth]{outerPCB_TempTest_20151109_1550_tempAtOnePoint_heightvstemp_sensLH_hist.pdf}
			
			\centering			
			{\small
				height vs temperature
							
				(sensor at laser head)
			}
	\end{columns}
}

\frame{\frametitle{Additional : Measureing the Height at one Point (Engine turned off and Position prepaired)} 
	\begin{columns}
		\column{.33\textwidth}
			\includegraphics[width=\textwidth]{outerPCB_TempTest_20151113_1740_tempAtOnePoint_tempvstime_Sens0.pdf}

			\centering			
			{\tiny
				height and temperature vs time
				\smallskip
				(sensor 0)
			}
		\column{.33\textwidth}
			\includegraphics[width=\textwidth]{outerPCB_TempTest_20151113_1740_tempAtOnePoint_tempvstime_SensLH.pdf}

			\centering			
			{\tiny
				height and temperature vs time
				\smallskip
				(sensor at laser head)
			}
		\column{.33\textwidth}
			\includegraphics[width=\textwidth]{outerPCB_TempTest_20151113_1740_tempAtOnePoint_heightvstime.pdf}
			
			\centering			
			{\tiny
				height vs time
				\vspace{6mm}
			}
	\end{columns}
	\begin{columns}
		\column{.50\textwidth}
			\centering
			\includegraphics[width=.7\textwidth]{outerPCB_TempTest_20151113_1740_tempAtOnePoint_heightandtempvstime_sens0.pdf}

			\centering			
			{\tiny
				height and temperature vs time
				\smallskip
				(sensor 0)
			}
		\column{.50\textwidth}
			\centering
			\includegraphics[width=.7\textwidth]{outerPCB_TempTest_20151113_1740_tempAtOnePoint_heightandtempvstime_sensLH.pdf}

			\centering			
			{\tiny
				height and temperature vs time
				\smallskip		
				(sensor at laser head)
			}
	\end{columns}
}

\frame{\frametitle{Additional : ploting Height against Temperature (Engine turned off and Position prepaired)} 
	\begin{columns}
		\column{.50\textwidth}
			\centering
			\includegraphics[width=0.7\textwidth]{outerPCB_TempTest_20151113_1740_tempAtOnePoint_heightvstemp_sens0_graph_fit.pdf}

			\centering
			\includegraphics[width=0.7\textwidth]{outerPCB_TempTest_20151113_1740_tempAtOnePoint_heightvstemp_sens0_hist.pdf}
			
			\centering			
			{\small
				height vs temperature
				
				(sensor 0)
			}
		\column{.50\textwidth}
			\centering
			\includegraphics[width=0.7\textwidth]{outerPCB_TempTest_20151113_1740_tempAtOnePoint_heightvstemp_sensLH_graph_fit.pdf}

			\centering
			\includegraphics[width=0.7\textwidth]{outerPCB_TempTest_20151113_1740_tempAtOnePoint_heightvstemp_sensLH_hist.pdf}
			
			\centering			
			{\small
				height vs temperature
							
				(sensor at laser head)
			}
	\end{columns}
}

\end{document}
