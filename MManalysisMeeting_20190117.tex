\documentclass[usenamees,dvipsnames]{beamer}
\usetheme{Madrid}
\usecolortheme{spruce}
\definecolor{green(pigment)}{rgb}{0.0, 0.65, 0.31}
\setbeamercolor*{item}{fg=green(pigment)}
\definecolor{Green}{rgb}{0.00, 1.00, 0.00}
\definecolor{Red}{rgb}{1.00, 0.00, 0.00}
\definecolor{Blue}{rgb}{0.00, 0.00, 1.00}
\usepackage[utf8]{inputenc}
\usepackage{amsmath}
\usepackage{amsfonts}
\usepackage{amssymb}
\usepackage[german]{babel}
\usepackage{graphicx}
\usepackage{rotating}
\usepackage{textcomp}
\usepackage{multirow,bigdelim,dcolumn,booktabs}
%\usepackage{beamerthemeshadow}
\usepackage{subfigure} 
\usepackage{siunitx}
\usepackage{appendixnumberbeamer}
\usepackage{hyperref}
\usepackage{mdframed}
\usepackage{xcolor}
\usepackage[absolute,overlay]{textpos}
%\beamersetuncovermixins{\opaqueness<1>{25}}{\opaqueness<2->{15}}
\beamertemplatenavigationsymbolsempty

\usepackage{tikz}
\usetikzlibrary{decorations.text}
\usetikzlibrary{trees}
\usetikzlibrary{decorations.pathmorphing}
\usetikzlibrary{decorations.markings}
\usetikzlibrary{patterns}

\graphicspath{
	{pictures/}
%	{/home/m/Maximilian.Herrmann/Bilder/forLatex/plots/}
%	{/home/m/Maximilian.Herrmann/Bilder/forLatex/sketches/}
%	{/home/m/Maximilian.Herrmann/Bilder/forLatex/pictures/}
}

\begin{document}
\title[SM2 in CRF]{SM2 Micromegas in the Cosmic Ray Facility}  
\author[M. Herrmann]{ Maximilian Herrmann}
\institute[LMU Munich]{Ludwig-Maximilians-Universit\"at M\"unchen - Lehrstuhl Schaile}
\date[17.01.2019]{MM Analysis Meeting 17.01.2019} 

\frame{\titlepage} 

\frame{\frametitle{Outline}
	\tableofcontents
} 

\section{Measurement Setup \& Data Processing}

\frame{\frametitle{CRF Trigger Scheme and Data Streams}
		
	\begin{columns}
		\column{.50\textwidth}
			\centering
			trigger logic
			\vspace{2mm}
			
			\includegraphics[width=0.95\textwidth]{CRFtriggerScheme-crop.pdf}
		\column{.50\textwidth}
			\centering
			data streams
			\vspace{2mm}
			
			\includegraphics[width=0.95\textwidth]{CRFdataStreams-crop.pdf}
	\end{columns}
	\vspace{3mm}
	\footnotesize
	
	\begin{itemize}
		\item
			MDT and Scintillator data recorded separately (with TDC data)
		\item
			UDP packages for each FEC collected and online zerosuppressed
	\end{itemize}
	
	\centering
	$\Rightarrow$ data stream alignment crucial
		
	\begin{itemize}
		\item
			MDT and Scintillator analysis $\Rightarrow$ root trees with 3D track
		\item
			sorting of APV root trees for combination with MDT data
	\end{itemize}
}

\frame{\frametitle{CRF Data - Reference Tracks}
		
	\begin{columns}
		\column{.50\textwidth}
			\centering
			hit distribution
			\vspace{2mm}
			
			\includegraphics[width=0.9\textwidth]{CRF_hits.pdf}
		\column{.50\textwidth}
			\centering
			slope distribution
			\vspace{2mm}
			
			\includegraphics[width=0.9\textwidth]{CRF_slopes.pdf}
	\end{columns}
	
	\footnotesize
	\begin{itemize}
		\item
			coarse resolution in X due to \SI{10}{cm} scintillators
		\item
			missing upper part of Y axis due to Phase 2 tests (C. Valderanis)
		\item
			separate segments in X due to trigger-logic
		\item
			not centered slope distribution 
			
			due to missing area (Y) and scintillator placement (X)
	\end{itemize}
	
}

\frame{\frametitle{Analysis 3 Step Approach\only<2,3>{ - 1. Step}\only<4>{ - 2. Step}\only<5>{ - 3. Step}}
	\begin{itemize}
		\item[1]
			\only<1,4,5>{signal fit and clustering}
			\only<2,3>{\textbf{signal fit and clustering}}
		\item[2]
			\only<1,2,3,5>{tracking}
			\only<4>{\textbf{tracking}}
		\item[3]
			\only<1,2,3,4>{evaluation of variable distributions}
			\only<5>{\textbf{evaluation of variable distributions}}
	\end{itemize}
	\footnotesize
	\vspace{4mm}
	
	\only<2,3>{MM data tree}
	
	\only<2>{
		$\Rightarrow$ strip data
		\begin{itemize}
			\item
				check for maximum charge
			\item
				inverse Fermi fit
				
				$\Rightarrow$ discard noise
				
				$\Rightarrow$ evaluate timing
			\item
				clustering
		\end{itemize}
	}
	
	
	\only<3>{
		$\Rightarrow$ cluster data
		\begin{itemize}
			\item
				calculate basic properties:
				\begin{itemize}
					\footnotesize
					\item
						charge sum
					\item
						centroid
					\item
						first and last signaltime
					\item
						charge averaged signaltime
				\end{itemize}
			\item
				\textmu TPC fit
				\begin{itemize}
					\footnotesize
					\item
						time (in 25 ns) VS stripnumber (without pitch)
					\item
						time error from fermi fit
					\item
						position error \`a la Kostas
				\end{itemize}
		\end{itemize}
	}		
		
	\only<2,3>{$\Rightarrow$ fill basic histograms and save trees to be further processed}
	
	\only<4>{
		for each layer
		\begin{itemize}
			\item
				select leading cluster
			\item
				calculate residual:
				\begin{itemize}
					\footnotesize
					\item
						evaluate CRF track at layer
					\item	
						transform the layer hit into the CRF coordinate system
				\end{itemize}
				$\Rightarrow$ residual = CT( centroid $\cdot$ pitch ) - track @ layer
			\item
				fill residuals as function of:
				\begin{itemize}
					\footnotesize
					\item
						reference track slope (by MDTs)
					\item
						position (precision and non-precision)
					\item
						charge averaged signaltime (separate for angles)
				\end{itemize}
		\end{itemize}
	}
	
	\only<5>{
		alignment
		\begin{itemize}
			\footnotesize
			\item
				shift : residual mean 
			\item
				rotation : residual as function of non-precision coordinate
			\item
				shift perpendicular to anode : 
				
				slope of residual VS track slope distribution
			\item
				other alignment parameter explained on next slides
		\end{itemize}
		
		repeat step 2 and 3 until data is aligned ( $\hat{=}$ residual width minimized)
		
		afterwards evaluate timing , charge and efficiency
	}
}

\section{3D Alignment}

\frame{\frametitle{Concept of 3D Alignment with 1D Detector}
	\begin{columns}
		\column{.5\textwidth}
			\centering
			\includegraphics[width=\textwidth]{CRF_tracks-crop.pdf}
			\hspace{2mm}
		\column{.5\textwidth}
			\footnotesize
			rotation matrix to describe active area
			
			$\mathbf{R} = 
			\mathbf{R}_{z}(\alpha_z) \cdot \mathbf{R}_{y}(\alpha_y) \cdot \mathbf{R}_{x}(\alpha_x)$
			
			(in combination with detector origin $\vec{d}$)
			
			\vspace{3mm}
			intersection of line with plane (at $\vec{p}$)
			
			$ \mathbf{R} \left[ \vec{p} - \vec{d} \right] \stackrel{!}= \vec{0} $
			\vspace{2mm}
			
			$ \mathbf{R} \left[ 
				\begin{pmatrix} m_x \\ m_y \\ 1 \end{pmatrix} \cdot z + \begin{pmatrix} t_x \\ t_y \\ 0 \end{pmatrix} - \vec{d}\; 
			\right] \stackrel{!}= \vec{0}$
			\vspace{2mm}
			
			$\Rightarrow$ z calculated for each track
	\end{columns}
	\vspace{6mm}

	residual is calculated with respect to Y coordinate (MDT precision) 
	\vspace{2mm}
	
	$\Rightarrow$ plot as function of $m_y$ to extract z position of detector
	
}

\frame{\frametitle{Alignment using Reference Tracks}
	\centering 
	\textbf{Concept:}
	\begin{columns}
		\column{.50\textwidth}
			\centering
			\includegraphics[width=0.90\textwidth]{positionshift2-crop.pdf}
		\column{.50\textwidth}
			\centering
			\includegraphics[width=0.90\textwidth]{verticalshift-crop.pdf}
	\end{columns}
	\vspace{2mm}
	\textbf{Implementation:}
	\begin{columns}
		\column{.55\textwidth}
			\centering
			\footnotesize before alignment
			
			\includegraphics[width=0.75\textwidth]{resVSslope_blue.pdf}
		\column{.45\textwidth}
			\small
			\textbf{residual} = pos\textsubscript{measured} - pos\textsubscript{reference}
%					\textbf{residual} = $\mathrm{pos}_{\mathrm{measured}} - \mathrm{pos}_{\mathrm{reference}}$

			\vspace{3mm}
			$\Rightarrow$ residual vs. slope {\scriptsize(reference track)}
	 
	 		\vspace{3mm}
			$\Rightarrow$ {\color{red}linear fit}
			
	 		\vspace{3mm}
			shift\textsubscript{horizontal} = intercept\textsubscript{fit}
%					$\mathrm{shift}_{\mathrm{horizontal}} = \mathrm{intercept}_{\mathrm{fit}}$

			\vspace{3mm}
			shift\textsubscript{vertical} = slope\textsubscript{fit}	
%					$\mathrm{shift}_{\mathrm{vertical}} \;\;\; = \mathrm{slope}_{\mathrm{fit}}$
	\end{columns}
}

\frame{\frametitle{Impact of 3D Alignment}
	\begin{columns}
		\column{.5\textwidth}
			\centering
			before
			
			\includegraphics[width=0.65\textwidth]{m3_ei_deltaZ_beforeAlignment.pdf}
		\column{.5\textwidth}
			\centering
			after
			
			\includegraphics[width=0.65\textwidth]{m3_ei_deltaZ_afterAlignment_out.pdf}
	\end{columns}
	
	\vspace{2mm}
	\begin{columns}
		\column{.5\textwidth}
			\centering
			rotation around strips
			
			\includegraphics[width=0.7\textwidth]{m3_ei_deltaZvsX_beforeAlignment.pdf}
		\column{.5\textwidth}
			\centering
			\includegraphics[width=0.75\textwidth]{m3_ei_residual_beforeNafterAlignment.pdf}
	\end{columns}

	$\Rightarrow$ alignment of rotation around X and Y axis
}

\section{Stereo Reconstruction}

\frame{\frametitle{Position Reconstruction with Stereo Readout}
	\centering
	\includegraphics<1>[width=0.8\textwidth]{stereoBoards_angleNshift-crop.pdf}
	\includegraphics<2>[width=0.8\textwidth]{stereoBoards_angleNshift_centerNcoord-crop.pdf}
	\vspace{2mm}
			
	\begin{columns}
		\column{.5\textwidth}
			\centering
			\includegraphics<1>[width=0.9\textwidth]{m1_stereo_posDifVSscinX.pdf}
			\includegraphics<2>[width=0.9\textwidth]{m1_stereo_posDifVSscinX_wCenter.pdf}
		\column{.5\textwidth}
			\textcolor{blue}{precision} 
			
			\hspace{1mm} = stereos mean / $\tan \alpha$ 
			\vspace{1.5mm}
						
			\textcolor{red}{non-precision}
			
			\hspace{1mm} = stereos difference / 2 $/ \tan \textcolor{Brown}{\alpha}$
			\vspace{1.5mm}
						
			$\Rightarrow$ \textcolor{Brown}{alignment} 
			
			\hspace{4.5mm} non-precision coordinate
	\end{columns}
}

\frame{\frametitle{Stereo Residual Dependence on Track Inclination}
	
	\begin{columns}
		\column{.5\textwidth}
			\centering
			spherical coordinates
			
			\includegraphics[width=0.8\textwidth]{Kugelkoord-def.pdf}
			
			{\tiny (image: wikipedia)}
			
			\vspace{5mm}
			non-precision correction
			\vspace{1.5mm}
						
			= $\dfrac{\bigtriangleup z \cdot \tan \theta \cdot \sin \phi}{2 \cdot \tan \alpha}$
			
		\column{.5\textwidth}
			\centering
			\includegraphics<1>[width=0.7\textwidth]{m1_stereo_resXvsTheta_woAddTerm_rebin.pdf}
			\includegraphics<2>[width=0.7\textwidth]{m1_stereo_resXvsTheta_likeBF_zcor.pdf}
						
			\includegraphics<1>[width=0.7\textwidth]{m1_stereo_resXvsPhi_woAddTerm_rebin.pdf}
			\includegraphics<2>[width=0.7\textwidth]{m1_stereo_resXvsPhi_likeBF_zcor.pdf}
	\end{columns}
}

\frame{\frametitle{\large Influence of Scintillator Resolution on Stereo Layer Residual}
	\footnotesize
	\begin{columns}
		\column{.5\textwidth}
%			\centering	
			\includegraphics[width=0.9\textwidth]{m1_si_resVSposAlongStripsByStereos_slopeX1e-2.pdf}
			\vspace{3mm}
			
			scintillator information ONLY
			
			 $\Rightarrow$ segmentation visible
		\column{.5\textwidth}
%			\centering	
			\includegraphics[width=\textwidth]{m1_si_resdiual_wNwoNewXtrack.pdf}
			\vspace{10mm}
			
			improved X track by stereo information
			
			$\Rightarrow$ only required if module not aligned well
	\end{columns}
}

\section{Signal Time Evaluation}

\frame{\frametitle{Influence of (Online) Zerosuppression}
	\centering	
	\large
	event-wise zerosuppression
	\vspace{2mm}
	
	\begin{columns}
		\column{.33\textwidth}
			\centering	
			\includegraphics[width=0.85\textwidth]{m3_ei_hits_550V_1201.pdf}
		\column{.33\textwidth}
			\centering	
			\includegraphics[width=0.85\textwidth]{m6_eo_chargeVSsignalSTDV_oldZeroSup.pdf}
		\column{.33\textwidth}
			\centering	
			\includegraphics[width=0.85\textwidth]{m6_ei_risetimeVScharge_550V_1201.pdf}
	\end{columns}
	
	\vspace{2mm}
	pedestal by rawdata
	\vspace{2mm}
	
	\begin{columns}
		\column{.33\textwidth}
			\centering	
			\includegraphics[width=0.85\textwidth]{m3_ei_hits_560V_1401.pdf}
		\column{.33\textwidth}
			\centering	
			\includegraphics[width=0.85\textwidth]{m6_eo_chargeVSsignalSTDV_pedestalFile.pdf}
		\column{.33\textwidth}
			\centering	
			\includegraphics[width=0.85\textwidth]{m6_ei_risetimeVScharge_560V_1401.pdf}
	\end{columns}
}

\frame{\frametitle{Strip Time Spectra for Ar:CO$_{2}$ 93:7 vol\% @ $U_{\mathrm{drift}} = $\SI{300}{V}}
%		\column{.5\textwidth}
%			\centering
%			single strip signal
%			
%			\includegraphics[width=0.75\textwidth]{pulseheight_differentSignaltimes.pdf}
	
	\begin{columns}
		\column{.33\textwidth}
			\centering
			inflection
			
			\includegraphics[width=\textwidth]{m1_ei_stripTime_firstNlast_20180601_turntime_wDifferences.pdf}
			
			\includegraphics[width=0.8\textwidth]{m1_ei_timeDifVsangle_20180601_turntime.pdf}
		\column{.33\textwidth}
			\centering
			baseline
			
			\includegraphics[width=\textwidth]{m1_ei_stripTime_firstNlast_20180601_wFWHM_wMaxDist.pdf}
			
			\includegraphics[width=0.8\textwidth]{m1_ei_timeDifVsangle_20180601.pdf}
		\column{.33\textwidth}
			\centering
			maximum
			
			\includegraphics[width=\textwidth]{m1_ei_stripTime_firstNlast_20180601_up_differences.pdf}
			
			\includegraphics[width=0.8\textwidth]{m1_ei_timeDifVsangle_20180601_up.pdf}
	\end{columns}
	
}

\frame{\frametitle{\textmu TPC Angle Reconstruction (nominal $v_{\mathrm{drift}}$)}
	
	\begin{columns}
		\column{.33\textwidth}
			\centering
			inflection
			
			\includegraphics[width=0.9\textwidth]{m1_ei_uTPCangleVSangle_20180601_turntime.pdf}
		\column{.33\textwidth}
			\centering
			baseline
			
			\includegraphics[width=0.9\textwidth]{m1_ei_uTPCangleVSangle_20180601.pdf}
		\column{.33\textwidth}
			\centering
			maximum
			
			\includegraphics[width=0.9\textwidth]{m1_ei_uTPCangleVSangle_20180601_uptime.pdf}
	\end{columns}
		
	\vspace{5mm}
	\begin{columns}
		\column{.33\textwidth}
			\centering
			\footnotesize
			reference angle $\in $ [ \SI{20}{\degree} , \SI{22}{\degree} ]
			
			\includegraphics[width=\textwidth]{m1_ei_uTPCangle_20-22degree_20180601_differentSignaltimes.pdf}
		\column{.33\textwidth}
			\centering
			most probable value
			
			\includegraphics[width=\textwidth]{m1_ei_20180601_uTPCangleMPVvsAngle_differentSignaltimes.pdf}
		\column{.33\textwidth}
			\centering
			width
			
			\includegraphics[width=\textwidth]{m1_ei_20180601_uTPCangleWidthVSangle_differentSignaltimes.pdf}
	\end{columns}
}

\frame{
	
	code can be found at:
	
	https://github.com/Harsger/analysis
	
	\vspace{10mm}
	
	talk on last muon week
	
	\tiny
	https://indico.cern.ch/event/763166/contributions/3179490/attachments/1741327/2817416/SM2inCRF\_MH\_MuonWeekOct18.pdf
	
}

\appendix

\frame{\centering \Huge Backup}

\frame{\frametitle{Recapitulation \textmu TPC Formulas}
	\centering
	\includegraphics[width=0.7\textwidth]{timeVSstrip_meshNcathodeNstrips_fitNtimeNangle.pdf}
			
	\begin{itemize}
		\item
			angle reconstruction: 
			{\boldmath$\textcolor{Plum}{\theta}$}$\; = \arctan \left( \tfrac{\mathrm{pitch}}{\textcolor{red}{\mathrm{slope}} \;\cdot\; v_{\mathrm{drift}}\;\cdot\;25 \mathrm{ns}} \right)$
			
			\vspace{3mm}
		\item
			position reconstruction:
			$
			\mathrm{pos}_{\mu\mathrm{TPC}}
			=
			\dfrac{\textcolor{Cyan}{t_{\mathrm{mid}}} - \textcolor{red}{\mathrm{intercept}}}{\textcolor{red}{\mathrm{slope}}}
			\cdot \textcolor{brown}{\mathrm{pitch}}
			$
	\end{itemize}
	
}

\frame{\frametitle{Determination of $t_{\mathrm{mid}}$ via Residual Dependence}
	\centering
	\textmu TPC residual VS $1 /\!$ slope ($\propto \tan(\theta)$)
	\vspace{2mm}
	
	\begin{columns}
		\column{.5\textwidth}
			\centering
			\includegraphics[width=0.8\textwidth]{m1_ei_uTPCresVSuTPCslope_0601_logz.pdf}
		\column{.5\textwidth}
			\centering
			\includegraphics[width=0.8\textwidth]{m1_ei_uTPCresVSuTPCslope_0601_zoom.pdf}
	\end{columns}
	\vspace{4mm}
	
	$\mathrm{residual} = \dfrac{t_{\mathrm{mid}} - \mathrm{intercept}}{\mathrm{slope}} \cdot \mathrm{pitch}\; - \mathrm{reference}$
}

\frame{\frametitle{Concept for Charge Weighted Timing Reconstruction}

	\centering
	\includegraphics<1,2>[width=0.5\textwidth]{inhomogeneousIonization_bE_cNm_timeNcentroidShift_deltaXnTnAngle.pdf}
	
	\only<1>{
		\vspace{4mm}
		\begin{tabular}{ccccc}
			$\bigtriangleup x$ & = & $\bigtriangleup z$ & $\cdot$ & $\tan \theta$
			\\
			 & & & &
			\\
			 & = & $( t_{\mathrm{c}} - t_{\mathrm{mid}} ) \cdot v_{\mathrm{drift}}$ &  $\cdot$ & $\mathrm{slope}_{\mathrm{ref}}$
		\end{tabular}
	}
	\only<2>{
		\vspace{4mm}
		\begin{tabular}{ccccc}
			$\mathrm{centroid}$ & = & $\sum\limits_{\scriptscriptstyle\mathrm{strips}} \mathrm{strip} \cdot q_{\scriptscriptstyle\mathrm{strip}}$ & $/$ & $\sum\limits_{\scriptscriptstyle\mathrm{strips}} q_{\scriptscriptstyle\mathrm{strip}}$
			\\
			 & & & &
			\\
			$t_{\mathrm{c}}$ & = & $\sum\limits_{\scriptscriptstyle\mathrm{strips}} t_{\scriptscriptstyle\mathrm{strip}} \cdot q_{\scriptscriptstyle\mathrm{strip}}$ &  $/$ & $\sum\limits_{\scriptscriptstyle\mathrm{strips}} q_{\scriptscriptstyle\mathrm{strip}}$
		\end{tabular}
	}
			
}

\frame{\frametitle{Residual Dependence on Charge Averaged Clustertime}
	
	\begin{columns}
		\column{.5\textwidth}
			\centering
			$\mathrm{angle}_{\mathrm{ref}} = -26.4^{\circ}$
						
			$\mathrm{slope}_{\mathrm{ref}} = -0.46$
			\vspace{2.5mm}
			
			\includegraphics[width=0.72\textwidth]{m3_ei_C150V_resVSclutime_slope-dot46.pdf}
		\column{.5\textwidth}
			\centering
			fitted correlation VS $\mathrm{slope}_{\mathrm{ref}}$
			\vspace{5mm}
			
			\includegraphics[width=\textwidth]{m3_ei_driftScan_slopeFitresVSclutime.pdf}
	\end{columns}
	\vspace{4mm}
	
	\centering
	
	$\bigtriangleup x = ( t_{\mathrm{c}} - t_{\mathrm{mid}} ) \cdot v_{\mathrm{drift}} \cdot \mathrm{slope}_{\mathrm{ref}}$
	
	\vspace{3mm}
	
	$\Rightarrow v_{\mathrm{drift}} = \dfrac{\bigtriangleup x}{( t_{\mathrm{c}} - t_{\mathrm{mid}} )}\;\; / \;\; \mathrm{slope}_{\mathrm{ref}}$
}

\frame{\frametitle{\large Resolution with Charge Averaged Clustertime Correction}
	
	\begin{columns}
		\column{.5\textwidth}
			\centering
			\includegraphics[width=0.9\textwidth]{m1_ei_allResolutionsVSangle_20180601.pdf}
		\column{.5\textwidth}
			\centering
			\includegraphics[width=\textwidth]{m3_driftVelocitiesFromCTCnSimulation.pdf}
	\end{columns}
	
	\vspace{4mm}
	\small
	
	improvement of centroid resolution for large angles
	
	$\Rightarrow$ better resolution expected (see Bernhard's Thesis)
	\vspace{4mm}
	
	reconstruction of drift velocities fails
	
	$\Rightarrow$ systematics can not yet be excluded
}

\frame{\frametitle{Influence of Multiple Scattering}
	\footnotesize
	
	\begin{columns}
		\column{.50\textwidth}
			\centering
			MDT residual width VS angle
						
			\includegraphics[width=0.8\textwidth]{MDTresidualVSangle_slopeDifCuts.pdf}
		\column{.50\textwidth}
			\centering
			MDT residual perpendicular incident
						
			\includegraphics[width=0.8\textwidth]{MDTresidual_slopeDifCuts.pdf}
	\end{columns}
	\vspace{3mm}
		
	\begin{columns}
		\column{.50\textwidth}
			\centering
			Module 0 residual narrow width VS angle
						
			\includegraphics[width=0.8\textwidth]{SM2-M0_CRFafterH8_eta-out_residualVSangle_slopeDifCuts.pdf}
		\column{.50\textwidth}
			\centering
			Module 0 residual perpendicular incident
						
			\includegraphics[width=0.8\textwidth]{SM2-M0_CRFafterH8_eta-out_residual_cutsMDTslopeDif.pdf}
	\end{columns}
}

\frame{\frametitle{Usage of Multiple Scattering - Muon Tomography (L1)}

	\begin{columns}
		\column{.50\textwidth}
			\centering
			\includegraphics[width=0.85\textwidth]{trackIntercept_yz_3e-2_1e-1_blackWhite.pdf}
		\column{.50\textwidth}
			\centering
			\includegraphics[width=0.7\textwidth]{CRFmounTomography_yz-crop.pdf}
	\end{columns}
	
	\begin{columns}
		\column{.50\textwidth}
			\centering
			\includegraphics[width=0.9\textwidth]{scatteredFraction_xy_blackWhite.pdf}
		\column{.50\textwidth}
			\centering
			\includegraphics[width=0.9\textwidth]{L1inCRF_muonTomoTop-crop.pdf}
	\end{columns}
	
}

\frame{\frametitle{\large Correction due to Capacitive Coupling of Neighboring Strips}

	charge spread due to capacitive coupling between resistive/readout strips
	
	\vspace{3mm}
	
	\begin{columns}
		\column{.50\textwidth}
			\centering
			\textbf{concept} \hspace{8mm}
			
			\includegraphics[width=0.9\textwidth]{capacitiveCoupling_withResStrips_smallQ.png}
		\column{.50\textwidth}
			\centering
			\textbf{simulation} \hspace{9mm}
			\vspace{3mm}
			
			\includegraphics[width=0.9\textwidth]{capacitveCoupling_circuit_simulation.png}
	\end{columns}
	
	\vspace{5mm}
	
	\textbf{implementation}
	
	\vspace{1mm}
	
	\footnotesize
		
	\textbf{loop} strips in cluster
	
	\hspace{3mm} \textbf{loop} timebins from signalstart to maximum
	
	\hspace{3mm} \hspace{3mm} \textbf{loop} neighbors from 1 to 3
	
	\hspace{3mm} \hspace{3mm} \hspace{3mm} neighbor charge \hspace{5mm} - 0.29$^n$ central strip charge
	
	\hspace{3mm} \hspace{3mm} \hspace{3mm} central strip charge + 0.29$^n$ central strip charge
	
}

\frame{\frametitle{Impact of Capacitive Coupling Correction}

	\begin{columns}
		\column{.5\textwidth}
			\centering
			reference angle $\in$ $[ 20^{\circ} , 22^{\circ} ]$
			
			\includegraphics[width=0.85\textwidth]{m1_ei_uTPCangle_20to22degree_20180601_CCCscan.pdf}
			
			\textmu TPC position resolution
			
			\includegraphics[width=0.85\textwidth]{m1_ei_uTPCresolutionVSrefAngle_20180601_CCCscan.pdf}
		\column{.5\textwidth}
			\centering
			
			\textmu TPC angular resolution
			
			\includegraphics[width=0.8\textwidth]{m1_ei_uTPCangleWidthVSrefAngle_20180601_CCCscan.pdf}
			
			residual dependence on clustertime
			
			\includegraphics[width=0.85\textwidth]{m1_ei_uTPCresVSclutimeSlope_20180601_CCCscan_better.pdf}
	\end{columns}
	
}

\end{document}
