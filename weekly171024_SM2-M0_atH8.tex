\documentclass{beamer}
\usetheme{Madrid}
\usecolortheme{spruce}
\definecolor{green(pigment)}{rgb}{0.0, 0.65, 0.31}
\setbeamercolor*{item}{fg=green(pigment)}
\definecolor{Green}{rgb}{0.00, 1.00, 0.00}
\definecolor{Red}{rgb}{1.00, 0.00, 0.00}
\definecolor{Blue}{rgb}{0.00, 0.00, 1.00}
\usepackage[utf8]{inputenc}
\usepackage{amsmath}
\usepackage{amsfonts}
\usepackage{amssymb}
\usepackage[german]{babel}
\usepackage{graphicx}
\usepackage{rotating}
\usepackage{textcomp}
\usepackage{multirow,bigdelim,dcolumn,booktabs}
%\usepackage{beamerthemeshadow}
\usepackage{subfigure} 
\usepackage{siunitx}
\usepackage{appendixnumberbeamer}
\usepackage{hyperref}
%\beamersetuncovermixins{\opaqueness<1>{25}}{\opaqueness<2->{15}}
\beamertemplatenavigationsymbolsempty

\usepackage{tikz}
\usetikzlibrary{decorations.text}
\usetikzlibrary{trees}
\usetikzlibrary{decorations.pathmorphing}
\usetikzlibrary{decorations.markings}
\usetikzlibrary{patterns}

\graphicspath{
	{/home/m/Maximilian.Herrmann/Bilder/forLatex/plots/}
	{/home/m/Maximilian.Herrmann/Bilder/forLatex/sketches/}
	{/home/m/Maximilian.Herrmann/Bilder/forLatex/pictures/}
}

\begin{document}
\title[SM2 Module 0 @ H8 2017]{\textmu TPC Reconstruction with the SM2 Module 0}  
\author[M. Herrmann]{ Maximilian Herrmann}
\institute[LMU Munich]{Ludwig-Maximilians-Universit\"at M\"unchen - Lehrstuhl Schaile}
\date[24.10.2017 MM Weekly]{24.10.2017, MicroMegas Weekly Meeting} 

\frame{\titlepage} 

%\frame{\frametitle{Outline}\tableofcontents}

\frame{\frametitle{Large Area Homogeneous Pulse Height Distribution in the Cosmic Ray Test Facility in Munich}

	\begin{columns}
		\column{.33\textwidth}
			\centering
			\textbf{eta out {\color{red} M0}}
						
			\includegraphics[width=0.9\textwidth]{SM2-M0_CRFafterH8_eta-out_meanCluQ_activeArea.pdf}
		\column{.33\textwidth}
			\centering
			\textbf{eta in}
			
			\includegraphics[width=0.9\textwidth]{SM2-M0_CRFafterH8_eta-in_meanCluQ.pdf}
		\column{.33\textwidth}
			\centering
			\textbf{stereo in}
			
			\includegraphics[width=0.9\textwidth]{SM2-M0_CRFafterH8_stereo-in_meanCluQ.pdf}
	\end{columns}
	
	\begin{columns}
		\column{.66\textwidth}
			\footnotesize
			\begin{itemize}
				%\item
				%	x reconstruction via scintillator hodoscope
				%	
				%	y reconstruction via two drift tube reference chambers (precision direction)
				%\item
				%	module only inside of greenish areas 
				%	
				%	$\Rightarrow$ outside electronic noise
				\item
					2 MDT chambers $\Rightarrow$ y precision coordinate 
					
					Scintillators \hspace{7mm} $\Rightarrow$ x
				\item
					2/3 of the active area visible (current trigger problem)
				\item
					HV: \SI{600}{V}, one sector at \SI{560}{V} (stereo out)
				\item	
					homogeneous pulse height (some inefficient spots)
				\item	
					measurement duration: 12 hour
			\end{itemize}
		\column{.33\textwidth}
			\centering
			\textbf{stereo out}
						
			\includegraphics[width=0.9\textwidth]{SM2-M0_CRFafterH8_stereo-out_meanCluQ.pdf}
	\end{columns}
}

\frame{\frametitle{\textmu TPC Analysis of SM2 Module 0 at the H8 Beamtime}

	\begin{columns}
		\column{.55\textwidth}
			\centering
						
			\includegraphics[width=0.9\textwidth]{timevsstrip_description.pdf}
		\column{.45\textwidth}
			\begin{itemize}
				\item
					angle reconstruction: 
						
					\vspace{2mm}
					
					$\theta = \tfrac{\mathrm{pitch}}{m_{\mu \mathrm{TPC}} \;\; v_{\mathrm{drift}}}$
						
					\vspace{2mm}
					
					$m_{\mu \mathrm{TPC}}$ : slope \textmu TPC fit
				\item
					position reconstruction:
						
					\vspace{2mm}
								
					$
					\mathrm{pos}_{\mu\mathrm{TPC}}
					=
					\dfrac{t_{0} - t_{\mu \mathrm{TPC}}}{m_{\mu \mathrm{TPC}}}
					$
					
					$t_{\mu \mathrm{TPC}}$ : intercept $\mu \mathrm{TPC}$ fit
			\end{itemize}
	\end{columns}
	
	\vspace{3mm}
	
	\begin{itemize}
		\item
			determination of $t_{0}$
		\item
			optimization of timing between master/slave APV (PLL phase)
		\item
			time jitter correction (\SI{40}{MHz} clock of APV/FEC)
		\item
			capacitive coupling between strips
	\end{itemize}
}

\frame{\frametitle{two possibilities for determination of $t_{0}$}

	\footnotesize
	
	\begin{columns}
		\column{.50\textwidth}
			\centering
			\textbf{timing earliest and latest strips}
			
			\includegraphics[width=0.9\textwidth]{SM2-M0_H8run109-angle20_stripTime_fasNslow.pdf}
			
			\vspace{3mm}
						
			$t_{0} = \tfrac{1}{2} \cdot \left( \overline{t}_{\mathrm{early}} + \overline{t}_{\mathrm{late}} \right)$
						
			\begin{itemize}
				\item
					offline determination:
					
					calculated with distributions of 
					
					earliest and latest strip
				\item
					online determination:
					
					for each event separately by current earliest and latest strip
			\end{itemize}
		\column{.50\textwidth}
			\centering
			\textbf{residual without $t_{0}$}
			
			\includegraphics[width=0.8\textwidth]{SM2-M0_H8run109-angle20_uTPCresVSuTPCslope_zoomed.pdf}
						
			\vspace{3mm}
			
			$
			\mathrm{res} = 
			\mathrm{pos}_\mathrm{ref} - 
			\underbrace{
				\tfrac{0 - \mathrm{intercept}_{\mu\mathrm{TPC}}}{\mathrm{slope}_{\mu\mathrm{TPC}}}
			}_{\mu \mathrm{TPC \;\; position} \;\; t_{0}=0}
			$ 
			
			\vspace{3mm}
			
			$ \Rightarrow t_{0} = - \mathrm{slope}_\mathrm{distribution} / \mathrm{pitch}$
			
			\vspace{8mm}
	\end{columns}
	
}

\section{PLL phase adjustment}

\frame{\frametitle{Optimization of the PLL phase to minimize eltx. Crosstalk}

	\small

	reconstructed charge and time influenced by differences of master and slave APVs
	
	$\Rightarrow$ adjust PLL phase for homogeneous detector properties
	
	$\Rightarrow$ test with L1 chamber in the cosmic ray facility (CRF) in Munich

	\begin{columns}
		\column{.50\textwidth}
			\centering
			\textbf{PLL0}
						
			\includegraphics[width=0.9\textwidth]{L1_MPVcluQ_pll0.pdf}
		\column{.50\textwidth}
			\centering
			\textbf{PLL3}
			
			\includegraphics[width=0.9\textwidth]{L1_MPVcluQ_pll3.pdf}
	\end{columns}
	
	PLL0 : row difference from master - slave differences
	
	PLL3 : centeral increased cluster charge due to a blow up  
}

\frame{\frametitle{Time Jitter Correction}
	
	\begin{columns}
		\column{.33\textwidth}
			\centering
			\textbf{APV principle}
			\vspace{7.5mm}
			
			\includegraphics[width=0.9\textwidth]{APV_principle.png}
		\column{.34\textwidth}
			\centering
			\textbf{signal fit}
			
			\includegraphics[width=\textwidth]{pulseheight_withExtrapolationLine_wPoint3.pdf}
		\column{.33\textwidth}
			\centering
			\textbf{jitter recording}
			\vspace{5.5mm}
			
			\includegraphics[width=0.8\textwidth]{FECjitterRecording.pdf}
	\end{columns}
		
	\begin{columns}
		\column{.33\textwidth}
			\footnotesize
			\begin{itemize}
				\item
					time jitter due to APV sampling in \SI{25}{ns} steps
				\item
					recorded via TDC
				\item
					slope of 
					
					\textmu TPC residual 
					
					VS time jitter 
					
					distribution 
					
					drift time dependent
					
			\end{itemize}
		\column{.34\textwidth}
			\centering
			
			\includegraphics[width=\textwidth]{jitter_25ns.pdf}
		\column{.33\textwidth}
			\centering
			
			\includegraphics[width=0.9\textwidth]{SM2-M0_H8run109-angle20_uTPCresVSjitter.pdf}
	\end{columns}
}

\frame{\frametitle{Correction due to Capacitive Coupling of Neighboring Strips}

	charge spread due to capacitive coupling between resistive/readout strips
	
	\vspace{3mm}
	
	\begin{columns}
		\column{.50\textwidth}
			\centering
			\textbf{concept} \hspace{8mm}
			
			\includegraphics[width=0.9\textwidth]{capacitiveCoupling_withResStrips_smallQ.png}
		\column{.50\textwidth}
			\centering
			\textbf{LT spice simulation} \hspace{9mm}
			\vspace{3mm}
			
			\includegraphics[width=0.9\textwidth]{capacitveCoupling_circuit_simulation.png}
	\end{columns}
	
	\vspace{1mm}
	
	\textbf{implementation}
	
	\footnotesize
	
	\begin{itemize}
		\item
			correction done per strip $\Rightarrow$ start with earliest strip
		\item
			consider 3 neighboring strips on both sides
		\item
			coupling factor 0.29, 0.29$^2$, 0.29$^3$
		\item
			correct each timebin
		\item
			redo fit of starting times
	\end{itemize}
	
}

\section{Results}

\frame{\frametitle{Tracking for \textmu TPC Residual}
	\begin{columns}
		\column{.50\textwidth}
			\centering
			\textbf{tracking}
			
			\includegraphics[width=0.9\textwidth]{H8tracking-crop.pdf}
			
			\vspace{6mm}
			
			\textbf{residual distribution}
			
			\includegraphics[width=0.8\textwidth]{SM2-M0_H8run109-angle20_uTPCresidual_correction.pdf}
		\column{.50\textwidth}
			\small
			\begin{itemize}
				\item
					\SI{70}{\micro\m} track uncertainty at module for \textbf{5 reference detectors}
				\item
					run parameter:
					\begin{itemize}	
						\item
							\SI{590}{V} amplification voltage
						\item
							\SI{300}{V} drift voltage
						\item
							muons
						\item
							\SI{20}{\degree} incident angle
					\end{itemize}
				\item
					corrections improve width of residual distribution
			\end{itemize}
	\end{columns}	
}

\frame{\frametitle{\textmu TPC Results}
	\begin{columns}
		\column{.50\textwidth}
			\centering
			\textbf{residual distribution}
			
			\includegraphics[width=0.8\textwidth]{SM2-M0_H8run109-angle20_uTPCresidualFitted.pdf}
			
			\textbf{reconstructed angle}
						
			\includegraphics[width=0.8\textwidth]{SM2-M0_H8run109-angle20_uTPCangle_eta-in.pdf}
		\column{.50\textwidth}
			\begin{itemize}	
				\item
					residual distribution fitted with double Gaussian
				\item
					width of narrow Gaussian: \SI{326}{\micro\m}
					
					$\Rightarrow$ improvement of about \SI{140}{\micro\m}
					
					(compared to no corrections)
				\item
					average reconstructed angle: \SI{23}{\degree}
			\end{itemize}
	\end{columns}	
}

\frame{\frametitle{Conclusions}
	\begin{itemize}
		\item
			homogeneous pulse height distribution for the full SM2 Module 0 measured in the cosmic ray test facility
		\item
			\textmu TPC analysis:
			
			\begin{itemize}
				\item
					determination of $t_{0}$
				\item
					optimization of the PLL phase
				\item
					time jitter correction with TDC signal
				\item
					correction of capacitive coupling between neighboring strips
			\end{itemize}
		\item
			improvement of the width of the \textmu TPC residual distribution of about \SI{140}{\micro\m} using the corrections 
	\end{itemize}
}

\appendix

\frame{\centering \Huge Backup}

\frame{\frametitle{Readout with Jitter Correction Possibility}
			\centering
			
			\includegraphics[width=\textwidth]{SM2-M0_H8aug17_triggerNreadout-crop.pdf}
}

\frame{\frametitle{Implementation of Capacitve Coupling Correction}
	
	\begin{itemize}
		\item
			start on the side of the cluster, where the earliest strip should be
		\item
			use signalstart time and maximum time from initial fit 
			
			(without correction)
	\end{itemize}
	\vspace{3mm}
	
	\textbf{loop} strips in cluster
	
	\hspace{3mm} \textbf{loop} timebins from signalstart to maximum
	
	\hspace{3mm} \hspace{3mm} \textbf{loop} neighbors from 1 to 3
	
	\hspace{3mm} \hspace{3mm} \hspace{3mm} neighbor charge \hspace{5mm} - 0.29$^n$ central strip charge
	
	\hspace{3mm} \hspace{3mm} \hspace{3mm} central strip charge + 0.29$^n$ central strip charge
	
	
}

\frame{\frametitle{SM2 Module 0 Behavior at the H8 Beamtime}
	
	\begin{columns}
		\column{.50\textwidth}
			\centering
				\textbf{number of cluster cluster}
				
				\includegraphics[width=0.7\textwidth]{SM2-M0_H8_nCluster_layer-crop.pdf}
		\column{.50\textwidth}
			\centering
			\textbf{strips leading cluster}
			
			\includegraphics[width=0.7\textwidth]{SM2-M0_H8_nStrips_angles-crop.pdf}
	\end{columns}
		
	\begin{columns}
		\column{.50\textwidth}
			\footnotesize
			cuts:
			\begin{itemize}
				\item
					maximum charge $>$ 100 ADC channel
				\item
					rise time $>$ 0.1
					
					(parameter of inverse fermi fit)
				\item
					for cluster no gaps between strips larger than 2
			\end{itemize}
		\column{.50\textwidth}
			\centering
				\textbf{strip time VS maximum charge}
				
				\includegraphics[width=0.7\textwidth]{SM2-M0_H8run109-angle20_stripTimeVScharge.pdf}
	\end{columns}
	
}

\end{document}
