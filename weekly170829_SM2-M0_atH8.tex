\documentclass{beamer}
\usetheme{Madrid}
\usecolortheme{spruce}
\definecolor{green(pigment)}{rgb}{0.0, 0.65, 0.31}
\setbeamercolor*{item}{fg=green(pigment)}
\definecolor{Green}{rgb}{0.00, 1.00, 0.00}
\definecolor{Red}{rgb}{1.00, 0.00, 0.00}
\definecolor{Blue}{rgb}{0.00, 0.00, 1.00}
\usepackage[utf8]{inputenc}
\usepackage{amsmath}
\usepackage{amsfonts}
\usepackage{amssymb}
\usepackage[german]{babel}
\usepackage{graphicx}
\usepackage{rotating}
\usepackage{multirow,bigdelim,dcolumn,booktabs}
%\usepackage{beamerthemeshadow}
\usepackage{subfigure} 
\usepackage{siunitx}
\usepackage{appendixnumberbeamer}
\usepackage{hyperref}
%\beamersetuncovermixins{\opaqueness<1>{25}}{\opaqueness<2->{15}}
\beamertemplatenavigationsymbolsempty

\usepackage{tikz}
\usetikzlibrary{decorations.text}
\usetikzlibrary{trees}
\usetikzlibrary{decorations.pathmorphing}
\usetikzlibrary{decorations.markings}
\usetikzlibrary{patterns}

\graphicspath{
	{/home/m/Maximilian.Herrmann/Bilder/forLatex/plots/}
	{/home/m/Maximilian.Herrmann/Bilder/forLatex/sketches/}
	{/home/m/Maximilian.Herrmann/Bilder/forLatex/pictures/}
}

\begin{document}
\title[SM2 Module 0 @ H8 2017]{Preliminary Results for the SM2 Module 0 at the H8 Testbeam in August 2017}  
\author[M. Herrmann]{ Maximilian Herrmann}
\institute[LMU Munich]{Ludwig-Maximilians-Universit\"at M\"unchen - Lehrstuhl Schaile}
\date[29.08.2017 MM Weekly]{29.08.2017, MicroMegas Weekly Meeting} 

\frame{\titlepage} 

%\frame{\frametitle{Outline}\tableofcontents}

\section{Measurement Setups}

\frame{\frametitle{Measurement Setup}
	\begin{figure}
		\begin{tikzpicture}
			\node[anchor=south west,inner sep=0] (image) at (0,0) {\includegraphics[width=0.5\linewidth]{SM2-M0_atH8_fromAbove_tilted.jpg}};
			\node[anchor=south west,inner sep=0] (image) at (6.3,0) {\includegraphics[width=0.5\linewidth]{SM2-M0_atH8_withTelescope_flipped.jpg}};
			
			%\node[right,black] at (0.5,5.0) {\large Granite Table with CMM};
			%\node[right,black] at (7.0,5.0) {\large CNC Drilling Machine};
			
			%\node[right,blue] at (1.4,3.5) {\large tilt};
			%\draw[arrows=<->,blue,ultra thick](1.7,3.1)--(2.3,3.3);
			%\node[right,blue] at (1.5,2.1) {\large vertical};
			%\draw[arrows=<->,blue,ultra thick](2.7,1.4)--(2.7,1.9);
			%\node[right,blue] at (0.9,0.6) {\large horizontal};
			%\draw[arrows=<->,blue,ultra thick](2.6,0.8)--(1.3,1.5);
		\end{tikzpicture}
	\end{figure}
		
	\begin{itemize}
		\item
			module on movable table for horizontal (along strips) and vertical (perpendicular to strips) translation
		\item
			aluminum frame for fixation and tilting of the module
	\end{itemize}
}

\frame{\frametitle{Setup for Tracking}
	\begin{columns}
		\column{.50\textwidth}
			\centering
			\includegraphics[width=0.9\textwidth]{setup-crop.pdf}
		\column{.50\textwidth}
			\centering
			\includegraphics[width=0.9\textwidth]{SM2-M0_H8aug17_setup.png}
	\end{columns}
	
	\vspace{7mm}
	
	tracking telescope with scintillator trigger:
			
	\begin{itemize}
		\item
			3 twodimensional GEMs
		\item
			2 twodimensional TMMs
	\end{itemize}
}

\frame{\frametitle{Readout}
	\begin{itemize}
		\item
			module fully equipped with 96 APVs
			
			\vspace{3mm}
		\item 
			1024 strips for each layer were read out by two FEC cards
			
			%$\Rightarrow$ 1/3 of the module connected 
			
			%(, which depending on beam position)
						\vspace{3mm}
		\item
			two further FEC cards for 28 APVs of tracking telescope
			
			$\Rightarrow$ 4 FECs read out @ \SI{220}{Hz} %(during spills)
						\vspace{3mm}
		\item
			acquisition of time jitter via TDC (VME)
	\end{itemize}
}

\section{Results}

\frame{\frametitle{High Voltage}
	\centering
	\includegraphics[width=0.75\textwidth]{SM2-M0_H8aug17_voltNcurrentVStime-crop.pdf}
	
	\begin{itemize}
		\item
			still conditioning effects visible, while increasing voltage
		\item	
			small due to module conditioned in Munich
		\item
			all except two sectors worked well at operational voltages
		\item
			at first glance no spills visible in current (analysis ongoing)
	\end{itemize}
	
}

\frame{\frametitle{Pulse Height}
	\centering
	\includegraphics[width=0.75\textwidth]{SM2-M0_H8aug17_ampScan_nice.pdf}
	
	\begin{itemize}
		\item
			gas amplification depended on layer 
		\item
			gas amplification for stereo layers lower
			
			(first panel glued) 
	\end{itemize}
}

\frame{\frametitle{Spatial Resolution}
	\begin{columns}
		\column{.50\textwidth}
			\centering
			track uncertainty
			
			\includegraphics[width=0.8\textwidth]{SM2-M0_H8aug17_trackUncertainty-crop.pdf}
			
			centroid distribution
			
			\includegraphics[width=0.8\textwidth]{SM2-M0_H8aug17_residualDistribution-crop.pdf}
		\column{.50\textwidth}
			\begin{itemize}
				\item
					\SI{70}{\micro\m} track uncertainty for 5 reference detectors at module
				\item
					residual distribution fitted by double Gaussian
					
					$\Rightarrow$ sigmas 
						\begin{itemize}
							\item
								corrected by the track uncertainty 
							\item
								weighted by integral of the respective Gaussian
						\end{itemize}
				\item
					resolution : 86$\pm$\SI{5}{\mu\m}
					
					for eta out @  
					
					amplification \SI{580}{V}
					
					drift \SI{-300}{V}
			\end{itemize}
	\end{columns}	
}



\frame{\frametitle{Dependence on Drift and Amplification Voltage}
	\begin{columns}
		\column{.50\textwidth}
			\centering
			
			\includegraphics[width=0.8\textwidth]{SM2-M0_H8aug17_resolutionVSampVolt-crop.pdf}
			
			\includegraphics[width=0.8\textwidth]{SM2-M0_H8aug17_resolutionVSdriftVolt_threeAngles-crop.pdf}
		\column{.50\textwidth}
			\begin{itemize}
				\item
					for both eta layers the same resolution is achieved
					
					$\Rightarrow$ below \SI{90}{\micro\m}
										
					\vspace{5mm}
				\item
					resolution independent of amplification voltage
					
					\vspace{5mm}
				\item
					resolution independent of drift voltage
			\end{itemize}
	\end{columns}	
}

\frame{\frametitle{Alignment relative to Eta In}
	\begin{columns}
		\column{.50\textwidth}
			\centering
			
			\includegraphics[width=1.0\textwidth]{Stereo_in.png}
			
			\includegraphics[width=1.0\textwidth]{Stereo_out.png}
		\column{.50\textwidth}
					\centering
					
			\includegraphics[width=1.0\textwidth]{Eta_in.png}
			\vspace{10mm}
			
			\begin{itemize}
				\item
					preliminary: deviation to Eta In
				\item
					has to be compared to mechanical measurements from Freiburg
			\end{itemize}
	\end{columns}	
}

\frame{\frametitle{Conclusions}
	\begin{itemize}
		\item
			successful measurements during testbeam due to stable beam conditions and good working Module 0
		\item
			gas flux constant at \SI{5}{ln/h}
		%\item
		%	HV properties after transport to CERN similar to properties in Munich
			
		%	$\Rightarrow$ no new conditioning required
		\item
			optimal amplification voltage depending on layer
			
			$\Rightarrow$ between \SI{580}{V} and \SI{605}{V}
		\item
			spatial resolution better than \SI{100}{\micro\m} for perpendicular incidence angle
		\item
			analysis ongoing
			%TO DO:
			%\begin{itemize}
			%	\item
			%		$\mu$TPC resolution (and combination with centroid)
			%	\item
			%		reconstruction with stereo layers
			%	\item
			%		alignment of the PCBs
			%\end{itemize}
		%\item
		%	some data is stored (already parsed to root trees) at:
			
			%\url{{/afs/cern.ch/work/m/mgherrma/public/H8_aug17}}
		%\item
		%	the Logbook (for an overview of the measurements) can be found at:
			
			%\url{{https://docs.google.com/spreadsheets/d/1DJ7HuTrgH2J9HP0Qs-uLwFL9Ibl65aM765TQcAHnThk/edit#gid=0}}
	\end{itemize}
}

\appendix

\frame{\centering \Huge Backup}

\frame{\frametitle{Italian Analysis}
	\centering
	\includegraphics[width=0.9\textwidth]{ExtrapolationYMod0eta-crop.pdf}
}

\frame{\frametitle{Alignment eta in}
	\centering
	\includegraphics[width=0.9\textwidth]{Eta_in.png}
}

\frame{\frametitle{Alignment Stereo In}
	\centering
	\includegraphics[width=0.9\textwidth]{Stereo_in.png}
}

\frame{\frametitle{Alignment Stereo Out}
	\centering
	\includegraphics[width=0.9\textwidth]{Stereo_out.png}
}

\frame{\frametitle{Measurement Program}
	view from telescope against beam direction
	\vspace{3mm}
	
	\begin{columns}
		\column{.50\textwidth}
			\centering
			\includegraphics[width=0.9\textwidth]{SM2-M0_pointsAtH8_stereofront-crop.pdf}
		\column{.50\textwidth}
			\centering
			\includegraphics[width=0.9\textwidth]{SM2-M0_pointsAtH8_etafront-crop.pdf}
	\end{columns}
	\vspace{3mm}
	
	\begin{itemize}
		\item
			scan of a large part of the active area to investigate the efficiency  
			
			$\Rightarrow$ dead or noisy areas (for example between the PCBs)
		\item
			resolution as function of amplification and drift voltage, as well as incidence angle
		\item
			PCB alignment
	\end{itemize}
}

\end{document}
